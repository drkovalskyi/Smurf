\documentclass{cmspaper}
\usepackage{graphicx}
\usepackage{amsmath}
\usepackage{amssymb}
\usepackage{subfigure}
\usepackage{multirow}
\usepackage[pdfborder=0 0 0,
            colorlinks,
            urlcolor = blue,
            linkcolor = black,
            citecolor = black,
            menucolor = black,]
           {hyperref}
%% \usepackage[colorlinks]{hyperref}
%% \usepackage{url}
\usepackage[toc,page]{appendix}
\usepackage{varioref} 
\renewcommand{\appendixname}{Appendix}
%% \renewcommand{\appendixtocname}{List of appendices}

% % useful definitions

% processes
\def\dyee {\ensuremath{Z/\gamma^*\to ee}}
\def\dymm {\ensuremath{Z/\gamma^*\to\mu\mu}}
\def\dytt {\ensuremath{Z/\gamma^*\to\tau\tau}}
\def\zee {\ensuremath{Z\to ee}}
\def\zmm {\ensuremath{Z\to\mu\mu}}
\def\ztt {\ensuremath{Z\to\tau\tau}}
\def\ttbar {\ensuremath{t\bar{t}}}
\def\wwll {\ensuremath{WW\to l^+l^-}}
\def\wwlulu{\ensuremath{WW\to l^+\nu l^-\bar{\nu}}}
\def\ww {\ensuremath{WW}}
\def\wz{\ensuremath{WZ}}
\def\zz{\ensuremath{ZZ}}
\def\wgamma{\ensuremath{W\gamma}}
\def\wjets{\ensuremath{W+}jets} 
\def\tw{\ensuremath{tW}} 
\def\singletopt{\ensuremath{t} ($t$-chan)} 
\def\singletops{\ensuremath{t} ($s$-chan)} 
\def\all{all}
\def\ee{\ensuremath{ee}}
\def\emu{\ensuremath{e\mu}}
\def\mm{\ensuremath{\mu\mu}}

%units

%others
\def\pt{\ensuremath{p_T}}
\def\ipb{pb\ensuremath{^{-1}}}
\def\ifb{fb\ensuremath{^{-1}}}
\def\et{\ensuremath{E_T}}
\def\met{\ensuremath{E\!\!\!\!/_T}}
\def\fBrem{\ensuremath{f_{\rm brem}}}
\def\pin{\ensuremath{p_{\rm in}}}
\def\pout{\ensuremath{p_{\rm out}}}

\input{commands.tex}

\setcounter{topnumber}{1}
\setcounter{bottomnumber}{1}
\setcounter{secnumdepth}{6}
%===================================================================================================
\begin{document}
\begin{titlepage}

  \analysisnote{2012/XXX}

  \date{\today}

  \title{A 2-dimensional Shape Analysis for the Higgs Boson Search in the Fully Leptonic $W^+W^-$ Final State}

  \input{authors.tex}

  \begin{abstract}
    This note describes an analysis method for the Higgs boson decay $W^+W^- \to 2\ell2\nu$
    using the di-lepton invariant mass and transverse mass as discriminating variables.
    The two dimensional distribution of the data in these two variables is used to separate signal
    from background. The method is applied to the most sensitive analysis categories containing
    different flavor leptons (electrons or muons).
    We report the expected
    significance for the Higgs boson signal in a dataset corresponding to 
    \intlumiEightTeV~ at $\sqrt s = 8~\TeV$ as well as projecting to 20~\ifb.
  \end{abstract} 

\end{titlepage}
\tableofcontents
%\listoftables
%\listoffigures
\newpage 

%===================================================================================================
\section{Introduction}
  \label{sec:overview}
  Drell-Yan (\dyll) events represent the major background to a \hww~ signal in the same flavor final state.
After applying tight cuts on \met, they are highly suppressed but the expected signal yield is significantly 
reduced and the remaining \dyll~ background is difficult to estimate.
The net effect is that the sensitivity of the \hww~ analysis is dominated by the opposite flavor final state.

In the published 2011 analysis\cite{ref:hwwpaper}, the \dyll~ background is estimated using a method called \routin\cite{ref:hwwsmurfs}, 
which extrapolates to the signal region the \dyll~ yield in the $Z$ peak region. 
Results from the \routin method are generally stable but suffer from large statistical and systematic uncertainties.

In addition, the analysis relied on MC for deriving the shapes used in the BDT analysis; 
given that the Drell-Yan Monte Carlo sample with generator cut \mll$<$20 \GeVcc contained too few events and the statistical uncertainty 
in that region was too large, in the same flavor final state a \mll$>$ 20\GeVcc cut was applied and a significant fraction 
of a low-mass Higgs signal was lost.

The present note describes a new method for \dyll~estimation constituting a valid alternative 
to \routin, that can cross-check its results, provide shapes from data and possibly reduce the uncertainties.
The main idea is to use a ``fake-rate'' mathod, where the rate for \dyll\ events passing the final \met\ selection
is evaluated on a \gjets\ sample and applied to same flavor dilepton events in a loose \met\ region. 


%
\newpage
\section{Method}
  \label{sec:method}
  This section describes the details of the 2D analysis method:

\begin{itemize}
    \item Definition of variables used.
    \item Construction of 2D templates.
    \item Unrolling (a technical note on implementing a binned 2D shape analysis
by using the standard Higgs group combination tools).
    \item Illustration of templates of all processes.
    \item Systematic uncertainties on 2D templates.
\end{itemize}



  \subsection{Definition of variables}
  \label{sec:choice_var}
  We use \mll~and \mt~to construct 2D templates. Exact definitions of 
them are as follows.  

%
\begin{itemize}
\item the dilepton mass $\mll$;
\item transverse Higgs mass, 
$\mt^{\ell\ell\met} = \sqrt{2\pt^{ll}\met(1-cos(\Delta\phi_{\ell\ell-\met}))}$ where 
$\Delta\phi_{\ell\ell-\met}$ is the angle between dilepton
direction and \met in the transverse plane.
\end{itemize} 

Figure~\ref{fig:templates_125_ex} shows 2D templates for 
signal at \mHi = 125 \GeV~(left) and \qqww (right).

%
\begin{figure}[!hbtp]
	
	%
	\centering
	\subfigure[Signal]{
	\centering
	\label{subfig:template_signal_125}
		\includegraphics[width=.35\textwidth]{figures/templates/sig_2D_mH125_0j_of.pdf}
	}
	\subfigure[\qqww]{
	\centering
	\label{subfig:template_qqWW_125}
		\includegraphics[width=.35\textwidth]{figures/templates/qqWW_2D_mH125_0j_of.pdf}
	}
	
	\caption{2D templates at \mHi = 125 \GeV} 
	\label{fig:templates_125_ex}

\end{figure}
	

  
  \subsection{Consturction of templates}
  \label{sec:evt_selection}
  The 2D templates are made with selections to increase S/B.   
As done in the cut-based and BDT-based analyses, we first 
choose events with two W bosons \cite{HWW2012ICHEP}. 
%Given the different kinematics in different \mHi hypotheses, 
%we make low-\mHi ($\mHi<300~\GeV$) and high-\mHi ($\mHi\ge 300~\GeV$) templates individually.   
The motivation of 2D analysis is to have a single template 
across all \mHi hypotheses. However, since event kinematics changes by \mHi, 
we make separate templates for low-\mHi ($\mHi<300~\GeV$) and high-\mHi ($\mHi\ge 300~\GeV$) hypotheses.   
On top of WW preselection, we apply  $\ptlmax > 50~\GeV$ for high-\mHi~hypotheses.
For VBF channel, we apply additional requirements as follows to select VBF events. 

\begin{itemize}
	\item two reconstructed jets with $\pt~>~30~\GeV$ and no other jets between
		  them with $\pt~>~30~\GeV$;
    \item neither of the jets must be $b$-tagged;
  	\item $\Delta\eta (j_1-j_2) > 3.5$;
    \item $m_{j_1j_2} > 500\:\GeV$; 
    \item $30~\GeV < m_{T}^{\ell\ell\met}$.
\end{itemize}  

In order to perform a maximum likelihood fit with templates, 
it is important to avoid empty bins in the background templates. 
Therefore, one can not have as small bins as they want. 
Also, if S/B does not change by making bin size smaller
then we can use larger bins to have better statistics in each bin. 
In appendix~\ref{app:binsize} it is shown that S/B stays at the same level 
by changing the bin size by factor 4. 

\begin{table}[!htb] 
	\centering
	\begin{tabular}{c | c | c | c }
   	\hline \hline
	Channel 					& Variable	& $\mHi< 300~\GeV$ 	& $\mHi\ge 300~\GeV$ 	\\ 
   	\hline \hline
	\multirow{2}{*}{0/1 jet}  	& \mt 		& [80,280] 10 bins 	& [80,380] 10 bins		\\	
	 							& \mll 		& [0,200] 8 bins	& [0,600] 8 bins		\\	
   	\hline
	\multirow{2}{*}{VBF}  		& \mt 		& [40,280] 4 bins 	& [30,330] 2 bins		\\	
	 							& \mll 		& [0,200] 4 bins	& [0,450] 3 bins		\\	
   	\hline \hline
	\end{tabular}
	\label{tab:binning_range}
	\caption{Summary of binning and range. For high-\mHi~templates, overflow to \mt/\mll=600~\GeV~ is allowed.}
\end{table}

Considering all these, we construct templates as summarized in Table~\ref{tab:binning_range}.

  
  \subsection{Unrolling}
  \label{sec:unrolling}
  In order to perform a statistical interpretation, we have to prepare cards 
along with central and alternative shapes(histograms). The available tools \cite{lands}, \cite{combine}
do not take 2-dimensional histogram as an input, but 1-dimensional histograms
transformed from 2-dimensional ones. Since, the tools do nat care about 
how bins are arranged, we can unroll a 2-dimensional histogram to 1-dimensional one.

\begin{figure}[htp]
	\centering
	\includegraphics[width=0.8\textwidth]{figures/unrolling_schematic.pdf}
	\caption{ Schematic of how unrolling is done from 4x4 2D to 1D histogram. 
			  Same numbers in the 2D and 1D histograms indicates same bins.} 
  	\label{fig:unrolling_sch}
\end{figure}

Figure~\ref{fig:unrolling_sch} shows how unrolling is done for this study. 
In case of a 4x4 2D template, the first column in the 2D histogram 
becomes the 1st - 4th bins in the 1D histogram. The second column becomes 
5th - 8th bins in the 1D. Same thing goes until the last column.  

\begin{figure}[htp]
	\centering
	\includegraphics[width=0.8\textwidth]{figures/unrolling_example.pdf}
	\caption{ Example of how unrolling is done from 4x4 2D to 1D histogram.} 
  	\label{fig:unrolling_ex}
\end{figure}

Figure~\ref{fig:unrolling_ex} shows an example of 2D template and 
the corresponding unrolled template.  

On technical point of view, we can replace BDT template by unrolled 
template and run the same machinery.  


  
  \subsection{Templates}
  \label{sec:templates}
  Having decided on the template bin size and range, we now illustrate
the templates for signal and background processes.  This shows
that the main signal and background templates are sufficiently 
filled and bins which are empty or have a very large statistical
uncertainty are avoided where possible.

The 2D templates (left) and their statistical uncertainties (right) 
in the zero jet channel are shown in 
Figures \ref{fig:templates_125_0j_1} - \ref{fig:templates_125_0j_3}
for the loss mass (signal \mHi = 125 \GeV~shown) analysis and in Figures
\ref{fig:templates_600_0j_1} - \ref{fig:templates_600_0j_3}
for the high mass (signal \mHi = 600 \GeV~shown) analysis.
Likewise, the templates for the VBF channel 
are shown in Figures  \ref{fig:templates_125_vbf_1} - \ref{fig:templates_125_vbf_3}
for the low mass analysis and 
in Figures \ref{fig:templates_600_vbf_1} - \ref{fig:templates_600_vbf_3} 
for the high mass analysis.
The statistical uncertainties are relative to the total background yields.

%%% mH=125 0jet
\begin{figure}[!hbtp]
	
	%
	\centering
	\subfigure[Signal]{
	\centering
	\label{subfig:template_signal_125}
		\includegraphics[width=.35\textwidth]{figures/templates/sig_2D_mH125_0j_of.pdf}
	}
	\subfigure[Signal statistical uncertainty]{
	\centering
	\label{subfig:template_signalerr_125}
		\includegraphics[width=.35\textwidth]{figures/templates/sigerr_2D_mH125_0j_of.pdf}
	}
	
	%
	\centering
	\subfigure[qqWW]{
	\centering
	\label{subfig:template_qqWW_125}
		\includegraphics[width=.35\textwidth]{figures/templates/qqWW_2D_mH125_0j_of.pdf}
	}
	\subfigure[qqWW statistical uncertainty]{
	\centering
	\label{subfig:template_qqWWerr_125}
		\includegraphics[width=.35\textwidth]{figures/templates/qqWWerr_2D_mH125_0j_of.pdf}
	}

	%
	\centering
	\subfigure[ggWW]{
	\centering
	\label{subfig:template_ggWW_125}
		\includegraphics[width=.35\textwidth]{figures/templates/ggWW_2D_mH125_0j_of.pdf}
	}
	\subfigure[ggWW statistical uncertainty]{
	\centering
	\label{subfig:template_ggWWerr_125}
		\includegraphics[width=.35\textwidth]{figures/templates/ggWWerr_2D_mH125_0j_of.pdf}
	}

	\caption{2D templates at \mHi = 125 \GeV in 0 jet bin} 
	\label{fig:templates_125_0j_1}

\end{figure}

\begin{figure}[!hbtp]
	
	%
	\centering
	\subfigure[Wjets]{
	\centering
	\label{subfig:template_Wjets_125}
		\includegraphics[width=.35\textwidth]{figures/templates/Wjets_2D_mH125_0j_of.pdf}
	}
	\subfigure[Wjets statistical uncertainty]{
	\centering
	\label{subfig:template_Wjetserr_125}
		\includegraphics[width=.35\textwidth]{figures/templates/Wjetserr_2D_mH125_0j_of.pdf}
	}
	
	%
	\centering
	\subfigure[Top]{
	\centering
	\label{subfig:template_Top_125}
		\includegraphics[width=.35\textwidth]{figures/templates/Top_2D_mH125_0j_of.pdf}
	}
	\subfigure[Top statistical uncertainty]{
	\centering
	\label{subfig:template_Toperr_125}
		\includegraphics[width=.35\textwidth]{figures/templates/Toperr_2D_mH125_0j_of.pdf}
	}

	%
	\centering
	\subfigure[VV]{
	\centering
	\label{subfig:template_VV_125}
		\includegraphics[width=.35\textwidth]{figures/templates/VV_2D_mH125_0j_of.pdf}
	}
	\subfigure[VV statistical uncertainty]{
	\centering
	\label{subfig:template_VVerr_125}
		\includegraphics[width=.35\textwidth]{figures/templates/VVerr_2D_mH125_0j_of.pdf}
	}

	\caption{2D templates at \mHi = 125 \GeV in 0jet bin.} 
	\label{fig:templates_125_0j_2}

\end{figure}

\begin{figure}[!hbtp]
	
	%
	\centering
	\subfigure[Zjets]{
	\centering
	\label{subfig:template_Zjets_125}
		\includegraphics[width=.35\textwidth]{figures/templates/Zjets_2D_mH125_0j_of.pdf}
	}
	\subfigure[Zjets statistical uncertainty]{
	\centering
	\label{subfig:template_Zjetserr_125}
		\includegraphics[width=.35\textwidth]{figures/templates/Zjetserr_2D_mH125_0j_of.pdf}
	}

	%
	\centering
	\subfigure[Wgamma]{
	\centering
	\label{subfig:template_Wgamma_125}
		\includegraphics[width=.35\textwidth]{figures/templates/Wgamma_2D_mH125_0j_of.pdf}
	}
	\subfigure[Wgamma statistical uncertainty]{
	\centering
	\label{subfig:template_Wgammaerr_125}
		\includegraphics[width=.35\textwidth]{figures/templates/Wgammaerr_2D_mH125_0j_of.pdf}
	}

	\caption{2D templates at \mHi = 125 \GeV in 0jet bin.} 
	\label{fig:templates_125_0j_3}

\end{figure} 

\begin{figure}[!hbtp]
	
	%
	\centering
	\subfigure[Stacked unrolled template linear]{
	\centering
	\label{subfig:template_unroll_stack_lin}
		\includegraphics[width=.45\textwidth]{figures/templates/2D_mH125_0j_of_stack_lin.pdf}
	}
	\subfigure[Overlaid unrolled template linear]{
	\centering
	\label{subfig:template_unroll_overlay_lin}
		\includegraphics[width=.45\textwidth]{figures/templates/2D_mH125_0j_of_overlay_lin.pdf}
	}

	%
	\centering
	\subfigure[Stacked unrolled template in log scale]{
	\centering
	\label{subfig:template_unroll_stack_log}
		\includegraphics[width=.45\textwidth]{figures/templates/2D_mH125_0j_of_stack_log.pdf}
	}
	\subfigure[Overlaid unrolled template in log scale]{
	\centering
	\label{subfig:template_unroll_overlay_log}
		\includegraphics[width=.45\textwidth]{figures/templates/2D_mH125_0j_of_overlay_log.pdf}
	}

	\caption{Unrolled templates at \mHi = 125 \GeV in 0jet bin.} 
	\label{fig:templates_125_0j_unroll}

\end{figure} 

\newpage
%%% mH=600 0jet
\begin{figure}[!hbtp]
	
	%
	\centering
	\subfigure[Signal]{
	\centering
	\label{subfig:template_signal_600}
		\includegraphics[width=.35\textwidth]{figures/templates/sig_2D_mH600_0j_of.pdf}
	}
	\subfigure[Signal statistical uncertainty]{
	\centering
	\label{subfig:template_signalerr_600}
		\includegraphics[width=.35\textwidth]{figures/templates/sigerr_2D_mH600_0j_of.pdf}
	}
	
	%
	\centering
	\subfigure[qqWW]{
	\centering
	\label{subfig:template_qqWW_600}
		\includegraphics[width=.35\textwidth]{figures/templates/qqWW_2D_mH600_0j_of.pdf}
	}
	\subfigure[qqWW statistical uncertainty]{
	\centering
	\label{subfig:template_qqWWerr_600}
		\includegraphics[width=.35\textwidth]{figures/templates/qqWWerr_2D_mH600_0j_of.pdf}
	}

	%
	\centering
	\subfigure[ggWW]{
	\centering
	\label{subfig:template_ggWW_600}
		\includegraphics[width=.35\textwidth]{figures/templates/ggWW_2D_mH600_0j_of.pdf}
	}
	\subfigure[ggWW statistical uncertainty]{
	\centering
	\label{subfig:template_ggWWerr_600}
		\includegraphics[width=.35\textwidth]{figures/templates/ggWWerr_2D_mH600_0j_of.pdf}
	}

	\caption{2D templates at \mHi = 600 \GeV in 0 jet bin} 
	\label{fig:templates_600_0j_1}

\end{figure}

\begin{figure}[!hbtp]
	
	%
	\centering
	\subfigure[Wjets]{
	\centering
	\label{subfig:template_Wjets_600}
		\includegraphics[width=.35\textwidth]{figures/templates/Wjets_2D_mH600_0j_of.pdf}
	}
	\subfigure[Wjets statistical uncertainty]{
	\centering
	\label{subfig:template_Wjetserr_600}
		\includegraphics[width=.35\textwidth]{figures/templates/Wjetserr_2D_mH600_0j_of.pdf}
	}
	
	%
	\centering
	\subfigure[Top]{
	\centering
	\label{subfig:template_Top_600}
		\includegraphics[width=.35\textwidth]{figures/templates/Top_2D_mH600_0j_of.pdf}
	}
	\subfigure[Top statistical uncertainty]{
	\centering
	\label{subfig:template_Toperr_600}
		\includegraphics[width=.35\textwidth]{figures/templates/Toperr_2D_mH600_0j_of.pdf}
	}

	%
	\centering
	\subfigure[VV]{
	\centering
	\label{subfig:template_VV_600}
		\includegraphics[width=.35\textwidth]{figures/templates/VV_2D_mH600_0j_of.pdf}
	}
	\subfigure[VV statistical uncertainty]{
	\centering
	\label{subfig:template_VVerr_600}
		\includegraphics[width=.35\textwidth]{figures/templates/VVerr_2D_mH600_0j_of.pdf}
	}

	\caption{2D templates at \mHi = 600 \GeV in 0jet bin.} 
	\label{fig:templates_600_0j_2}

\end{figure}

\begin{figure}[!hbtp]
	
	%
	\centering
	\subfigure[Zjets]{
	\centering
	\label{subfig:template_Zjets_600}
		\includegraphics[width=.35\textwidth]{figures/templates/Zjets_2D_mH600_0j_of.pdf}
	}
	\subfigure[Zjets statistical uncertainty]{
	\centering
	\label{subfig:template_Zjetserr_600}
		\includegraphics[width=.35\textwidth]{figures/templates/Zjetserr_2D_mH600_0j_of.pdf}
	}

	%
	\centering
	\subfigure[Wgamma]{
	\centering
	\label{subfig:template_Wgamma_600}
		\includegraphics[width=.35\textwidth]{figures/templates/Wgamma_2D_mH600_0j_of.pdf}
	}
	\subfigure[Wgamma statistical uncertainty]{
	\centering
	\label{subfig:template_Wgammaerr_600}
		\includegraphics[width=.35\textwidth]{figures/templates/Wgammaerr_2D_mH600_0j_of.pdf}
	}

	\caption{2D templates at \mHi = 600 \GeV in 0jet bin.} 
	\label{fig:templates_600_0j_3}

\end{figure} 

\begin{figure}[!hbtp]
	
	%
	\centering
	\subfigure[Stacked unrolled template linear]{
	\centering
	\label{subfig:template_unroll_stack_lin}
		\includegraphics[width=.45\textwidth]{figures/templates/2D_mH600_0j_of_stack_lin.pdf}
	}
	\subfigure[Overlaid unrolled template linear]{
	\centering
	\label{subfig:template_unroll_overlay_lin}
		\includegraphics[width=.45\textwidth]{figures/templates/2D_mH600_0j_of_overlay_lin.pdf}
	}

	%
	\centering
	\subfigure[Stacked unrolled template in log scale]{
	\centering
	\label{subfig:template_unroll_stack_log}
		\includegraphics[width=.45\textwidth]{figures/templates/2D_mH600_0j_of_stack_log.pdf}
	}
	\subfigure[Overlaid unrolled template in log scale]{
	\centering
	\label{subfig:template_unroll_overlay_log}
		\includegraphics[width=.45\textwidth]{figures/templates/2D_mH600_0j_of_overlay_log.pdf}
	}

	\caption{Unrolled templates at \mHi = 600 \GeV in 0jet bin.} 
	\label{fig:templates_600_0j_unroll}

\end{figure} 

\newpage
%%% mH=125 VBF 
\begin{figure}[!hbtp]
	
	%
	\centering
	\subfigure[Signal]{
	\centering
	\label{subfig:template_signal_125}
		\includegraphics[width=.35\textwidth]{figures/templates/sig_2D_mH125_2j_of.pdf}
	}
	\subfigure[Signal statistical uncertainty]{
	\centering
	\label{subfig:template_signalerr_125}
		\includegraphics[width=.35\textwidth]{figures/templates/sigerr_2D_mH125_2j_of.pdf}
	}
	
	%
	\centering
	\subfigure[qqWW]{
	\centering
	\label{subfig:template_qqWW_125}
		\includegraphics[width=.35\textwidth]{figures/templates/qqWW_2D_mH125_2j_of.pdf}
	}
	\subfigure[qqWW statistical uncertainty]{
	\centering
	\label{subfig:template_qqWWerr_125}
		\includegraphics[width=.35\textwidth]{figures/templates/qqWWerr_2D_mH125_2j_of.pdf}
	}

	%
	\centering
	\subfigure[ggWW]{
	\centering
	\label{subfig:template_ggWW_125}
		\includegraphics[width=.35\textwidth]{figures/templates/ggWW_2D_mH125_2j_of.pdf}
	}
	\subfigure[ggWW statistical uncertainty]{
	\centering
	\label{subfig:template_ggWWerr_125}
		\includegraphics[width=.35\textwidth]{figures/templates/ggWWerr_2D_mH125_2j_of.pdf}
	}

	\caption{2D templates at \mHi = 125 \GeV in VBF channel} 
	\label{fig:templates_125_vbf_1}

\end{figure}

\begin{figure}[!hbtp]
	
	%
	\centering
	\subfigure[Wjets]{
	\centering
	\label{subfig:template_Wjets_125}
		\includegraphics[width=.35\textwidth]{figures/templates/Wjets_2D_mH125_2j_of.pdf}
	}
	\subfigure[Wjets statistical uncertainty]{
	\centering
	\label{subfig:template_Wjetserr_125}
		\includegraphics[width=.35\textwidth]{figures/templates/Wjetserr_2D_mH125_2j_of.pdf}
	}
	
	%
	\centering
	\subfigure[Top]{
	\centering
	\label{subfig:template_Top_125}
		\includegraphics[width=.35\textwidth]{figures/templates/Top_2D_mH125_2j_of.pdf}
	}
	\subfigure[Top statistical uncertainty]{
	\centering
	\label{subfig:template_Toperr_125}
		\includegraphics[width=.35\textwidth]{figures/templates/Toperr_2D_mH125_2j_of.pdf}
	}

	%
	\centering
	\subfigure[VV]{
	\centering
	\label{subfig:template_VV_125}
		\includegraphics[width=.35\textwidth]{figures/templates/VV_2D_mH125_2j_of.pdf}
	}
	\subfigure[VV statistical uncertainty]{
	\centering
	\label{subfig:template_VVerr_125}
		\includegraphics[width=.35\textwidth]{figures/templates/VVerr_2D_mH125_2j_of.pdf}
	}

	\caption{2D templates at \mHi = 125 \GeV in VBF channel.} 
	\label{fig:templates_125_vbf_2}

\end{figure}

\begin{figure}[!hbtp]
	
	%
	\centering
	\subfigure[Zjets]{
	\centering
	\label{subfig:template_Zjets_125}
		\includegraphics[width=.35\textwidth]{figures/templates/Zjets_2D_mH125_2j_of.pdf}
	}
	\subfigure[Zjets statistical uncertainty]{
	\centering
	\label{subfig:template_Zjetserr_125}
		\includegraphics[width=.35\textwidth]{figures/templates/Zjetserr_2D_mH125_2j_of.pdf}
	}

	%
%	\centering
%	\subfigure[Wgamma]{
%	\centering
%	\label{subfig:template_Wgamma_125}
%		\includegraphics[width=.35\textwidth]{figures/templates/Wgamma_2D_mH125_2j_of.pdf}
%	}
%	\subfigure[Wgamma statistical uncertainty]{
%	\centering
%	\label{subfig:template_Wgammaerr_125}
%		\includegraphics[width=.35\textwidth]{figures/templates/Wgammaerr_2D_mH125_2j_of.pdf}
%	}

	\caption{2D templates at \mHi = 125 \GeV in VBF channel.} 
	\label{fig:templates_125_vbf_3}

\end{figure} 

\begin{figure}[!hbtp]
	
	%
	\centering
	\subfigure[Stacked unrolled template linear]{
	\centering
	\label{subfig:template_unroll_stack_lin}
		\includegraphics[width=.45\textwidth]{figures/templates/2D_mH125_2j_of_stack_lin.pdf}
	}
	\subfigure[Overlaid unrolled template linear]{
	\centering
	\label{subfig:template_unroll_overlay_lin}
		\includegraphics[width=.45\textwidth]{figures/templates/2D_mH125_2j_of_overlay_lin.pdf}
	}

	%
	\centering
	\subfigure[Stacked unrolled template in log scale]{
	\centering
	\label{subfig:template_unroll_stack_log}
		\includegraphics[width=.45\textwidth]{figures/templates/2D_mH125_2j_of_stack_log.pdf}
	}
	\subfigure[Overlaid unrolled template in log scale]{
	\centering
	\label{subfig:template_unroll_overlay_log}
		\includegraphics[width=.45\textwidth]{figures/templates/2D_mH125_2j_of_overlay_log.pdf}
	}

	\caption{Unrolled templates at \mHi = 125 \GeV in VBF channel.} 
	\label{fig:templates_125_vbf_unroll}

\end{figure} 

\newpage
%%% mH=600 VBF 
\begin{figure}[!hbtp]
	
	%
	\centering
	\subfigure[Signal]{
	\centering
	\label{subfig:template_signal_600}
		\includegraphics[width=.35\textwidth]{figures/templates/sig_2D_mH600_2j_of.pdf}
	}
	\subfigure[Signal statistical uncertainty]{
	\centering
	\label{subfig:template_signalerr_600}
		\includegraphics[width=.35\textwidth]{figures/templates/sigerr_2D_mH600_2j_of.pdf}
	}
	
	%
	\centering
	\subfigure[qqWW]{
	\centering
	\label{subfig:template_qqWW_600}
		\includegraphics[width=.35\textwidth]{figures/templates/qqWW_2D_mH600_2j_of.pdf}
	}
	\subfigure[qqWW statistical uncertainty]{
	\centering
	\label{subfig:template_qqWWerr_600}
		\includegraphics[width=.35\textwidth]{figures/templates/qqWWerr_2D_mH600_2j_of.pdf}
	}

	%
	\centering
	\subfigure[ggWW]{
	\centering
	\label{subfig:template_ggWW_600}
		\includegraphics[width=.35\textwidth]{figures/templates/ggWW_2D_mH600_2j_of.pdf}
	}
	\subfigure[ggWW statistical uncertainty]{
	\centering
	\label{subfig:template_ggWWerr_600}
		\includegraphics[width=.35\textwidth]{figures/templates/ggWWerr_2D_mH600_2j_of.pdf}
	}

	\caption{2D templates at \mHi = 600 \GeV in 0 jet bin} 
	\label{fig:templates_600_vbf_1}

\end{figure}

\begin{figure}[!hbtp]
	
	%
	\centering
	\subfigure[Wjets]{
	\centering
	\label{subfig:template_Wjets_600}
		\includegraphics[width=.35\textwidth]{figures/templates/Wjets_2D_mH600_2j_of.pdf}
	}
	\subfigure[Wjets statistical uncertainty]{
	\centering
	\label{subfig:template_Wjetserr_600}
		\includegraphics[width=.35\textwidth]{figures/templates/Wjetserr_2D_mH600_2j_of.pdf}
	}
	
	%
	\centering
	\subfigure[Top]{
	\centering
	\label{subfig:template_Top_600}
		\includegraphics[width=.35\textwidth]{figures/templates/Top_2D_mH600_2j_of.pdf}
	}
	\subfigure[Top statistical uncertainty]{
	\centering
	\label{subfig:template_Toperr_600}
		\includegraphics[width=.35\textwidth]{figures/templates/Toperr_2D_mH600_2j_of.pdf}
	}

	%
	\centering
	\subfigure[VV]{
	\centering
	\label{subfig:template_VV_600}
		\includegraphics[width=.35\textwidth]{figures/templates/VV_2D_mH600_2j_of.pdf}
	}
	\subfigure[VV statistical uncertainty]{
	\centering
	\label{subfig:template_VVerr_600}
		\includegraphics[width=.35\textwidth]{figures/templates/VVerr_2D_mH600_2j_of.pdf}
	}

	\caption{2D templates at \mHi = 600 \GeV in VBF channel.} 
	\label{fig:templates_600_vbf_2}

\end{figure}

\begin{figure}[!hbtp]
	
	%
	\centering
	\subfigure[Zjets]{
	\centering
	\label{subfig:template_Zjets_600}
		\includegraphics[width=.35\textwidth]{figures/templates/Zjets_2D_mH600_2j_of.pdf}
	}
	\subfigure[Zjets statistical uncertainty]{
	\centering
	\label{subfig:template_Zjetserr_600}
		\includegraphics[width=.35\textwidth]{figures/templates/Zjetserr_2D_mH600_2j_of.pdf}
	}

	%
%	\centering
%	\subfigure[Wgamma]{
%	\centering
%	\label{subfig:template_Wgamma_600}
%		\includegraphics[width=.35\textwidth]{figures/templates/Wgamma_2D_mH600_2j_of.pdf}
%	}
%	\subfigure[Wgamma statistical uncertainty]{
%	\centering
%	\label{subfig:template_Wgammaerr_600}
%		\includegraphics[width=.35\textwidth]{figures/templates/Wgammaerr_2D_mH600_2j_of.pdf}
%	}

	\caption{2D templates at \mHi = 600 \GeV in VBF channel.} 
	\label{fig:templates_600_vbf_3}

\end{figure} 

\begin{figure}[!hbtp]
	
	%
	\centering
	\subfigure[Stacked unrolled template linear]{
	\centering
	\label{subfig:template_unroll_stack_lin}
		\includegraphics[width=.45\textwidth]{figures/templates/2D_mH600_2j_of_stack_lin.pdf}
	}
	\subfigure[Overlaid unrolled template linear]{
	\centering
	\label{subfig:template_unroll_overlay_lin}
		\includegraphics[width=.45\textwidth]{figures/templates/2D_mH600_2j_of_overlay_lin.pdf}
	}

	%
	\centering
	\subfigure[Stacked unrolled template in log scale]{
	\centering
	\label{subfig:template_unroll_stack_log}
		\includegraphics[width=.45\textwidth]{figures/templates/2D_mH600_2j_of_stack_log.pdf}
	}
	\subfigure[Overlaid unrolled template in log scale]{
	\centering
	\label{subfig:template_unroll_overlay_log}
		\includegraphics[width=.45\textwidth]{figures/templates/2D_mH600_2j_of_overlay_log.pdf}
	}

	\caption{Unrolled templates at \mHi = 600 \GeV in VBF channel.} 
	\label{fig:templates_600_vbf_unroll}

\end{figure} 


  
  \subsection{Shape Uncertainties}
  \label{sec:shape_uncert}
  For the shape variation, we use the same technique used by BDT shape analysis \cite{MVASyst}.
\textcolor{red}{Alternative shapes for lepton \pt and MET resolutions, and lepton efficiency 
are not implemented yet. Only is normalization included for now. Effect of missing these 
alternative shapes to the expected significance is smaller than 10 \%. These will be added 
in a near future.}


%
\newpage
\section{Performance}
  \label{sec:performance}
  To compare the performance of the BDT shape analysis and the 2D method,
we calculate expected significance using the standard Combination tool \cite{combine}. 
The comparison is performed for the the 0 and 1-jet categories, as well
as their combination with the VBF category. 
In all cases only the different lepton flavor
final state is considered.


  
  \subsection{Expected Significance at \intlumiEightTeV }
  \label{sec:exp_significance_12fb}
  Table~\ref{tab:exp_sig_0j}, \ref{tab:exp_sig_1j}, and \ref{tab:exp_sig_alljet} 
show expected signicances in 0jet, 1jet, and 0/1/2jet bins, respectively.
We compare two BDT analyses. One is the ICHEP analysis(ICHEP BDT) and 
the other(new BDT) is using \mt and \mll ranges as 2D analysis. 
Since BDT analysis is performed only in 0/1 jet bins, 
we use cut-based analysis in the 2 jet bin for combination of jet bins. 
In the 2-jet bin, we require tighter cuts than the ICHEP analysis
in order to obtain better sensitivity. Additional cuts are 
mass-dependent cuts on \ptlmax, \ptlmin, \mll, and \delphill~ 
applied in 0/1 jet bins for cut-based analysis. 

\begin{table}[!htb] 
	\centering
	\begin{tabular}{c | c c c  }
   	\hline \hline
	\mHi & ICHEP BDT & new BDT & 2D \\
	\hline
	110	&0.4	&0.4	&0.4 	\\
	115	&1.4	&1.4	&1.1 	\\
	120	&1.2	&1.2	&1.4 	\\
	125	&1.7	&1.8	&2.2 	\\
	130	&2.4	&2.6	&3.1 	\\
	140	&3.9	&4.3	&4.9 	\\
	150	&5.8	&7.0	&7.1	\\
	160	&10.1	&12.1	&11.1 	\\
	170	&5.6	&7.0	&6.0 	\\	
	180	&6.2	&7.8	&7.4    \\ 
	190	&4.7	&4.8	&4.9    \\
	200	&3.0	&3.1	&3.8    \\
	250	&1.9	&1.8	&2.2    \\
	300	&1.8	&1.9	&1.6    \\
	350	&1.9	&2.0	&1.9    \\
	400	&1.8	&1.8	&1.8    \\
	450	&1.5	&1.6	&1.4    \\
	500	&1.2	&1.2	&1.0    \\
	550	&1.0	&1.0	&0.8    \\
	600	&0.8	&0.8	&0.6    \\
   	\hline \hline
	\end{tabular}
	\label{tab:exp_sig_0j}
	\caption{Expected significances in 0 jet bin for BDT and 2D methods}
\end{table}

\begin{table}[!htb] 
	\centering
	\begin{tabular}{c | c c c }
   	\hline \hline
	\mHi & ICHEP BDT & new BDT & 2D \\
	\hline 
	110	&0.3	&0.3	&0.3 	\\
	115	&1.1	&1.1	&1.0 	\\
	120	&0.8	&0.8	&1.0	\\
	125	&1.1	&1.2	&1.6	\\
	130	&1.5	&1.5	&2.0	\\
	140	&2.4	&2.6	&3.1	\\
	150	&4.1	&4.5	&4.3	\\
	160	&8.6	&9.1	&7.2	\\
	170	&5.0	&5.3	&4.3	\\
	180	&4.7	&5.0	&4.3	\\
	190	&3.6	&3.9	&3.4	\\
	200	&2.4	&2.7	&2.4	\\
	250	&1.8	&1.8	&1.7	\\
	300	&1.3	&1.3	&1.4	\\
	350	&1.6	&1.6	&1.7	\\
	400	&1.5	&1.5	&1.5	\\
	450	&1.7	&1.6	&1.4	\\
	500	&1.2	&1.2	&1.0 	\\
	550	&1.2	&1.3	&0.9 	\\
	600	&1.0	&1.0	&0.9 	\\
   	\hline \hline
	\end{tabular}
	\label{tab:exp_sig_1j}
	\caption{Expected significances in 1 jet bin for BDT and 2D methods}
\end{table}

\begin{table}[!htb] 
	\centering
	\begin{tabular}{c | c c c }
   	\hline \hline
	\mHi & ICHEP BDT & new BDT & 2D \\
	\hline 
	110	&0.5	&0.6	&0.6 	\\	
	115	&1.8	&1.8	&1.5	\\
	120	&1.4	&1.4	&1.8	\\
	125	&2.4	&2.5	&3.1	\\
	130	&2.6	&2.9	&3.7    \\
	140	&4.4	&4.9	&5.9    \\
	150	&7.6	&8.6	&8.3    \\
	160	&14.7	&15.9	&13.6    \\
	170	&8.9	&9.8	&8.4    \\
	180	&8.7	&9.6	&8.4    \\
	190	&6.6	&6.8	&6.5    \\
	200	&3.9	&4.1	&4.4    \\
	250	&3.1	&3.0	&3.2    \\
	300	&2.4	&2.4	&2.2    \\
	350	&2.6	&2.6	&2.5    \\
	400	&2.6	&2.5	&2.4    \\
	450	&2.6	&2.6	&2.3    \\
	500	&1.7	&1.7	&1.4    \\
	550	&1.8	&1.8	&1.5    \\
	600	&1.4	&1.4	&1.3    \\
   	\hline \hline
	\end{tabular}
	\label{tab:exp_sig_alljet}
	\caption{Expected significances in 0+1+2 jet bin for BDT and 2D methods}
\end{table}

2D analysis gives better expected sensitivity in low \mHi hypotheses. 
At \mHi=125\GeV~, expected significance in opposite flavor 
final state is improved by 25 \% ( $2.5 \sigma \rightarrow 3.1\sigma$) with respect to new BDT analysis. 

%Now, we observe 25 \% improvement with respect to BDT analysis at \mHi = 125 \GeV. 
%Because we use difference bin size and  $[\mt,\mll]$ ranges,
%we test the effect of them to the performance.  
%Table~\ref{tab:exp_sig_understand} shows expected significance 
%with different bin size (top) and different ranges (bottom).
%For testing bin size, the range is kept same and for testing 
%ranges, bin size is kept same.  
%The expected significance changes by less than 10 \% 
%by increasing number of bins by factor 4, from 4 to 16
%and it is comparable with BDT analysis which gives 1.80.
%On the other hand, by extending ranges expected significance 
%increases by 20 \% from BDT region to larger $[\mt,\mll]$ region.    
%
%\begin{table}[!htb] 
%	\centering
%	\begin{tabular}{ccc }
%   	\hline \hline
% 	\multicolumn{3}{c}{Same range and different bin size}  						\\ 
%	\hline
%	Binning ($\mt \times \mll $) & Range & expected significance 				\\ 
%   	\hline \hline
%	\multirow{2}{*}{$2 \times 2$} 	& $80 < \mt < 125$	& \multirow{2}{*}{1.71} \\ 
%									& $12 < \mll < 80$	& 						\\ 
%	\hline
%	\multirow{2}{*}{$3 \times 3$} 	& $80 < \mt < 125$	& \multirow{2}{*}{1.69} \\ 
%									& $12 < \mll < 80$	& 						\\ 
%	\hline
%	\multirow{2}{*}{$4 \times 4$} 	& $80 < \mt < 125$	& \multirow{2}{*}{1.75} \\ 
%									& $12 < \mll < 80$	& 						\\ 
%   	\hline \hline 
% 	\multicolumn{3}{c}{}  														\\ 
%   	\hline \hline 
% 	\multicolumn{3}{c}{Different range and same bin size} 						\\
%	\hline
%	Binning ($\mt \times \mll $) 	& Range 			& expected significance \\
%   	\hline \hline
%	\multirow{2}{*}{$2 \times 4$}   & $80 < \mt < 120$  & \multirow{2}{*}{1.81}	\\
%									& $0 < \mll < 100$  & 				   		\\
%	\hline
%	\multirow{2}{*}{$5 \times 4$}   & $80 < \mt < 180$  & \multirow{2}{*}{1.81}	\\
%									& $0 < \mll < 100$  & 				   		\\
%	\hline
%	\multirow{2}{*}{$8 \times 6$}   & $80 < \mt < 240$  & \multirow{2}{*}{2.07}	\\
%									& $0 < \mll < 150$  & 				   		\\
%   	\hline \hline 
%	\end{tabular}
%	\label{tab:exp_sig_understand}
%	\caption{Top table shows expected significance with the same range but different 
%	bin sizes. Bottom table shows expected significance with the same bin size but different 
%	ranges. Expected significance from BDT analysis is 1.64.}
%\end{table} 

  
%  \newpage
%  \subsection{Expected Significance at 20~\ifb }
%  \label{sec:exp_significance_20fb}
%  For the projection to 20~\ifb, top normalization uncertainty is halved
because it is driven by data statistics. Table~\ref{tab:exp_sig_20fb} shows
that the expected significance in 20~\ifb~is 80-90 \% larger than that in 5~\ifb
at all mass points. This means that significance is scaled almost statistically.
At \mHi = 125 \GeV the expected  significance in 20~\ifb is 4$\sigma$.

\begin{table}[htb] 
	\centering
	\begin{tabular}{c | c c | c }
   	\hline \hline
	\mHi & 5.1~\ifb & 20~\ifb & Improvement(\%) \\
	\hline 
	110 & 0.46 & 0.85 & 85 \\
	115 & 0.86 & 1.58 & 83 \\
	120 & 1.44 & 2.62 & 82 \\
	125 & 2.22 & 4.01 & 80 \\
	130 & 3.02 & 5.41 & 79 \\
	135 & 4.04 & 7.34 & 82 \\
	140 & 4.78 & 8.74 & 83 \\
	150 & 6.13 & 11.66 & 90 \\
	160 & 9.86 & 18.35 & 86 \\
	170 & 8.45 & 15.44 & 83 \\
	180 & 5.98 & 11.11 & 86 \\
	190 & 3.88 & 7.20 & 85 \\
	200 & 3.14 & 5.79 & 85 \\
   	\hline \hline
	\end{tabular}
	\caption{Expected significances in 0+1 jet bins in 5.1~\ifb and 20~\ifb}
	\label{tab:exp_sig_20fb}
\end{table} 


%
\newpage
\section{Summary}
     \label{sec:summary}
     In summary, we described an analysis to study the spin of a single narrow 
resonance at 125 GeV by the gluon fusion through the decays into $WW\to 2\ell2\nu$.  
The analysis is based on a two-dimensional templates $m_T-m_{\ell\ell}$. 
The expected sensitivity to distinguish between between SM Higgs hypothesis and 
spin 2 Graviton like resonance with minimal coupling 
$2_\text{min}^+$ is $1.7\sigma$ for \intlumiEightTeV. 
Scaling by luminosity the projected separation is about $2.0\sigma$ for 25~$\ifb$. 


%===================================================================================================
\clearpage

\vspace*{-0.2cm}
\thebibliography{12}

\bibitem{pdg}
 K. Nakamura et al. (Particle Data Group), "Review of particle physics", J. Phys.G37 , 2010.

\bibitem{Higgs1}
F. Englert and R. Brout, "Broken symmetries and the masses of gauge bosons", Phys. Rev. Lett. 13,  1964.

\bibitem{Higgs2}
P. W. Higgs, "Broken symmetry and the mass of gauge vector mesons", Phys. Rev. Lett. 13, 1964.

\bibitem{Higgs3}
Guralnik, G.S. and Hagen, C.R. and Kibble, T.W.B., "Global Conservation Laws and Massless Particles", 
Phys.Rev.Lett. 13, 1964.

\bibitem{HWW2010}
CMS Collaboration, "Title: Measurement of WW Production and Search for the Higgs Boson in 
pp Collisions at $\sqrt{s}$ = 7 TeV", arXiv:1102.5429

\bibitem{VBTFCrossSectionNote}
J. Alcaraz Maestre, \textit{et al.}, "Updated Measurements of Inclusive W and Z Cross Sections 
at $\sqrt{s}=7$ TeV", CMS AN-2010/264.

\bibitem{ggWWError}
F.~ Stoeckli, "http://indico.cern.ch/getFile.py/access?contribId=0\&resId=1\&materialId=slides\&confId=49009", 
EWK Diboson meeting of March 12 2009.

\bibitem{json}
{\small
/afs/cern.ch/cms/CAF/CMSCOMM/COMM\_DQM/certification/Collisions11/7TeV/Prompt/Cert\_160404-163869\_7TeV\_PromptReco\_Collisions11\_JSON.txt
}

\bibitem{ElIso}
A. Vartak, M. LeBourgeois, V. Sharma, "Lepton Isolation in the CMS Tracker, ECAL and HCAL", CMS AN-2010/106.

\bibitem{PVDA}
W. Erdmann, M. LeBourgeois, B. Mangano, 
https://indico.cern.ch/getFile.py/access?contribId=5\&sessionId=3\&resId=1\&materialId=slides\&confId=127127, 
note in preparation.

\bibitem{NExpHits}
B. Mangano \textit{et al.}, "Improvement in Photon Conversion Rejection Performance Using 
Advanced Tracking Tools", AN-10-283.

\bibitem{fakeLeptonNote1}
S.~Xie, \textit{et al.}", "Study of Data-Driven Methods for Estimation of Fake Lepton Backgrounds", 
CMS AN-2009/120.

\bibitem{fakeLeptonNote2}
W.~Andrews, \textit{et al.}, "Fake Rates for dilepton Analyses", CMS AN-2010/257.

\bibitem{fakeLeptonBkgSpillage1}
 F. Golf, D. Evans, J. Mulmenstadt  \textit{et al.}, ``Expectations for observation of top quark pair production in the dilepton final state with the early CMS data'', CMS AN-2009/050.

\bibitem{dyestnote}
W. Andrews, et al., “A Method to Measure the Contribution of $\dyll$ to a di-lepton+ MET Selection”, CMS AN-2009/023 (2009).

\bibitem{jes}
CMS Collaboration, "Jet Energy Calibration with Photon+Jet Events", PAS JME-09-004.

\bibitem{jetpas}
CMS Collaboration, "Jet Performance in pp Collisions at $\sqrt{s}=7 \rm\ TeV$", PAS JME-10-003.

\bibitem{btag}
CMS collaboration, "Commissioning of b-jet identification with pp collisions at $\sqrt{s}=7~\TeV$, BTV-10-001.

\bibitem{antikt}
Cacciari, Matteo and Salam, Gavin P. and Soyez, Gregory, "The anti-$k_t$ jet clustering 
algorithm", JHEP 04,  2008.

\bibitem{ConversionNote}
W.~Andrews, \textit{et al.}, "Study of photon conversion rejection at CMS", CMS AN-2009/159.

\bibitem{tmva}
A. Hoecker, \textit{et al.}, "TMVA - Toolkit for Multivariate Data Analysis", arXiv:physics/0703039, 2007.

\bibitem{XS}
CMS Generator group, Standard Model Cross Sections for CMS at 7 TeV, 2010.

\bibitem{PDF4LHC}
PDF4LHC Working Group, 
{\tt http://www.hep.ucl.ac.uk/pdf4lhc/PDF4LHCrecom.pdf}

\bibitem{Nadolsky:2008zw}
Nadolsky, Pavel M. and others, "Implications of CTEQ global analysis for 
collider observables", Phys. Rev. D78 2008.

\bibitem{Martin:2009iq}
Martin, A. D. and Stirling, W. J. and Thorne, R. S. and Watt, G., "Parton 
distributions for the LHC, Eur. Phys. J. C63 2009.

\bibitem{Ball:2010de}
Ball, Richard D. and others, "A first unbiased global NLO determination 
of parton distributions and their uncertainties", arXiv 1002.4407.

\bibitem{bayesian}
A. O'Hagan and J.J. Forster, "Bayesian Inference", Kendall's Advanced Theory of Statistics, 
Arnold, London, 2B, 2004.

\bibitem{ref:tagprobe_mit_w}
G. Bauer {\it et. al.}, "Lepton ef?iencies for the inclusive W cross section measurement with 36.1pb$^{-1}$", AN2011/097

\bibitem{ref:tagprobe_snt_top}
W. Andrews {\it et. al.}, "Uncertainties on the Lepton Selection Efficiency for t$t\bar{t}$ Cross Section Analysis", AN2010/274

\bibitem{LHCHiggsCrossSectionWorkingGroup:2011ti}
LHC Higgs Cross Section Working Group, "Handbook of LHC Higgs Cross Sections: 
Inclusive Observables", CERN-2011-002, 2011.

\bibitem{PFMET} 
CMS Collaboration, ``CMS MET Performance in Events Containing Electroweak Bosons from pp Collisions at $\sqrt{s}=7$ TeV'', CMS PAS JME-2010-005 (2010)


\bibitem{trkMET} 
Marco Zanetti, ``MET with PU in $\hww\to2\ell$'', https://indico.cern.ch/conferenceDisplay.py?confId=131580
Benjamin Hooberman, ``MET with PU in MC and First 2011 Data'', https://indico.cern.ch/contributionDisplay.py?contribId=5\&confId=132579. 


\bibitem{lands}
Mingshui Chen and Andrey Korytov, https://mschen.web.cern.ch/mschen/lands/

\bibitem{MCFMHiggsProduction}
J. Campbell, R.K. Ellis, G. Zanderighi, ``Next-to-Leading order Higgs + 2 jet production via gluon fusion.'', JHEP 0610:028 (2006), hep-ph/0608194
%===================================================================================================

\clearpage 
\appendix
\appendixpage
%  \section{Templates at \mHi = 160 \GeV~and 200 \GeV}
%     \label{app:templates_more}
%     In this appendix 2D templates at \mHi = 160 \GeV~and 200 \GeV~are shown.

%%%%%% mH=160 GeV
\begin{figure}[!hbtp]
	
	%
	\centering
	\subfigure[Signal]{
	\centering
	\label{subfig:template_signal_160}
		\includegraphics[width=.35\textwidth]{figures/templates/sig_2D_mH160_0j_of.pdf}
	}
	\subfigure[Signal statistical uncertainty]{
	\centering
	\label{subfig:template_signalerr_160}
		\includegraphics[width=.35\textwidth]{figures/templates/sigerr_2D_mH160_0j_of.pdf}
	}
	
	%
	\centering
	\subfigure[qqWW]{
	\centering
	\label{subfig:template_qqWW_160}
		\includegraphics[width=.35\textwidth]{figures/templates/qqWW_2D_mH160_0j_of.pdf}
	}
	\subfigure[qqWW statistical uncertainty]{
	\centering
	\label{subfig:template_qqWWerr_160}
		\includegraphics[width=.35\textwidth]{figures/templates/qqWWerr_2D_mH160_0j_of.pdf}
	}

	%
	\centering
	\subfigure[ggWW]{
	\centering
	\label{subfig:template_ggWW_160}
		\includegraphics[width=.35\textwidth]{figures/templates/ggWW_2D_mH160_0j_of.pdf}
	}
	\subfigure[ggWW statistical uncertainty]{
	\centering
	\label{subfig:template_ggWWerr_160}
		\includegraphics[width=.35\textwidth]{figures/templates/ggWWerr_2D_mH160_0j_of.pdf}
	}

	\caption{2D templates at \mHi = 160 \GeV} 
	\label{fig:templates_160_1}

\end{figure}

\begin{figure}[!hbtp]
	
	%
	\centering
	\subfigure[Wjets]{
	\centering
	\label{subfig:template_Wjets_160}
		\includegraphics[width=.35\textwidth]{figures/templates/Wjets_2D_mH160_0j_of.pdf}
	}
	\subfigure[Wjets statistical uncertainty]{
	\centering
	\label{subfig:template_Wjetserr_160}
		\includegraphics[width=.35\textwidth]{figures/templates/Wjetserr_2D_mH160_0j_of.pdf}
	}
	
	%
	\centering
	\subfigure[Top]{
	\centering
	\label{subfig:template_Top_160}
		\includegraphics[width=.35\textwidth]{figures/templates/Top_2D_mH160_0j_of.pdf}
	}
	\subfigure[Top statistical uncertainty]{
	\centering
	\label{subfig:template_Toperr_160}
		\includegraphics[width=.35\textwidth]{figures/templates/Toperr_2D_mH160_0j_of.pdf}
	}

	%
	\centering
	\subfigure[VV]{
	\centering
	\label{subfig:template_VV_160}
		\includegraphics[width=.35\textwidth]{figures/templates/VV_2D_mH160_0j_of.pdf}
	}
	\subfigure[VV statistical uncertainty]{
	\centering
	\label{subfig:template_VVerr_160}
		\includegraphics[width=.35\textwidth]{figures/templates/VVerr_2D_mH160_0j_of.pdf}
	}

	\caption{2D templates at \mHi = 160 \GeV} 
	\label{fig:templates_160_2}

\end{figure}

\begin{figure}[!hbtp]
	
	%
	\centering
	\subfigure[Zjets]{
	\centering
	\label{subfig:template_Zjets_160}
		\includegraphics[width=.35\textwidth]{figures/templates/Zjets_2D_mH160_0j_of.pdf}
	}
	\subfigure[Zjets statistical uncertainty]{
	\centering
	\label{subfig:template_Zjetserr_160}
		\includegraphics[width=.35\textwidth]{figures/templates/Zjetserr_2D_mH160_0j_of.pdf}
	}

	%
	\centering
	\subfigure[Wgamma]{
	\centering
	\label{subfig:template_Wgamma_160}
		\includegraphics[width=.35\textwidth]{figures/templates/Wgamma_2D_mH160_0j_of.pdf}
	}
	\subfigure[Wgamma statistical uncertainty]{
	\centering
	\label{subfig:template_Wgammaerr_160}
		\includegraphics[width=.35\textwidth]{figures/templates/Wgammaerr_2D_mH160_0j_of.pdf}
	}

	\caption{2D templates at \mHi = 160 \GeV} 
	\label{fig:templates_160_3}

\end{figure} 

\begin{figure}[!hbtp]
	
	%
	\centering
	\subfigure[Stacked unrolled template linear]{
	\centering
	\label{subfig:template_unroll_stack_lin}
		\includegraphics[width=.45\textwidth]{figures/templates/2D_mH160_0j_of_stack_lin.pdf}
	}
	\subfigure[Overlaid unrolled template linear]{
	\centering
	\label{subfig:template_unroll_overlay_lin}
		\includegraphics[width=.45\textwidth]{figures/templates/2D_mH160_0j_of_overlay_lin.pdf}
	}

	%
	\centering
	\subfigure[Stacked unrolled template in log scale]{
	\centering
	\label{subfig:template_unroll_stack_log}
		\includegraphics[width=.45\textwidth]{figures/templates/2D_mH160_0j_of_stack_log.pdf}
	}
	\subfigure[Overlaid unrolled template in log scale]{
	\centering
	\label{subfig:template_unroll_overlay_log}
		\includegraphics[width=.45\textwidth]{figures/templates/2D_mH160_0j_of_overlay_log.pdf}
	}

	\caption{Unrolled templates at \mHi = 160 \GeV} 
	\label{fig:templates_160_unroll}

\end{figure} 

%%%%%% mH=200 GeV
\begin{figure}[!hbtp]
	
	%
	\centering
	\subfigure[Signal]{
	\centering
	\label{subfig:template_signal_200}
		\includegraphics[width=.35\textwidth]{figures/templates/sig_2D_mH200_0j_of.pdf}
	}
	\subfigure[Signal statistical uncertainty]{
	\centering
	\label{subfig:template_signalerr_200}
		\includegraphics[width=.35\textwidth]{figures/templates/sigerr_2D_mH200_0j_of.pdf}
	}
	
	%
	\centering
	\subfigure[qqWW]{
	\centering
	\label{subfig:template_qqWW_200}
		\includegraphics[width=.35\textwidth]{figures/templates/qqWW_2D_mH200_0j_of.pdf}
	}
	\subfigure[qqWW statistical uncertainty]{
	\centering
	\label{subfig:template_qqWWerr_200}
		\includegraphics[width=.35\textwidth]{figures/templates/qqWWerr_2D_mH200_0j_of.pdf}
	}

	%
	\centering
	\subfigure[ggWW]{
	\centering
	\label{subfig:template_ggWW_200}
		\includegraphics[width=.35\textwidth]{figures/templates/ggWW_2D_mH200_0j_of.pdf}
	}
	\subfigure[ggWW statistical uncertainty]{
	\centering
	\label{subfig:template_ggWWerr_200}
		\includegraphics[width=.35\textwidth]{figures/templates/ggWWerr_2D_mH200_0j_of.pdf}
	}

	\caption{2D templates at \mHi = 200 \GeV} 
	\label{fig:templates_200_1}

\end{figure}

\begin{figure}[!hbtp]
	
	%
	\centering
	\subfigure[Wjets]{
	\centering
	\label{subfig:template_Wjets_200}
		\includegraphics[width=.35\textwidth]{figures/templates/Wjets_2D_mH200_0j_of.pdf}
	}
	\subfigure[Wjets statistical uncertainty]{
	\centering
	\label{subfig:template_Wjetserr_200}
		\includegraphics[width=.35\textwidth]{figures/templates/Wjetserr_2D_mH200_0j_of.pdf}
	}
	
	%
	\centering
	\subfigure[Top]{
	\centering
	\label{subfig:template_Top_200}
		\includegraphics[width=.35\textwidth]{figures/templates/Top_2D_mH200_0j_of.pdf}
	}
	\subfigure[Top statistical uncertainty]{
	\centering
	\label{subfig:template_Toperr_200}
		\includegraphics[width=.35\textwidth]{figures/templates/Toperr_2D_mH200_0j_of.pdf}
	}

	%
	\centering
	\subfigure[VV]{
	\centering
	\label{subfig:template_VV_200}
		\includegraphics[width=.35\textwidth]{figures/templates/VV_2D_mH200_0j_of.pdf}
	}
	\subfigure[VV statistical uncertainty]{
	\centering
	\label{subfig:template_VVerr_200}
		\includegraphics[width=.35\textwidth]{figures/templates/VVerr_2D_mH200_0j_of.pdf}
	}

	\caption{2D templates at \mHi = 200 \GeV} 
	\label{fig:templates_200_2}

\end{figure}

\begin{figure}[!hbtp]
	
	%
	\centering
	\subfigure[Zjets]{
	\centering
	\label{subfig:template_Zjets_200}
		\includegraphics[width=.35\textwidth]{figures/templates/Zjets_2D_mH200_0j_of.pdf}
	}
	\subfigure[Zjets statistical uncertainty]{
	\centering
	\label{subfig:template_Zjetserr_200}
		\includegraphics[width=.35\textwidth]{figures/templates/Zjetserr_2D_mH200_0j_of.pdf}
	}

	%
	\centering
	\subfigure[Wgamma]{
	\centering
	\label{subfig:template_Wgamma_200}
		\includegraphics[width=.35\textwidth]{figures/templates/Wgamma_2D_mH200_0j_of.pdf}
	}
	\subfigure[Wgamma statistical uncertainty]{
	\centering
	\label{subfig:template_Wgammaerr_200}
		\includegraphics[width=.35\textwidth]{figures/templates/Wgammaerr_2D_mH200_0j_of.pdf}
	}

	\caption{2D templates at \mHi = 200 \GeV} 
	\label{fig:templates_200_3}

\end{figure} 

\begin{figure}[!hbtp]
	
	%
	\centering
	\subfigure[Stacked unrolled template linear]{
	\centering
	\label{subfig:template_unroll_stack_lin}
		\includegraphics[width=.45\textwidth]{figures/templates/2D_mH200_0j_of_stack_lin.pdf}
	}
	\subfigure[Overlaid unrolled template linear]{
	\centering
	\label{subfig:template_unroll_overlay_lin}
		\includegraphics[width=.45\textwidth]{figures/templates/2D_mH200_0j_of_overlay_lin.pdf}
	}

	%
	\centering
	\subfigure[Stacked unrolled template in log scale]{
	\centering
	\label{subfig:template_unroll_stack_log}
		\includegraphics[width=.45\textwidth]{figures/templates/2D_mH200_0j_of_stack_log.pdf}
	}
	\subfigure[Overlaid unrolled template in log scale]{
	\centering
	\label{subfig:template_unroll_overlay_log}
		\includegraphics[width=.45\textwidth]{figures/templates/2D_mH200_0j_of_overlay_log.pdf}
	}

	\caption{Unrolled templates at \mHi = 200 \GeV} 
	\label{fig:templates_200_unroll}

\end{figure} 

  
  \newpage
  \section{S/B with various bin sizes}
     \label{app:binsize}
     \begin{figure}[htp]
	\centering
	\includegraphics[width=1.0\textwidth]{figures/binsize.pdf}
	\caption{ S/B with various bin sizes.}
  	\label{fig:app_binsize}
\end{figure}

Figure~\ref{fig:app_binsize} shows S/B in the unrolled templates. 
Bins with large signal contribution are shown with different bin sizes. 
S/B does not change much by the size of bins.  
Thus, we choose the largest bin size of the choices to get maximum statistics in each bin.  

  
%  \newpage
%  \section{Data and MC templates}
%     \label{app:templates_eyefit}
%     
\begin{table}[!htb] 
	\centering
	\begin{tabular}{c|cccccccc}
	\hline \hline 
  		Process   & Signal(125)	& Signal(150)	& qqWW 	& ggWW 	& VV  	& Top 	& Wjets	& Wgamma \\
	\hline 
	  	Yields($\mt>80~\GeV$)    & 36.4 		& 165.6			& 696.2	& 47.2 	& 16.3 	& 119.4	&84.9	& 13.6 \\  % yields with mT > 80 GeV 
	  	Yields($\mt>60~\GeV$)   & 40.2 		& 174.0			& 731.9	& 48.5 	& 17.7 	& 124.7	&104.1	& 18.6 \\  % yields with mT > 60 GeV 
	\hline \hline 
	\end{tabular}
	\label{tab:exp_sig_understand}
	\caption{Signal and Background yields in 5 \ifb.} 
\end{table} 

\begin{figure}[!hbtp]
	
	%
	\centering
	\subfigure[Data]{
	\centering
	\label{subfig:template_data_125}
		\includegraphics[width=.35\textwidth]{figures/templates/data_2D_mH125_0j_of.pdf}
	}
	
	%
	\centering
	\subfigure[Signal \mHi=125 \GeV]{
	\centering
	\label{subfig:template_sig_125}
		\includegraphics[width=.30\textwidth]{figures/templates/sig_2D_mH125_0j_of.pdf}
	}
	\subfigure[Signal \mHi=150 \GeV]{
	\centering
	\label{subfig:template_sig_150}
		\includegraphics[width=.30\textwidth]{figures/templates/sig_2D_mH150_0j_of.pdf}
	}
	\subfigure[qqWW]{
	\centering
	\label{subfig:template_qqWW_125}
		\includegraphics[width=.30\textwidth]{figures/templates/qqWW_2D_mH125_0j_of.pdf}
	}

	%
	\centering
	\subfigure[ggWW]{
	\centering
	\label{subfig:template_ggWW_125}
		\includegraphics[width=.30\textwidth]{figures/templates/ggWW_2D_mH125_0j_of.pdf}
	}
	\subfigure[Top]{
	\centering
	\label{subfig:template_top_125}
		\includegraphics[width=.30\textwidth]{figures/templates/Top_2D_mH125_0j_of.pdf}
	}
	\subfigure[Wjets]{
	\centering
	\label{subfig:template_Wjets_125}
		\includegraphics[width=.30\textwidth]{figures/templates/Wjets_2D_mH125_0j_of.pdf}
	}
	
	%
	\centering
	\subfigure[VV]{
	\centering
	\label{subfig:template_VV_125}
		\includegraphics[width=.30\textwidth]{figures/templates/VV_2D_mH125_0j_of.pdf}
	}
	\centering
	\subfigure[Wgamma]{
	\centering
	\label{subfig:template_Wgamma_125}
		\includegraphics[width=.30\textwidth]{figures/templates/Wgamma_2D_mH125_0j_of.pdf}
	}



	\caption{2D templates for data, signal, and backgrounds in 5 \ifb. 
			Templates are normalized to unit area.} 
	\label{fig:templates_eyefit}

\end{figure}







\end{document}
