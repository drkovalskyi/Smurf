Before filling the 2D templates, we first apply a 
preselection to increase the overall S/B.
Identically to the cut-based and BDT-based analyses, we first
choose events with two W bosons \cite{HWW2012ICHEP}.

Because the signal event kinematics change with \mHi, it is necessary
to change the preselection with \mHi. In the BDT-based analysis
the \mll~and \mt~ selection changes with each mass point.  In the 
2D analysis, by defining an appropriate range and binning 
for these two variables we are able to use a single template definition
for low-\mHi ($\mHi<300~\GeV$) and 
a single template definition for high-\mHi ($\mHi\ge 300~\GeV$) hypotheses.
This is advantageous because it avoids unnecessary statistical fluctuations
between the data selected for nearby hypotheses.

On top of WW preselection, we apply  $\ptlmax > 50~\GeV$ for high-\mHi~hypotheses.
For the VBF channel, we apply additional requirements as follows to preselect VBF events. 

\begin{itemize}
	\item two reconstructed jets with $\pt~>~30~\GeV$ and no other jets between
		  them with $\pt~>~30~\GeV$;
    \item neither of the jets must be $b$-tagged;
  	\item $\Delta\eta (j_1-j_2) > 3.5$;
    \item $m_{j_1j_2} > 500\:\GeV$; 
    \item $m_{T}^{\ell\ell\met}>30\GeV$.
\end{itemize}  

When defining the templates, we take care to avoid empty
bins in the background templates, given the available
simulation and control sample statistics.
Thus, the bins cannot be arbitrarily small.
The parameters used to construct the templates
are summarized in Table~\ref{tab:binning_range}
In Appendix~\ref{app:binsize} we show that
changing the chosen bin size by a factor of 4
does not change the S/B.

\begin{table}[!htb] 
	\centering
	\begin{tabular}{c | c | c | c }
   	\hline \hline
	Channel 					& Variable	& $\mHi< 300~\GeV$ 	& $\mHi\ge 300~\GeV$ 	\\ 
   	\hline \hline
	\multirow{2}{*}{0/1 jet}  	& \mt 		& [80,280] 10 bins 	& [80,380] 10 bins		\\	
	 							& \mll 		& [0,200] 8 bins	& [0,450] 8 bins		\\	
   	\hline
	\multirow{2}{*}{VBF}  		& \mt 		& [30,280] 4 bins 	& [30,330] 2 bins		\\	
	 							& \mll 		& [0,200] 4 bins	& [0,450] 3 bins		\\	
   	\hline \hline
	\end{tabular}
	\label{tab:binning_range}
	\caption{Summary of template parameters. For the high-\mHi~templates, overflow to \mt/\mll=600~\GeV~ is included
in the content of the last bin.}
\end{table}

