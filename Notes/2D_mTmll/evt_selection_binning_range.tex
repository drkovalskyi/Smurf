The 2D templates are made with selections to increase S/B.   
As done in the cut-based and BDT-based analyses, we first 
choose events with two W bosons \cite{HWW2012ICHEP}. 
%Given the different kinematics in different \mHi hypotheses, 
%we make low-\mHi ($\mHi<300~\GeV$) and high-\mHi ($\mHi\ge 300~\GeV$) templates individually.   
The motivation of 2D analysis is to have a single template 
across all \mHi hypotheses. However, since event kinematics changes by \mHi, 
we make separate templates for low-\mHi ($\mHi<300~\GeV$) and high-\mHi ($\mHi\ge 300~\GeV$) hypotheses.   
On top of WW preselection, we apply  $\ptlmax > 50~\GeV$ for high-\mHi~hypotheses.
For VBF channel, we apply additional requirements as follows to select VBF events. 

\begin{itemize}
	\item two reconstructed jets with $\pt~>~30~\GeV$ and no other jets between
		  them with $\pt~>~30~\GeV$;
    \item neither of the jets must be $b$-tagged;
  	\item $\Delta\eta (j_1-j_2) > 3.5$;
    \item $m_{j_1j_2} > 500\:\GeV$; 
    \item $30~\GeV < m_{T}^{\ell\ell\met}$.
\end{itemize}  

In order to perform a maximum likelihood fit with templates, 
it is important to avoid empty bins in the background templates. 
Therefore, one can not have as small bins as they want. 
Also, if S/B does not change by making bin size smaller
then we can use larger bins to have better statistics in each bin. 
In appendix~\ref{app:binsize} it is shown that S/B stays at the same level 
by changing the bin size by factor 4. 

\begin{table}[!htb] 
	\centering
	\begin{tabular}{c | c | c | c }
   	\hline \hline
	Channel 					& Variable	& $\mHi< 300~\GeV$ 	& $\mHi\ge 300~\GeV$ 	\\ 
   	\hline \hline
	\multirow{2}{*}{0/1 jet}  	& \mt 		& [80,280] 10 bins 	& [80,380] 10 bins		\\	
	 							& \mll 		& [0,200] 8 bins	& [0,600] 8 bins		\\	
   	\hline
	\multirow{2}{*}{VBF}  		& \mt 		& [40,280] 4 bins 	& [30,330] 2 bins		\\	
	 							& \mll 		& [0,200] 4 bins	& [0,450] 3 bins		\\	
   	\hline \hline
	\end{tabular}
	\label{tab:binning_range}
	\caption{Summary of binning and range. For high-\mHi~templates, overflow to \mt/\mll=600~\GeV~ is allowed.}
\end{table}

Considering all these, we construct templates as summarized in Table~\ref{tab:binning_range}.
