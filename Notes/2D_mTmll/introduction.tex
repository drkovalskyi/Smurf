The current $\Hi\To\WW \To \Lep\Lprime\Nu\Nubar$ analysis \cite{HWW2012ICHEP}
is performed with two complementary approaches, cut-based and shape-based. 
The cut-based analysis is a single bin counting experiment and is
limited in sensitivity by systematic uncertainties. In the shape-based
approach a multivariate discriminator (BDT) is constructed to separate
signal and background. To maximise the sensitivity to the signal,
a binned fit is performed on the BDT output distribution.

The shape-based analysis is more sensitive than the cut-based analysis
for two reasons:

\begin{itemize}
    \item The BDT exploits the correlation between multiple
input variables described in Ref. \cite{HWW2012ICHEP} to 
separate signal and background more efficiently than orthogonal cuts.
    \item The fit to the BDT output allows the inclusion of additional
low signal to background ratio bins, and consequently additional
constrains on the background.
\end{itemize}

One of the drawbacks of the BDT approach is the 
non-trivial dependence of the discriminator value on 
the input variables. This means that physical interpretation
of the data in the BDT discriminator can be difficult. Additionally,
for Higgs boson masses below 140 GeV, most of the additional
sensitivity of the shape-analysis cames from the fit
rather than the BDT itself. This has been shown
by performing $m_{ll}$ shape analyis with 2011 data \cite{HWW2011CommonNote}
which gives about 90 \% of the sensitivity of the BDT approach.

In this document we present an alternative approach that allows for a
simpler physical interpretation of the observed data with a sensitivity that is
better than the default BDT shape-analysis for low \mHi hypotheses.
We have developed a 2-dimensional shape analysis using two independent variables: $m_{ll}$
and $m_T$. We have applied the method to the $e\mu$ final state,
which is the most sensitive. Same-flavor final states contribute
marginally due to the severe Drell-Yan background.

The 2-dimensional method is described in section~\ref{sec:method} 
and its performance is shown in section~\ref{sec:performance}.

