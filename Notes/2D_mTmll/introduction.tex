Current $\Hi\To\WW \To \Lep\Lprime\Nu\Nubar$ analysis is performed
with two complementary approaches, cut-based and shape-based. The
cut-based analysis is a simple cut-and-counting experiment and its
sensitivity is limited by systematic uncertainties.  The shape-based
approach uses the full shape of BDT output to maximize sensitivity.

The BDT shape analysis uses both multivariate technique and maximum
likelihood fit to the shape of BDT discriminator to improve
sensitivity of the analysis. One of the drawbacks of a multi-variate
discriminator like BDT is that when something unexpected is observed
in data it is hard to interpret the results since MVA output has
non-trivial dependence on its input variables. Additionally, for Higgs
mass below 140 GeV the power of BDT shape analysis comes mostly from
using shapes rather than using multivariate technique. This is shown
by performing $m_{ll}$ shape analyis with 2011 data which gives about
90 \% sensitivity compared to the BDT approach.

In this document we present an alternative approach that allows for a
simple interpretation of observed data with a sensitivity that is
better than the default BDT-based shape analysis. We developed a
2-dimensional shape analysis using two independent variables: $m_{ll}$
and $m_T$. In this study we use only $e\mu$ final state in the 0 and 1
jet bins. Same-flavor final states have marginal contribution due to
sever Drell-Yan background. It should be trivial to expand the
analysis to VBF case (2-jets) when adequate Monte Carlo simulations
are available.

The 2D method is described in section~\ref{sec:method} 
and its performance is shown in section~\ref{sec:performance}.  
