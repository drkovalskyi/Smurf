Current $\Hi\To\WW \To \Lep\Lprime\Nu\Nubar$ analysis is performed with two 
complementary approaches, cut-based and shape-based. The cut-based analysis 
is a simple cut-and-counting experiment and its sensitivity is limited by systematic uncertainties. 
The shape-based approach uses the full shape of BDT output to maximize sensitivity.  

The BDT shape analysis uses both multivariate technique and maximum likelihood fit 
to the shape of BDT discriminator to maximize S/B(signal to background ratio). 
But, when it comes to the interpretation of result, it is impossible to know 
the characteristics of events (\textit{i.e.} kinematic variables ) in a BDT discriminator bin.
Additionally, the power of BDT shape analysis comes from using shapes rather than 
using multivariate technique. This is shown by performing $m_{ll}$ shape analyis 
with 2011 data which gives about 80 - 90 \% sensitivity compared to the BDT approach. 

Therefore, we test 2-dimensional shape analysis using two independent variables
to obtain better sensitivity than the 1-dimensional one. In this study we use 
only $e\mu$ final state in the 0 and 1 jet bins because adding 2 jet bins 
and $ee$/$\mu\mu$ channels do not improve sensitivity much.  

The 2D method is described in section~\ref{sec:method} 
and its performance is shown in section~\ref{sec:performance}.  
Our goal is that 2D method performs as good as BDT shape analysis
in terms of expected significance at low Higgs masses.
