A narrow resonance around 125 GeV has been discovered mainly through the 
diboson decay modes $\gamma\gamma$, $WW$ and $ZZ$~\cite{discovery-atlas,discovery-cms}. 
The determination of the quantum numbers such as the spin and parity 
of this Higgs-like particle has become one of the most important tasks 
in the high energy physics. 
The scattering amplitudes of the Higgs~\cite{Higgs1,Higgs2,Higgs3} or 
exotic boson decays into the $WW$ final state 
have been studied phenomenologically~\cite{Ellis2012,xww}. 
It is demonstrated in Reference~\cite{xww} that the $WW$ channel has 
a good sensitivity to distinguish between the standard model Higgs 
hypothesis and the spin 2 resonance which couples with the 
dibosons through minimal couplings. 

Due to the two missing neutrinos the leptonic final states are not 
fully reconstructed. However kinematic observables such as 
the opening angle between the two reconstructed leptons, dilepton invariant 
mass and transverse mass 
$m_T = ({2p_T^{\ell\ell}\met(1-\cos\Delta\phi_{\ell\ell-\met})})^{1/2}$,
where $\Delta\phi_{\ell\ell-\met}$ is the angle between the direction of 
the di-lepton pair and the missing energy $\met$ vector in the transverse plane, 
can be used to distinguish between the SM Higgs hypothesis and other 
exotic resonaces with alternative spin or parity. 

In this note we document the studies on distinguishing between the 
SM Higgs hypothesis and the spin 2 KK-graviton based spin 2 resonance with 
minimal couplings~\cite{xww}. 
