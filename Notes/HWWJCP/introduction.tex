A narrow resonance with an invariant mass of approximately 125 GeV 
has been discovered by ATLAS and CMS, 
mainly through the diboson decay modes $\gamma\gamma$, 
$WW$ and $ZZ$~\cite{discovery-atlas,discovery-cms}. 
The determination of the quantum numbers of this Higgs-like particle,
such as its spin and parity, is currently one of the most important
tasks in high energy physics.

Phenomenological studies of the scattering 
amplitudes of Higgs~\cite{Higgs1,Higgs2,Higgs3} or 
exotic boson decays to the $WW$ final state 
have been performed, and are described in References~\cite{Ellis2012,xww}. 
It is demonstrated in Reference~\cite{xww} that the $WW$ channel has 
a good sensitivity to distinguish between the standard model Higgs boson
hypothesis and a spin-2 resonance, which couples to the 
dibosons through minimal couplings, referred to as $2_\text{min}^+$.

The $WW$ final state cannot be fully reconstructed if 
both $W$ bosons decay leptonically, because the two neutrinos
are not detected.
However, kinematic observables such as 
the opening angle between the two reconstructed leptons in the transverse plane
the dilepton invariant mass and transverse mass
  \footnote{The transverse mass is defined as 
  $m_T = ({2p_T^{\ell\ell}\met(1-\cos\Delta\phi_{\ell\ell-\met})})^{1/2}$,
where $\Delta\phi_{\ell\ell-\met}$ is the angle between the direction of 
the di-lepton pair and the missing energy $\met$ vector in the transverse plane} 
can be used to distinguish  between the SM Higgs boson hypothesis 
and other exotic resonances with different spin or parity.
In this note we report on studies of the separation 
between the SM Higgs boson hypothesis and a KK-graviton 
based spin 2-resonance with minimal couplings~\cite{xww}.

