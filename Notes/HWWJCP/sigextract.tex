We use the same event selections as in the SM Higgs search 
~\cite{HWWHCP2012} with the exception of two changes.
To improve the selection efficiency for the $gg\to X\to WW$ hypothesis,
we apply dilepton $p_T>$ 30 GeV and transverse mass $m_T>$60 GeV.
To improve the signal purity we use events in the 0-Jet bin 
and only consider the different lepton flavor final state.

We analyse the data by constructing two dimensional templates based on $m_T-m_{\ell\ell}$.
This is the same procedure used in the SM Higgs search.
In this analysis, the templates defined as follows:

%%%%%%%%%%%%%%%%%
\begin{itemize}
\item $m_T$ 14 bins: [60,70,80,90,100,110,120,140,160,180,200,220,240,260,280],
\item $m_{\ell\ell}$ bins: [12,30,45,60,75,100,125,150,175,200].
\end{itemize}
%%%%%%%%%%%%%%%%%

This template binning is chosen to ensure the following conditions are met:

%%%%%%%%%%%%%%%%%
\begin{itemize}
    \item enough statistics for the main signal and background processes, 
    \item enough granuarity in the signal enriched regions to distinguish between 
different hypotheses, 
    \item the same binning in the background enriched (mainly $WW$ and $Top$) regions 
as in the SM Higgs search~\cite{HWWHCP2012}. 
\end{itemize}
%%%%%%%%%%%%%%%%%

The $m_T-m_{\ell\ell}$ templates for the SM Higgs boson and 
the spin-2 graviton like resonance $2_\text{min}^+$, zoomed in the 
signal enriched regions, are shown normalised to \intlumiEightTeV 
in Figure~\ref{fig:mtvsmll_sig}.
The corresponding background temlates are shown in 
Figure~\ref{fig:mtvsmll_bkg}, and the expected signal and
background yields for \intlumiEightTeV are shown in Table~\ref{tab:yield}.

To separate the standard model and $gg\to X\to WW$ hypotheses,
we construct a signal plus background model for each hypothesis.
The same event selections and background predictions are used for both models.
We then perform the CLs fit for both models to the data, letting the 
signal strength float.
The difference in the best fit likelihoods is then used 
to quantify the consistency between data and each signal hypothesis. 

To calculate the expected separation between the SM Higgs and other 
exotic resonances, we use the same number of events for the exotic resonance
after the full selections as in the SM Higgs case. 

%%%%%%%%%%%%%%%%%%%%%%%%%%%%%%%%%%%%%%%%%%%%%
\begin{figure}[!hbtp]
\centering
\subfigure[SM Higgs]{
\centering
\label{subfig:hww}
\includegraphics[width=.45\textwidth]{figures/mtvsmll_hww.pdf}
}
\subfigure[Graviton ($2_m^+$)]{
\centering
\label{subfig:xww}
\includegraphics[width=.45\textwidth]{figures/mtvsmll_xww.pdf}
}\\
\caption{The $m_T-m_{\ell\ell}$ templates for the SM Higgs and 
spin 2 graviton like resonances, zoomed in 
the signal regions. The distributions are 
normalized to the expectations for \intlumiEightTeV.}
\label{fig:mtvsmll_sig}
\end{figure}
%%%%%%%%%%%%%%%%%%%%%%%%%%%%%%%%%%%%%%%%%%%%%


%%%%%%%%%%%%%%%%%%%%%%%%%%%%%%%%%%%%%%%%%%%%%
\begin{figure}[!hbtp]
\centering
\subfigure[$qq\to WW$]{
\centering
\label{subfig:qqww}
\includegraphics[width=.45\textwidth]{figures/mtvsmll_qqWW.pdf}
}
\subfigure[W+Jets]{
\centering
\label{subfig:wjets}
\includegraphics[width=.45\textwidth]{figures/mtvsmll_Wjets.pdf}
}\\
\subfigure[$W\gamma$]{
\centering
\label{subfig:wgamma}
\includegraphics[width=.45\textwidth]{figures/mtvsmll_Wgamma.pdf}
}
\subfigure[$W\gamma^*$]{
\centering
\label{subfig:wgst}
\includegraphics[width=.45\textwidth]{figures/mtvsmll_Wgstar.pdf}
}\\
\subfigure[Top]{
\centering
\label{subfig:top}
\includegraphics[width=.45\textwidth]{figures/mtvsmll_Top.pdf}
}
\subfigure[VV]{
\centering
\label{subfig:vv}
\includegraphics[width=.45\textwidth]{figures/mtvsmll_VV.pdf}
}\\

\caption{The $m_T-m_{\ell\ell}$ templates for the backgrounds, zoomed in 
the signal regions. The distributions are 
normalized to the expectations for \intlumiEightTeV.}
\label{fig:mtvsmll_bkg}
\end{figure}
%%%%%%%%%%%%%%%%%%%%%%%%%%%%%%%%%%%%%%%%%%%%%

\begin{table}[!hbtp]
{%\footnotesize
 %\tiny
 \begin{center}

 \begin{tabular}{| c | c c c c c }
 \hline
 ggH & qqWW & ggWW & VV & Top & Wjets \\ \hline
$210.6\pm43.7$ & $3616.2\pm334.7$ & $190.7\pm58.8$ & $120.8\pm8.7$ & $390.8\pm81.3$ & $831.4\pm299.3$ \\
\hline
\end{tabular}

\vspace{10pt}

 \begin{tabular}{ c c c  | c |}
 \hline
 Wgamma & Wg3l & Ztt & $\sum$Bkg \\ \hline
$100.7\pm30.8$ & $164.7\pm50.4$ & $39.1\pm4.3$ & $5454.4\pm463.9$ \\ 
\hline
\end{tabular}


\end{center}
}
\caption{Expected signal and background yields for \intlumiEightTeV.}
\label{tab:yield}
\end{table}
