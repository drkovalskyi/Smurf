\RequirePackage{lineno}
\documentclass{cmspaper}
\usepackage{graphicx}
\usepackage{amsmath}
\usepackage{amssymb}
\usepackage{subfigure}
\usepackage{multirow}
% % useful definitions

% processes
\def\dyee {\ensuremath{Z/\gamma^*\to ee}}
\def\dymm {\ensuremath{Z/\gamma^*\to\mu\mu}}
\def\dytt {\ensuremath{Z/\gamma^*\to\tau\tau}}
\def\zee {\ensuremath{Z\to ee}}
\def\zmm {\ensuremath{Z\to\mu\mu}}
\def\ztt {\ensuremath{Z\to\tau\tau}}
\def\ttbar {\ensuremath{t\bar{t}}}
\def\wwll {\ensuremath{WW\to l^+l^-}}
\def\wwlulu{\ensuremath{WW\to l^+\nu l^-\bar{\nu}}}
\def\ww {\ensuremath{WW}}
\def\wz{\ensuremath{WZ}}
\def\zz{\ensuremath{ZZ}}
\def\wgamma{\ensuremath{W\gamma}}
\def\wjets{\ensuremath{W+}jets} 
\def\tw{\ensuremath{tW}} 
\def\singletopt{\ensuremath{t} ($t$-chan)} 
\def\singletops{\ensuremath{t} ($s$-chan)} 
\def\all{all}
\def\ee{\ensuremath{ee}}
\def\emu{\ensuremath{e\mu}}
\def\mm{\ensuremath{\mu\mu}}

%units

%others
\def\pt{\ensuremath{p_T}}
\def\ipb{pb\ensuremath{^{-1}}}
\def\ifb{fb\ensuremath{^{-1}}}
\def\et{\ensuremath{E_T}}
\def\met{\ensuremath{E\!\!\!\!/_T}}
\def\fBrem{\ensuremath{f_{\rm brem}}}
\def\pin{\ensuremath{p_{\rm in}}}
\def\pout{\ensuremath{p_{\rm out}}}

\input{commands}

\setcounter{topnumber}{1}
\setcounter{bottomnumber}{1}


\begin{document}
% \setpagewiselinenumbers
\modulolinenumbers[1]
% \linenumbers
\date{\today} \title{Update of the search for the Higgs boson in WW$\to2l2\nu$ final state for the Lepton-Photon 2011 conference.}
\input{authors_compact}
\maketitle

%===================================================================================================
\section{Update Summary}

This document describes the ``reload'' of the H$\to$WW$\to2l2\nu$
analysis for the Lepton Photon 2011 conference.  The reload primarily
adds extra data to the existing analysis, but also includes minor
technical changes such as the residual energy scale correction, Monte
Carlo re-weighting for new pile-up conditions and a new luminosity
estimation.

The final dataset (LP) used for the analysis corresponds to \lpintlumi
of data. For comparison with the EPS conference results we split the
data sample in two parts: EPS (\intlumi) and post-EPS (0.4 fb$^{-1}$).

To update the analysis we first validated the post-EPS data. We find
the post-EPS data to be consistent with the EPS data with regard to
trigger and lepton efficiencies
(Tables~\ref{tab:lp_eff_electrons}-\ref{tab:lp_eff_muons}). Because we
do not observe any significant change in the lepton selection
efficiencies, we perform the LP analysis using the same data to
simulation scale factors as the EPS analysis. We find good agreement
between data and MC for the distribution of the number of
reconstructed vertices per event after re-weighting based on the
expected number of pile-up events.  We inspected the relevant
kinematic variables and find no significant discrepancies between the
two data samples (Appendix~\ref{app:lp_postEPSdist}).

We then performed the background estimations. The rate for each
background process is found to be stable in the EPS, post-EPS and full
LP data
samples. Tables~\ref{tab:routin_lp_data_0j}-\ref{tab:lp_periods_ww}
summarize the results for all dominant backgrounds. Observed
background yields per unit of luminosity are consistent for EPS
and post-EPS data. In the light of recent discussions of the
W$\gamma^{(*)}$ background, it is worth to point out that the closure
test on same-sign events gives good agreement with the expectations,
as shown in Table~\ref{tab:lp_FakeLeptonBkgPrediction_SameSignSample}.

The observed and expected upper limits correspoinding to full LP data
set are shown in Figure~\ref{fig:limits_final}.

\begin{figure}[!htbp]
\centering
\includegraphics[width=0.48\textwidth]{lp_figures/limits_nj_cut_ana_v6_1500pb_LP.pdf}
\includegraphics[width=0.48\textwidth]{lp_figures/limits_nj_shape_ana_v6_1500pb_LP.pdf}
\caption{Cut-based analysis (left) and multivariate-based analysis (right) upper limits at 95\% C.L. using data corresponding to 1.5~$\ifb$.
}
\label{fig:limits_final}
\end{figure}

In order to better understand the results we compared upper limit
results for EPS, post-EPS and full LP datasets in
Figures~\ref{fig:limits_0j_cut}-\ref{fig:limits_1j_shape}.

We notice that overall the post-EPS data show much better agreement
between observed and expected limits. In 0-jet case for same-flavor
events we see a downward fluctuation in both cut based and mva based
analyses (Figure~\ref{subfig:post_0j_sf_cut} and
\ref{subfig:post_0j_sf_shape}). In 1-jet case for same-flavor events
we see the observed limits to be consistent with the expected limits
(Figure~\ref{subfig:post_1j_sf_cut} and \ref{subfig:post_1j_sf_shape}
in both cut based and shape based analyses. It is different from the
EPS result (Figure~\ref{subfig:eps_1j_sf_cut} and
\ref{subfig:eps_1j_sf_shape}) that showed around 2$\sigma$ deviation
from the expected limits. This observation supports the idea that the
excess in 1-jet same-flavor events is likely to be just a fluctuation
and suggests that Drell-Yan background is under control.

\section{Additional Transverse Mass Selection Requirement}

We propose an additional selection requirement that can
be used to give extra confidence to the analysis. In general, a
requirement that $80 < M_T < m^{Higgs}$ would remove poorly understood
regions with only a negligible reduction in sensitivity.
Specifically, we refer to the following:

\begin{itemize}
    \item There is a concern that $W\gamma^*$ FSR background with
      $W\rightarrow \ell\nu$ and $\gamma^*\rightarrow\ell\ell$ could be
      a significant unaccounted background.  The $M_T$ of the
      reconstructed $e^{+}e^{-}$+MET system should be less than or
      equal to the expected $M_T$ from the decay of a $W$ boson.
    \item The \dytt~background is estimated entirely from simulation.
      This background manifests itself at low $M_T$, as illustrated in
      Figure~\ref{fig:ww_mthiggs_lp}.
    \item Other backgrounds such as W+jets and WW are reduced by
      removing the low $M_T$ region.
\end{itemize}

The yield cross check using same-sign events bounds the $W/\gamma*$ background 
at something less than 10 events at WW preselection level.  
This bound has the caveat that if the background is large and the 
fake rate method understimates the real background then the two
effects can cancel.

The default conversion rejection algorithm used in this analysis
includes a 2 cm flight distance (Lxy) cut, so would not reject this background
from apparently prompt conversions.
Taking into account the expected efficiency of our conversion
rejection algorithm, and removing the Lxy cut, 
we would expect to reject $30-50$\% of the residual 
$\gamma*\rightarrow e^{+}e^{-}$ background.
Performing this test we remove four events at the WW preselection level.
This reduction in the observed yield is consistent with expectations
from WW, W+jet and W+$\gamma$ simulation.
The events removed are distributed randomly in $M_T$.

Although we do not find evidence of a large unaccounted $W/\gamma*$ background,
we cannot conclusively prove that it is absent either.
We note that the proposed cut would reduce the \dytt~background from an expectation
of around 13 events to below one event in the one-jet bin $e-\mu$ final state at
the MVA preselection level.

\begin{figure}[!htbp]
\centering
\includegraphics[width=0.48\textwidth]{lp_figures/limits_nj_cut_ana_v6_1500pb_LP_MTCUT80.pdf}
\includegraphics[width=0.48\textwidth]{lp_figures/limits_nj_shape_ana_v6_1500pb_LP_MTCUT80.pdf}
\caption{Cut-based analysis (left) and multivariate-based analysis (right) upper limits at 95\% C.L. 
using data corresponding to 1.5~$\ifb$ applying the additional $m_T$ cut.}
\label{fig:limits_final_mt80}
\end{figure}

Figures \ref{fig:limits_final_mt80}, \ref{fig:limits_lp_mtcut80_cut} and
\ref{fig:limits_lp_mtcut80_shape} show results with the $M_T$ cut
applied. It has negligible effect on the expected analysis sensitivity
at all mass points. The observed limits get closer to the expected
ones at low Higgs mass points. This can be either a background
reduction effected or a statistical fluctuation.

\section{Conclusions}

The H$\to$WW$\to2l2\nu$ analysis was successfully updated for the
Lepton Photon 2011 conference. The results obtained in the post-EPS
dataset were found to be consistent with the results from the EPS
analysis, thus the two were combined to perform the update.

The EPS counting analysis indicated an excess in the same flavor final
state, both in the 0-jet Higgs boson mass range ~150-170GeV region and
the 1-jet analysis over the full mass range.  This trend is not
evident in the post-EPS dataset.

We conclude that adding the post-EPS dataset to this analysis 
supports the hypothesis that same flavor excesses in the EPS
analysis are more likely attributed to statistical fluctuation
than a systematic bias in Drell-Yan estimation.

Evidence has been presented that the robustness of the analysis can be
improved by the addition of a cut to remove events with an $M_T$
inconsistent with the signal hypothesis. We advocate to use this
selection requirement to reduce contribution of poorly understood
backgrounds to the analysis.

The Higgs exclusion limits are [140-200] GeV for the cut-based analysis and
[144-200] for the multivariate analysis. 

%%%%%%%%%%%%%%%%%%%%%%%%%%%%%%
\begin{figure}[!htbp]
\centering
\subfigure[]{
\centering
\label{subfig:lp_0j_sf_cut}
\includegraphics[width=0.48\textwidth]{lp_figures/limits_0j_sf_cut.pdf}}
\subfigure[]{
\centering
\label{subfig:lp_0j_of_cut}
\includegraphics[width=0.48\textwidth]{lp_figures/limits_0j_of_cut.pdf}}
\subfigure[]{
\centering
\label{subfig:eps_0j_sf_cut}
\includegraphics[width=0.48\textwidth]{lp_figures/limits_0j_sf_cut_ana_v6_1500pb_LP_EPS.pdf}}
\subfigure[]{
\centering
\label{subfig:eps_0j_of_cut}
\includegraphics[width=0.48\textwidth]{lp_figures/limits_0j_of_cut_ana_v6_1500pb_LP_EPS.pdf}}
\subfigure[]{
\centering
\label{subfig:post_0j_sf_cut}
\includegraphics[width=0.48\textwidth]{lp_figures/limits_0j_sf_cut_ana_v6_1500pb_LP_POSTEPS.pdf}}
\subfigure[]{
\centering
\label{subfig:post_0j_of_cut}
\includegraphics[width=0.48\textwidth]{lp_figures/limits_0j_of_cut_ana_v6_1500pb_LP_POSTEPS.pdf}}
\caption{Cut-based analysis upper limits at 95\% C.L. using LP, EPS and post-EPS datasets for 0-jet events.
\subref{subfig:lp_0j_sf_cut}: LP same-flavor; \subref{subfig:lp_0j_of_cut}: LP opposite-flavor; 
\subref{subfig:eps_0j_sf_cut}: EPS same-flavor; \subref{subfig:eps_0j_of_cut}: EPS opposite-flavor; 
\subref{subfig:post_0j_sf_cut}: post-EPS same-flavor; \subref{subfig:post_0j_of_cut}: post-EPS opposite-flavor; 
}
\label{fig:limits_0j_cut}
\end{figure}


\begin{figure}[!htbp]
\centering
\subfigure[]{
\centering
\label{subfig:lp_0j_sf_shape}
\includegraphics[width=0.48\textwidth]{lp_figures/limits_0j_sf_shape.pdf}}
\subfigure[]{
\centering
\label{subfig:lp_0j_of_shape}
\includegraphics[width=0.48\textwidth]{lp_figures/limits_0j_of_shape.pdf}}
\subfigure[]{
\centering
\label{subfig:eps_0j_sf_shape}
\includegraphics[width=0.48\textwidth]{lp_figures/limits_0j_sf_shape_ana_v6_1500pb_LP_EPS.pdf}}
\subfigure[]{
\centering
\label{subfig:eps_0j_of_shape}
\includegraphics[width=0.48\textwidth]{lp_figures/limits_0j_of_shape_ana_v6_1500pb_LP_EPS.pdf}}
\subfigure[]{
\centering
\label{subfig:post_0j_sf_shape}
\includegraphics[width=0.48\textwidth]{lp_figures/limits_0j_sf_shape_ana_v6_1500pb_LP_POSTEPS.pdf}}
\subfigure[]{
\centering
\label{subfig:post_0j_of_shape}
\includegraphics[width=0.48\textwidth]{lp_figures/limits_0j_of_shape_ana_v6_1500pb_LP_POSTEPS.pdf}}
\caption{Multivariate-based analysis upper limits at 95\% C.L. using LP, EPS and post-EPS datasets for 0-jet events.
\subref{subfig:lp_0j_sf_shape}: LP same-flavor; \subref{subfig:lp_0j_of_shape}: LP opposite-flavor; 
\subref{subfig:eps_0j_sf_shape}: EPS same-flavor; \subref{subfig:eps_0j_of_shape}: EPS opposite-flavor; 
\subref{subfig:post_0j_sf_shape}: post-EPS same-flavor; \subref{subfig:post_0j_of_shape}: post-EPS opposite-flavor; 
}
\label{fig:limits_0j_shape}
\end{figure}

\begin{figure}[!htbp]
\centering
\subfigure[]{
\centering
\label{subfig:lp_1j_sf_cut}
\includegraphics[width=0.48\textwidth]{lp_figures/limits_1j_sf_cut.pdf}}
\subfigure[]{
\centering
\label{subfig:lp_1j_of_cut}
\includegraphics[width=0.48\textwidth]{lp_figures/limits_1j_of_cut.pdf}}
\subfigure[]{
\centering
\label{subfig:eps_1j_sf_cut}
\includegraphics[width=0.48\textwidth]{lp_figures/limits_1j_sf_cut_ana_v6_1500pb_LP_EPS.pdf}}
\subfigure[]{
\centering
\label{subfig:eps_1j_of_cut}
\includegraphics[width=0.48\textwidth]{lp_figures/limits_1j_of_cut_ana_v6_1500pb_LP_EPS.pdf}}
\subfigure[]{
\centering
\label{subfig:post_1j_sf_cut}
\includegraphics[width=0.48\textwidth]{lp_figures/limits_1j_sf_cut_ana_v6_1500pb_LP_POSTEPS.pdf}}
\subfigure[]{
\centering
\label{subfig:post_1j_of_cut}
\includegraphics[width=0.48\textwidth]{lp_figures/limits_1j_of_cut_ana_v6_1500pb_LP_POSTEPS.pdf}}
\caption{Cut-based analysis upper limits at 95\% C.L. using LP, EPS and post-EPS datasets for 1-jet events.
\subref{subfig:lp_1j_sf_cut}: LP same-flavor; \subref{subfig:lp_1j_of_cut}: LP opposite-flavor; 
\subref{subfig:eps_1j_sf_cut}: EPS same-flavor; \subref{subfig:eps_1j_of_cut}: EPS opposite-flavor; 
\subref{subfig:post_1j_sf_cut}: post-EPS same-flavor; \subref{subfig:post_1j_of_cut}: post-EPS opposite-flavor; 
}
\label{fig:limits_1j_cut}
\end{figure}

\begin{figure}[!htbp]
\centering
\subfigure[]{
\centering
\label{subfig:lp_1j_sf_shape}
\includegraphics[width=0.48\textwidth]{lp_figures/limits_1j_sf_shape.pdf}}
\subfigure[]{
\centering
\label{subfig:lp_1j_of_shape}
\includegraphics[width=0.48\textwidth]{lp_figures/limits_1j_of_shape.pdf}}
\subfigure[]{
\centering
\label{subfig:eps_1j_sf_shape}
\includegraphics[width=0.48\textwidth]{lp_figures/limits_1j_sf_shape_ana_v6_1500pb_LP_EPS.pdf}}
\subfigure[]{
\centering
\label{subfig:eps_1j_of_shape}
\includegraphics[width=0.48\textwidth]{lp_figures/limits_1j_of_shape_ana_v6_1500pb_LP_EPS.pdf}}
\subfigure[]{
\centering
\label{subfig:post_1j_sf_shape}
\includegraphics[width=0.48\textwidth]{lp_figures/limits_1j_sf_shape_ana_v6_1500pb_LP_POSTEPS.pdf}}
\subfigure[]{
\centering
\label{subfig:post_1j_of_shape}
\includegraphics[width=0.48\textwidth]{lp_figures/limits_1j_of_shape_ana_v6_1500pb_LP_POSTEPS.pdf}}
\caption{Multivariate-based analysis upper limits at 95\% C.L. using LP, EPS and post-EPS datasets for 1-jet events.
\subref{subfig:lp_1j_sf_shape}: LP same-flavor; \subref{subfig:lp_1j_of_shape}: LP opposite-flavor; 
\subref{subfig:eps_1j_sf_shape}: EPS same-flavor; \subref{subfig:eps_1j_of_shape}: EPS opposite-flavor; 
\subref{subfig:post_1j_sf_shape}: post-EPS same-flavor; \subref{subfig:post_1j_of_shape}: post-EPS opposite-flavor; 
}
\label{fig:limits_1j_shape}
\end{figure}


%%%%%%%%%%%%%%%%%%%%%%%%%%%%%%
\begin{figure}[!htbp]
\centering
\subfigure[]{
\centering
\label{subfig:0j_sf}
\includegraphics[width=0.48\textwidth]{lp_figures/limits_0j_sf_cut_ana_v6_1500pb_LP_MTCUT80.pdf}}
\subfigure[]{
\centering
\label{subfig:0j_of}
\includegraphics[width=0.48\textwidth]{lp_figures/limits_0j_of_cut_ana_v6_1500pb_LP_MTCUT80.pdf}}
\subfigure[]{
\centering
\label{subfig:1j_sf}
\includegraphics[width=0.48\textwidth]{lp_figures/limits_1j_sf_cut_ana_v6_1500pb_LP_MTCUT80.pdf}}
\subfigure[]{
\centering
\label{subfig:1j_of}
\includegraphics[width=0.48\textwidth]{lp_figures/limits_1j_of_cut_ana_v6_1500pb_LP_MTCUT80.pdf}}
\subfigure[]{
\centering
\label{subfig:2j}
\includegraphics[width=0.48\textwidth]{lp_figures/limits_2j_cut_ana_v6_1500pb_LP_MTCUT80.pdf}}
\subfigure[]{
\centering
\label{subfig:njcomb}
\includegraphics[width=0.48\textwidth]{lp_figures/limits_nj_cut_ana_v6_1500pb_LP_MTCUT80.pdf}}
\caption{Cut-based analysis upper limits at 95\% C.L. using data corresponding to 1.5~$\ifb$ applying the additional $m_T$ cut.
The limits are shown in 4 final states separately. \subref{subfig:0j_sf}: SF in 0 Jet bin; 
\subref{subfig:0j_of}: OF in 0 Jet bin; \subref{subfig:1j_sf}: SF in 1 Jet bin; 
\subref{subfig:1j_of}: OF in 1 Jet bin; \subref{subfig:2j}: 2 Jet bin; \subref{subfig:njcomb}: 0/1/2 Jets combined; }
\label{fig:limits_lp_mtcut80_cut}
\end{figure}
%%%%%%%%%%%%%%%%%%%%%%%%%%%%%%
%%%%%%%%%%%%%%%%%%%%%%%%%%%%%%
\begin{figure}[!htbp]
\centering
\subfigure[]{
\centering
\label{subfig:0j_sf}
\includegraphics[width=0.48\textwidth]{lp_figures/limits_0j_sf_shape_ana_v6_1500pb_LP_MTCUT80.pdf}
}
\subfigure[]{
\centering
\label{subfig:0j_of}
\includegraphics[width=0.48\textwidth]{lp_figures/limits_0j_of_shape_ana_v6_1500pb_LP_MTCUT80.pdf}
}
\subfigure[]{
\centering
\label{subfig:1j_sf}
\includegraphics[width=0.48\textwidth]{lp_figures/limits_1j_sf_shape_ana_v6_1500pb_LP_MTCUT80.pdf}
}
\subfigure[]{
\centering
\label{subfig:1j_of}
\includegraphics[width=0.48\textwidth]{lp_figures/limits_1j_of_shape_ana_v6_1500pb_LP_MTCUT80.pdf}
}
\subfigure[]{
\centering
\label{subfig:nj}
\includegraphics[width=0.48\textwidth]{lp_figures/limits_nj_shape_ana_v6_1500pb_LP_MTCUT80.pdf}
}
\caption{Multivariate based analysis upper limits at 95\% C.L. using data corresponding to 1.5~$\ifb$, 
applying the additional $m_T$ cut.
The limits are shown in 4 final states separately. \subref{subfig:0j_sf}: SF in 0 Jet bin; 
\subref{subfig:0j_of}: OF in 0 Jet bin; \subref{subfig:1j_sf}: SF in 1 Jet bin; 
\subref{subfig:1j_of}: OF in 1 Jet bin; \subref{subfig:nj}: 0/1/2 Jets combined;
}
\label{fig:limits_lp_mtcut80_shape}
\end{figure}
%%%%%%%%%%%%%%%%%%%%%%%%%%%%%%


\clearpage

\input{lp2011_info}

\appendix

\section{Data Validation Plots}
\label{app:lp_postEPSdist}
This is a EPS vs Post-EPS distributions


%%      \subsubsection{Lepton and Trigger Efficiency}
%%      
We used the tag and probe method to study the lepton selection
efficiencies in the EPS and post-EPS datasets.
The post-EPS dataset was initially available in the PromptReco V5 processing,
which is known to contain ECAL calibration issues leading to large fake MET.
These studies were performed on the PromptReco V5 before the data was 
reprocessed because the calibration issues are not expected to bias the
lepton efficiencies.

The electron and muon selections efficiencies in the two datasets
and compared in Tables \ref{tab:lp_eff_electrons} and \ref{tab:lp_eff_muons}
respectively.
Because we do not observe any significant change in the
lepton selection efficiencies, we perform the LP analysis
using the same data to simulation scale factors as the EPS analysis.

\begin{table}[!ht]
\begin{center}
\begin{tabular} {c|ccc}
\hline
          & $10<p_T<15$ & $15<p_T<20$ & $p_T>20$ \\
\hline
\multicolumn{4}{c} {Barrel ($|\eta|<1.479$)} \\ \hline
post-EPS       & $0.3955 \pm 0.0411$ & $0.5443 \pm 0.0090$ & $0.6808 \pm 0.0029$ \\
EPS            & $0.4064 \pm 0.0250$ & $0.5184 \pm 0.0076$ & $0.6891 \pm 0.0013$ \\ \hline
\multicolumn{4}{c} {Endcap ($|\eta|>1.479$)} \\ \hline
post-EPS       & $0.1935 \pm 0.0265$ & $0.3320 \pm 0.0218$ & $0.6808 \pm 0.0029$ \\
EPS            & $0.2079 \pm 0.0103$ & $0.3219 \pm 0.0101$ & $0.6891 \pm 0.0013$ \\
\hline
\end{tabular}
\caption{Comparison of the electron efficiencies in the EPS and post-EPS datasets.}
\label{tab:lp_eff_electrons}
\end{center}
\end{table}

\begin{table}[!ht]
\begin{center}
\begin{tabular} {c|ccc}
\hline
          & $10<p_T<15$ & $15<p_T<20$ & $p_T>20$ \\
\hline
\multicolumn{4}{c} {Barrel ($|\eta|<1.479$)} \\ \hline
post-EPS       & N/A                 & $0.7651 \pm 0.0188$ & $0.6808 \pm 0.0029$ \\
EPS            & $0.6620 \pm 0.0108$ & $0.7306 \pm 0.0047$ & $0.6891 \pm 0.0013$ \\ \hline
\multicolumn{4}{c} {Endcap ($|\eta|>1.479$)} \\ \hline
post-EPS       & N/A                 & $0.7543 \pm 0.0197$ & $0.9193 \pm 0.0010$ \\
EPS            & $0.7079 \pm 0.0110$ & $0.7235 \pm 0.0073$ & $0.9138 \pm 0.0008$ \\
\hline
\end{tabular}
\caption{Comparison of the muon efficiencies in the EPS and post-EPS datasets.
Note that the low $p_T$ efficiencies were not available for the post-EPS dataset at the time of writing
with the code used to produce these results.
An independent measurement of the data to simulation scale factors found
$0.91 \pm 0.01$ ($0.91 \pm 0.02$) for the EPS dataset and $0.95 \pm 0.02$ ($0.90 \pm 0.03$)
for the post-EPS dataset in the Barrel (Endcap) regions.}
\label{tab:lp_eff_muons}
\end{center}
\end{table}


%%      \subsubsection{PU reweighting}
%%      
We perform a reweighting to take into account the difference 
between the pile-up model used to generate the simulated data samples.
Because the expected pile-up distribution depends on the instantaneous 
luminosity, we reweight to the target distribution that corresponds to 
the dataset used.
The result of applying this procedure to the EPS, post-EPS and LP datasets
is shown in Figures \ref{subfig:lp_pureweight_eps}, \ref{subfig:lp_pureweight_posteps}
and \ref{subfig:lp_pureweight_lp} respectively. 
In all cases we find good agreement between the data and the simulation
after reweighting.

\begin{figure}[!hbtp]
\centering
\subfigure[]{
\centering
\label{subfig:lp_pureweight_eps}
\includegraphics[width=.4\textwidth]{lp_figures/puReweight/histo_nvtx_ww0j_allhwwcuts_EPS.pdf}}
\subfigure[]{
\centering
\label{subfig:lp_pureweight_posteps}
\includegraphics[width=.4\textwidth]{lp_figures/puReweight/histo_nvtx_ww0j_allhwwcuts_POSTEPS.pdf}}\\
\subfigure[]{
\centering
\label{subfig:lp_pureweight_lp}
\includegraphics[width=.4\textwidth]{lp_figures/puReweight/histo_nvtx_ww0j_allhwwcuts.pdf}}
\caption{The number of reconstructed vertices in data and simulation 
at the WW preselection level after pile-up reweighting
according to the expected pile-up multiplicity
\subref{subfig:lp_pureweight_eps} in the EPS dataset;
\subref{subfig:lp_pureweight_posteps} in the post-EPS dataset;
\subref{subfig:lp_pureweight_lp} in the LP dataset;
}
\label{fig:lp_ww0j_dilep}
\end{figure}


%%     \subsubsection{Summary of Data Validation}
%%     We performed a validation of the post-EPS dataset by comparing it to the EPS data and to the MC. 
We found that everything is consistent, see sections \ref{app:lp_postEPSdist} for details.



%%      \subsection{Background Estimation for \lpintlumi}
%%     \label{app:lp_bkgestim}
%%      This section describes the results of the data-driven background estimation with \lpintlumi. 
The rate for each background process is proven to be reasonably stable in the EPS, post-EPS and full LP data samples.

\subsubsection{Drell-Yan}

Table~XX shows the results for the Drell-Yan estimation in the LP dataset. 
The method is described in Section~\ref{sec:bkg_dy}.

YY should insert the summary table for DY estimation with \lpintlumi.

The scale factors obtained in the EPS, post-EPS and LP datasets are compatible within the large uncertainties (Table~\ref{tab:lp_periods_dy}).

\begin{table}[!htbp]
\begin{center}
\begin{tabular}{c c c c} 
\hline
Period & 0-jet bin & 1-jet bin & 2-jet bin \\ 
\hline
EPS      & 3.47 $\pm$ 2.26 & 2.69 $\pm$ 1.37 & 4.81 $\pm$ 1.84 \\
post-EPS & 2.71 $\pm$ 3.50 & 3.17 $\pm$ 1.84 & 4.90 $\pm$ 1.97 \\
LP       & 3.02 $\pm$ 1.84 & 2.81 $\pm$ 1.40 & 4.84 $\pm$ 1.83 \\
\hline
\end{tabular}
\caption{Comparison of data-MC scale factors for the DY background for the EPS, post-EPS and LP (LP=EPS+post-EPS) data samples.}
\label{tab:lp_periods_dy}
\end{center}
\end{table}


\subsubsection{Fake-induced background}

The fake-induced background estimation on the LP dataset is reported in Table~\ref{tab:lp_fake_est} 
(see Section~\ref{sec:bkg_fakes} for details on the method).
The closure test on same-sign events gives good agreement with the expectations, as shown in Table~\ref{tab:lp_FakeLeptonBkgPrediction_SameSignSample}.

A summary of the rate of fake events per \ifb of data in the EPS, post-EPS and LP is in Table~\ref{tab:lp_periods_fake}. 
The level of agreement is at the 1-2 sigma level.

\begin{table}[!htbp]
\begin{center}
\begin{tabular}{c c c c c c} 
\hline
jet-bin &	 $\mu\mu$ &	 $\mu e$ &	 $e\mu$ &	 $ee$ &	 total \\ 
\hline
0 &	 12.76 $\pm$ 2.24 &	 43.67 $\pm$ 2.72 &	 66.67 $\pm$ 4.22 &	 15.34 $\pm$ 1.47 &	 138.46 $\pm$ 5.69 \\
1 &	 6.70 $\pm$ 1.86  &      16.45 $\pm$ 1.87 &      28.18 $\pm$ 2.86 &       4.91 $\pm$ 0.93 &       56.27 $\pm$ 4.00 \\
2 &	 2.04 $\pm$ 1.26  &       2.85 $\pm$ 0.61 &      13.39 $\pm$ 1.90 &       1.42 $\pm$ 0.55 &       22.62 $\pm$ 2.66 \\
\hline
\end{tabular}
\caption{Predictions of the fake-induced background contribution 
in the data-driven estimation after the $WW$ preselection. 
The analyzed data correspond to \lpintlumi.
The uncertainties are statistical only.}
\label{tab:lp_fake_est}
\end{center}
\end{table}

\begin{table}[!htbp]
\begin{center}
\begin{tabular}{|l|c|c|}
\hline
Type                                                             & Yield \\
\hline
Estimated events                                                 &  $68.4\pm3.8^{+15.6}_{-10.6}$  \\
\hline
Observed same-sign events in Data                                &  $92$        \\
Monte Carlo estimate of non-fake contribution (WZ \& W$\gamma)$  & $28.0\pm1.9$ \\
Fake background observed                                         & $64.0\pm9.7$ \\
\hline
\end{tabular}
\caption{Summary of fake lepton background yields in the same sign sample after WW selection (\lpintlumi). }
\label{tab:lp_FakeLeptonBkgPrediction_SameSignSample}
\end{center}
\end{table}

\begin{table}[!htbp]
\begin{center}
\begin{tabular}{c c c c} 
\hline
Period & 0-jet bin & 1-jet bin & 2-jet bin \\ 
\hline
EPS      & 94.4 $\pm$ 4.4 & 34.2 $\pm$ 2.9 & 13.9 $\pm$ 1.9 \\
post-EPS & 76.5 $\pm$ 6.3 & 42.6 $\pm$ 5.3 & 16.9 $\pm$ 3.6 \\
LP       & 89.6 $\pm$ 3.7 & 36.4 $\pm$ 2.6 & 14.6 $\pm$ 1.7 \\
\hline
\end{tabular}
\caption{Comparison of the number of fake-induced events per \ifb of data in the EPS, post-EPS and LP (LP=EPS+post-EPS) data samples.}
\label{tab:lp_periods_fake}
\end{center}
\end{table}

\subsubsection{Top}

Table~\ref{tab:lp_ttbar_est} summarizes the results for the top background estimation in the LP data sample. 
The method described in details in Section~\ref{sec:bkg_top}.

The scale factors for top background in EPS and post-EPS data are compatible within one sigma (Table~\ref{tab:lp_periods_top}).

\begin{table}[!htbp]
\begin{center}
\begin{tabular}{l c c c}
\hline
Sample                                        &   0-jet          & 1-jet          \\
\hline
Estimated top events in simulation  	      &  48.1 $\pm$ 2.9  & 147.8 $\pm$ 4.4 \\
tagging efficiency (\%)                       &  52.5 $\pm$ 4.6  &  68.1 $\pm$ 1.6 \\
top-tagged events in data           	      &          106     &    386          \\
background events in control region           &  26.3 $\pm$ 4.6  &  21.1 $\pm$ 3.7 \\
Data-driven top background estimate           &  72.1 $\pm$ 17.5 & 170.5 $\pm$ 15.3\\
\hline
\end{tabular}
\caption{Predictions of the top background contribution compared 
with observed event counts in the data-driven estimation after the $WW$ preselection. 
The analyzed data correspond to \lpintlumi.
The uncertainties are statistical only. The systematic uncertainties are expected to be 
negligible at this level or precision.}
\label{tab:lp_ttbar_est}
\end{center}
\end{table}

\begin{table}[!htbp]
\begin{center}
\begin{tabular}{c c c} 
\hline
Period & 0-jet bin & 1-jet bin \\ 
\hline
EPS      & 1.67 $\pm$ 0.25 & 1.20 $\pm$ 0.10 \\
post-EPS & 1.12 $\pm$ 0.40 & 1.02 $\pm$ 0.18 \\
LP       & 1.50 $\pm$ 0.24 & 1.15 $\pm$ 0.09 \\
\hline
\end{tabular}
\caption{Comparison of data-MC scale factors for the top background for the EPS, post-EPS and LP (LP=EPS+post-EPS) data samples.}
\label{tab:lp_periods_top}
\end{center}
\end{table}

\subsubsection{WW}

The data driven estimation of the WW background for the cut based and MVA based analyses are summaried in Tables~\ref{tab:lp_wwEstimResData} 
and \ref{tab:lp_wwEstimResDataMVA} respectively. They correspond to the full LP dataset (\lpintlumi) and follow the 
method described in Section~\ref{sec:bkg_ww}.

The WW rate is stable in the EPS and post-EPS data samples (Table~\ref{tab:lp_periods_ww}).

\begin{table}[!htbp]
\begin{center}
\begin{tabular}{c | c c c | c c c}
\hline
$m_H$ & \multicolumn{3}{c}{0-jet bin} & \multicolumn{3}{|c}{1-jet bin} \\
$[\GeVcc]$ & estimate & expected & scale factor & estimate  & expected & scale factor \\ \hline
115 & 46.6 $\pm$  5.3 & 43.7 $\pm$  0.7 & 1.07 $\pm$ 0.12 & 10.9 $\pm$  2.9 &  9.9 $\pm$  0.3 & 1.11 $\pm$ 0.29 \\
120 & 61.0 $\pm$  6.9 & 57.2 $\pm$  0.8 & 1.07 $\pm$ 0.12 & 14.4 $\pm$  3.8 & 13.0 $\pm$  0.4 & 1.11 $\pm$ 0.29 \\
130 & 70.5 $\pm$  8.0 & 65.8 $\pm$  0.8 & 1.07 $\pm$ 0.12 & 16.7 $\pm$  4.4 & 15.0 $\pm$  0.4 & 1.11 $\pm$ 0.29 \\
140 & 64.1 $\pm$  7.2 & 60.9 $\pm$  0.8 & 1.05 $\pm$ 0.12 & 15.1 $\pm$  3.9 & 13.6 $\pm$  0.4 & 1.11 $\pm$ 0.29 \\
150 & 41.3 $\pm$  5.0 & 42.0 $\pm$  0.6 & 0.98 $\pm$ 0.12 & 13.3 $\pm$  3.6 & 12.2 $\pm$  0.3 & 1.09 $\pm$ 0.30 \\
160 & 28.7 $\pm$  3.5 & 29.2 $\pm$  0.5 & 0.99 $\pm$ 0.12 & 11.3 $\pm$  3.1 & 10.4 $\pm$  0.3 & 1.09 $\pm$ 0.30 \\
170 & 22.7 $\pm$  2.8 & 23.0 $\pm$  0.5 & 0.98 $\pm$ 0.12 &  9.5 $\pm$  2.6 &  8.7 $\pm$  0.3 & 1.10 $\pm$ 0.30 \\
180 & 26.0 $\pm$  3.2 & 26.5 $\pm$  0.5 & 0.98 $\pm$ 0.12 & 11.5 $\pm$  3.1 & 10.3 $\pm$  0.3 & 1.12 $\pm$ 0.30 \\
190 & 39.5 $\pm$  4.7 & 39.9 $\pm$  0.6 & 0.99 $\pm$ 0.12 & 16.7 $\pm$  4.4 & 14.9 $\pm$  0.4 & 1.12 $\pm$ 0.30 \\
200 & 40.6 $\pm$  4.9 & 41.5 $\pm$  0.6 & 0.98 $\pm$ 0.12 & 18.7 $\pm$  4.9 & 16.5 $\pm$  0.4 & 1.14 $\pm$ 0.30 \\
\hline
\end{tabular}
\caption{Cut-based analysis: data driven WW estimation for different Higgs masses in the 0- and 1-jet bins (\lpintlumi). 
Only statistical uncertainties are reported.}
\label{tab:lp_wwEstimResData}
\end{center}
\end{table}

\begin{table}[!htbp]
\begin{center}
\begin{tabular}{c | c c c | c c c}
\hline
$m_H$ & \multicolumn{3}{c}{0-jet bin} & \multicolumn{3}{|c}{1-jet bin} \\
$[\GeVcc]$ & estimate & expected & scale factor & estimate  & expected & scale factor \\ \hline
115 & 274.9 $\pm$ 31.1 & 257.8 $\pm$  1.6 & 1.07 $\pm$ 0.12 & 84.8 $\pm$ 22.3 & 76.8 $\pm$  0.9 & 1.11 $\pm$ 0.29 \\
120 & 274.9 $\pm$ 31.1 & 257.8 $\pm$  1.6 & 1.07 $\pm$ 0.12 & 84.8 $\pm$ 22.3 & 76.8 $\pm$  0.9 & 1.11 $\pm$ 0.29 \\
130 & 317.4 $\pm$ 35.9 & 297.6 $\pm$  1.7 & 1.07 $\pm$ 0.12 & 98.8 $\pm$ 26.0 & 89.4 $\pm$  0.9 & 1.11 $\pm$ 0.29 \\
140 & 344.9 $\pm$ 39.0 & 323.4 $\pm$  1.8 & 1.07 $\pm$ 0.12 & 107.5 $\pm$ 28.3 & 97.3 $\pm$  1.0 & 1.11 $\pm$ 0.29 \\
150 & 368.4 $\pm$ 41.7 & 345.5 $\pm$  1.8 & 1.07 $\pm$ 0.12 & 114.5 $\pm$ 30.1 & 103.6 $\pm$  1.0 & 1.11 $\pm$ 0.29 \\
160 & 368.4 $\pm$ 41.7 & 345.5 $\pm$  1.8 & 1.07 $\pm$ 0.12 & 114.5 $\pm$ 30.1 & 103.6 $\pm$  1.0 & 1.11 $\pm$ 0.29 \\
170 & 368.4 $\pm$ 41.7 & 345.5 $\pm$  1.8 & 1.07 $\pm$ 0.12 & 114.5 $\pm$ 30.1 & 103.6 $\pm$  1.0 & 1.11 $\pm$ 0.29 \\
180 & 392.4 $\pm$ 44.4 & 368.0 $\pm$  1.9 & 1.07 $\pm$ 0.12 & 122.2 $\pm$ 32.1 & 110.6 $\pm$  1.0 & 1.11 $\pm$ 0.29 \\
190 & 417.6 $\pm$ 47.2 & 391.6 $\pm$  2.0 & 1.07 $\pm$ 0.12 & 130.9 $\pm$ 34.4 & 118.5 $\pm$  1.1 & 1.11 $\pm$ 0.29 \\
200 & 438.2 $\pm$ 49.5 & 410.9 $\pm$  2.0 & 1.07 $\pm$ 0.12 & 137.4 $\pm$ 36.1 & 124.3 $\pm$  1.1 & 1.11 $\pm$ 0.29 \\\hline
\end{tabular}
\caption{MVA analysis: data driven WW estimation for different Higgs masses in the 0- and 1-jet bin (\lpintlumi). 
Only statistical uncertainties are reported.}
\label{tab:lp_wwEstimResDataMVA}
\end{center}
\end{table}

\begin{table}[!htbp]
\begin{center}
\begin{tabular}{c c c} 
\hline
Period & 0-jet bin & 1-jet bin \\ 
\hline
EPS      & 1.05 $\pm$ 0.14 & 1.14 $\pm$ 0.33 \\
post-EPS & 1.09 $\pm$ 0.20 & 1.00 $\pm$ 0.51 \\
LP       & 1.07 $\pm$ 0.12 & 1.11 $\pm$ 0.29 \\
\hline
\end{tabular}
\caption{Comparison of data-MC scale factors for the WW background (MVA analysis) for the EPS, post-EPS and LP (LP=EPS+post-EPS) data samples.}
\label{tab:lp_periods_ww}
\end{center}
\end{table}


\clearpage

%%      \subsection{Final limits for \lpintlumi}
%%     \label{app:lp_limits}
%%      

\subsection{Results in different final states}

We compare the upper limits obtained using the data with the run number $<170826$ ( referred to 
as the EPS data) and the one after (referred to as the post-EPS data). 
The comparisons are done in the 0 and 1 jet bins separating the 
same flavor ($ee/\mu\mu$) and opposite flavor ($e\mu$) final states.
The observed and expected upper limits corresponding to the EPS data are shown in 
Figure~\ref{fig:limits_eps_cut}-\ref{fig:limits_eps_shape} for cut-based and multivariate 
based analyeses respectively, with the results tabulated in 
Table~\ref{tab:limits_eps_cut}-\ref{tab:limits_eps_shape} respectively.
The observed and expected upper limits corresponding to the post-EPS data are shown in 
Figure~\ref{fig:limits_posteps_cut}-\ref{fig:limits_posteps_shape} for cut-based and multivariate 
based analyeses respectively, with the results tabulated in 
Table~\ref{tab:limits_posteps_cut}-\ref{tab:limits_posteps_shape} respectively.



%%%%%%%%%%%%%%%%%%%%%%%%%%%%%%
\begin{figure}[!htbp]
\centering
\subfigure[]{
\centering
\label{subfig:0j_sf}
\includegraphics[width=0.48\textwidth]{lp_figures/limits_0j_sf_cut_ana_v6_1500pb_LP_EPS.pdf}}
\subfigure[]{
\centering
\label{subfig:0j_of}
\includegraphics[width=0.48\textwidth]{lp_figures/limits_0j_of_cut_ana_v6_1500pb_LP_EPS.pdf}}
\subfigure[]{
\centering
\label{subfig:1j_sf}
\includegraphics[width=0.48\textwidth]{lp_figures/limits_1j_sf_cut_ana_v6_1500pb_LP_EPS.pdf}}
\subfigure[]{
\centering
\label{subfig:1j_of}
\includegraphics[width=0.48\textwidth]{lp_figures/limits_1j_of_cut_ana_v6_1500pb_LP_EPS.pdf}}
\caption{Cut-based analysis upper limits at 95\% C.L. using the EPS data (run $<=$ 170826) corresponding to 1.1~$\ifb$.
The limits are shown in 4 final states separately. \subref{subfig:0j_sf}: SF in 0 Jet bin; 
\subref{subfig:0j_of}: OF in 0 Jet bin; \subref{subfig:1j_sf}: SF in 1 Jet bin; 
\subref{subfig:1j_of}: OF in 1 Jet bin; 
}
\label{fig:limits_eps_cut}
\end{figure}
%%%%%%%%%%%%%%%%%%%%%%%%%%%%%%
\clearpage
%%%%%%%%%%%%%%%%%%%%%%%%%%%%%%
\begin{figure}[!htbp]
\centering
\subfigure[]{
\centering
\label{subfig:0j_sf}
\includegraphics[width=0.48\textwidth]{lp_figures/limits_0j_sf_shape_ana_v6_1500pb_LP_EPS.pdf}}
\subfigure[]{
\centering
\label{subfig:0j_of}
\includegraphics[width=0.48\textwidth]{lp_figures/limits_0j_of_shape_ana_v6_1500pb_LP_EPS.pdf}}
\subfigure[]{
\centering
\label{subfig:1j_sf}
\includegraphics[width=0.48\textwidth]{lp_figures/limits_1j_sf_shape_ana_v6_1500pb_LP_EPS.pdf}}
\subfigure[]{
\centering
\label{subfig:1j_of}
\includegraphics[width=0.48\textwidth]{lp_figures/limits_1j_of_shape_ana_v6_1500pb_LP_EPS.pdf}}
\caption{Multivariate based analysis upper limits at 95\% C.L. using the EPS data (run $<=$ 170826) corresponding to 1.1~$\ifb$.
The limits are shown in 4 final states separately. \subref{subfig:0j_sf}: SF in 0 Jet bin; 
\subref{subfig:0j_of}: OF in 0 Jet bin; \subref{subfig:1j_sf}: SF in 1 Jet bin; 
\subref{subfig:1j_of}: OF in 1 Jet bin; 
}
\label{fig:limits_eps_shape}
\end{figure}
%%%%%%%%%%%%%%%%%%%%%%%%%%%%%%
\clearpage

%%%%%%%%%%%%%%%%%%%%%%%%%%%%%%
\begin{figure}[!htbp]
\centering
\subfigure[]{
\centering
\label{subfig:0j_sf}
\includegraphics[width=0.48\textwidth]{lp_figures/limits_0j_sf_cut_ana_v6_1500pb_LP_POSTEPS.pdf}}
\subfigure[]{
\centering
\label{subfig:0j_of}
\includegraphics[width=0.48\textwidth]{lp_figures/limits_0j_of_cut_ana_v6_1500pb_LP_POSTEPS.pdf}}
\subfigure[]{
\centering
\label{subfig:1j_sf}
\includegraphics[width=0.48\textwidth]{lp_figures/limits_1j_sf_cut_ana_v6_1500pb_LP_POSTEPS.pdf}}
\subfigure[]{
\centering
\label{subfig:1j_of}
\includegraphics[width=0.48\textwidth]{lp_figures/limits_1j_of_cut_ana_v6_1500pb_LP_POSTEPS.pdf}}
\caption{Cut-based analysis upper limits at 95\% C.L. using the post-EPS data (run $>$ 170826) corresponding to 0.4~$\ifb$.
The limits are shown in 4 final states separately. \subref{subfig:0j_sf}: SF in 0 Jet bin; 
\subref{subfig:0j_of}: OF in 0 Jet bin; \subref{subfig:1j_sf}: SF in 1 Jet bin; 
\subref{subfig:1j_of}: OF in 1 Jet bin; 
}
\label{fig:limits_posteps_cut}
\end{figure}
%%%%%%%%%%%%%%%%%%%%%%%%%%%%%%
\clearpage

%%%%%%%%%%%%%%%%%%%%%%%%%%%%%%
\begin{figure}[!htbp]
\centering
\subfigure[]{
\centering
\label{subfig:0j_sf}
\includegraphics[width=0.48\textwidth]{lp_figures/limits_0j_sf_shape_ana_v6_1500pb_LP_POSTEPS.pdf}}
\subfigure[]{
\centering
\label{subfig:0j_of}
\includegraphics[width=0.48\textwidth]{lp_figures/limits_0j_of_shape_ana_v6_1500pb_LP_POSTEPS.pdf}}
\subfigure[]{
\centering
\label{subfig:1j_sf}
\includegraphics[width=0.48\textwidth]{lp_figures/limits_1j_sf_shape_ana_v6_1500pb_LP_POSTEPS.pdf}}
\subfigure[]{
\centering
\label{subfig:1j_of}
\includegraphics[width=0.48\textwidth]{lp_figures/limits_1j_of_shape_ana_v6_1500pb_LP_POSTEPS.pdf}}
\caption{Mutivariate based analysis upper limits at 95\% C.L. using the post-EPS data (run $>$ 170826) corresponding to 0.4~$\ifb$.
The limits are shown in 4 final states separately. \subref{subfig:0j_sf}: SF in 0 Jet bin; 
\subref{subfig:0j_of}: OF in 0 Jet bin; \subref{subfig:1j_sf}: SF in 1 Jet bin; 
\subref{subfig:1j_of}: OF in 1 Jet bin; 
}
\label{fig:limits_posteps_shape}
\end{figure}
%%%%%%%%%%%%%%%%%%%%%%%%%%%%%%
\clearpage


%%%%%%%%%%%%%%%%%%%%%%%%%%%%%%
\begin{table}
\begin{center}
\begin{tabular}{c c c c c}
\hline\hline
 $m_H$ (GeV) & Observed & Median Expected & 68\% C.L. Band & 95\% C.L. Band \\ \hline
\hline
\multicolumn{5}{c} {0-Jet Bin Same Flavor} \\
\hline
115 & 4.1 & 8.8 & [5.7, 15.4] & [4.0, 36.0] \\
120 & 4.1 & 5.0 & [3.2, 8.4] & [2.3, 17.6] \\
130 & 2.5 & 2.3 & [1.5, 3.4] & [1.0, 5.4] \\
140 & 1.6 & 1.5 & [1.0, 2.2] & [0.7, 3.3] \\
150 & 1.8 & 1.0 & [0.7, 1.5] & [0.5, 2.4] \\
160 & 1.0 & 0.6 & [0.4, 0.9] & [0.3, 1.6] \\
170 & 1.0 & 0.7 & [0.5, 1.0] & [0.3, 1.6] \\
180 & 0.6 & 1.0 & [0.7, 1.5] & [0.5, 2.2] \\
190 & 1.6 & 1.5 & [1.0, 2.1] & [0.7, 3.1] \\
200 & 2.1 & 2.2 & [1.6, 3.3] & [1.1, 4.6] \\
250 & 3.3 & 5.8 & [4.0, 8.5] & [3.0, 12.2] \\
300 & 4.3 & 5.5 & [3.9, 8.3] & [2.8, 11.9] \\
\hline
\multicolumn{5}{c} {0-Jet Bin Opposite Flavor} \\
\hline
115 & 7.6 & 6.6 & [4.5, 9.8] & [3.2, 14.4] \\
120 & 4.9 & 3.8 & [2.6, 5.7] & [1.9, 8.2] \\
130 & 2.8 & 1.9 & [1.3, 2.9] & [1.0, 4.2] \\
140 & 1.5 & 1.3 & [0.9, 1.8] & [0.6, 2.6] \\
150 & 0.9 & 1.0 & [0.7, 1.5] & [0.5, 2.2] \\
160 & 0.5 & 0.5 & [0.4, 0.8] & [0.3, 1.2] \\
170 & 0.7 & 0.7 & [0.5, 1.0] & [0.3, 1.4] \\
180 & 0.9 & 0.9 & [0.7, 1.4] & [0.5, 2.0] \\
190 & 1.4 & 1.3 & [0.9, 2.0] & [0.7, 2.8] \\
200 & 2.3 & 1.9 & [1.3, 2.8] & [1.0, 4.2] \\
250 & 3.1 & 3.9 & [2.8, 5.8] & [2.0, 8.2] \\
300 & 3.8 & 4.5 & [3.1, 6.6] & [2.3, 9.5] \\
\hline
\multicolumn{5}{c} {1-Jet Bin Same Flavor} \\
\hline
115 & 19.7 & 17.7 & [11.4, 26.5] & [8.6, 43.9] \\
120 & 12.4 & 9.4 & [6.4, 15.3] & [4.7, 27.8] \\
130 & 8.3 & 4.6 & [3.1, 7.2] & [2.1, 14.5] \\
140 & 5.7 & 2.7 & [1.7, 4.1] & [1.2, 7.0] \\
150 & 4.2 & 2.1 & [1.3, 3.2] & [1.0, 5.2] \\
160 & 2.3 & 1.1 & [0.8, 1.8] & [0.6, 2.7] \\
170 & 2.7 & 1.4 & [1.0, 2.1] & [0.7, 3.2] \\
180 & 2.9 & 1.8 & [1.3, 2.6] & [0.9, 3.8] \\
190 & 4.3 & 2.6 & [1.7, 4.0] & [1.3, 6.1] \\
200 & 6.5 & 4.0 & [2.7, 6.0] & [2.0, 9.4] \\
250 & 11.2 & 8.6 & [6.0, 13.2] & [4.4, 19.7] \\
300 & 12.1 & 8.7 & [6.0, 13.1] & [4.3, 19.5] \\
\hline
\multicolumn{5}{c} {1-Jet Bin Opposite Flavor} \\
\hline
115 & 19.4 & 10.2 & [6.9, 15.4] & [5.1, 22.9] \\
120 & 10.4 & 6.0 & [4.0, 8.9] & [3.0, 13.0] \\
130 & 4.1 & 3.2 & [2.1, 4.7] & [1.5, 7.2] \\
140 & 2.2 & 2.0 & [1.4, 3.0] & [1.0, 4.6] \\
150 & 1.6 & 1.6 & [1.1, 2.3] & [0.8, 3.5] \\
160 & 1.1 & 1.0 & [0.6, 1.4] & [0.5, 2.1] \\
170 & 1.2 & 1.2 & [0.8, 1.7] & [0.6, 2.7] \\
180 & 1.6 & 1.6 & [1.1, 2.5] & [0.9, 3.6] \\
190 & 2.1 & 2.6 & [1.8, 3.8] & [1.3, 5.6] \\
200 & 3.1 & 3.2 & [2.2, 4.8] & [1.6, 7.1] \\
250 & 6.2 & 6.1 & [4.2, 9.1] & [3.1, 13.6] \\
300 & 7.8 & 6.9 & [4.8, 10.4] & [3.5, 15.2] \\
\hline\hline
\end{tabular}
\end{center}
\caption{Multivariate based upper limits at 95\% C.L. in 0 and 1 Jet final state, 
using the post-EPS data (run $<=$ 170826) corresponding to  1.1~$\ifb$ 
shown in Figure~\ref{fig:limits_eps_cut}.}
\label{tab:limits_eps_cut}
\end{table}
%%%%%%%%%%%%%%%%%%%%%%%%%%%%%%

%%%%%%%%%%%%%%%%%%%%%%%%%%%%%%
\begin{table}
\begin{center}
\begin{tabular}{c c c c c}
\hline\hline
 $m_H$ (GeV) & Observed & Median Expected & 68\% C.L. Band & 95\% C.L. Band \\ \hline
\hline
\multicolumn{5}{c} {0-Jet Bin Same Flavor} \\
\hline
115 & 3.7 & 6.0 & [4.2, 8.7] & [3.1, 12.6] \\
120 & 2.6 & 3.6 & [2.5, 5.2] & [1.8, 7.5] \\
130 & 2.1 & 1.8 & [1.2, 2.6] & [0.9, 3.7] \\
140 & 1.9 & 1.2 & [0.8, 1.7] & [0.6, 2.4] \\
150 & 1.5 & 0.8 & [0.6, 1.2] & [0.4, 1.7] \\
160 & 1.0 & 0.5 & [0.3, 0.7] & [0.2, 1.0] \\
170 & 0.7 & 0.5 & [0.4, 0.8] & [0.3, 1.1] \\
180 & 0.8 & 0.7 & [0.5, 1.1] & [0.4, 1.6] \\
190 & 1.0 & 1.2 & [0.8, 1.8] & [0.6, 2.6] \\
200 & 1.5 & 1.8 & [1.3, 2.7] & [0.9, 3.9] \\
250 & 3.0 & 3.5 & [2.5, 5.2] & [1.8, 7.5] \\
300 & 3.5 & 3.9 & [2.8, 5.8] & [2.0, 8.2] \\
\hline
\multicolumn{5}{c} {0-Jet Bin Opposite Flavor} \\
\hline
115 & 10.9 & 5.0 & [3.4, 7.3] & [2.5, 10.5] \\
120 & 5.9 & 2.9 & [2.0, 4.3] & [1.4, 5.9] \\
130 & 2.4 & 1.5 & [1.1, 2.2] & [0.8, 3.2] \\
140 & 1.3 & 1.0 & [0.7, 1.4] & [0.5, 2.1] \\
150 & 1.1 & 0.7 & [0.5, 1.0] & [0.4, 1.5] \\
160 & 0.7 & 0.4 & [0.3, 0.6] & [0.2, 0.9] \\
170 & 0.7 & 0.5 & [0.3, 0.7] & [0.3, 1.1] \\
180 & 0.8 & 0.7 & [0.5, 1.0] & [0.4, 1.5] \\
190 & 1.3 & 1.1 & [0.7, 1.6] & [0.5, 2.2] \\
200 & 1.8 & 1.4 & [1.0, 2.1] & [0.7, 3.0] \\
250 & 2.2 & 2.6 & [1.8, 3.7] & [1.3, 5.3] \\
300 & 2.7 & 3.2 & [2.2, 4.7] & [1.6, 6.6] \\
\hline
\multicolumn{5}{c} {1-Jet Bin Same Flavor} \\
\hline
115 & 24.4 & 14.8 & [10.3, 21.7] & [7.5, 30.8] \\
120 & 11.5 & 8.3 & [5.7, 12.2] & [4.2, 17.9] \\
130 & 6.5 & 3.7 & [2.5, 5.5] & [1.8, 7.8] \\
140 & 4.1 & 2.1 & [1.5, 3.1] & [1.1, 4.6] \\
150 & 2.6 & 1.4 & [1.0, 2.2] & [0.7, 3.2] \\
160 & 1.6 & 0.9 & [0.6, 1.3] & [0.4, 2.0] \\
170 & 1.8 & 1.0 & [0.7, 1.5] & [0.5, 2.3] \\
180 & 2.5 & 1.3 & [0.8, 1.9] & [0.6, 2.9] \\
190 & 4.0 & 2.0 & [1.3, 3.0] & [0.9, 4.6] \\
200 & 5.8 & 3.0 & [2.0, 4.6] & [1.4, 6.9] \\
250 & 6.7 & 5.6 & [3.8, 8.4] & [2.7, 12.5] \\
300 & 7.2 & 5.8 & [4.0, 8.5] & [2.9, 12.2] \\
\hline
\multicolumn{5}{c} {1-Jet Bin Opposite Flavor} \\
\hline
115 & 15.8 & 8.7 & [6.1, 12.8] & [4.4, 18.3] \\
120 & 8.3 & 5.5 & [3.8, 8.0] & [2.8, 11.6] \\
130 & 3.8 & 2.8 & [1.9, 4.2] & [1.4, 6.0] \\
140 & 2.1 & 1.7 & [1.1, 2.5] & [0.8, 3.6] \\
150 & 1.5 & 1.2 & [0.9, 1.9] & [0.6, 2.7] \\
160 & 1.0 & 0.8 & [0.5, 1.2] & [0.4, 1.7] \\
170 & 1.2 & 0.9 & [0.6, 1.4] & [0.4, 2.0] \\
180 & 1.5 & 1.2 & [0.8, 1.8] & [0.6, 2.8] \\
190 & 2.0 & 1.9 & [1.2, 2.9] & [0.9, 4.4] \\
200 & 2.9 & 2.5 & [1.7, 3.9] & [1.2, 5.9] \\
250 & 5.1 & 4.2 & [2.8, 6.3] & [1.9, 9.4] \\
300 & 4.9 & 4.3 & [3.0, 6.4] & [2.1, 9.3] \\
\hline\hline
\end{tabular}
\end{center}
\caption{Multivariate based upper limits at 95\% C.L. in 0 and 1 Jet final state, 
using the post-EPS data (run $<=$ 170826) corresponding to  1.1~$\ifb$ 
shown in Figure~\ref{fig:limits_eps_shape}.}
\label{tab:limits_eps_shape}
\end{table}
%%%%%%%%%%%%%%%%%%%%%%%%%%%%%%

%%%%%%%%%%%%%%%%%%%%%%%%%%%%%%
\begin{table}
\begin{center}
\begin{tabular}{c c c c c}
\hline\hline
 $m_H$ (GeV) & Observed & Median Expected & 68\% C.L. Band & 95\% C.L. Band \\ \hline
\hline
\multicolumn{5}{c} {0-Jet Bin Same Flavor} \\
\hline
115 & 4.1 & 8.8 & [5.7, 15.4] & [4.0, 36.0] \\
120 & 4.1 & 5.0 & [3.2, 8.4] & [2.3, 17.6] \\
130 & 2.5 & 2.3 & [1.5, 3.4] & [1.0, 5.4] \\
140 & 1.6 & 1.5 & [1.0, 2.2] & [0.7, 3.3] \\
150 & 1.8 & 1.0 & [0.7, 1.5] & [0.5, 2.4] \\
160 & 1.0 & 0.6 & [0.4, 0.9] & [0.3, 1.6] \\
170 & 1.0 & 0.7 & [0.5, 1.0] & [0.3, 1.6] \\
180 & 0.6 & 1.0 & [0.7, 1.5] & [0.5, 2.2] \\
190 & 1.6 & 1.5 & [1.0, 2.1] & [0.7, 3.1] \\
200 & 2.1 & 2.2 & [1.6, 3.3] & [1.1, 4.6] \\
250 & 3.3 & 5.8 & [4.0, 8.5] & [3.0, 12.2] \\
300 & 4.3 & 5.5 & [3.9, 8.3] & [2.8, 11.9] \\
\hline
\multicolumn{5}{c} {0-Jet Bin Opposite Flavor} \\
\hline
115 & 7.6 & 6.6 & [4.5, 9.8] & [3.2, 14.4] \\
120 & 4.9 & 3.8 & [2.6, 5.7] & [1.9, 8.2] \\
130 & 2.8 & 1.9 & [1.3, 2.9] & [1.0, 4.2] \\
140 & 1.5 & 1.3 & [0.9, 1.8] & [0.6, 2.6] \\
150 & 0.9 & 1.0 & [0.7, 1.5] & [0.5, 2.2] \\
160 & 0.5 & 0.5 & [0.4, 0.8] & [0.3, 1.2] \\
170 & 0.7 & 0.7 & [0.5, 1.0] & [0.3, 1.4] \\
180 & 0.9 & 0.9 & [0.7, 1.4] & [0.5, 2.0] \\
190 & 1.4 & 1.3 & [0.9, 2.0] & [0.7, 2.8] \\
200 & 2.3 & 1.9 & [1.3, 2.8] & [1.0, 4.2] \\
250 & 3.1 & 3.9 & [2.8, 5.8] & [2.0, 8.2] \\
300 & 3.8 & 4.5 & [3.1, 6.6] & [2.3, 9.5] \\
\hline
\multicolumn{5}{c} {1-Jet Bin Same Flavor} \\
\hline
115 & 19.7 & 17.7 & [11.4, 26.5] & [8.6, 43.9] \\
120 & 12.4 & 9.4 & [6.4, 15.3] & [4.7, 27.8] \\
130 & 8.3 & 4.6 & [3.1, 7.2] & [2.1, 14.5] \\
140 & 5.7 & 2.7 & [1.7, 4.1] & [1.2, 7.0] \\
150 & 4.2 & 2.1 & [1.3, 3.2] & [1.0, 5.2] \\
160 & 2.3 & 1.1 & [0.8, 1.8] & [0.6, 2.7] \\
170 & 2.7 & 1.4 & [1.0, 2.1] & [0.7, 3.2] \\
180 & 2.9 & 1.8 & [1.3, 2.6] & [0.9, 3.8] \\
190 & 4.3 & 2.6 & [1.7, 4.0] & [1.3, 6.1] \\
200 & 6.5 & 4.0 & [2.7, 6.0] & [2.0, 9.4] \\
250 & 11.2 & 8.6 & [6.0, 13.2] & [4.4, 19.7] \\
300 & 12.1 & 8.7 & [6.0, 13.1] & [4.3, 19.5] \\
\hline
\multicolumn{5}{c} {1-Jet Bin Opposite Flavor} \\
\hline
115 & 19.4 & 10.2 & [6.9, 15.4] & [5.1, 22.9] \\
120 & 10.4 & 6.0 & [4.0, 8.9] & [3.0, 13.0] \\
130 & 4.1 & 3.2 & [2.1, 4.7] & [1.5, 7.2] \\
140 & 2.2 & 2.0 & [1.4, 3.0] & [1.0, 4.6] \\
150 & 1.6 & 1.6 & [1.1, 2.3] & [0.8, 3.5] \\
160 & 1.1 & 1.0 & [0.6, 1.4] & [0.5, 2.1] \\
170 & 1.2 & 1.2 & [0.8, 1.7] & [0.6, 2.7] \\
180 & 1.6 & 1.6 & [1.1, 2.5] & [0.9, 3.6] \\
190 & 2.1 & 2.6 & [1.8, 3.8] & [1.3, 5.6] \\
200 & 3.1 & 3.2 & [2.2, 4.8] & [1.6, 7.1] \\
250 & 6.2 & 6.1 & [4.2, 9.1] & [3.1, 13.6] \\
300 & 7.8 & 6.9 & [4.8, 10.4] & [3.5, 15.2] \\
\hline\hline
\end{tabular}
\end{center}
\caption{Cut-based upper limits at 95\% C.L. in 0 and 1 Jet final state, 
using the post-EPS data (run $>$ 170826) corresponding to  0.4~$\ifb$ 
shown in Figure~\ref{fig:limits_posteps_cut}.}
\label{tab:limits_posteps_cut}
\end{table}
%%%%%%%%%%%%%%%%%%%%%%%%%%%%%%
%%%%%%%%%%%%%%%%%%%%%%%%%%%%%%
\begin{table}
\begin{center}
\begin{tabular}{c c c c c}
\hline\hline
 $m_H$ (GeV) & Observed & Median Expected & 68\% C.L. Band & 95\% C.L. Band \\ \hline
\hline
\multicolumn{5}{c} {0-Jet Bin Same Flavor} \\
\hline
115 & 10.2 & 10.2 & [7.3, 14.7] & [5.4, 20.9] \\
120 & 4.8 & 5.9 & [4.2, 8.5] & [3.1, 12.4] \\
130 & 2.7 & 2.8 & [2.0, 4.1] & [1.5, 5.9] \\
140 & 1.2 & 1.8 & [1.2, 2.6] & [0.9, 3.7] \\
150 & 0.8 & 1.2 & [0.9, 1.8] & [0.6, 2.6] \\
160 & 0.5 & 0.8 & [0.6, 1.1] & [0.4, 1.6] \\
170 & 0.5 & 0.9 & [0.6, 1.2] & [0.4, 1.8] \\
180 & 0.8 & 1.2 & [0.8, 1.8] & [0.6, 2.5] \\
190 & 1.5 & 1.9 & [1.3, 2.8] & [1.0, 4.1] \\
200 & 2.6 & 2.9 & [2.0, 4.2] & [1.5, 6.1] \\
250 & 6.9 & 5.9 & [4.1, 8.6] & [3.0, 12.3] \\
300 & 5.2 & 6.7 & [4.7, 9.8] & [3.6, 14.0] \\
\hline
\multicolumn{5}{c} {0-Jet Bin Opposite Flavor} \\
\hline
115 & 9.8 & 7.9 & [5.5, 11.5] & [4.1, 16.5] \\
120 & 4.8 & 4.7 & [3.3, 6.9] & [2.4, 9.8] \\
130 & 3.3 & 2.4 & [1.7, 3.6] & [1.2, 5.2] \\
140 & 2.0 & 1.6 & [1.1, 2.4] & [0.8, 3.4] \\
150 & 1.8 & 1.2 & [0.8, 1.7] & [0.6, 2.4] \\
160 & 0.9 & 0.8 & [0.5, 1.1] & [0.4, 1.6] \\
170 & 0.9 & 0.9 & [0.6, 1.2] & [0.4, 1.8] \\
180 & 1.0 & 1.2 & [0.8, 1.7] & [0.6, 2.4] \\
190 & 1.5 & 1.7 & [1.2, 2.5] & [0.8, 3.5] \\
200 & 1.9 & 2.3 & [1.6, 3.4] & [1.2, 4.9] \\
250 & 4.9 & 4.3 & [3.0, 6.2] & [2.2, 8.9] \\
300 & 6.4 & 5.2 & [3.7, 7.7] & [2.7, 10.9] \\
\hline
\multicolumn{5}{c} {1-Jet Bin Same Flavor} \\
\hline
115 & 26.5 & 26.2 & [18.5, 38.7] & [14.1, 55.0] \\
120 & 13.4 & 15.0 & [10.5, 22.0] & [7.9, 32.3] \\
130 & 6.5 & 6.5 & [4.6, 9.6] & [3.4, 13.9] \\
140 & 4.2 & 3.8 & [2.6, 5.6] & [1.9, 8.2] \\
150 & 2.5 & 2.5 & [1.7, 3.7] & [1.3, 5.5] \\
160 & 1.5 & 1.6 & [1.1, 2.4] & [0.8, 3.4] \\
170 & 1.4 & 1.7 & [1.2, 2.5] & [0.9, 3.7] \\
180 & 1.6 & 2.1 & [1.4, 3.1] & [1.0, 4.5] \\
190 & 2.4 & 3.2 & [2.2, 4.7] & [1.6, 7.0] \\
200 & 3.3 & 4.7 & [3.2, 7.0] & [2.3, 10.3] \\
250 & 5.9 & 9.1 & [6.3, 13.6] & [4.7, 19.9] \\
300 & 8.3 & 9.8 & [6.9, 14.4] & [5.1, 20.9] \\
\hline
\multicolumn{5}{c} {1-Jet Bin Opposite Flavor} \\
\hline
115 & 20.2 & 15.9 & [11.1, 23.5] & [8.2, 33.6] \\
120 & 11.0 & 9.7 & [6.7, 14.3] & [4.9, 20.9] \\
130 & 4.9 & 4.7 & [3.3, 7.1] & [2.4, 10.3] \\
140 & 3.2 & 2.9 & [2.0, 4.2] & [1.4, 6.0] \\
150 & 2.1 & 2.0 & [1.4, 3.0] & [1.0, 4.4] \\
160 & 1.7 & 1.4 & [1.0, 2.0] & [0.7, 3.0] \\
170 & 1.7 & 1.5 & [1.0, 2.3] & [0.8, 3.4] \\
180 & 2.1 & 2.0 & [1.3, 2.9] & [1.0, 4.4] \\
190 & 3.5 & 2.9 & [2.0, 4.4] & [1.4, 6.6] \\
200 & 4.7 & 3.9 & [2.6, 5.9] & [1.9, 8.8] \\
250 & 7.1 & 6.4 & [4.3, 9.6] & [3.1, 14.1] \\
300 & 12.3 & 7.1 & [4.9, 10.4] & [3.6, 15.4] \\
\hline\hline
\end{tabular}
\end{center}
\caption{Multivariate based upper limits at 95\% C.L. in 0 and 1 Jet final state, 
using the post-EPS data (run $>$ 170826) corresponding to  0.4~$\ifb$ 
shown in Figure~\ref{fig:limits_posteps_shape}.}
\label{tab:limits_posteps_shape}
\end{table}
%%%%%%%%%%%%%%%%%%%%%%%%%%%%%%





%%      \subsection{Conclusions}
%%      
We have recorded the supporting material for the
``reload'' of this analysis for the Lepton Photon 2011 conference.
This LP analysis comprises the addition of the post-EPS dataset
to this analysis, and a number of small technical changes.

The technical changes were confirmed to have no significant 
impact on the results in the EPS dataset.  
The data to simulation scale factors and background estimations
were derived for both the EPS and post-EPS datasets.
The results obtained in the post-EPS dataset were found to be
consistent with the results from the EPS analysis, thus
the two were combined to perform the reload.

The EPS counting analysis indicated an excess in the same flavor final state,
both in the 0-jet Higgs boson mass range ~150-170GeV region
and the 1-jet analysis over the full mass range.
This trend is not evident in the post-EPS dataset.

FIXME discussion of the limits when combining the 
EPS and post-EPS datasets and quote the exclusion limits FIXME

FIXME discussion of the effect of the MT cut FIXME

We conclude that adding the post-EPS dataset to this analysis 
supports the hypothesis that same flavor excesses in the EPS
analysis are more likely attributed to statistical fluctuation
than a systematic bias in Drell-Yan estimation.
Evidence has been presented that the robustness of the analysis 
can be improved by the addition of a cut to remove events
with an $M_T$ inconsistent with the signal hypothesis.




%%     \subsection{MVA output plots}
%%     \label{app:lp_mvaplots}
%%     \subsection{EPS distributions}

This section contains the MVA output plots for the EPS dataset for $m_H$=115, 120, 130, 140, 150, 160, 200 GeV analyses split in opposte and same flavor, 
0-jet and 1-jet bin (Figures~\ref{fig:lp_mva_115_EPS}-\ref{fig:lp_mva_200_EPS}).

\begin{figure}[!hbtp]
\centering
\subfigure[]{
\centering
\label{subfig:lp_mva_115_0j_of_EPS}
\includegraphics[width=.40\textwidth]{lp_figures/histo_mva_115_0j_of_EPS.png}}
\subfigure[]{
\centering
\label{subfig:lp_mva_115_0j_sf_EPS}
\includegraphics[width=.40\textwidth]{lp_figures/histo_mva_115_0j_sf_EPS.png}}\\
\subfigure[]{
\centering
\label{subfig:lp_mva_115_1j_of_EPS}
\includegraphics[width=.40\textwidth]{lp_figures/histo_mva_115_1j_of_EPS.png}}
\subfigure[]{
\centering
\label{subfig:lp_mva_115_1j_sf_EPS}
\includegraphics[width=.40\textwidth]{lp_figures/histo_mva_115_1j_sf_EPS.png}}
\caption{
MVA output for $m_H$=115 GeV EPS analysis: 
0-jet OF \subref{subfig:lp_mva_115_0j_of_EPS},
0-jet SF \subref{subfig:lp_mva_115_0j_sf_EPS},
1-jet OF \subref{subfig:lp_mva_115_1j_of_EPS},
1-jet SF \subref{subfig:lp_mva_115_1j_sf_EPS}
.}
\label{fig:lp_mva_115_EPS}
\end{figure}

\begin{figure}[!hbtp]
\centering
\subfigure[]{
\centering
\label{subfig:lp_mva_120_0j_of_EPS}
\includegraphics[width=.40\textwidth]{lp_figures/histo_mva_120_0j_of_EPS.png}}
\subfigure[]{
\centering
\label{subfig:lp_mva_120_0j_sf_EPS}
\includegraphics[width=.40\textwidth]{lp_figures/histo_mva_120_0j_sf_EPS.png}}\\
\subfigure[]{
\centering
\label{subfig:lp_mva_120_1j_of_EPS}
\includegraphics[width=.40\textwidth]{lp_figures/histo_mva_120_1j_of_EPS.png}}
\subfigure[]{
\centering
\label{subfig:lp_mva_120_1j_sf_EPS}
\includegraphics[width=.40\textwidth]{lp_figures/histo_mva_120_1j_sf_EPS.png}}
\caption{
MVA output for $m_H$=120 GeV EPS analysis: 
0-jet OF \subref{subfig:lp_mva_120_0j_of_EPS},
0-jet SF \subref{subfig:lp_mva_120_0j_sf_EPS},
1-jet OF \subref{subfig:lp_mva_120_1j_of_EPS},
1-jet SF \subref{subfig:lp_mva_120_1j_sf_EPS}
.}
\label{fig:lp_mva_120_EPS}
\end{figure}

\begin{figure}[!hbtp]
\centering
\subfigure[]{
\centering
\label{subfig:lp_mva_130_0j_of_EPS}
\includegraphics[width=.40\textwidth]{lp_figures/histo_mva_130_0j_of_EPS.png}}
\subfigure[]{
\centering
\label{subfig:lp_mva_130_0j_sf_EPS}
\includegraphics[width=.40\textwidth]{lp_figures/histo_mva_130_0j_sf_EPS.png}}\\
\subfigure[]{
\centering
\label{subfig:lp_mva_130_1j_of_EPS}
\includegraphics[width=.40\textwidth]{lp_figures/histo_mva_130_1j_of_EPS.png}}
\subfigure[]{
\centering
\label{subfig:lp_mva_130_1j_sf_EPS}
\includegraphics[width=.40\textwidth]{lp_figures/histo_mva_130_1j_sf_EPS.png}}
\caption{
MVA output for $m_H$=130 GeV EPS analysis: 
0-jet OF \subref{subfig:lp_mva_130_0j_of_EPS},
0-jet SF \subref{subfig:lp_mva_130_0j_sf_EPS},
1-jet OF \subref{subfig:lp_mva_130_1j_of_EPS},
1-jet SF \subref{subfig:lp_mva_130_1j_sf_EPS}
.}
\label{fig:lp_mva_130_EPS}
\end{figure}

\begin{figure}[!hbtp]
\centering
\subfigure[]{
\centering
\label{subfig:lp_mva_140_0j_of_EPS}
\includegraphics[width=.40\textwidth]{lp_figures/histo_mva_140_0j_of_EPS.png}}
\subfigure[]{
\centering
\label{subfig:lp_mva_140_0j_sf_EPS}
\includegraphics[width=.40\textwidth]{lp_figures/histo_mva_140_0j_sf_EPS.png}}\\
\subfigure[]{
\centering
\label{subfig:lp_mva_140_1j_of_EPS}
\includegraphics[width=.40\textwidth]{lp_figures/histo_mva_140_1j_of_EPS.png}}
\subfigure[]{
\centering
\label{subfig:lp_mva_140_1j_sf_EPS}
\includegraphics[width=.40\textwidth]{lp_figures/histo_mva_140_1j_sf_EPS.png}}
\caption{
MVA output for $m_H$=140 GeV EPS analysis: 
0-jet OF \subref{subfig:lp_mva_140_0j_of_EPS},
0-jet SF \subref{subfig:lp_mva_140_0j_sf_EPS},
1-jet OF \subref{subfig:lp_mva_140_1j_of_EPS},
1-jet SF \subref{subfig:lp_mva_140_1j_sf_EPS}
.}
\label{fig:lp_mva_140_EPS}
\end{figure}

\begin{figure}[!hbtp]
\centering
\subfigure[]{
\centering
\label{subfig:lp_mva_150_0j_of_EPS}
\includegraphics[width=.40\textwidth]{lp_figures/histo_mva_150_0j_of_EPS.png}}
\subfigure[]{
\centering
\label{subfig:lp_mva_150_0j_sf_EPS}
\includegraphics[width=.40\textwidth]{lp_figures/histo_mva_150_0j_sf_EPS.png}}\\
\subfigure[]{
\centering
\label{subfig:lp_mva_150_1j_of_EPS}
\includegraphics[width=.40\textwidth]{lp_figures/histo_mva_150_1j_of_EPS.png}}
\subfigure[]{
\centering
\label{subfig:lp_mva_150_1j_sf_EPS}
\includegraphics[width=.40\textwidth]{lp_figures/histo_mva_150_1j_sf_EPS.png}}
\caption{
MVA output for $m_H$=150 GeV EPS analysis: 
0-jet OF \subref{subfig:lp_mva_150_0j_of_EPS},
0-jet SF \subref{subfig:lp_mva_150_0j_sf_EPS},
1-jet OF \subref{subfig:lp_mva_150_1j_of_EPS},
1-jet SF \subref{subfig:lp_mva_150_1j_sf_EPS}
.}
\label{fig:lp_mva_150_EPS}
\end{figure}

\begin{figure}[!hbtp]
\centering
\subfigure[]{
\centering
\label{subfig:lp_mva_160_0j_of_EPS}
\includegraphics[width=.40\textwidth]{lp_figures/histo_mva_160_0j_of_EPS.png}}
\subfigure[]{
\centering
\label{subfig:lp_mva_160_0j_sf_EPS}
\includegraphics[width=.40\textwidth]{lp_figures/histo_mva_160_0j_sf_EPS.png}}\\
\subfigure[]{
\centering
\label{subfig:lp_mva_160_1j_of_EPS}
\includegraphics[width=.40\textwidth]{lp_figures/histo_mva_160_1j_of_EPS.png}}
\subfigure[]{
\centering
\label{subfig:lp_mva_160_1j_sf_EPS}
\includegraphics[width=.40\textwidth]{lp_figures/histo_mva_160_1j_sf_EPS.png}}
\caption{
MVA output for $m_H$=160 GeV EPS analysis: 
0-jet OF \subref{subfig:lp_mva_160_0j_of_EPS},
0-jet SF \subref{subfig:lp_mva_160_0j_sf_EPS},
1-jet OF \subref{subfig:lp_mva_160_1j_of_EPS},
1-jet SF \subref{subfig:lp_mva_160_1j_sf_EPS}
.}
\label{fig:lp_mva_160_EPS}
\end{figure}

\begin{figure}[!hbtp]
\centering
\subfigure[]{
\centering
\label{subfig:lp_mva_200_0j_of_EPS}
\includegraphics[width=.40\textwidth]{lp_figures/histo_mva_200_0j_of_EPS.png}}
\subfigure[]{
\centering
\label{subfig:lp_mva_200_0j_sf_EPS}
\includegraphics[width=.40\textwidth]{lp_figures/histo_mva_200_0j_sf_EPS.png}}\\
\subfigure[]{
\centering
\label{subfig:lp_mva_200_1j_of_EPS}
\includegraphics[width=.40\textwidth]{lp_figures/histo_mva_200_1j_of_EPS.png}}
\subfigure[]{
\centering
\label{subfig:lp_mva_200_1j_sf_EPS}
\includegraphics[width=.40\textwidth]{lp_figures/histo_mva_200_1j_sf_EPS.png}}
\caption{
MVA output for $m_H$=200 GeV EPS analysis: 
0-jet OF \subref{subfig:lp_mva_200_0j_of_EPS},
0-jet SF \subref{subfig:lp_mva_200_0j_sf_EPS},
1-jet OF \subref{subfig:lp_mva_200_1j_of_EPS},
1-jet SF \subref{subfig:lp_mva_200_1j_sf_EPS}
.}
\label{fig:lp_mva_200_EPS}
\end{figure}

\clearpage

%%     \subsection{Post-EPS distributions}

This section contains the MVA output plots for the post-EPS dataset for $m_H$=115, 120, 130, 140, 150, 160, 200 GeV analyses split in opposte and same flavor, 
0-jet and 1-jet bin (Figures~\ref{fig:lp_mva_115_POSTEPS}-\ref{fig:lp_mva_200_POSTEPS}).

\begin{figure}[!hbtp]
\centering
\subfigure[]{
\centering
\label{subfig:lp_mva_115_0j_of_POSTEPS}
\includegraphics[width=.40\textwidth]{lp_figures/histo_mva_115_0j_of_POSTEPS.png}}
\subfigure[]{
\centering
\label{subfig:lp_mva_115_0j_sf_POSTEPS}
\includegraphics[width=.40\textwidth]{lp_figures/histo_mva_115_0j_sf_POSTEPS.png}}\\
\subfigure[]{
\centering
\label{subfig:lp_mva_115_1j_of_POSTEPS}
\includegraphics[width=.40\textwidth]{lp_figures/histo_mva_115_1j_of_POSTEPS.png}}
\subfigure[]{
\centering
\label{subfig:lp_mva_115_1j_sf_POSTEPS}
\includegraphics[width=.40\textwidth]{lp_figures/histo_mva_115_1j_sf_POSTEPS.png}}
\caption{
MVA output for $m_H$=115 GeV post-EPS analysis: 
0-jet OF \subref{subfig:lp_mva_115_0j_of_POSTEPS},
0-jet SF \subref{subfig:lp_mva_115_0j_sf_POSTEPS},
1-jet OF \subref{subfig:lp_mva_115_1j_of_POSTEPS},
1-jet SF \subref{subfig:lp_mva_115_1j_sf_POSTEPS}
.}
\label{fig:lp_mva_115_POSTEPS}
\end{figure}

\begin{figure}[!hbtp]
\centering
\subfigure[]{
\centering
\label{subfig:lp_mva_120_0j_of_POSTEPS}
\includegraphics[width=.40\textwidth]{lp_figures/histo_mva_120_0j_of_POSTEPS.png}}
\subfigure[]{
\centering
\label{subfig:lp_mva_120_0j_sf_POSTEPS}
\includegraphics[width=.40\textwidth]{lp_figures/histo_mva_120_0j_sf_POSTEPS.png}}\\
\subfigure[]{
\centering
\label{subfig:lp_mva_120_1j_of_POSTEPS}
\includegraphics[width=.40\textwidth]{lp_figures/histo_mva_120_1j_of_POSTEPS.png}}
\subfigure[]{
\centering
\label{subfig:lp_mva_120_1j_sf_POSTEPS}
\includegraphics[width=.40\textwidth]{lp_figures/histo_mva_120_1j_sf_POSTEPS.png}}
\caption{
MVA output for $m_H$=120 GeV post-EPS analysis: 
0-jet OF \subref{subfig:lp_mva_120_0j_of_POSTEPS},
0-jet SF \subref{subfig:lp_mva_120_0j_sf_POSTEPS},
1-jet OF \subref{subfig:lp_mva_120_1j_of_POSTEPS},
1-jet SF \subref{subfig:lp_mva_120_1j_sf_POSTEPS}
.}
\label{fig:lp_mva_120_POSTEPS}
\end{figure}

\begin{figure}[!hbtp]
\centering
\subfigure[]{
\centering
\label{subfig:lp_mva_130_0j_of_POSTEPS}
\includegraphics[width=.40\textwidth]{lp_figures/histo_mva_130_0j_of_POSTEPS.png}}
\subfigure[]{
\centering
\label{subfig:lp_mva_130_0j_sf_POSTEPS}
\includegraphics[width=.40\textwidth]{lp_figures/histo_mva_130_0j_sf_POSTEPS.png}}\\
\subfigure[]{
\centering
\label{subfig:lp_mva_130_1j_of_POSTEPS}
\includegraphics[width=.40\textwidth]{lp_figures/histo_mva_130_1j_of_POSTEPS.png}}
\subfigure[]{
\centering
\label{subfig:lp_mva_130_1j_sf_POSTEPS}
\includegraphics[width=.40\textwidth]{lp_figures/histo_mva_130_1j_sf_POSTEPS.png}}
\caption{
MVA output for $m_H$=130 GeV post-EPS analysis: 
0-jet OF \subref{subfig:lp_mva_130_0j_of_POSTEPS},
0-jet SF \subref{subfig:lp_mva_130_0j_sf_POSTEPS},
1-jet OF \subref{subfig:lp_mva_130_1j_of_POSTEPS},
1-jet SF \subref{subfig:lp_mva_130_1j_sf_POSTEPS}
.}
\label{fig:lp_mva_130_POSTEPS}
\end{figure}

\begin{figure}[!hbtp]
\centering
\subfigure[]{
\centering
\label{subfig:lp_mva_140_0j_of_POSTEPS}
\includegraphics[width=.40\textwidth]{lp_figures/histo_mva_140_0j_of_POSTEPS.png}}
\subfigure[]{
\centering
\label{subfig:lp_mva_140_0j_sf_POSTEPS}
\includegraphics[width=.40\textwidth]{lp_figures/histo_mva_140_0j_sf_POSTEPS.png}}\\
\subfigure[]{
\centering
\label{subfig:lp_mva_140_1j_of_POSTEPS}
\includegraphics[width=.40\textwidth]{lp_figures/histo_mva_140_1j_of_POSTEPS.png}}
\subfigure[]{
\centering
\label{subfig:lp_mva_140_1j_sf_POSTEPS}
\includegraphics[width=.40\textwidth]{lp_figures/histo_mva_140_1j_sf_POSTEPS.png}}
\caption{
MVA output for $m_H$=140 GeV post-EPS analysis: 
0-jet OF \subref{subfig:lp_mva_140_0j_of_POSTEPS},
0-jet SF \subref{subfig:lp_mva_140_0j_sf_POSTEPS},
1-jet OF \subref{subfig:lp_mva_140_1j_of_POSTEPS},
1-jet SF \subref{subfig:lp_mva_140_1j_sf_POSTEPS}
.}
\label{fig:lp_mva_140_POSTEPS}
\end{figure}

\begin{figure}[!hbtp]
\centering
\subfigure[]{
\centering
\label{subfig:lp_mva_150_0j_of_POSTEPS}
\includegraphics[width=.40\textwidth]{lp_figures/histo_mva_150_0j_of_POSTEPS.png}}
\subfigure[]{
\centering
\label{subfig:lp_mva_150_0j_sf_POSTEPS}
\includegraphics[width=.40\textwidth]{lp_figures/histo_mva_150_0j_sf_POSTEPS.png}}\\
\subfigure[]{
\centering
\label{subfig:lp_mva_150_1j_of_POSTEPS}
\includegraphics[width=.40\textwidth]{lp_figures/histo_mva_150_1j_of_POSTEPS.png}}
\subfigure[]{
\centering
\label{subfig:lp_mva_150_1j_sf_POSTEPS}
\includegraphics[width=.40\textwidth]{lp_figures/histo_mva_150_1j_sf_POSTEPS.png}}
\caption{
MVA output for $m_H$=150 GeV post-EPS analysis: 
0-jet OF \subref{subfig:lp_mva_150_0j_of_POSTEPS},
0-jet SF \subref{subfig:lp_mva_150_0j_sf_POSTEPS},
1-jet OF \subref{subfig:lp_mva_150_1j_of_POSTEPS},
1-jet SF \subref{subfig:lp_mva_150_1j_sf_POSTEPS}
.}
\label{fig:lp_mva_150_POSTEPS}
\end{figure}

\begin{figure}[!hbtp]
\centering
\subfigure[]{
\centering
\label{subfig:lp_mva_160_0j_of_POSTEPS}
\includegraphics[width=.40\textwidth]{lp_figures/histo_mva_160_0j_of_POSTEPS.png}}
\subfigure[]{
\centering
\label{subfig:lp_mva_160_0j_sf_POSTEPS}
\includegraphics[width=.40\textwidth]{lp_figures/histo_mva_160_0j_sf_POSTEPS.png}}\\
\subfigure[]{
\centering
\label{subfig:lp_mva_160_1j_of_POSTEPS}
\includegraphics[width=.40\textwidth]{lp_figures/histo_mva_160_1j_of_POSTEPS.png}}
\subfigure[]{
\centering
\label{subfig:lp_mva_160_1j_sf_POSTEPS}
\includegraphics[width=.40\textwidth]{lp_figures/histo_mva_160_1j_sf_POSTEPS.png}}
\caption{
MVA output for $m_H$=160 GeV post-EPS analysis: 
0-jet OF \subref{subfig:lp_mva_160_0j_of_POSTEPS},
0-jet SF \subref{subfig:lp_mva_160_0j_sf_POSTEPS},
1-jet OF \subref{subfig:lp_mva_160_1j_of_POSTEPS},
1-jet SF \subref{subfig:lp_mva_160_1j_sf_POSTEPS}
.}
\label{fig:lp_mva_160_POSTEPS}
\end{figure}

\begin{figure}[!hbtp]
\centering
\subfigure[]{
\centering
\label{subfig:lp_mva_200_0j_of_POSTEPS}
\includegraphics[width=.40\textwidth]{lp_figures/histo_mva_200_0j_of_POSTEPS.png}}
\subfigure[]{
\centering
\label{subfig:lp_mva_200_0j_sf_POSTEPS}
\includegraphics[width=.40\textwidth]{lp_figures/histo_mva_200_0j_sf_POSTEPS.png}}\\
\subfigure[]{
\centering
\label{subfig:lp_mva_200_1j_of_POSTEPS}
\includegraphics[width=.40\textwidth]{lp_figures/histo_mva_200_1j_of_POSTEPS.png}}
\subfigure[]{
\centering
\label{subfig:lp_mva_200_1j_sf_POSTEPS}
\includegraphics[width=.40\textwidth]{lp_figures/histo_mva_200_1j_sf_POSTEPS.png}}
\caption{
MVA output for $m_H$=200 GeV post-EPS analysis: 
0-jet OF \subref{subfig:lp_mva_200_0j_of_POSTEPS},
0-jet SF \subref{subfig:lp_mva_200_0j_sf_POSTEPS},
1-jet OF \subref{subfig:lp_mva_200_1j_of_POSTEPS},
1-jet SF \subref{subfig:lp_mva_200_1j_sf_POSTEPS}
.}
\label{fig:lp_mva_200_POSTEPS}
\end{figure}

\clearpage

%%     \subsubsection{LP distributions}

This section contains the MVA output plots for the LP dataset for $m_H$=115, 120, 130, 140, 150, 160, 200 GeV analyses split in opposte and same flavor, 
0-jet and 1-jet bin (Figures~\ref{fig:lp_mva_115}-\ref{fig:lp_mva_200}).

\begin{figure}[!hbtp]
\centering
\subfigure[]{
\centering
\label{subfig:lp_mva_115_0j_of}
\includegraphics[width=.40\textwidth]{lp_figures/histo_mva_115_0j_of.png}}
\subfigure[]{
\centering
\label{subfig:lp_mva_115_0j_sf}
\includegraphics[width=.40\textwidth]{lp_figures/histo_mva_115_0j_sf.png}}\\
\subfigure[]{
\centering
\label{subfig:lp_mva_115_1j_of}
\includegraphics[width=.40\textwidth]{lp_figures/histo_mva_115_1j_of.png}}
\subfigure[]{
\centering
\label{subfig:lp_mva_115_1j_sf}
\includegraphics[width=.40\textwidth]{lp_figures/histo_mva_115_1j_sf.png}}
\caption{
MVA output for $m_H$=115 GeV LP analysis: 
0-jet OF \subref{subfig:lp_mva_115_0j_of},
0-jet SF \subref{subfig:lp_mva_115_0j_sf},
1-jet OF \subref{subfig:lp_mva_115_1j_of},
1-jet SF \subref{subfig:lp_mva_115_1j_sf}
.}
\label{fig:lp_mva_115}
\end{figure}

\begin{figure}[!hbtp]
\centering
\subfigure[]{
\centering
\label{subfig:lp_mva_120_0j_of}
\includegraphics[width=.40\textwidth]{lp_figures/histo_mva_120_0j_of.png}}
\subfigure[]{
\centering
\label{subfig:lp_mva_120_0j_sf}
\includegraphics[width=.40\textwidth]{lp_figures/histo_mva_120_0j_sf.png}}\\
\subfigure[]{
\centering
\label{subfig:lp_mva_120_1j_of}
\includegraphics[width=.40\textwidth]{lp_figures/histo_mva_120_1j_of.png}}
\subfigure[]{
\centering
\label{subfig:lp_mva_120_1j_sf}
\includegraphics[width=.40\textwidth]{lp_figures/histo_mva_120_1j_sf.png}}
\caption{
MVA output for $m_H$=120 GeV LP analysis: 
0-jet OF \subref{subfig:lp_mva_120_0j_of},
0-jet SF \subref{subfig:lp_mva_120_0j_sf},
1-jet OF \subref{subfig:lp_mva_120_1j_of},
1-jet SF \subref{subfig:lp_mva_120_1j_sf}
.}
\label{fig:lp_mva_120}
\end{figure}

\begin{figure}[!hbtp]
\centering
\subfigure[]{
\centering
\label{subfig:lp_mva_130_0j_of}
\includegraphics[width=.40\textwidth]{lp_figures/histo_mva_130_0j_of.png}}
\subfigure[]{
\centering
\label{subfig:lp_mva_130_0j_sf}
\includegraphics[width=.40\textwidth]{lp_figures/histo_mva_130_0j_sf.png}}\\
\subfigure[]{
\centering
\label{subfig:lp_mva_130_1j_of}
\includegraphics[width=.40\textwidth]{lp_figures/histo_mva_130_1j_of.png}}
\subfigure[]{
\centering
\label{subfig:lp_mva_130_1j_sf}
\includegraphics[width=.40\textwidth]{lp_figures/histo_mva_130_1j_sf.png}}
\caption{
MVA output for $m_H$=130 GeV LP analysis: 
0-jet OF \subref{subfig:lp_mva_130_0j_of},
0-jet SF \subref{subfig:lp_mva_130_0j_sf},
1-jet OF \subref{subfig:lp_mva_130_1j_of},
1-jet SF \subref{subfig:lp_mva_130_1j_sf}
.}
\label{fig:lp_mva_130}
\end{figure}

\begin{figure}[!hbtp]
\centering
\subfigure[]{
\centering
\label{subfig:lp_mva_140_0j_of}
\includegraphics[width=.40\textwidth]{lp_figures/histo_mva_140_0j_of.png}}
\subfigure[]{
\centering
\label{subfig:lp_mva_140_0j_sf}
\includegraphics[width=.40\textwidth]{lp_figures/histo_mva_140_0j_sf.png}}\\
\subfigure[]{
\centering
\label{subfig:lp_mva_140_1j_of}
\includegraphics[width=.40\textwidth]{lp_figures/histo_mva_140_1j_of.png}}
\subfigure[]{
\centering
\label{subfig:lp_mva_140_1j_sf}
\includegraphics[width=.40\textwidth]{lp_figures/histo_mva_140_1j_sf.png}}
\caption{
MVA output for $m_H$=140 GeV LP analysis: 
0-jet OF \subref{subfig:lp_mva_140_0j_of},
0-jet SF \subref{subfig:lp_mva_140_0j_sf},
1-jet OF \subref{subfig:lp_mva_140_1j_of},
1-jet SF \subref{subfig:lp_mva_140_1j_sf}
.}
\label{fig:lp_mva_140}
\end{figure}

\begin{figure}[!hbtp]
\centering
\subfigure[]{
\centering
\label{subfig:lp_mva_150_0j_of}
\includegraphics[width=.40\textwidth]{lp_figures/histo_mva_150_0j_of.png}}
\subfigure[]{
\centering
\label{subfig:lp_mva_150_0j_sf}
\includegraphics[width=.40\textwidth]{lp_figures/histo_mva_150_0j_sf.png}}\\
\subfigure[]{
\centering
\label{subfig:lp_mva_150_1j_of}
\includegraphics[width=.40\textwidth]{lp_figures/histo_mva_150_1j_of.png}}
\subfigure[]{
\centering
\label{subfig:lp_mva_150_1j_sf}
\includegraphics[width=.40\textwidth]{lp_figures/histo_mva_150_1j_sf.png}}
\caption{
MVA output for $m_H$=150 GeV LP analysis: 
0-jet OF \subref{subfig:lp_mva_150_0j_of},
0-jet SF \subref{subfig:lp_mva_150_0j_sf},
1-jet OF \subref{subfig:lp_mva_150_1j_of},
1-jet SF \subref{subfig:lp_mva_150_1j_sf}
.}
\label{fig:lp_mva_150}
\end{figure}

\begin{figure}[!hbtp]
\centering
\subfigure[]{
\centering
\label{subfig:lp_mva_160_0j_of}
\includegraphics[width=.40\textwidth]{lp_figures/histo_mva_160_0j_of.png}}
\subfigure[]{
\centering
\label{subfig:lp_mva_160_0j_sf}
\includegraphics[width=.40\textwidth]{lp_figures/histo_mva_160_0j_sf.png}}\\
\subfigure[]{
\centering
\label{subfig:lp_mva_160_1j_of}
\includegraphics[width=.40\textwidth]{lp_figures/histo_mva_160_1j_of.png}}
\subfigure[]{
\centering
\label{subfig:lp_mva_160_1j_sf}
\includegraphics[width=.40\textwidth]{lp_figures/histo_mva_160_1j_sf.png}}
\caption{
MVA output for $m_H$=160 GeV LP analysis: 
0-jet OF \subref{subfig:lp_mva_160_0j_of},
0-jet SF \subref{subfig:lp_mva_160_0j_sf},
1-jet OF \subref{subfig:lp_mva_160_1j_of},
1-jet SF \subref{subfig:lp_mva_160_1j_sf}
.}
\label{fig:lp_mva_160}
\end{figure}

\begin{figure}[!hbtp]
\centering
\subfigure[]{
\centering
\label{subfig:lp_mva_200_0j_of}
\includegraphics[width=.40\textwidth]{lp_figures/histo_mva_200_0j_of.png}}
\subfigure[]{
\centering
\label{subfig:lp_mva_200_0j_sf}
\includegraphics[width=.40\textwidth]{lp_figures/histo_mva_200_0j_sf.png}}\\
\subfigure[]{
\centering
\label{subfig:lp_mva_200_1j_of}
\includegraphics[width=.40\textwidth]{lp_figures/histo_mva_200_1j_of.png}}
\subfigure[]{
\centering
\label{subfig:lp_mva_200_1j_sf}
\includegraphics[width=.40\textwidth]{lp_figures/histo_mva_200_1j_sf.png}}
\caption{
MVA output for $m_H$=200 GeV LP analysis: 
0-jet OF \subref{subfig:lp_mva_200_0j_of},
0-jet SF \subref{subfig:lp_mva_200_0j_sf},
1-jet OF \subref{subfig:lp_mva_200_1j_of},
1-jet SF \subref{subfig:lp_mva_200_1j_sf}
.}
\label{fig:lp_mva_200}
\end{figure}

\clearpage

%%     \subsubsection{LP distributions with $80<m_T<m_H$}

This section contains the MVA output plots for the LP dataset for $m_H$=115, 120, 130, 140, 150, 160, 200 GeV analyses split in opposte and same flavor, 
0-jet and 1-jet bin requiring a transverse mass value greater than 80 GeV and smaller than the considered Higgs mass 
(Figures~\ref{fig:lp_mva_115_MTCUTGT80}-\ref{fig:lp_mva_200_MTCUTGT80}).

\begin{figure}[!hbtp]
\centering
\subfigure[]{
\centering
\label{subfig:lp_mva_115_0j_of_MTCUTGT80}
\includegraphics[width=.40\textwidth]{lp_figures/histo_mva_115_0j_of_MTCUTGT80.png}}
\subfigure[]{
\centering
\label{subfig:lp_mva_115_0j_sf_MTCUTGT80}
\includegraphics[width=.40\textwidth]{lp_figures/histo_mva_115_0j_sf_MTCUTGT80.png}}\\
\subfigure[]{
\centering
\label{subfig:lp_mva_115_1j_of_MTCUTGT80}
\includegraphics[width=.40\textwidth]{lp_figures/histo_mva_115_1j_of_MTCUTGT80.png}}
\subfigure[]{
\centering
\label{subfig:lp_mva_115_1j_sf_MTCUTGT80}
\includegraphics[width=.40\textwidth]{lp_figures/histo_mva_115_1j_sf_MTCUTGT80.png}}
\caption{
MVA output for $m_H$=115 GeV LP ($80<m_T<m_H$) analysis: 
0-jet OF \subref{subfig:lp_mva_115_0j_of_MTCUTGT80},
0-jet SF \subref{subfig:lp_mva_115_0j_sf_MTCUTGT80},
1-jet OF \subref{subfig:lp_mva_115_1j_of_MTCUTGT80},
1-jet SF \subref{subfig:lp_mva_115_1j_sf_MTCUTGT80}
.}
\label{fig:lp_mva_115_MTCUTGT80}
\end{figure}

\begin{figure}[!hbtp]
\centering
\subfigure[]{
\centering
\label{subfig:lp_mva_120_0j_of_MTCUTGT80}
\includegraphics[width=.40\textwidth]{lp_figures/histo_mva_120_0j_of_MTCUTGT80.png}}
\subfigure[]{
\centering
\label{subfig:lp_mva_120_0j_sf_MTCUTGT80}
\includegraphics[width=.40\textwidth]{lp_figures/histo_mva_120_0j_sf_MTCUTGT80.png}}\\
\subfigure[]{
\centering
\label{subfig:lp_mva_120_1j_of_MTCUTGT80}
\includegraphics[width=.40\textwidth]{lp_figures/histo_mva_120_1j_of_MTCUTGT80.png}}
\subfigure[]{
\centering
\label{subfig:lp_mva_120_1j_sf_MTCUTGT80}
\includegraphics[width=.40\textwidth]{lp_figures/histo_mva_120_1j_sf_MTCUTGT80.png}}
\caption{
MVA output for $m_H$=120 GeV LP ($80<m_T<m_H$) analysis: 
0-jet OF \subref{subfig:lp_mva_120_0j_of_MTCUTGT80},
0-jet SF \subref{subfig:lp_mva_120_0j_sf_MTCUTGT80},
1-jet OF \subref{subfig:lp_mva_120_1j_of_MTCUTGT80},
1-jet SF \subref{subfig:lp_mva_120_1j_sf_MTCUTGT80}
.}
\label{fig:lp_mva_120_MTCUTGT80}
\end{figure}

\begin{figure}[!hbtp]
\centering
\subfigure[]{
\centering
\label{subfig:lp_mva_130_0j_of_MTCUTGT80}
\includegraphics[width=.40\textwidth]{lp_figures/histo_mva_130_0j_of_MTCUTGT80.png}}
\subfigure[]{
\centering
\label{subfig:lp_mva_130_0j_sf_MTCUTGT80}
\includegraphics[width=.40\textwidth]{lp_figures/histo_mva_130_0j_sf_MTCUTGT80.png}}\\
\subfigure[]{
\centering
\label{subfig:lp_mva_130_1j_of_MTCUTGT80}
\includegraphics[width=.40\textwidth]{lp_figures/histo_mva_130_1j_of_MTCUTGT80.png}}
\subfigure[]{
\centering
\label{subfig:lp_mva_130_1j_sf_MTCUTGT80}
\includegraphics[width=.40\textwidth]{lp_figures/histo_mva_130_1j_sf_MTCUTGT80.png}}
\caption{
MVA output for $m_H$=130 GeV LP ($80<m_T<m_H$) analysis: 
0-jet OF \subref{subfig:lp_mva_130_0j_of_MTCUTGT80},
0-jet SF \subref{subfig:lp_mva_130_0j_sf_MTCUTGT80},
1-jet OF \subref{subfig:lp_mva_130_1j_of_MTCUTGT80},
1-jet SF \subref{subfig:lp_mva_130_1j_sf_MTCUTGT80}
.}
\label{fig:lp_mva_130_MTCUTGT80}
\end{figure}

\begin{figure}[!hbtp]
\centering
\subfigure[]{
\centering
\label{subfig:lp_mva_140_0j_of_MTCUTGT80}
\includegraphics[width=.40\textwidth]{lp_figures/histo_mva_140_0j_of_MTCUTGT80.png}}
\subfigure[]{
\centering
\label{subfig:lp_mva_140_0j_sf_MTCUTGT80}
\includegraphics[width=.40\textwidth]{lp_figures/histo_mva_140_0j_sf_MTCUTGT80.png}}\\
\subfigure[]{
\centering
\label{subfig:lp_mva_140_1j_of_MTCUTGT80}
\includegraphics[width=.40\textwidth]{lp_figures/histo_mva_140_1j_of_MTCUTGT80.png}}
\subfigure[]{
\centering
\label{subfig:lp_mva_140_1j_sf_MTCUTGT80}
\includegraphics[width=.40\textwidth]{lp_figures/histo_mva_140_1j_sf_MTCUTGT80.png}}
\caption{
MVA output for $m_H$=140 GeV LP ($80<m_T<m_H$) analysis: 
0-jet OF \subref{subfig:lp_mva_140_0j_of_MTCUTGT80},
0-jet SF \subref{subfig:lp_mva_140_0j_sf_MTCUTGT80},
1-jet OF \subref{subfig:lp_mva_140_1j_of_MTCUTGT80},
1-jet SF \subref{subfig:lp_mva_140_1j_sf_MTCUTGT80}
.}
\label{fig:lp_mva_140_MTCUTGT80}
\end{figure}

\begin{figure}[!hbtp]
\centering
\subfigure[]{
\centering
\label{subfig:lp_mva_150_0j_of_MTCUTGT80}
\includegraphics[width=.40\textwidth]{lp_figures/histo_mva_150_0j_of_MTCUTGT80.png}}
\subfigure[]{
\centering
\label{subfig:lp_mva_150_0j_sf_MTCUTGT80}
\includegraphics[width=.40\textwidth]{lp_figures/histo_mva_150_0j_sf_MTCUTGT80.png}}\\
\subfigure[]{
\centering
\label{subfig:lp_mva_150_1j_of_MTCUTGT80}
\includegraphics[width=.40\textwidth]{lp_figures/histo_mva_150_1j_of_MTCUTGT80.png}}
\subfigure[]{
\centering
\label{subfig:lp_mva_150_1j_sf_MTCUTGT80}
\includegraphics[width=.40\textwidth]{lp_figures/histo_mva_150_1j_sf_MTCUTGT80.png}}
\caption{
MVA output for $m_H$=150 GeV LP ($80<m_T<m_H$) analysis: 
0-jet OF \subref{subfig:lp_mva_150_0j_of_MTCUTGT80},
0-jet SF \subref{subfig:lp_mva_150_0j_sf_MTCUTGT80},
1-jet OF \subref{subfig:lp_mva_150_1j_of_MTCUTGT80},
1-jet SF \subref{subfig:lp_mva_150_1j_sf_MTCUTGT80}
.}
\label{fig:lp_mva_150_MTCUTGT80}
\end{figure}

\begin{figure}[!hbtp]
\centering
\subfigure[]{
\centering
\label{subfig:lp_mva_160_0j_of_MTCUTGT80}
\includegraphics[width=.40\textwidth]{lp_figures/histo_mva_160_0j_of_MTCUTGT80.png}}
\subfigure[]{
\centering
\label{subfig:lp_mva_160_0j_sf_MTCUTGT80}
\includegraphics[width=.40\textwidth]{lp_figures/histo_mva_160_0j_sf_MTCUTGT80.png}}\\
\subfigure[]{
\centering
\label{subfig:lp_mva_160_1j_of_MTCUTGT80}
\includegraphics[width=.40\textwidth]{lp_figures/histo_mva_160_1j_of_MTCUTGT80.png}}
\subfigure[]{
\centering
\label{subfig:lp_mva_160_1j_sf_MTCUTGT80}
\includegraphics[width=.40\textwidth]{lp_figures/histo_mva_160_1j_sf_MTCUTGT80.png}}
\caption{
MVA output for $m_H$=160 GeV LP ($80<m_T<m_H$) analysis: 
0-jet OF \subref{subfig:lp_mva_160_0j_of_MTCUTGT80},
0-jet SF \subref{subfig:lp_mva_160_0j_sf_MTCUTGT80},
1-jet OF \subref{subfig:lp_mva_160_1j_of_MTCUTGT80},
1-jet SF \subref{subfig:lp_mva_160_1j_sf_MTCUTGT80}
.}
\label{fig:lp_mva_160_MTCUTGT80}
\end{figure}

\begin{figure}[!hbtp]
\centering
\subfigure[]{
\centering
\label{subfig:lp_mva_200_0j_of_MTCUTGT80}
\includegraphics[width=.40\textwidth]{lp_figures/histo_mva_200_0j_of_MTCUTGT80.png}}
\subfigure[]{
\centering
\label{subfig:lp_mva_200_0j_sf_MTCUTGT80}
\includegraphics[width=.40\textwidth]{lp_figures/histo_mva_200_0j_sf_MTCUTGT80.png}}\\
\subfigure[]{
\centering
\label{subfig:lp_mva_200_1j_of_MTCUTGT80}
\includegraphics[width=.40\textwidth]{lp_figures/histo_mva_200_1j_of_MTCUTGT80.png}}
\subfigure[]{
\centering
\label{subfig:lp_mva_200_1j_sf_MTCUTGT80}
\includegraphics[width=.40\textwidth]{lp_figures/histo_mva_200_1j_sf_MTCUTGT80.png}}
\caption{
MVA output for $m_H$=200 GeV LP ($80<m_T<m_H$) analysis: 
0-jet OF \subref{subfig:lp_mva_200_0j_of_MTCUTGT80},
0-jet SF \subref{subfig:lp_mva_200_0j_sf_MTCUTGT80},
1-jet OF \subref{subfig:lp_mva_200_1j_of_MTCUTGT80},
1-jet SF \subref{subfig:lp_mva_200_1j_sf_MTCUTGT80}
.}
\label{fig:lp_mva_200_MTCUTGT80}
\end{figure}

%%     \subsection{LP distributions with $m_T<80$}

This section contains the MVA output plots for the LP dataset for $m_H$=115, 120, 130, 140, 150, 160, 200 GeV analyses split in opposte and same flavor, 
0-jet and 1-jet bin requiring a transverse mass value smaller than 80 GeV (Figures~\ref{fig:lp_mva_115_MTCUTLT80}-\ref{fig:lp_mva_200_MTCUTLT80}).

\begin{figure}[!hbtp]
\centering
\subfigure[]{
\centering
\label{subfig:lp_mva_115_0j_of_MTCUTLT80}
\includegraphics[width=.40\textwidth]{lp_figures/histo_mva_115_0j_of_MTCUTLT80.png}}
\subfigure[]{
\centering
\label{subfig:lp_mva_115_0j_sf_MTCUTLT80}
\includegraphics[width=.40\textwidth]{lp_figures/histo_mva_115_0j_sf_MTCUTLT80.png}}\\
\subfigure[]{
\centering
\label{subfig:lp_mva_115_1j_of_MTCUTLT80}
\includegraphics[width=.40\textwidth]{lp_figures/histo_mva_115_1j_of_MTCUTLT80.png}}
\subfigure[]{
\centering
\label{subfig:lp_mva_115_1j_sf_MTCUTLT80}
\includegraphics[width=.40\textwidth]{lp_figures/histo_mva_115_1j_sf_MTCUTLT80.png}}
\caption{
MVA output for $m_H$=115 GeV LP ($m_T<80$) analysis: 
0-jet OF \subref{subfig:lp_mva_115_0j_of_MTCUTLT80},
0-jet SF \subref{subfig:lp_mva_115_0j_sf_MTCUTLT80},
1-jet OF \subref{subfig:lp_mva_115_1j_of_MTCUTLT80},
1-jet SF \subref{subfig:lp_mva_115_1j_sf_MTCUTLT80}
.}
\label{fig:lp_mva_115_MTCUTLT80}
\end{figure}

\begin{figure}[!hbtp]
\centering
\subfigure[]{
\centering
\label{subfig:lp_mva_120_0j_of_MTCUTLT80}
\includegraphics[width=.40\textwidth]{lp_figures/histo_mva_120_0j_of_MTCUTLT80.png}}
\subfigure[]{
\centering
\label{subfig:lp_mva_120_0j_sf_MTCUTLT80}
\includegraphics[width=.40\textwidth]{lp_figures/histo_mva_120_0j_sf_MTCUTLT80.png}}\\
\subfigure[]{
\centering
\label{subfig:lp_mva_120_1j_of_MTCUTLT80}
\includegraphics[width=.40\textwidth]{lp_figures/histo_mva_120_1j_of_MTCUTLT80.png}}
\subfigure[]{
\centering
\label{subfig:lp_mva_120_1j_sf_MTCUTLT80}
\includegraphics[width=.40\textwidth]{lp_figures/histo_mva_120_1j_sf_MTCUTLT80.png}}
\caption{
MVA output for $m_H$=120 GeV LP ($m_T<80$) analysis: 
0-jet OF \subref{subfig:lp_mva_120_0j_of_MTCUTLT80},
0-jet SF \subref{subfig:lp_mva_120_0j_sf_MTCUTLT80},
1-jet OF \subref{subfig:lp_mva_120_1j_of_MTCUTLT80},
1-jet SF \subref{subfig:lp_mva_120_1j_sf_MTCUTLT80}
.}
\label{fig:lp_mva_120_MTCUTLT80}
\end{figure}

\begin{figure}[!hbtp]
\centering
\subfigure[]{
\centering
\label{subfig:lp_mva_130_0j_of_MTCUTLT80}
\includegraphics[width=.40\textwidth]{lp_figures/histo_mva_130_0j_of_MTCUTLT80.png}}
\subfigure[]{
\centering
\label{subfig:lp_mva_130_0j_sf_MTCUTLT80}
\includegraphics[width=.40\textwidth]{lp_figures/histo_mva_130_0j_sf_MTCUTLT80.png}}\\
\subfigure[]{
\centering
\label{subfig:lp_mva_130_1j_of_MTCUTLT80}
\includegraphics[width=.40\textwidth]{lp_figures/histo_mva_130_1j_of_MTCUTLT80.png}}
\subfigure[]{
\centering
\label{subfig:lp_mva_130_1j_sf_MTCUTLT80}
\includegraphics[width=.40\textwidth]{lp_figures/histo_mva_130_1j_sf_MTCUTLT80.png}}
\caption{
MVA output for $m_H$=130 GeV LP ($m_T<80$) analysis: 
0-jet OF \subref{subfig:lp_mva_130_0j_of_MTCUTLT80},
0-jet SF \subref{subfig:lp_mva_130_0j_sf_MTCUTLT80},
1-jet OF \subref{subfig:lp_mva_130_1j_of_MTCUTLT80},
1-jet SF \subref{subfig:lp_mva_130_1j_sf_MTCUTLT80}
.}
\label{fig:lp_mva_130_MTCUTLT80}
\end{figure}

\begin{figure}[!hbtp]
\centering
\subfigure[]{
\centering
\label{subfig:lp_mva_140_0j_of_MTCUTLT80}
\includegraphics[width=.40\textwidth]{lp_figures/histo_mva_140_0j_of_MTCUTLT80.png}}
\subfigure[]{
\centering
\label{subfig:lp_mva_140_0j_sf_MTCUTLT80}
\includegraphics[width=.40\textwidth]{lp_figures/histo_mva_140_0j_sf_MTCUTLT80.png}}\\
\subfigure[]{
\centering
\label{subfig:lp_mva_140_1j_of_MTCUTLT80}
\includegraphics[width=.40\textwidth]{lp_figures/histo_mva_140_1j_of_MTCUTLT80.png}}
\subfigure[]{
\centering
\label{subfig:lp_mva_140_1j_sf_MTCUTLT80}
\includegraphics[width=.40\textwidth]{lp_figures/histo_mva_140_1j_sf_MTCUTLT80.png}}
\caption{
MVA output for $m_H$=140 GeV LP ($m_T<80$) analysis: 
0-jet OF \subref{subfig:lp_mva_140_0j_of_MTCUTLT80},
0-jet SF \subref{subfig:lp_mva_140_0j_sf_MTCUTLT80},
1-jet OF \subref{subfig:lp_mva_140_1j_of_MTCUTLT80},
1-jet SF \subref{subfig:lp_mva_140_1j_sf_MTCUTLT80}
.}
\label{fig:lp_mva_140_MTCUTLT80}
\end{figure}

\begin{figure}[!hbtp]
\centering
\subfigure[]{
\centering
\label{subfig:lp_mva_150_0j_of_MTCUTLT80}
\includegraphics[width=.40\textwidth]{lp_figures/histo_mva_150_0j_of_MTCUTLT80.png}}
\subfigure[]{
\centering
\label{subfig:lp_mva_150_0j_sf_MTCUTLT80}
\includegraphics[width=.40\textwidth]{lp_figures/histo_mva_150_0j_sf_MTCUTLT80.png}}\\
\subfigure[]{
\centering
\label{subfig:lp_mva_150_1j_of_MTCUTLT80}
\includegraphics[width=.40\textwidth]{lp_figures/histo_mva_150_1j_of_MTCUTLT80.png}}
\subfigure[]{
\centering
\label{subfig:lp_mva_150_1j_sf_MTCUTLT80}
\includegraphics[width=.40\textwidth]{lp_figures/histo_mva_150_1j_sf_MTCUTLT80.png}}
\caption{
MVA output for $m_H$=150 GeV LP ($m_T<80$) analysis: 
0-jet OF \subref{subfig:lp_mva_150_0j_of_MTCUTLT80},
0-jet SF \subref{subfig:lp_mva_150_0j_sf_MTCUTLT80},
1-jet OF \subref{subfig:lp_mva_150_1j_of_MTCUTLT80},
1-jet SF \subref{subfig:lp_mva_150_1j_sf_MTCUTLT80}
.}
\label{fig:lp_mva_150_MTCUTLT80}
\end{figure}

\begin{figure}[!hbtp]
\centering
\subfigure[]{
\centering
\label{subfig:lp_mva_160_0j_of_MTCUTLT80}
\includegraphics[width=.40\textwidth]{lp_figures/histo_mva_160_0j_of_MTCUTLT80.png}}
\subfigure[]{
\centering
\label{subfig:lp_mva_160_0j_sf_MTCUTLT80}
\includegraphics[width=.40\textwidth]{lp_figures/histo_mva_160_0j_sf_MTCUTLT80.png}}\\
\subfigure[]{
\centering
\label{subfig:lp_mva_160_1j_of_MTCUTLT80}
\includegraphics[width=.40\textwidth]{lp_figures/histo_mva_160_1j_of_MTCUTLT80.png}}
\subfigure[]{
\centering
\label{subfig:lp_mva_160_1j_sf_MTCUTLT80}
\includegraphics[width=.40\textwidth]{lp_figures/histo_mva_160_1j_sf_MTCUTLT80.png}}
\caption{
MVA output for $m_H$=160 GeV LP ($m_T<80$) analysis: 
0-jet OF \subref{subfig:lp_mva_160_0j_of_MTCUTLT80},
0-jet SF \subref{subfig:lp_mva_160_0j_sf_MTCUTLT80},
1-jet OF \subref{subfig:lp_mva_160_1j_of_MTCUTLT80},
1-jet SF \subref{subfig:lp_mva_160_1j_sf_MTCUTLT80}
.}
\label{fig:lp_mva_160_MTCUTLT80}
\end{figure}

\begin{figure}[!hbtp]
\centering
\subfigure[]{
\centering
\label{subfig:lp_mva_200_0j_of_MTCUTLT80}
\includegraphics[width=.40\textwidth]{lp_figures/histo_mva_200_0j_of_MTCUTLT80.png}}
\subfigure[]{
\centering
\label{subfig:lp_mva_200_0j_sf_MTCUTLT80}
\includegraphics[width=.40\textwidth]{lp_figures/histo_mva_200_0j_sf_MTCUTLT80.png}}\\
\subfigure[]{
\centering
\label{subfig:lp_mva_200_1j_of_MTCUTLT80}
\includegraphics[width=.40\textwidth]{lp_figures/histo_mva_200_1j_of_MTCUTLT80.png}}
\subfigure[]{
\centering
\label{subfig:lp_mva_200_1j_sf_MTCUTLT80}
\includegraphics[width=.40\textwidth]{lp_figures/histo_mva_200_1j_sf_MTCUTLT80.png}}
\caption{
MVA output for $m_H$=200 GeV LP ($m_T<80$) analysis: 
0-jet OF \subref{subfig:lp_mva_200_0j_of_MTCUTLT80},
0-jet SF \subref{subfig:lp_mva_200_0j_sf_MTCUTLT80},
1-jet OF \subref{subfig:lp_mva_200_1j_of_MTCUTLT80},
1-jet SF \subref{subfig:lp_mva_200_1j_sf_MTCUTLT80}
.}
\label{fig:lp_mva_200_MTCUTLT80}
\end{figure}


  \clearpage
\end{document}
