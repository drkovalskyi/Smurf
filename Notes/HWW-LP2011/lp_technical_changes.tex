
The LP analysis makes the following additions or changes to the EPS analysis
described in the body of this document:

\begin{itemize}
    \item The integrated luminosity estimate for EPS dataset is revised from 1.092 fb$^{-1}$ to 1.143 fb$^{-1}$.
    \item The \dytt and \dyll~backgrounds are considered separately.
    \item The data to simulation scale factors for pile-up, lepton and trigger efficiencies are 
always applied to the simulation both when deriving and applying scale factors for background estimation.
    \item The residual jet energy corrections are included.
\end{itemize}

The effect of these changes is evaluated on the EPS dataset in Section \ref{sec:lp_limits}.

\subsubsection{Additional MT Preselection}

In general, a requirement that $80 < M_T < m^{Higgs}$ would remove poorly
understood regions with only a negligible reduction in sensitivity.
Specifically, we refer to the following:

\begin{itemize}
    \item There is a concern that $W/\gamma*$ with $W\rightarrow \ell\nu$
and $\gamma*\rightarrow\ell\ell$ could be a significant unaccounted background.
The $M_T$ of the reconstructed $e^{+}e^{-}$+MET system should be less than
    or equal to the expected $M_T$ from the decay of a $W$ boson.
    \item The \dytt~background is estimated entirely from simulation.  
This background manifests itself at low $M_T$, as illustrated in Section \ref{app:lp_postEPSdist}.
    \item The poorly modelled W+jets background would be reduced by removing the low $M_T$ region,
as illustrated in Section \ref{app:lp_postEPSdist}.
\end{itemize}

The yield cross check using same-sign events bounds the $W/\gamma*$ background 
at something less than 10 events at WW preselection level.  
This bound has the caveat that if the background is large and the 
fake rate method understimates the real background then the two
effects can cancel.

The default conversion rejection algorithm used in this analysis
includes a 2 cm flight (Lxy) cut, so would not reject this background
from apparently prompt conversions.
Taking into account the expected efficiency of our conversion
rejection algorithm, and removing the Lxy cut, 
we would expect to reject $30-50$\% of the residual 
$\gamma*\rightarrow e^{+}e^{-}$ background.
Performing this test we remove four events at the WW preselection level.
This reduction in the observed yield is consistent with expectations
from WW, W+jet and W+$\gamma$ simulation.
The events removed are distributed randomly in $M_T$.

Although we do not find evidence of a large unaccounted $W/\gamma*$ background,
we cannot conclusively prove that it is absent either.
We note that the proposed cut would reduce the \dytt~background from an expectation
of around 13 events to below one event in the one-jet bin $e-\mu$ final state at
the MVA preselection level.

In general, a requirement that $80 < M_T < m^{Higgs}$ would remove poorly
understood regions with only a negligible reduction in sensitivity.
This is illustrated in Section \ref{app:lp_limits}.
We think this is important to the analysis because shapes for the types of background
this cut would remove are not well understood. Reducing our exposure to these types of 
background is the right thing to do.

