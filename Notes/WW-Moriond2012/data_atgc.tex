\subsection{Anomalous $WW\gamma$ and $WWZ$ couplings}

Self-interaction between gauge bosons is a
direct consequence of the non-Abelian nature of the
electroweak sector of the standard model (SM). The
SM provides exact values of the self-interaction
couplings, and any deviation in measured values
from the SM predictions would be indications
of new physics. The $pp\to WW$ $s$-channel production
is governed by two such vertices, $WW\gamma$ and $WWZ$.
Anomalous values of these couplings would cause
the $WW$ production cross section and kinematics
to differ from the SM prediction. Therefore, a precise
study of the $pp\to WW$ production is not only
an important test of the electroweak sector of the SM, but
is imperative for the upcoming searches for the
Higgs boson production in one of its most sensitive
discovery channel, $H\to WW$.

The most general Lorentz invariant effective
Lagrangian that describes $WW\gamma$ and $WWZ$ couplings
has 14 independent parameters~\cite{hagiwara1,hagiwara2}, seven
for each vertex. Assuming $C$ and $P$ conservation, the number of
independent parameters reduces to six, resulting in
an effective Lagrangian normalized
by the electroweak coupling of the form:
\begin{equation}
\label{eq:atgc_lagrangian}
\frac{\cal L_{WWV}}{g_{WWV}} =
  ig_1^V(W_{\mu\nu}^\dagger W^\mu V^\nu - W^\dagger_\mu V_\nu W^{\mu\nu})
 + i\kappa_V W^\dagger_\mu W_\nu V^{\mu\nu} +
   \frac{i\lambda_V}{M_W^2}W^\dagger_{\delta\nu}W^\mu _\nu V^{\nu\delta},
\end{equation}
where $V = \gamma$ or $Z$, $W^\mu$ is the $W^-$ field,
$W_{\mu\nu} = \partial_\mu W_\nu - \partial_\nu W_\mu$, and the overall couplings
$g_{WW\gamma} = -e$ and $g_{WWZ} = -e \cot\theta_W$ where $\theta_W$
is the Weinberg angle. Assuming electromagnetic gauge
invariance, $g_1^\gamma = 1$, and we are left with five parameters to
describe the $WW\gamma$ and $WWZ$ couplings: $g_1^Z$, $\kappa_Z$,
$\kappa_\gamma$, $\lambda_Z$, and $\lambda_\gamma$. In the SM,
$\lambda_Z = \lambda_\gamma = 0$ and $g_1^Z = \kappa_Z = \kappa_\gamma = 1$.
In this analysis, we follow a convention to describe
the couplings in terms of their deviation from the SM values:
$\Delta g_1^Z \equiv g_1^Z - 1$, $\Delta \kappa_Z \equiv \kappa_Z - 1$,
and $\Delta \kappa_\gamma \equiv \kappa_\gamma -1$.

As the interaction Lagrangian in Eq.~\ref{eq:atgc_lagrangian}
with non-SM couplings violates partial wave unitarity at high energies,
we follow a conventional approach to scale the couplings by a form factor:
\begin{equation}
\label{eq:atgc_formfactor}
\alpha(\hat{s}) = \frac{\alpha_0}{(1 + \hat{s}/\Lambda^2)^2}.
\end{equation}
Here, $\alpha_0$ is a low-energy approximation of the coupling
$\alpha(\hat{s})$, $\hat{s}$ is the square of the invariant mass
of the $WW$ system, and $\Lambda$ is the form factor scale,
an energy at which new physics cancels divergences in the $WWV$
vertex.

Both LEP and Tevatron experiments have been active in measuring
the $WW\gamma$ and $WWZ$ couplings~\cite{atgc_lepResults1,
atgc_lepResults2, atgc_lepResults3,atgc_tevatronResults1,
atgc_tevatronResults2,atgc_tevatronResults3,atgc_tevatronResults4,
atgc_tevatronResults5} and observe no evidence for the aTGC values.
This study provides the first measurement of the $WW\gamma$ and $WWZ$
couplings in the $pp \to WW$ production at a center of mass energy
of 7 TeV.

\subsection{Method}

\subsection{Results}


