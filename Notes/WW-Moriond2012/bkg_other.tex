The $WZ$ and $ZZ$ events with lepton pairs from a resonant $Z$ boson are suppressed 
by the $Z$ veto. The remaining contribution is estimated from simulation, 
after applying the proper data to simulation correction factors for the 
lepton and trigger efficiencies. 
%

The $W+\gamma$ background, where the $\gamma$ fakes an electron through
an asymmetric conversion is difficult to estimate from data. Additional
cross-checks can be performed to place data based constraints on this estimate. 
For instance, applying the same standard selection, but requiring two same-sign 
leptons, gives a sample dominated by $\Wjets$ and $W+\gamma$ events. Again, the 
expected contribution is very small, due to stringent $\gamma$ conversion 
requirements explained in Sec.~\ref{sec:sel_electrons}.

The electroweak process \Wgstar\ enters the signal region if one of the leptons 
from the $\gamma^*$ is lost. This background is normally covered in Monte Carlo
simulations as a part of the \WZ\ process. However in the simulation, 
there is a generator level cut of $m_{\gamma^*}>12$ GeV, and there is
significant rate of events at lower values~\cite{wgstar}. 
A dedicated $W\gamma^*$ sample is simulated in Madgraph 
with $m_{\gamma^*} < 12$ GeV. The prediction from this simulation is then normalized to 
data according to the rate in the $W\gamma^*$ enriched region, as 
detailed in Ref.~\cite{HWW2011Final}. 


The $\dytt$ background is suppressed signficiantly by the 
projected $\met$ requirements as the $\met$ tend to be aligned with 
one of the leptons. However the large amount of pileup interactions 
may lead to fake \met\ that is larger than the natural \met\
in \dytt\ events. Given that the fake \met\ can not be reliably 
estimated in simulation, we developped two data-driven methods to estimate the 
$\dytt$ background as documented in Ref.~\cite{HWW2011Final}. 
Both methods find that the \dytt\ rate in data is a factor of 4 larger than 
the value predicted by simulation. 

