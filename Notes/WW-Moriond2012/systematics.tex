
FIXME: Table \ref{tab:systww} values to be confirmed still 
appropriate for WW.

We have taken into account the following systematic uncertainties:

\begin{itemize}
\item {\it Luminosity:} We assume an uncertainty of 4.5\% since at this moment there are some
concerns about the actual luminosity measurements at CMS.

\item {\it Lepton identification and trigger efficiencies:}
We measure the efficiencies in data using the tag and probe method that is described
in detail in Section~\ref{sec:efficiency}.
The estimated uncertainty is about $2\%$ per lepton leg.

\item {\it Momentum scale:}
Due to several factors, the energy scale for electrons and the momentum
scale for muons have relatively large uncertainties in the current data
processing.
We assign a systematic uncertainty by varying the transverse momentum of the muons by $1\%$,
and $2\%$ and $5\%$ for electrons in the barrel and the endcap, respectively.
The contribution to the uncertainty on the dilepton efficiency is about $1.5\%$.

\item {\it $\met$ modeling:} We use a data-driven method to estimate the $\dyll$
background, which is affected by the $\met$ resolution.
Events with neutrinos giving real $\met$ in the final state also have a small uncertainty.
We assess this uncertainty on the event selection efficiency by varying the $\met$ in signal events
by an additional $10\%$. We find an uncertainty on the event selection efficiency of around 2\%.

\item {\it Background estimation:}
The methods to estimate the different backgrounds are explained in
Section~\ref{sec:backgrounds}.
Here we summarize the systematic uncertainties associated with the methods used.

  \begin{itemize}
  \item Jet induced backgrounds, $\Wjets$ and $QCD$: the associated systematic
    uncertainty is 36\%.
  \item Top background: this background is estimated using $b$-tagged events and
    the $b$-tagging efficiency, which is measured in control regions in data.
    The associated systematic uncertainties are below $5\%$,
    while the statistical component is about $25\%$ for $\intlumi$.
  \item Drell-Yan background: The uncertainty arises from the limited knowledge of
    events with large $\met$ tails.
    We conservatively quantify such uncertainty from the variation of the ratio $R_{out/in}$
    (Eq.\ref{eq:dyest}) as a function of the $\met$ requirement,
    leading to an estimate of about $43\%$.
%    As very few $\dyll$ events are selected, this has anyhow a small affect on the final analysis results ($\sim xxx\%$).
  \item Other backgrounds: The sub-dominant backgrounds are estimated from simulation
    with appropriate systematic uncertainties on their cross section.
    The theoretical uncertainties on the $\WZ$ and $\ZZ$ are estimated the same way as in the signal detailed in below. 
%    We take $3\%$ for $\WZ$ and $\ZZ$ events and $10\%$ for $W+\gamma$ events.
    These uncertainties must be augmented by the luminosity normalization uncertainty.
  \end{itemize}

\item {\it Pileup:} an incorrect modeling of the pileup in the Monte Carlo samples
can bias the expected event yields. The simulated events have been re-weighted
on the basis of the number of reconstructed
primary vertices. The re-weighting procedure affects only slightly the results of the analysis,
the event yields changing by $\sim1\%$. The latter is conservatively assumed as
the corresponding systematic uncertainty.

\item {\it Jet veto efficiency:}
This is dominated by the theoretical uncertainty, and is found to be 4.6\%. We add in 
quardrature another 1\% coming from the difference of jet veto efficiency measured in data 
from simulation. 

\item {\it Theoretical uncertainties:}

We evalute the theoretical uncertainties due to the uncertainties in the PDFs and 
QCD high order effects to the selection efficiency and normalizations 
for the $WW$ signal and the backgrounds that are evaluated from MC ($WZ$ and $ZZ$). 

In order to assign systematic uncertainties due to PDFs, 
we follow the strategy defined by the CMS Generator 
Group described in~\cite{XS}, which is consistent with the latest 
PDF4LHC recommendations~\cite{PDF4LHC}. The 68\% CL of the positive and negative uncertainties 
obtained with CTEQ66~\cite{Nadolsky:2008zw}, MSTW2008NLO~\cite{Martin:2009iq} and
NNPDF2.0~\cite{Ball:2010de} sets are considered, adopting the 
specific recommended recipes in each case. The final assigned systematic uncertainties
corresponds to half of the maximum difference observed between positive and 
negative variations for any combination of the three sets. The maximum 
difference corresponds to a positive variation from one set minus a negative 
variation from a different set, since central values from different sets are 
typically of the size of the uncertainties within a set. Uncertainties due 
to $\alpha_s$ are also considered. An uncertainty of about 2.3\% has been estimated for the signal. 

We have also computed the change of the acceptance of having two lepton in 
the fiducial region when varying renormalization and factorization QCD scales. 
The test serves to estimate the accuracy of the cross-section via the stability of the perturbative 
expansion. In this course it is common to vary the renormalization and 
factorization scales around some default scale choice, which is usually 
chosen to be at the order of the scale of the hard-scattering process. 
For each process we thus define some default scale $ \mu_0$, which is the mass
of the boson or particle under study, and vary 
in the range $ \mu_0/2 < \mu_F=\mu_R < 2\times\mu_0$. 
We find an uncertainty due to the change on the kinematical requirements of about 
1.5\% percent due these effects to the $WW$ signal. 


\item {\it Monte Carlo statistics:} We also take into account the
size of the simulated event samples. 
\end{itemize}

\begin{table}[ht!]
\begin{center}
\caption{\label{tab:systww} Summary of all systematic uncertainties (relative).}
\vspace{5pt}
{\small
\begin{tabular}{l|c|c|c|c|c|c|c|c}
\hline
%\multirow{2}{*}{Source} & $qq \to$ & $gg \to$  & non-$\Z$ resonant & top & DY & $\Wjets$ & $V(W/Z)+\gamma$    \\
%                        & $\WW$    & $\WW$       & $VV$              &     &         &          &                     \\
\multirow{2}{*}{Source} & $qq \to$ & $gg \to$  & $WZ$ & $ZZ$  & top & $Z/\gamma^*$ & $\Wjets$ & $W+\gamma$    \\
                        & $\WW$    & $\WW$     &      &       &     &              &          &               \\
\hline

\hline
Luminosity                    & 4.5 & 4.5 & 4.5 & 4.5 & --- & --- &  --- & ---  \\
Trigger efficiencies          & 1.5 & 1.5 & 1.5 & 1.5 & --- & --- &  --- & --- \\
Muon efficiency               & 1.5 & 1.5 & 1.5 & 1.5 & --- & --- &  --- & --- \\
Electron id efficiency        & 2.5 & 2.5 & 2.5 & 2.5 & --- & --- &  --- & --- \\
Momentum scale                & 1.5 & 1.5 & 1.5 & 1.5 & --- & --- &  --- & --- \\
$\met$ resolution             & 2.0 & 2.0 & 2.0 & 2.0 & --- & --- &  --- & --- \\
Jet veto                      & 4.7 & 4.7 & 4.7 & 4.7 & --- & --- &  --- & --- \\
PDF uncertainties             & 2.3 & 2.3 & 6.5 & 4.8 & --- & --- &  --- & --- \\
QCD scale uncertainties       & 1.5 & 1.5 & 4.2 & 1.8 & --- & --- &  --- & --- \\
Pile up                       & 1.0 & 1.0 & 1.0 & 1.0 & --- & --- &  --- & --- \\
$\Wjets$ norm.                & --- & --- & --- & --- & --- & --- &  36  & --- \\
top  norm.                    & --- & --- & --- & --- & 19  & --- &  --- & --- \\
$\dyll$ norm.                 & --- & --- & --- & --- & --- &  43 &  --- & --- \\
%$WZ/ZZ$ cross section         & --- & --- & 3.0 & 3.0 & --- & --- &  --- & --- \\
$W+\gamma$ cross section      & --- & --- & --- & --- & --- & --- &  --- & 30 \\
Monte Carlo statistics        &   1 &   1 &   2 &  2  & --- & --- &  --- & --- \\
\hline 
Total systematic uncertainty  &  8  &   8 &  11 &  9  & 19  &  43  & 36  & 30 \\ 
\hline
\end{tabular}
}
\end{center}
\end{table}

