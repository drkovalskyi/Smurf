The $W^+W^-$ diboson production process can be studied to perform
accurate tests of the description of electroweak and strong interactions
in the Standard Model (SM). Next-to-leading order calculations of $W^+W^-$
production in $pp$ collisions at $\sqrt{s} = 7~\TeV$ predict a cross-section
of $\sigma^{NLO} (pp \to WW \to 2\ell 2\nu) = 43.0 \pm 2.0~$pb~\cite{MCFM}~\cite{XS}.

Within the SM the dominant $W^+W^-$ production mechanisms are via the
s-channel and t-channel diagrams. With the s-channels we can measure $WWZ$
and $WW\gamma$ triple gauge couplings, which are sensitive to possible new
physics processes via anomalous couplings. In addition, they represent an
important background source for new particle searches, e.g. \hww{}
Higgs boson searches.

This note documents the first $W^+W^-$ production cross-section
measurement in $\intlumi$ of $pp$ collision data at $\sqrt{s} = $
7~$\TeV$ using leptonic decays, i.e. \wwlnln{}, with electrons
and muons in the final state. Fully leptonic tau decays are also
considered as a part of the signal, but the selection requirements are
not optimized for such events.  The work is based on previous studies
of \wwpm{}\cite{WWNote,WWPAS,WWNote7TeV}
and \hww{}\cite{HWWNote,HWWPAS} processes performed on simulation.

The note is structured as follows. A brief discussion about the data samples
used in the analysis is presented in Sec.~\ref{sec:datasets}. Trigger
selection, lepton selection, and all kinematic requirements are described in
Sec.~\ref{sec:selection}. A summary of the event data yields and the
expectations from Monte Carlo is shown in Sec.~\ref{sec:yields}, followed by
the estimation of the backgrounds in Sec.~\ref{sec:backgrounds}. The
cross-section measurement, together with all sources of systematic
uncertainties are explained in Sec.~\ref{sec:results}.
Finally, the anomalous couplings limits and conclusions are included in
Secs.~\ref{sec:couplings} and~\ref{sec:summary}, respectively.

