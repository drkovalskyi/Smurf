 
To determine the efficiency of the dilepton triggers, 
we derive the efficiency of the requirements imposed on each leg separately.
This requires a modification to the tag and probe method described above in some cases.
If the trigger objects are saved by the HLT before the requirement that there be two valid objects then
we can check each leg independently of the other using the usual tag and probe method.
If the trigger objects are saved after the requirement that there are two valid objects, then there is 
a 100\% correlation between the decision we can probe on each lepton.
This means that we must pick exactly one tag candidate for each event a priori, which we do 
randomly. 
If the randomly selected tag candidate meets the tight requirements then we are free to 
probe the other lepton.

%\subsubsection{Electron triggers}
%%%%%% double
The double electron trigger requires the higher $p_T$ leg to be seeded at Level-1.
The efficiency of the seeded and unseeded legs with respect to an electron passing
offline selection is tabulated in 
Table \ref{tab:eff_trigger_doubleEle}.
The efficiency of the single electron trigger with respect to
an electron passing offline selection is given in Table \ref{tab:eff_trigger_singleEle}.
The listed values represent the overall efficiencies averaged over the run range
of the dataset, absorbing changes in thresholds and seeding requirements
over time.
The efficiency of the leading and trailing legs of the double muon trigger
is summarized in Table \ref{tab:eff_trigger_doubleMu}.
The efficiency of the single
muon trigger is given in Table \ref{tab:eff_trigger_singleMu}.

In the case of the $e\mu$ triggers, we cross check the trigger efficiency 
against the leading and trailing legs of the double electron and
double muon triggers using  
dilepton $t\bar{t}$ events requiring that the event has missing transverse
energy greater than $20$ \GeV. The efficiency of 
the muon leg are measured using events passing the single electron trigger,
while the efficiency of the electron leg are measured using events passing the
single muon trigger. They are found to be consistent within statistical 
uncertainties. We thus take the single leg efficiencies from the double
electron and double muon triggers for the cross triggers as well.

 \begin{table}[!ht]
 \begin{center}
 \begin{tabular}{|c|c|c|}
 \hline
 Measurement &  $  0.0  \le |\eta| <   1.5$ & $  1.5  \le |\eta| <   2.5$ \\
 \hline
\multicolumn{3}{c}{Leading leg requirement.} \\
\hline
$ 20.0 < p_{T} \le  30.0$ &     0.9849 +/- 0.0003  &      0.9774 +/- 0.0007 \\
\hline
$ 30.0 < p_{T} $ &     0.9928 +/- 0.0001  &      0.9938 +/- 0.0001 \\
\hline
\multicolumn{3}{c}{Trailing leg requirement.} \\
 \hline
$ 20.0 < p_{T} \le  30.0$ &     0.9923 +/- 0.0002  &      0.9953 +/- 0.0003 \\
\hline
$ 30.0 < p_{T} $ &     0.9948 +/- 0.0001  &      0.9956 +/- 0.0001 \\
\hline
\end{tabular}
\caption{The per leg efficiency of the double electron trigger, averaged over the full 2011 dataset.}
\label{tab:eff_trigger_doubleEle}
\end{center}
\end{table}

 \begin{table}[!ht]
 \begin{center}
 \begin{tabular}{|c|c|c|}
 \hline
 Measurement &  $  0.0  \le |\eta| <   1.5$ & $  1.5  \le |\eta| <   2.5$ \\
 \hline
$ 20.0 < p_{T} \le  25.0$ &     0.0002 +/- 0.0001  &      0.0001 +/- 0.0002 \\
\hline
$ 25.0 < p_{T} \le  30.0$ &     0.0314 +/- 0.0006  &      0.0144 +/- 0.0007 \\
\hline
$ 30.0 < p_{T} \le  35.0$ &     0.1511 +/- 0.0009  &      0.1303 +/- 0.0016 \\
\hline
$ 35.0 < p_{T} \le  40.0$ &     0.2318 +/- 0.0008  &      0.2496 +/- 0.0017 \\
\hline
$ 40.0 < p_{T} \le  50.0$ &     0.2342 +/- 0.0006  &      0.2327 +/- 0.0011 \\
\hline
$ 50.0 < p_{T} \le  65.0$ &     0.2899 +/- 0.0013  &      0.2502 +/- 0.0022 \\
\hline
$ 65.0 < p_{T} \le  80.0$ &     0.8170 +/- 0.0027  &      0.5012 +/- 0.0065 \\
\hline
$ 80.0 < p_{T} $ &     0.9470 +/- 0.0020  &      0.9193 +/- 0.0048 \\
\hline
\end{tabular}
\caption{The efficiency of the single electron trigger, averaged over the full 2011 dataset.}
\label{tab:eff_trigger_singleEle}
\end{center}
\end{table}

 \begin{table}[!ht]
 \begin{center}
 \begin{tabular}{|c|c|c|c|c|}
 \hline
 Measurement &  $  0.0  \le |\eta| <   0.8$ & $  0.8  \le |\eta| <   1.2$ & $  1.2  \le |\eta| <   2.1$ & $  2.1  \le |\eta| <   2.4$ \\
 \hline
\multicolumn{5}{c}{Leading leg requirement.} \\
\hline
$ 20.0 < p_{T} \le  30.0$ &     0.9648 +/- 0.0007  &      0.9516 +/- 0.0013  &      0.9480 +/- 0.0009  &      0.8757 +/- 0.0026 \\
\hline
$ 30.0 < p_{T} $ &     0.9666 +/- 0.0003  &      0.9521 +/- 0.0005  &      0.9485 +/- 0.0004  &      0.8772 +/- 0.0012 \\
\hline
\multicolumn{5}{c}{Trailing leg requirement.} \\
 \hline
$ 20.0 < p_{T} \le  30.0$ &     0.9655 +/- 0.0007  &      0.9535 +/- 0.0013  &      0.9558 +/- 0.0009  &      0.9031 +/- 0.0023 \\
\hline
$ 30.0 < p_{T} $ &     0.9670 +/- 0.0003  &      0.9537 +/- 0.0005  &      0.9530 +/- 0.0004  &      0.8992 +/- 0.0011 \\
\hline
\end{tabular}
\caption{The per leg efficiency of the double muon trigger, averaged over the full 2011 dataset.}
\label{tab:eff_trigger_doubleMu}
\end{center}
\end{table}

 \begin{table}[!ht]
 \begin{center}
 \begin{tabular}{|c|c|c|c|c|}
 \hline
 Measurement &  $  0.0  \le |\eta| <   0.8$ & $  0.8  \le |\eta| <   1.5$ & $  1.5  \le |\eta| <   2.1$ & $  2.1  \le |\eta| <   2.4$ \\
 \hline
$ 15.0 < p_{T} \le  24.0$ &     0.2799 +/- 0.0022  &      0.2723 +/- 0.0023  &      0.2706 +/- 0.0024  &      0.2264 +/- 0.0034 \\
\hline
$ 24.0 < p_{T} \le  30.0$ &     0.4659 +/- 0.0016  &      0.4449 +/- 0.0019  &      0.4549 +/- 0.0021  &      0.3294 +/- 0.0030 \\
\hline
$ 30.0 < p_{T} \le  40.0$ &     0.9002 +/- 0.0005  &      0.8352 +/- 0.0008  &      0.8266 +/- 0.0009  &      0.3345 +/- 0.0019 \\
\hline
$ 40.0 < p_{T} $ &     0.9440 +/- 0.0003  &      0.8821 +/- 0.0005  &      0.8611 +/- 0.0007  &      0.3453 +/- 0.0017 \\
\hline
\end{tabular}
\caption{The efficiency of the single muon trigger, averaged over the full 2011 dataset.}
\label{tab:eff_trigger_singleMu}
\end{center}
\end{table}


Having measured the per lepton trigger efficiencies 
and for the double and single trigger,
we compute the efficiency for dilepton events to be selected.
We do this by taking into account the two ways an event can be selected: 
the double trigger can pass or the double trigger can fail because one leg is bad
but the good leg can pass the single trigger.
If both legs are bad in the double trigger they will also both be bad in the single trigger
because the requirements of the single trigger are tighter than any single leg of the double trigger.
Thus taking into account combinatorics, the event efficiency $\varepsilon_{\ell\ell'}(p_T,\:\eta,\:p'_T,\:\eta')$
is given in Equation \ref{eqn:evteff}, where $\varepsilon_{S}(p_T,\:\eta)$ is the single 
lepton trigger efficiency,
$\varepsilon_{\mathrm{D,leading}}(p_T,\:\eta)$ is the efficiency of the leading leg of the 
appropriate double trigger, and $\varepsilon_{\mathrm{D,trailing}}(p_T,\:\eta)$ is the 
efficiency of the trailing leg of the appropriate double trigger.

\begin{eqnarray}
\label{eqn:evteff}
\varepsilon_{\ell\ell'}(p_T,\:\eta,\:p'_T,\:\eta') & = & 1 - [(1-\varepsilon_{\mathrm{D,leading}}(p_T,\:\eta))(1-\varepsilon_{\mathrm{D,leading}}(p_T',\:\eta')) \\
               &   & +~\varepsilon_{\mathrm{D,leading}}(p_T,\:\eta)(1-\varepsilon_{\mathrm{D,trailing}}(p_T',\:\eta')) \\
               &   & +~\varepsilon_{\mathrm{D,leading}}(p_T',\:\eta')(1-\varepsilon_{\mathrm{D,trailing}}(p_T,\:\eta))] \\
               &   & +~\varepsilon_{S}(p_T',\:\eta')(1-\varepsilon_{\mathrm{D,trailing}}(p_T,\:\eta)) \nonumber\\
               &   & +~\varepsilon_{S}(p_T,\:\eta)(1-\varepsilon_{\mathrm{D,trailing}}(p_T',\:\eta'))
\end{eqnarray}

The procedure of Equation \ref{eqn:evteff} is applied to simulated $\WW$ decays to obtain an event-by-event weight factor. We find a 
trigger efficiency with respect to the offline selection of $98\%$ for both the $qq\to\WW$ and $qq\to\WW$ processes.

