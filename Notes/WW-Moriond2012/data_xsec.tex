
We now calculate the $\WW$ production cross section according to equation \ref{eq:mainformula},

\begin{equation}
\label{eq:mainformula}
\sigma_{WW \to 2\ell 2\nu}  = \frac{N_{data} - N_{bkg}}{\epsilon \cdot {\cal{L}} \cdot BR(WW \to \ell \nu \ell \nu)}
\end{equation}

Where $N_{data}$ is the number of events observed in data, $N_{bkg}$ is the estimated number
of background events, $\epsilon$ is the efficiency to select $\sigma_{WW \to 2\ell 2\nu}$
candidates, $\cal{L}$ is the integrated luminosity and $BR(WW \to \ell \nu \ell \nu)$ is the 
branching ratio for the $\ell \nu \ell \nu$ final state considered.
The number of events observed in data and the estimated number of events from the different
backgrounds are summarised in Table \ref{tab:data_yields}.
The efficiency, $\varepsilon$ is computed as the weighted mean of
the $qq\to\WW$ and $gg\to\WW$ efficiencies in simulation, assuming a 3\%
contribution from the $gg$ process.

\begin{table}[ht!]
  \begin{center}
 {\small
  \begin{tabular} {|c|c|c|c|c|c|c|}
\hline
          &   data & all bkg. & $qq \to \WW$ & $gg \to \WW$ &  $\ttbar+tW$   & $\Wjets$    \\
  \hline
  \hline
 $ee+\mu\mu$ &  462 & 96.27 $\pm$ 6.54 & 279.29 $\pm$ 4.11 & 16.65 $\pm$ 0.58 & 45.13 $\pm$ 2.26 & 16.40 $\pm$ 3.93 \\ 
  $e\mu + \mu e$ &  668 & 148.28 $\pm$ 4.69 & 449.10 $\pm$ 3.23 & 26.57 $\pm$ 0.46 & 80.14 $\pm$ 1.81 & 43.21 $\pm$ 3.20 \\ 
  Total & 1130 & 244.55 $\pm$ 6.54 & 728.40 $\pm$ 4.11 & 43.22 $\pm$ 0.58 & 125.27 $\pm$ 2.26 & 59.61 $\pm$ 3.93 \\ 
 \hline
 \hline
  \end{tabular}
  \begin{tabular} {|c|c|c|c|c|c|}
\hline
       & $WZ$/$ZZ$ not included in the $\dyll$ & $\dyll+WZ+ZZ$ & $W+\gamma$ & \dytt \\
  \hline
  \hline
 $ee+\mu\mu$ & 4.97 $\pm$ 0.14 & 24.79 $\pm$ 3.46 & 4.97 $\pm$ 1.31 & 0.00 $\pm$ 0.00 \\ 
 $e\mu + \mu e$ & 9.58 $\pm$ 0.20 & 0.99 $\pm$ 0.24 & 13.64 $\pm$ 2.90 & 0.72 $\pm$ 0.22 \\ 
 Total & 14.55 $\pm$ 0.25 & 25.78 $\pm$ 3.47 & 18.62 $\pm$ 3.18 & 0.72 $\pm$ 0.22 \\ 
 \hline
  \end{tabular}
  }
  \caption{Expected number of signal and background events from the data-driven methods for
  an integrated luminosity of \intlumi after applying the selection requirements.
  Statistical uncertainties only.}
   \label{tab:data_yields}
  \end{center}
\end{table}

\begin{table}[!ht]
\begin{center}
\begin{tabular}{|c|c|c|}
\hline
 variable      &  value & uncertainty \\
\hline
$N_{data}$     & $1130$ & ---\\
\hline
$N_{bkg}$      & $244.55$ & $6.54~\mathrm{(stat.)} \pm 33.52~\mathrm{(syst.)}$ \\
\hline
$\epsilon$ (\%) & $3.378$ & $0.233$  \\
\hline
$\cal{L}$ ($pb$) & $4630$ & $208$ \\
\hline
$BR(W \to \ell \nu)$ & $0.1080$ & $0.0009$ \\
\hline
\end{tabular}
\caption{Summary of the pieces to compute the $WW$ cross-section and its uncertainty.}
  \label{tab:xs_summary}
\end{center}
\end{table}

The final inputs to Equation \ref{eq:mainformula} are given in
Table \ref{tab:xs_summary}.  Using these we obtain the following 
$WW$ cross-section measurement:

\begin{equation*}
\sigma_{WW \to 2\ell 2\nu}  = 53.93 \pm 2.05~\mathrm{(stat.)} \pm 4.26~\mathrm{(syst.)} \pm 2.43~\mathrm{(lumi.)~pb},
\end{equation*}

To be compared with the standard model prediction \cite{Campbell:2011bn}:

\begin{equation*}
\sigma_{WW \to 2\ell 2\nu}  = 47.04^{+4.3}_{-3.2} ~\mathrm{pb}.
\end{equation*}



