Triggering on Higgs boson decays in the dilepton final state increases 
in difficulty with increasing instantaenous luminosity.
Single lepton triggers can only be sustained with very tight identification and
isolation requirements and large transverse momentum thresholds.
This means that double lepton triggers are the only viable option to maintain
sensitivity to a low mass Higgs boson, where the leptons transverse momentum
can be small.

We designed a suite of signal and control triggers appropriate for this analysis.
These dilepton triggers have a high efficiency to collect Higgs boson events
and are sufficiently loose to collect control events to estimate
fake lepton backgrounds and selection efficiencies with adequate precision.
We describe the features and motivations for the analysis triggers in Section~\ref{sec:mainTriggers},
and additional triggers to collect control events in 
Section~\ref{sec:utilityTriggers}.

\subsubsection{Analysis Triggers}
\label{sec:mainTriggers}

The main dielectron triggers, described in Table~\ref{tab:triggers_ee}, require two HLT electron
candidates with loose shower shape and calorimeter isolation requirements on both legs
and a match to a Level-1 seed for the leading leg.
To accomodate the offline selection of $E_{T}>20,10$~GeV for the leading and trailing
electrons, $E_{T}>17,8$~GeV is required at the HLT level.
Controlling the total trigger rate is most challenging in
the dielectron channel, due to large fake electron background rates.
Additional requirements must be added to the track-to-cluster matching
and track isolation to control the total trigger rate at 
instantaneous luminosities above $1\times10^{33}~cm^{-1}sec^{-1}$.

The identification and isolation requirements are described in Table~\ref{tab:HLTElectronCuts}.
Because the electron HLT uses simplified algorithms compared to the offline selections,
the variables used online and offline do not always correspond exactly.

\begin{table}[!ht]
  \caption{Analysis triggers for the $ee$ final state. 
The identification and isolation requirements are described in Table~\ref{tab:HLTElectronCuts}.}
    \vspace{5pt}
   \label{tab:triggers_ee}
  \begin{center}
 {\small
  \begin{tabular} {l|l|l|c}
\hline
  Dataset & Trigger name & L1 seed & Description\\
  \hline \hline
  \multirow{2}{*}{DoubleElectron} & HLT\_Ele17\_CaloIdL\_CaloIsoVL\_&  L1\_SingleEG12  & $p_T>17,8~\GeVc$ \\
                                  & Ele8\_CaloIdL\_CaloIsoVL &                  & \\

  \multirow{2}{*}{DoubleElectron} & HLT\_Ele17\_CaloIdT\_TrkIdVL\_CaloIsoVL\_TrkIsoVL\_ &  L1\_SingleEG12  & $p_T>17,8~\GeVc$ \\
                                  & Ele8\_CaloIdT\_TrkIdVL\_CaloIsoVL\_TrkIsoVL &                  & \\
  \hline
  \end{tabular}
}
  \end{center}
\end{table}

\begin{table}[!ht]
 \caption{Summary of requirements applied to electrons in the triggers used for this analysis.
The selection requirements are given for electrons in the barrel (endcap).
L=Loose, VL=Very loose, T=Tight, VT=Very Tight.}
    \vspace{5pt}
 \label{tab:HLTElectronCuts}
 \centering
 \begin{tabular}{l|c}
   \hline
   name                       &  criterion \\
   \hline \hline
   \multirow{2}{*}{CaloId\_L} & $\mathrm{H/E < 0.15 (0.10) }$ \\
                               & $\sigma_{\eta\eta}\mathrm{< 0.014\;(0.035)}$ \\
    \hline
   \multirow{2}{*}{CaloId\_VT} & $\mathrm{H/E < 0.05 (0.05) }$ \\
                               & $\sigma_{\eta\eta}\mathrm{< 0.011\;(0.031)}$ \\
    \hline \hline
    \multirow{2}{*}{TrkId\_VL} & $|\Delta\eta|\mathrm{< 0.01\; (0.01)}$ \\
                               & $\Delta\phi\mathrm{< 0.15\;(0.10)}$  \\
    \hline
    \multirow{2}{*}{TrkId\_T} & $|\Delta\eta|\mathrm{< 0.008\; (0.008)}$ \\
                               & $\Delta\phi\mathrm{< 0.07\;(0.05)}$  \\
    \hline \hline
    \multirow{2}{*}{CaloIso\_VL} & $\mathrm{ECalIso/E_T <0.2\;(0.2)}$ \\
                                 & $\mathrm{HCalIso/E_T <0.2\;(0.2)}$ \\
    \hline
    \multirow{2}{*}{CaloIso\_T} & $\mathrm{ECalIso/E_T <0.15\;(0.075)}$ \\
                                 & $\mathrm{HCalIso/E_T <0.15\;(0.075)}$ \\
    \hline
    \multirow{2}{*}{CaloIso\_VT} & $\mathrm{ECalIso/E_T <0.05\;(0.05)}$ \\
                                 & $\mathrm{HCalIso/E_T <0.05\;(0.05)}$ \\
    \hline \hline
    TrkIso\_VL                   & $\mathrm{TrkIso/E_T <0.2\;(0.2)}$ \\
    \hline
    TrkIso\_T                   & $\mathrm{TrkIso/E_T <0.15\;(0.075)}$ \\
   \hline
    TrkIso\_VT                   & $\mathrm{TrkIso/E_T <0.05\;(0.05)}$ \\
    \hline
 \end{tabular}
\end{table}

The main dimuon triggers
require two HLT muon candidates with transverse momentum greater than $7$~$\GeVc$ and
a match to a Level-1 seed is required for both legs.
These are described in Table~\ref{tab:triggers_mm}.

\begin{table}[!ht]
  \caption{Analysis triggers for the $\mu\mu$ final state. Triggers marked (*) are also used for efficiency studies.}
    \vspace{5pt}
   \label{tab:triggers_mm}
  \begin{center}
 {\small
  \begin{tabular} {l|l|l|c}
\hline
  Dataset & Trigger name & L1 seed & Description\\
  \hline \hline
  DoubleMu & HLT\_DoubleMu6 & L1\_DoubleMu3  & $p_T>6,6~\GeVc$\\
  DoubleMu & HLT\_DoubleMu7 & L1\_DoubleMu3  & $p_T>7,7~\GeVc$ \\
  DoubleMu & HLT\_Mu13\_Mu8 & L1\_DoubleMu3  & $p_T>13,8~\GeVc$ \\
  DoubleMu & HLT\_Mu17\_Mu8 & L1\_DoubleMu3  & $p_T>17,8~\GeVc$ \\
  \hline
  \end{tabular}
}
  \end{center}
\end{table}

In the electron-muon channel we use two complementary triggers, which require
both muon and electron HLT candidates.
These are summarised in Table~\ref{tab:triggers_em}.
Finally, to recover any residual inefficiency,
we also allow events that passed only the single electron
or single isolated muon triggers summarised in Table~\ref{tab:triggers_single}.

\begin{table}[!ht]
  \caption{Analysis triggers for the $e\mu$ final state.
The identification and isolation requirements for electrons are described in Table~\ref{tab:HLTElectronCuts}.}
    \vspace{5pt}
   \label{tab:triggers_em}
  \begin{center}
 {\small
  \begin{tabular} {l|l|l|c}
\hline
  Dataset & Trigger name & L1 seed & Description\\
  \hline \hline
  MuEG & HLT\_Mu17\_Ele8\_CaloIdL & L1\_Mu3\_EG5 & $p_T>17,8~\GeVc$ \\
  MuEG & HLT\_Mu8\_Ele17\_CaloIdL & L1\_Mu3\_EG5 & $p_T>8,17~\GeVc$ \\
 \hline
  \end{tabular}
}
  \end{center}
\end{table}

\begin{table}[!ht]
  \caption{Single lepton triggers to recover lost efficiency. These trigges are also used for efficiency studies.
The identification and isolation requirements for electrons are described in Table~\ref{tab:HLTElectronCuts}.}
    \vspace{5pt}
   \label{tab:triggers_single}
  \begin{center}
 {\small
  \begin{tabular} {l|l|l|c}
\hline
  Dataset & Trigger name & L1 seed & Description\\
  \hline \hline
  SingleEle & HLT\_Ele27\_CaloIdVT\_CaloIsoT\_TrkIdT\_TrkIsoT & L1\_SingleEG15  & $p_T>27~\GeVc$ \\
  \hline \hline
  SingleMu & HLT\_IsoMu12   & L1\_SingleMu7  & $p_T>12~\GeVc$ \\
  SingleMu & HLT\_IsoMu17   & L1\_SingleMu10 & $p_T>17~\GeVc$ \\
  SingleMu & HLT\_Mu15      & L1\_SingleMu10 & $p_T>15~\GeVc$ \\
  \hline 
  \end{tabular}
}
  \end{center}
\end{table}

\subsubsection{Utility Triggers}
\label{sec:utilityTriggers}

We now describe additional triggers that are introduced to collect control or
calibration events not covered by the main analysis triggers.

Because the main dielectron analysis triggers make requirements on
both legs, events collected with those triggers cannot be used to measure
efficiencies without introducing unacceptable bias.
Thus, to measure the electron selection and trigger efficiency
we introduce two specialised tag and probe triggers designed to maximize
the number of useful \dyll~events for both low and high $p_{T}$ electrons,
while keeping the total trigger rate at a reasonable level. 
The tag and probe method is described later in Section~\ref{sec:efficiency}.

The first trigger probes low $p_T$ electrons and applies very tight identification 
and isolation requirements on the tag leg to reduce the background rate.
The second trigger probes higher $p_{T}$ electrons.
Both triggers are described in Table~\ref{tab:triggers_util} and labeled "eff".

\begin{table}[!ht]
  \caption{Utility triggers.
The identification and isolation requirements for electrons are described in Table~\ref{tab:HLTElectronCuts}.
The identification and isolation requirements for photons are described in Table~\ref{tab:PhotonPlusLeptonTriggerCuts}.
Triggers labeled "eff" are used for efficiency studies and triggers labeled "FR" are used for fake rate studies.}
    \vspace{5pt}
   \label{tab:triggers_util}
  \begin{center}
 {\small
  \begin{tabular} {l|l|l|c}
\hline
  Dataset & Trigger name & L1 seed & Description\\
 \hline \hline
  \multirow{8}{*}{DoubleElectron} & HLT\_Ele17\_CaloIdVT\_CaloIsoVT\_ &  L1\_SingleEG12  & $p_T>17,8~\GeVc$, eff\\
                                    & TrkIdT\_TrkIsoVT\_SC8\_Mass30 &                  & \\
    & HLT\_Ele32\_CaloIdL\_CaloIsoVL\_SC17 & L1\_SingleEG20 & $p_T>32,17~\GeVc$, eff\\
    & HLT\_Ele8\_CaloIdL\_CaloIsoVL             & L1\_SingleEG5                 & $p_T>8~\GeVc$, FR \\
    & HLT\_Ele17\_CaloIdL\_CaloIsoVL            & L1\_SingleEG12                & $p_T>8~\GeVc$, FR\\
    & HLT\_Ele8\_CaloIdL\_CaloIsoVL\_Jet40      & L1\_SingleEG5      & $p_T>8~\GeVc$, FR\\
  \multirow{2}{*}{ } & HLT\_Photon20\_CaloIdVT\_IsoT\_ &  L1\_SingleEG12  & $p_T>8~\GeVc$, FR\\
                     & Ele8\_CaloIdL\_CaloIsoVL        &                  & \\
    \hline \hline
  \multirow{2}{*}{SingleMuon}     & HLT\_Mu8                                  &  L1\_SingleMu3                & $p_T>8~\GeVc$, FR\\
    & HLT\_Mu15                                 &  L1\_SingleMu10               & $p_T>8~\GeVc$, FR\\    
\hline
  DoubleMuon  & HLT\_Mu8\_Jet40                           &  L1\_Mu3\_Jet20               & $p_T>8~\GeVc$, FR \\
\hline
  MuEG  & HLT\_Mu8\_Photon20\_CaloIdVT\_IsoT        &  L1\_Mu3\_EG5                 & $p_T>8~\GeVc$, FR\\
  \hline
  \end{tabular}
}
  \end{center}
\end{table}

Another set of specialised triggers are used to record events
enriched in fake electrons and muons for the measurement of jet induced backgrounds.
This is done using the fake rate method, which is described in detail in
Section~\ref{sec:bkg_fakes}.

We introduce four triggers each for the electron and muon fake rate measurements,
described in Table~\ref{tab:triggers_util} and labeled "FR".
Because these triggers are prescaled, the first two impose different $p_T$ thresholds 
to collect a sufficient sample over a large $p_T$ range.
The third trigger requires an additional jet with corrected $E_{T}>40$~GeV
to perform systematic studies on the fake rate measurement.
The fourth trigger is used to collect $\mathrm{\gamma+jet}$ events to
make a second independent measurement of the fake rate.
This trigger imposes tight cuts on the photon to ensure purity.
The photon selection requiriements are summarised in Table~\ref{tab:PhotonPlusLeptonTriggerCuts}.

\begin{table}[htb]
 \caption{Summary of requirements applied in the photon triggers used for this analysis. 
Values in parentheses correspond to the values used in the endcaps when different than in
the barrel. T=Tight, VT=Very Tight.}
    \vspace{5pt}
  \label{tab:PhotonPlusLeptonTriggerCuts}
  \centering
  \begin{tabular}{l||c}
    \hline
    name                        &  criterion \\
    \hline \hline
    \multirow{2}{*}{CaloId\_VT} & $\mathrm{H/E < 0.05 }$ \\
                                & $\sigma_{\eta\eta}\mathrm{< 0.011\;(0.01)}$ \\
    \hline \hline
    \multirow{3}{*}{Iso\_T}     & $\mathrm{ECalIso} < 5.0 + 0.012*E_{T} $ \\
                                & $\mathrm{HCalIso} < 3.0 + 0.005*E_{T} $ \\
                                & $\mathrm{TrkIso}  < 3.0 + 0.002*E_{T} $ \\
    \hline
  \end{tabular}
\end{table}

