In previous studies of the \WW\ final state the \dytt\ contribution
was considered small and well reproduced by the simulation since this
final state has a natural source of \met\ - neutrinos from \Tau\
decays. The fact that \met\ tends to be alligned with one of the
leptons is explored in the projected \met\ variable definition to
reduce the background rate.

With a rapid increase in the number of multiple interactions per bunch
crossing in 2011 data the situation is changing. Large amount of
pileup may lead to fake \met\ that is larger than the natural \met\
in \dytt\ events. Given that the \dytt\ cross-section is large and the
fact that we use a lower \met\ threshold in \emu\ final state we need
to make sure that this background is under control and reliably
estimated. Since the simulations that we have at the moment do not
reproduce the fake \met\ observed in data, we need a data-driven
method for the \dytt\ background estimation.

In order to estimate the \dytt\ background from data we can use \zee\
and \zmm\ events replacing electrons and muons with a simulated
$\tau\to l\nu_\tau\bar{\nu_e}$ decay - final state leptons will
represent the dilepton pair and neutrinos will modify obseved \met{}.

Two methods were used to implement this idea. The first one is based
on replacing muons from \zmm\ decays with simulated \Tau\ decays at
the event processing stage leading to a set of ``pseudo'' \ztt\ events
in AOD format. The alternative is to do the replacement at the final
ntuple level performing trivial Monte Carlo simulation of \Tau\ decays
on the fly. Neglecting masses of electron, muon and neutrinos
and \Tau\ polarization the angular distribution of electron or muon
originating from \Tau\ decay in \Tau\ rest frame is
\begin{equation}
        \frac{d\Gamma}{dx}\sim x^2(3-2x)
\end{equation}
where $x=2E_l/m_\tau$ - reduced energy, i.e. the energy of the lepton
over its maximum allowed energy~\cite{pdg}.

Both methods lead to similar results. We see a factor of 4
larger \dytt\ rate in data compared with Monte Carlo in the 0-jet case
and a factor of 2 in the 1-jet case. 

Overall this background quickly becomes negligible with tighter
\mt\ and the trailing lepton momentum cuts.
