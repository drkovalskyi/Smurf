Because the production cross-section is substantially higher than the
$\WW$ cross-section, top backgrounds pose a significant challenge.
To reduce the top background, we introduce two top tagging methods.
Both methods rely on the fact that top quarks decay to $Wb$ with
almost certainty.

The first method vetoes events
containing soft muons from the $b$-quark decays.
The requirements used to select soft muons are:

\begin{itemize}
    \item $\pt > 3$ GeV;
    \item Reconstructed as a TrackerMuon
    \item Meets $\mathrm{TMLastStationAngTight}$ muon id requirements
    \item The number of valid inner tracker hits $>10$
    \item The transverse impact parameter with respect to the Primary Vertex, $|d_{0}| < 0.2$~cm,
    \item The longitudinal impact parameter with respect to the Primary Vertex $|d_{z}| <0.2$~cm. This 
    requirement has been loosened with respect to~\cite{HWW2011};
    \item Non-isolated $({\rm{Iso}_{Total}}/{\pt}~>~0.1)$ if $\pt>20\GeV$.
\end{itemize}

The second method uses standard $b$-jet tagging~\cite{HWW2011}.
In this method, events containing jets tagged with
 the $\mathrm{TrkCountingHighEff}$~\cite{btag} algorithm with
a discriminator value of greater than 2.1 are vetoed.
The algorithm is applied to jets with the same definition as Section \ref{sec:sel_jets},
with the exception that we consider jets with $\Et ~>~10~\GeV$, and we require 
$\frac{|\sum^i~d_z^i (\pt^i)^2|}{\sum^i~(\pt^i)^2}<2$, where the sum runs over all 
tracks that belongs to each jet. These two requirements are different with respect to~\cite{HWW2011}. 
By using the expected tagging efficiency for the two methods,
it is possible to estimate the residual top background after the vetoes
have been applied. This is described in detail in Section~\ref{sec:bkg_top}.
