The standard model (SM) of particle physics successfully describes the
majority of high-energy experimental data~\cite{pdg}. One of the key
remaining questions is the origin of the masses of $\W$~and
$\Z$~bosons.  In the SM in its simplest implementation it is
attributed to the spontaneous breaking of electroweak symmetry caused
by a new scalar field~\cite{Higgs1, Higgs2, Higgs3}. The existence of
the associated field quantum, the Higgs boson, has yet to be
experimentally confirmed.

The $\Hi\to\WW$ channel is particularly sensitive for Higgs boson
searches in the intermediate mass range
($120-200~\GeVcc$)~\cite{dittmar}. This document describes the search
for the Higgs boson in the $\Hi\To\WW \To \Lep\Lprime\Nu\Nubar$
channel, for Higgs boson masses in the range of $115-600~\GeVcc$ using
full 2011 dataset.
    
%% The main analysis strategy is to select events with two opposite
%% charged leptons, large missing energy and little jet activity. The two
%% leptons are required to be isolated electrons or muons, of moderately
%% high transverse momenta ($\pt$) and with a small opening angle in the
%% transverse plane.
The study is based on the analysis peformed with the early 2011
data~\cite{HWW2011}. Several modifications and improvements have been
added, both to cope with the more difficult conditions due to the
higher instantaneous luminosity regime in 2011, and also to extend the
sensitivity.

Given the importance of this analysis all steps of the analysis were
reproduced by at least two different analyzers using different tools
and frameworks starting from the AOD level. This gives us confidence
in the results of this complicated analysis.

If you are already familiar with the analysis you may want to skip the
event selection (Sec.~\ref{sec:selection}), signal extraction
(Sec.~\ref{sec:signal_selection}) and background estimation
techniques (Sec.~\ref{sec:backgrounds}). Here is a list of
important changes with respect to the summer 2011 analysis:
\begin{itemize}
\item 
New electron identification selection based on multivariate techniques
that allows to reduce fake elector rate by a factor of two at the same
signal efficiency~\cite{MVAElId}.
\item 
Re-optimized top tagging selection to reduce the dependence on
pile-up (Sec.~\vref{sec:sel_toptag}).
\item 
Introduced a pileup dependent selection requirement for $\met$ in
same-flavor final states (Sec.~\vref{sec:sel_met}).
\item 
Revised top background estimation procedure to take into account NLO
effects of \tw\ production (Sec.~\vref{sec:bkg_top}).
\item 
Refined $\dyll$ background estimation procedure (Sec.~\vref{sec:bkg_dy}).
\item 
Data-driven \dytt\ background estimation (Sec.~\vref{sec:bkg_dytt}).
\item 
Re-optimized cut-based analysis for better sensitivy at low mass
(trailing lepton $\pt>15$~\GeV\ for same-flavor events and di-lepton
$\pt>45$~\GeV{}).
\item 
Detail shape analysis systematic~\cite{MVASyst}.
\item 
Measured in data $\wgamma^*$ cross-section
(Sec.~\vref{sec:bkg_wgammastar}).
\end{itemize}

Since the last update of the analysis, total amount of usable data
increased by a factor of 3 (4.7~\ifb{}). With 4~\ifb\ of data
available we performed a study of the detector performance under the
new machine conditions. In this study we compared results for two
independent datasets corresponding to Run2011A and Run2011B data
taking periods. This also allowed us to review and refine our
background estimation techniques. The details of the study can be
found in Appendix~\vref{app:a_vs_b}.

After a detail review of the pile conditions
(Sec.~\ref{sec:pileupReweighting}), efficiency measurements
(Sec.~\ref{sec:alleff}) and general description of the systematic
uncertainties (Sec.~\ref{sec:systematics}) we show full 2011 data
results (Sec.~\vref{sec:dataresults}). In that section we summarize
all the background estimations and the expected background yields for
all Higgs mass hypothesis. At the end of the section the final results
for the Higgs search are shown
(Sec.~\vref{sec:search_results}). Summary of the analysis can be found
in Section.~\vref{sec:summary}.
