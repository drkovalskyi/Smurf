The standard model (SM) of particle physics successfully describes the
majority of high-energy experimental data~\cite{pdg}. One of the key
remaining questions is the origin of the masses of $\W$~and
$\Z$~bosons.  In the SM in its simplest implementation it is
attributed to the spontaneous breaking of electroweak symmetry caused
by a new scalar field~\cite{Higgs1, Higgs2, Higgs3}. The existence of
the associated field quantum, the Higgs boson, has yet to be
experimentally confirmed.

The $\Hi\to\WW$ channel is particularly sensitive for Higgs boson
searches in the intermediate mass range
($120-200~\GeVcc$)~\cite{dittmar}. This document describes the search
for the Higgs boson in the $\Hi\To\WW \To \Lep\Lprime\Nu\Nubar$
channel, for Higgs boson masses in the range of $115-600~\GeVcc$ using
full 2011 dataset.
    
The main analysis strategy is to select events with two opposite
charged leptons, large missing energy and little jet activity. The two
leptons are required to be isolated electrons or muons, of moderately
high transverse momenta ($\pt$) and with a small opening angle in the
transverse plane. The analysis is based on the analysis peformed with
the early 2011 data~\cite{HWW2011}. Several modifications and
improvements have been added, both to cope with the more difficult
conditions due to the higher instantaneous luminosity regime in 2011,
and also to extend the sensitivity.
\begin{itemize}
\item Revised top background estimation procedure to take into account NLO effects of \tw\ production.
\item New electron identification selection based on multivariate techniques.
\item Re-optimized top tagging selection to reduce the dependence on pile-up.
\item Re-optimized $\met$ requirement to reduce the $\dyll$ contamination.
\item Data-driven \dytt\ background estimation.
\item Re-optimized cut-based analysis for better sensitivy at low mass.
\item Detail shape analysis systematic.
\end{itemize}
