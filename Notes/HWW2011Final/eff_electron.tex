
The electron selection efficiency can be factorised into two contributions,
the efficiency from the electron reconstruction and from the additional
analysis selections that are described in Section~\ref{sec:sel_electrons}.

The electron reconstruction efficiency is defined as the efficiency for a
supercluster to be matched to a reconstructed ECAL driven GSF electron.
%The data to simulation scale factor was measured for the W and Z cross-section
%analysis~\cite{VBTFCrossSectionNote}~\cite{ref:tagprobe_mit_w},
%and found to be consistent with $1.0$ with a total uncertainty of
%$1.3\%$ and $1.5\%$ for the barrel and endcap, respectively.
The data to simulation scale factor was measure by the Egamma POG binned in
$p_T$ and $\eta$~\cite{ref:egamma_eff_gsf}. From these studies, we take an overall scale factor of
$0.99$ with an uncertainty of $2.0\%$.

We thus measure the efficiency of our offline analysis selection 
with respect to a reconstructed ECAL driven GSF electron denominator. 
The efficiency measurement results are tabulated in detail in Appendix 
\ref{app:efficiency_studies_electron}. For electrons with $p_{T}$ 
above $20$ \GeV\, the efficiency is roughly $80\%$ for the barrel and 
$65\%$ for the endcap, while for electrons with $p_{T}$ below $20$ \GeV\
the efficiency is about $40\%$ for the barrel and $20\%$ for the endcap.

The Monte Carlo to data scale factors for the electron selection efficiency are
on average near $1$ for all electrons except endcap electrons with $p_{T}$ 
below $20$ \GeV\ for which the scale factor is roughly $1.1$. There is a significant 
decrease from the Run2011A period to the Run2011B period
in signal efficiency for electrons with $p_{T}$ below $20$ \GeV due to the 
effect of pileup on the isolation requirement. A corresponding but smaller decrease 
in the Monte Carlo to data scale factor is observed. 


