The sensitivity of Higgs boson searches at CMS can be significantly
enhanced by optimal use of available information to discriminate the
Standard Model Higgs decays from the background processes. In most
cases simple cut-based selection requirements allow to achieve close
to optimal signal and background separation as long as the
discriminating variables have a sharp transition from a signal
dominated to a background dominated regions. In a number of cases,
such as the Higgs boson search in $\WW \to 2\ell2\nu$ and $\ZZ \to
2\ell2\nu$ final states, there is no good way to separate EWK $\WW$
and $\ZZ$ background processes from the signal ones. All the
discriminating variables, such as angular and kinematic distributions
are weak discriminators that only allow good background rejection at
the cost of major signal efficiency loss. In such cases splitting
events in categories with different signal-to-background ratios may
allow for more optimal use of available information than a simple cut.

A shape analysis is an analysis where statistical interpretation of
results is based on discrete (categories) or continious (pdfs) event
classifiers rather than simple orthogonal cuts. Such an analysis
relies on a good understanding of signal and background distributions
of the discriminating variables and it typically has more complicated
systematic uncertainties that need to be properly evaluated.

In practice it is more convinient to use a single discriminating
variable for the shape analysis. Therefore, in a case of multiple
weakly discriminating variables it is advisable to combine them into a
single one using some multi-variate analysis (MVA) technique such as
Likelihood, Neural Net (NN) or Boosted Decision Tree (BDT).

In this note we describes implementation and treatment of systematic
uncertainties of the multivariate shape analysis for Higgs boson
searches in the $\WW \to 2\ell2\nu$ and $\ZZ \to 2\ell2\nu$ final
states.

