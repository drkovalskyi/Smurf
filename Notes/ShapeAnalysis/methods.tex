Bayesian, CL$_S$-LHC, nuisance parameters, p-values etc 

\subsection{Tools}
Upper limits are derived on the product of the Higgs boson
production cross section and the $\Hi \to \WW (\Z\Z)$ branching fraction,
$\sigma_{\rm{H}} \times $BR($\Hi \to \WW (\Z\Z))$, with respect
to the SM expectation, i.e. $\sigma^{95\%}/\sigma^{SM}$. Two different
statistical methods are used, both using the same likelihood function
from the expected number of observed events modeled as a Poisson
random variable whose mean value is the sum of the contributions from
signal and background processes. The first method, known as $CL_{s}$,
is based on the hybrid Frequentist-Bayesian approach~\cite{cousins},
while the second one is based on Bayesian inference~\cite{bayesian}.
Both methods account for systematic uncertainties. Although not
identical, the upper limits obtained from both methods are similar. 
To perform the computation of the limits, the software packages
\texttt{RooStats}~\cite{rootstat} and \texttt{LandS}~\cite{lands} have 
been used. Whenever the same result was computed with both packages, the
results were found to agree.

\subsection{Systematic Uncertainties in Shape Analyses} 

There are three different ways to account for systematic uncertainties for a 
given source:
\begin{itemize}
  \item {\bf normalization uncertanty} - account only for the overall
    normalization assuming that the shape is perfectly known;
  \item {\bf statistical shape uncertainties} - account for limited
    number of events available for the shape extraction;
  \item {\bf shape variation uncertainties} - account for uncertainty
    on the shape itself.
\end{itemize}

Normalization uncertanties are the most straightforward to treat. They
are identical to those used in a cut-based analysis. To simplify the
analysis in some cases, such as background contributions with
large normalization uncertainties, it can be used as the only source of
systematic uncertainty ignoring the shape variation.

Statistical uncertainties on shape extraction are often
negligible. Only if the sample that is used for shape extraction has
small number of events this effect may become sizable. At the moment
only one of the two official tools used in the Higgs group supports
this type of uncertainty. There are a few ways to address this
issue. First we can use the tool that supports this type of
uncertainty (LandS) to estimate the impact of this effect on final
results. If it is small it can be ignored. If the effect is not
negligible we can split the sample in a number of categories and
perform multiple ``cut-based'' analyses. Both tools support that
option, but it is a bit complicated and should be used only if really
necessary.

Shape variation uncertainties are implemented using three shapes:
nominal, ``up''-alternative and ``down''-alternative. A single
nuisance parameter is used for all bins simultaneously.
