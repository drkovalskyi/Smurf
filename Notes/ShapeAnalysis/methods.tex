Upper limits are derived on the product of the Higgs boson production
cross section and the $\Hi \to \WW (\Z\Z)$ branching fraction,
$\sigma_{\rm{H}} \times $BR($\Hi \to \WW (\Z\Z))$, with respect to the
SM expectation, i.e. $\sigma^{95\%}/\sigma^{SM}$. Two different
statistical methods are used to report results. The first method,
known as $CL_{s}$, is the modified frequentist
approach~\cite{cls1,cls2}, while the second one is based on Bayesian
inference~\cite{bayesian}.

To make results easier to compare, CMS and ATLAS collaborations have
agreed on using the $CL_s$ method with the LHC type
statistic~\cite{cls_lhc} to report Higgs search results. The
likelihood function is defined as:
\begin{eqnarray}
  L(\rm{data}|\mu,\theta)&=&\rm{Poisson}(\rm{data}|\mu\cdot s(\theta)+b(\theta))\cdot p(\tilde{\theta}|\theta) \nonumber\\
 &=&\prod_i\frac{(\mu s_i+b_i)^{n_i}}{n_i!}e^{-\mu s_i-b_i}\cdot p(\tilde{\theta}|\theta)
\label{eq:likelihood}
\end{eqnarray}
where $\mu$ is the signal strength modifier which is often reported in
the upper limit results as a ratio of the cross-section upper limit
over the standard model cross-section and $\theta$ represents a full
set of nuisance parameters that are used to incorporate systematic
uncertainties. The test statistic for the method is defined as a
likelihood ratio:
\begin{equation}
\tilde{q_\mu}=-2\log\frac{L(\rm{data}|\mu,\hat\theta_\mu)}{L(\rm{data}|\hat\mu,\hat\theta)}
\end{equation}
where the numerator corresponds to the maximum likelihood for given
``data'' and $\mu$ profiling over the nuisance parameters and the
denominator corresponds to the maximum likelihood for given ``data''
profiling over the nuisance parameters and $\mu$. This test statistic
differs from the ones used at LEP (no profiling of systematic errors)
and at Tevatron (the denominator likelihood uses $\mu=0$ and only
systematic errors are profiled).

In order to perform hypothesis testing we need to determine the
probability distribution functions (PDFs) for the test statistic for
signal plus background and background only hypothesis keeping the
nuisance parameters fixed to the observed values. Based on these PDFs
we define:
\begin{equation}
CL_s(\mu)=\frac{p_\mu}{p_0}
\end{equation}
where $p_\mu$ is the probability to observe a test statistic larger or
equal to the actual observation for a given $\mu$ for
signal+background hypothesis, and $p_0$ is that of for background only
hypothesis. If $\mu=1$ and $CL_s<\alpha$ we say that the signal
hypothesis is rejected at $(1-\alpha)$ Confidence Level (C.L.).

The second method (Bayesian) is based on interpreting the likelihood
defined in Equation~\ref{eq:likelihood} as a probability distribution
function with a flat prior for the signal strength and a set of pdfs
for nuisance parameters, which are often approximated with the
log-normal distribution. Integrating over the nuisance parameters we
find the upper limit for the signal strength.

The results obtained from both methods may differ but in most cases
they are similar. To perform the computation of the limits, the
software packages
\texttt{RooStats}~\cite{rootstat} and \texttt{LandS}~\cite{lands} have 
been used.

\subsection{Systematic Uncertainties in Shape Analyses} 

There are three different ways to account for systematic uncertainties for a 
given source:
\begin{itemize}
  \item {\bf normalization uncertanty} - account only for the overall
    normalization assuming that the shape is perfectly known;
  \item {\bf statistical shape uncertainties} - account for limited
    number of events available for the shape extraction;
  \item {\bf shape variation uncertainties} - account for uncertainty
    on the shape itself.
\end{itemize}

Normalization uncertanties are the most straightforward to treat. They
are identical to those used in a cut-based analysis. To simplify the
analysis in some cases, such as background contributions with
large normalization uncertainties, it can be used as the only source of
systematic uncertainty ignoring the shape variation.

Statistical uncertainties on shape extraction are often
negligible. Only if the sample that is used for shape extraction has
small number of events this effect may become sizable. At the moment
only one of the two official tools used in the Higgs group supports
this type of uncertainty. There are a few ways to address this
issue. First we can use the tool that supports this type of
uncertainty (LandS) to estimate the impact of this effect on final
results. If it is small it can be ignored. If the effect is not
negligible we can split the sample in a number of categories and
perform multiple ``cut-based'' analyses. Both tools support that
option, but it is a bit complicated and should be used only if really
necessary.

Shape variation uncertainties are implemented using three shapes:
nominal, ``up''-alternative and ``down''-alternative. A single
nuisance parameter is used for all bins simultaneously.
