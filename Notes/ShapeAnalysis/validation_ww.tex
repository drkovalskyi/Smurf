To validate the procedures discussed in the previous section, detailed studies 
have been carried out using a data sample of $\mathcal{L}~=~1.55~\pm~0.07~\ifb$,
corresponding to the available dataset for the Lepton-Photon 2011 conference. We
use simulated events from the newer Summer11 CMSSW\_4\_2\_X production, and
therefore the whole analysis chain was re-evaluated. Nevertheless, small changes
were observed with respect to the results obtained earlier.

\subsubsection{Shape vs. Counting Analyses}
A first interesting study is to compare the improvement in the performance by
using the shape of a discriminant variable with respect to a simple
cut-and-count analysis. There is no single very powerful distribution, like
$m_{\gamma\gamma}$ for the $\Hi \to \gamma\gamma$ case, in this dilepton
final state. Nevertheless, we want to show that the major gain is coming from the
use of the shape of a discriminant variable, and not just by the use of the
correlations among different variables in a multivariate classifier, e.g. Boost
Decision Tree (BDT).

Table~\ref{tab:mvaseveral} shows the median expected cross section ratio 
limits for several approaches as a function of the Higgs mass. First, we see
that the BDT shape analysis is always about 30-40\% better than the simple
cut-based approach~\footnote{The systematic uncertainties related to the
shape variation of the multivariate classifier are not considered here, the full
comparision will be shown in the following section.}. 
Second, we see that a multivariate likelihood approach, 
where the correlations among the different variables are not used, gives a
degradation in the performance of about 10\% only with respect to the BDT
approach. Finally, we see that the use of single discriminant variables, like 
either $\mll$ or $\delphill$, gives a performance comparable to the likelihood
approach, and always better than the cut-based analysis. Therefore, the 
main improvement comes from the use of the shape of a discriminant variable, 
not by the correlations among the variables of the multivariate BDT approach.

\begin{table}[!ht]
  \begin{center}
 {\small
  \begin{tabular} {|c|c|c|c|c|c|c|c|c|}
  \hline
mass       & BDT  & likelihood  & cut-based &  \multicolumn{5}{c|}{single variable} \\
$[\GeVcc]$ &      &             &           &  $\mll$  & $m_{T}$  & $\delphill$  & $\pt^{\ell\ell}$  & $\pt^{leading}$ \\
  \hline
115 & 3.30 & 3.70 & 4.74 & 3.69 &  5.97 & 4.92 &  6.27 & 6.41 \\
120 & 2.03 & 2.27 & 2.82 & 2.21 &  3.42 & 2.95 &  3.57 & 3.65 \\
130 & 0.97 & 1.11 & 1.30 & 1.06 &  1.61 & 1.35 &  1.69 & 1.63 \\
140 & 0.62 & 0.70 & 0.78 & 0.69 &  0.99 & 0.83 &  1.04 & 0.98 \\
150 & 0.40 & 0.45 & 0.53 & 0.50 &  0.64 & 0.57 &  0.69 & 0.64 \\
160 & 0.22 & 0.26 & 0.30 & 0.35 &  0.42 & 0.37 &  0.44 & 0.42 \\
170 & 0.25 & 0.28 & 0.32 & 0.41 &  0.40 & 0.40 &  0.41 & 0.46 \\
180 & 0.39 & 0.44 & 0.50 & 0.63 &  0.52 & 0.60 &  0.55 & 0.63 \\
190 & 0.59 & 0.68 & 0.72 & 0.98 &  0.77 & 0.94 &  0.80 & 0.91 \\
200 & 0.80 & 0.90 & 0.97 & 1.29 &  1.00 & 1.30 &  1.02 & 1.13 \\
250 & 1.34 & 1.45 & 1.87 & 1.71 &  1.99 & 2.25 &  2.05 & 1.66 \\
300 & 1.51 & 1.60 & 2.19 & 1.84 &  2.64 & 2.21 &  2.77 & 1.77 \\
350 & 1.40 & 1.47 & 2.09 & 1.75 &  2.63 & 1.94 &  2.84 & 1.60 \\
400 & 1.52 & 1.65 & 2.29 & 1.93 &  2.98 & 2.12 &  3.29 & 1.72 \\
450 & 2.14 & 2.30 & 3.01 & 2.71 &  3.93 & 3.01 &  4.52 & 2.35 \\
500 & 3.15 & 3.43 & 4.44 & 4.00 &  5.54 & 4.52 &  6.27 & 3.46 \\
550 & 4.53 & 4.96 & 6.15 & 5.54 &  7.85 & 6.44 &  9.04 & 4.72 \\
600 & 6.60 & 7.14 & 9.07 & 8.03 & 10.96 & 9.51 & 12.76 & 6.82 \\
  \hline
  \end{tabular}
  \caption{Median expected cross section ratio limits for several 
  approaches as a function of the Higgs mass.}
  \label{tab:mvaseveral}
  }
  \end{center}
\end{table}

\subsubsection{Inclusion of Shape Uncertainties}
The median expected cross section ratio limits as a function 
of the Higgs mass, together with the 1/2-$\sigma$ uncertainty bands, for the 
Lepton-Photon 2011 dataset, after the inclusion of all the shape uncertainties 
are summarized in this section. Table~\ref{tab:mva_shapewithwithout} shows the effect 
on the results by including such uncertainties, while Table~\ref{tab:mva_shapevscuts} 
compares them with the cut-based approach. In summary, we see that the median expected 
limit is barely unaffected by the inclusion of these shape uncertainties, while 
the uncertainty bands get wider by $\sim$10-15\%. Therefore, the $\sim$30\% 
increase in the performance with respect to the cut-based analysis remains after the 
inclusion of such effects.

In order to study the effect of some uncertainties, we exclude some of them and 
recompute the limits. Tables~\ref{tab:mva_shapewithwithoutww},
~\ref{tab:mva_shapewithwithoutstat} and~\ref{tab:mva_shapewithwithoutlepjes} compare 
the default analysis with the one excluding the $\WW$ shape uncertainties, the statistical 
uncertainties and the lepton and jet energy shape uncertainties, respectively. All 
of them show a sizable, but not dominant, effect. Therefore we should consider them all. 
Finally, we compare in Table~\ref{tab:mva_shapealtstat} the default analysis with 
the one using an alternative approach for the evaluation of the statistical uncertainty, as 
described in Section~\ref{sec:systematic_ww}. No large degradation on the expected limits 
is seen.

\begin{table}[!ht]
\begin{center}
{\normalsize
\begin{tabular}{|l|c|c|c|c|c|c|}
\hline
      &  \multicolumn{3}{c|}{without shape uncertainties} &\multicolumn{3}{c|}{with shape uncertainties} \\
\hline
Mass  &  Median      &     68\% C.L. band &  95\% C.L. band &  Median	   &	 68\% C.L. band &  95\% C.L. band\\
      &  Expected    &                    &                 &  Expected    &			&		 \\
\hline
115 &  3.3 & [2.3, 4.8] & [1.7, 6.8]  &  3.3 & [2.1, 5.1] & [1.5, 7.6] \\
120 &  2.0 & [1.4, 3.0] & [1.0, 4.1]  &  2.0 & [1.3, 3.1] & [0.9, 4.5] \\
130 &  0.9 & [0.7, 1.4] & [0.5, 1.9]  &  0.9 & [0.6, 1.5] & [0.4, 2.2] \\
140 &  0.6 & [0.4, 0.9] & [0.3, 1.2]  &  0.6 & [0.4, 0.9] & [0.2, 1.3] \\
150 &  0.4 & [0.3, 0.6] & [0.2, 0.8]  &  0.4 & [0.2, 0.6] & [0.2, 0.8] \\
160 &  0.3 & [0.2, 0.3] & [0.1, 0.5]  &  0.2 & [0.1, 0.3] & [0.1, 0.5] \\
170 &  0.3 & [0.2, 0.4] & [0.1, 0.5]  &  0.2 & [0.1, 0.4] & [0.1, 0.6] \\
180 &  0.4 & [0.3, 0.6] & [0.2, 0.8]  &  0.4 & [0.2, 0.6] & [0.1, 0.9] \\
190 &  0.6 & [0.4, 0.9] & [0.3, 1.2]  &  0.5 & [0.3, 0.9] & [0.2, 1.3] \\
200 &  0.8 & [0.5, 1.2] & [0.4, 1.7]  &  0.7 & [0.4, 1.2] & [0.3, 1.9] \\
250 &  1.3 & [0.9, 2.0] & [0.7, 2.8]  &  1.4 & [0.9, 2.2] & [0.6, 3.2] \\
300 &  1.5 & [1.1, 2.2] & [0.8, 3.1]  &  1.5 & [1.0, 2.3] & [0.7, 3.5] \\
350 &  1.4 & [1.0, 2.0] & [0.7, 2.8]  &  1.4 & [0.9, 2.2] & [0.6, 3.1] \\
400 &  1.5 & [1.1, 2.1] & [0.8, 2.9]  &  1.5 & [1.0, 2.3] & [0.7, 3.3] \\
450 &  2.1 & [1.4, 3.0] & [1.1, 4.2]  &  2.0 & [1.4, 3.1] & [1.0, 4.5] \\
500 &  3.0 & [2.1, 4.2] & [1.6, 5.9]  &  3.0 & [2.0, 4.5] & [1.4, 6.5] \\
550 &  4.1 & [2.9, 5.9] & [2.2, 8.3]  &  4.1 & [2.8, 6.2] & [2.0, 8.9] \\
600 &  5.6 & [4.0, 8.0] & [3.0, 11.1] &  5.6 & [3.9, 8.4] & [2.8, 12.2]\\
\hline
\end{tabular}
}
\caption{Comparison of the median expected cross section ratio limits as a function 
of the Higgs mass, together with the 1/2-$\sigma$ uncertainty bands, without and with the 
shape uncertainties considered.}
\label{tab:mva_shapewithwithout}
\end{center}
\end{table}

\begin{table}[!ht]
\begin{center}
{\normalsize
\begin{tabular}{|l|c|c|c|c|c|c|}
\hline
      &  \multicolumn{3}{c|}{cut-based analysis} &\multicolumn{3}{c|}{BDT shape analysis} \\
\hline
Mass  &  Median      &     68\% C.L. band &  95\% C.L. band &  Median	   &	 68\% C.L. band &  95\% C.L. band\\
      &  Expected    &                    &                 &  Expected    &			&		 \\
\hline
115 &  4.7 & [3.0, 7.3] & [1.8, 11.0] &  3.3 & [2.1, 5.1] & [1.5, 7.6] \\
120 &  2.8 & [1.8, 4.3] & [1.1, 6.3]  &  2.0 & [1.3, 3.1] & [0.9, 4.5] \\
130 &  1.3 & [0.9, 1.9] & [0.6, 2.8]  &  0.9 & [0.6, 1.5] & [0.4, 2.2] \\
140 &  0.8 & [0.5, 1.2] & [0.4, 1.6]  &  0.6 & [0.4, 0.9] & [0.2, 1.3] \\
150 &  0.5 & [0.4, 0.8] & [0.2, 1.1]  &  0.4 & [0.2, 0.6] & [0.2, 0.8] \\
160 &  0.3 & [0.2, 0.4] & [0.1, 0.7]  &  0.2 & [0.1, 0.3] & [0.1, 0.5] \\
170 &  0.3 & [0.2, 0.5] & [0.2, 0.7]  &  0.2 & [0.1, 0.4] & [0.1, 0.6] \\
180 &  0.5 & [0.3, 0.8] & [0.2, 1.1]  &  0.4 & [0.2, 0.6] & [0.1, 0.9] \\
190 &  0.7 & [0.5, 1.1] & [0.4, 1.5]  &  0.5 & [0.3, 0.9] & [0.2, 1.3] \\
200 &  1.0 & [0.7, 1.4] & [0.5, 2.0]  &  0.7 & [0.4, 1.2] & [0.3, 1.9] \\
250 &  1.9 & [1.3, 2.7] & [0.9, 3.8]  &  1.4 & [0.9, 2.2] & [0.6, 3.2] \\
300 &  2.2 & [1.5, 3.1] & [1.1, 4.4]  &  1.5 & [1.0, 2.3] & [0.7, 3.5] \\
350 &  2.1 & [1.4, 3.0] & [1.0, 4.2]  &  1.4 & [0.9, 2.2] & [0.6, 3.1] \\
400 &  2.3 & [1.6, 3.3] & [1.1, 4.6]  &  1.5 & [1.0, 2.3] & [0.7, 3.3] \\
450 &  2.9 & [2.0, 4.1] & [1.5, 6.0]  &  2.0 & [1.4, 3.1] & [1.0, 4.5] \\
500 &  4.1 & [2.9, 6.0] & [2.1, 8.4]  &  3.0 & [2.0, 4.5] & [1.4, 6.5] \\
550 &  5.5 & [3.9, 8.0] & [2.8, 11.3] &  4.1 & [2.8, 6.2] & [2.0, 8.9] \\
600 &  7.6 & [5.4, 11.1]& [4.0, 15.5] &  5.6 & [3.9, 8.4] & [2.8, 12.2]\\
\hline
\end{tabular}
}
\caption{Comparison of the median expected cross section ratio limits as a function 
of the Higgs mass, together with the 1/2-$\sigma$ uncertainty bands, for the cut-based analysis 
and the full BDT shape analysis.}
\label{tab:mva_shapevscuts}
\end{center}
\end{table}

\begin{table}[!ht]
\begin{center}
{\normalsize
\begin{tabular}{|l|c|c|c|c|c|c|}
\hline
      &  \multicolumn{3}{c|}{without $\WW$ shape uncertainties} &\multicolumn{3}{c|}{BDT shape analysis} \\
\hline
Mass  &  Median      &     68\% C.L. band &  95\% C.L. band &  Median	   &	 68\% C.L. band &  95\% C.L. band\\
      &  Expected    &                    &                 &  Expected    &			&		 \\
\hline
115 &  3.3 & [2.2, 5.1] & [1.5, 7.5]  &  3.3 & [2.1, 5.1] & [1.5, 7.6] \\
120 &  2.1 & [1.4, 3.1] & [0.9, 4.5]  &  2.0 & [1.3, 3.1] & [0.9, 4.5] \\
130 &  1.0 & [0.6, 1.4] & [0.4, 2.1]  &  0.9 & [0.6, 1.5] & [0.4, 2.2] \\
140 &  0.6 & [0.4, 0.9] & [0.3, 1.3]  &  0.6 & [0.4, 0.9] & [0.2, 1.3] \\
150 &  0.4 & [0.2, 0.6] & [0.2, 0.8]  &  0.4 & [0.2, 0.6] & [0.2, 0.8] \\
160 &  0.2 & [0.1, 0.3] & [0.1, 0.5]  &  0.2 & [0.1, 0.3] & [0.1, 0.5] \\
170 &  0.2 & [0.1, 0.4] & [0.1, 0.6]  &  0.2 & [0.1, 0.4] & [0.1, 0.6] \\
180 &  0.4 & [0.2, 0.6] & [0.1, 0.8]  &  0.4 & [0.2, 0.6] & [0.1, 0.9] \\
190 &  0.5 & [0.3, 0.9] & [0.2, 1.3]  &  0.5 & [0.3, 0.9] & [0.2, 1.3] \\
200 &  0.7 & [0.4, 1.2] & [0.3, 1.8]  &  0.7 & [0.4, 1.2] & [0.3, 1.9] \\
250 &  1.3 & [0.9, 2.0] & [0.6, 2.9]  &  1.4 & [0.9, 2.2] & [0.6, 3.2] \\
300 &  1.5 & [1.0, 2.3] & [0.7, 3.3]  &  1.5 & [1.0, 2.3] & [0.7, 3.5] \\
350 &  1.4 & [0.9, 2.1] & [0.6, 3.1]  &  1.4 & [0.9, 2.2] & [0.6, 3.1] \\
400 &  1.5 & [1.0, 2.3] & [0.7, 3.2]  &  1.5 & [1.0, 2.3] & [0.7, 3.3] \\
450 &  2.1 & [1.4, 3.1] & [1.0, 4.5]  &  2.0 & [1.4, 3.1] & [1.0, 4.5] \\
500 &  3.0 & [2.1, 4.5] & [1.5, 6.4]  &  3.0 & [2.0, 4.5] & [1.4, 6.5] \\
550 &  4.2 & [2.9, 6.1] & [2.1, 8.6]  &  4.1 & [2.8, 6.2] & [2.0, 8.9] \\
600 &  5.7 & [3.9, 8.3] & [2.8, 12.0] &  5.6 & [3.9, 8.4] & [2.8, 12.2]\\
\hline
\end{tabular}
}
\caption{Comparison of the median expected cross section ratio limits as a function 
of the Higgs mass, together with the 1/2-$\sigma$ uncertainty bands, without and with the 
$\WW$ shape uncertainties considered.}
\label{tab:mva_shapewithwithoutww}
\end{center}
\end{table}

\begin{table}[!ht]
\begin{center}
{\normalsize
\begin{tabular}{|l|c|c|c|c|c|c|}
\hline
      &  \multicolumn{3}{c|}{without statistical uncertainties} &\multicolumn{3}{c|}{BDT shape analysis} \\
\hline
Mass  &  Median      &     68\% C.L. band &  95\% C.L. band &  Median	   &	 68\% C.L. band &  95\% C.L. band\\
      &  Expected    &                    &                 &  Expected    &			&		 \\
\hline
115 &  3.3 & [2.2, 4.9] & [1.5, 7.1]  &  3.3 & [2.1, 5.1] & [1.5, 7.6] \\
120 &  2.0 & [1.3, 3.0] & [0.9, 4.4]  &  2.0 & [1.3, 3.1] & [0.9, 4.5] \\
130 &  0.9 & [0.6, 1.4] & [0.4, 2.1]  &  0.9 & [0.6, 1.5] & [0.4, 2.2] \\
140 &  0.6 & [0.4, 0.9] & [0.3, 1.3]  &  0.6 & [0.4, 0.9] & [0.2, 1.3] \\
150 &  0.4 & [0.2, 0.6] & [0.2, 0.8]  &  0.4 & [0.2, 0.6] & [0.2, 0.8] \\
160 &  0.2 & [0.1, 0.3] & [0.1, 0.5]  &  0.2 & [0.1, 0.3] & [0.1, 0.5] \\
170 &  0.2 & [0.1, 0.4] & [0.1, 0.5]  &  0.2 & [0.1, 0.4] & [0.1, 0.6] \\
180 &  0.4 & [0.2, 0.6] & [0.1, 0.9]  &  0.4 & [0.2, 0.6] & [0.1, 0.9] \\
190 &  0.5 & [0.3, 0.9] & [0.2, 1.4]  &  0.5 & [0.3, 0.9] & [0.2, 1.3] \\
200 &  0.7 & [0.4, 1.2] & [0.2, 1.9]  &  0.7 & [0.4, 1.2] & [0.3, 1.9] \\
250 &  1.4 & [0.9, 2.2] & [0.6, 3.3]  &  1.4 & [0.9, 2.2] & [0.6, 3.2] \\
300 &  1.5 & [1.0, 2.3] & [0.7, 3.4]  &  1.5 & [1.0, 2.3] & [0.7, 3.5] \\
350 &  1.4 & [0.9, 2.1] & [0.6, 3.0]  &  1.4 & [0.9, 2.2] & [0.6, 3.1] \\
400 &  1.5 & [1.0, 2.3] & [0.7, 3.3]  &  1.5 & [1.0, 2.3] & [0.7, 3.3] \\
450 &  2.1 & [1.4, 3.1] & [1.0, 4.5]  &  2.0 & [1.4, 3.1] & [1.0, 4.5] \\
500 &  3.0 & [2.0, 4.5] & [1.4, 6.5]  &  3.0 & [2.0, 4.5] & [1.4, 6.5] \\
550 &  4.1 & [2.8, 6.1] & [2.0, 8.8]  &  4.1 & [2.8, 6.2] & [2.0, 8.9] \\
600 &  5.6 & [3.8, 8.2] & [2.7, 11.8] &  5.6 & [3.9, 8.4] & [2.8, 12.2]\\
\hline
\end{tabular}
}
\caption{Comparison of the median expected cross section ratio limits as a function 
of the Higgs mass, together with the 1/2-$\sigma$ uncertainty bands, without and with the statistical 
shape uncertainties considered.}
\label{tab:mva_shapewithwithoutstat}
\end{center}
\end{table}

\begin{table}[!ht]
\begin{center}
{\normalsize
\begin{tabular}{|l|c|c|c|c|c|c|}
\hline
      &  \multicolumn{3}{c|}{without lepton/jet shape uncertainties} &\multicolumn{3}{c|}{BDT shape analysis} \\
\hline
Mass  &  Median      &     68\% C.L. band &  95\% C.L. band &  Median	   &	 68\% C.L. band &  95\% C.L. band\\
      &  Expected    &                    &                 &  Expected    &			&		 \\
\hline
115 &  3.3 & [2.2, 5.0] & [1.5, 7.3]  &  3.3 & [2.1, 5.1] & [1.5, 7.6] \\
120 &  2.0 & [1.3, 3.1] & [0.9, 4.5]  &  2.0 & [1.3, 3.1] & [0.9, 4.5] \\
130 &  1.0 & [0.6, 1.4] & [0.4, 2.1]  &  0.9 & [0.6, 1.5] & [0.4, 2.2] \\
140 &  0.6 & [0.4, 0.9] & [0.3, 1.3]  &  0.6 & [0.4, 0.9] & [0.2, 1.3] \\
150 &  0.4 & [0.3, 0.6] & [0.2, 0.9]  &  0.4 & [0.2, 0.6] & [0.2, 0.8] \\
160 &  0.2 & [0.1, 0.3] & [0.1, 0.5]  &  0.2 & [0.1, 0.3] & [0.1, 0.5] \\
170 &  0.2 & [0.2, 0.4] & [0.1, 0.5]  &  0.2 & [0.1, 0.4] & [0.1, 0.6] \\
180 &  0.4 & [0.2, 0.6] & [0.2, 0.8]  &  0.4 & [0.2, 0.6] & [0.1, 0.9] \\
190 &  0.5 & [0.4, 0.9] & [0.2, 1.3]  &  0.5 & [0.3, 0.9] & [0.2, 1.3] \\
200 &  0.7 & [0.5, 1.2] & [0.3, 1.8]  &  0.7 & [0.4, 1.2] & [0.3, 1.9] \\
250 &  1.4 & [0.9, 2.1] & [0.6, 3.2]  &  1.4 & [0.9, 2.2] & [0.6, 3.2] \\
300 &  1.5 & [1.0, 2.2] & [0.7, 3.2]  &  1.5 & [1.0, 2.3] & [0.7, 3.5] \\
350 &  1.3 & [0.9, 2.0] & [0.6, 3.0]  &  1.4 & [0.9, 2.2] & [0.6, 3.1] \\
400 &  1.4 & [1.0, 2.1] & [0.7, 3.2]  &  1.5 & [1.0, 2.3] & [0.7, 3.3] \\
450 &  2.0 & [1.4, 3.0] & [1.0, 4.1]  &  2.0 & [1.4, 3.1] & [1.0, 4.5] \\
500 &  2.9 & [2.0, 4.3] & [1.4, 6.1]  &  3.0 & [2.0, 4.5] & [1.4, 6.5] \\
550 &  4.0 & [2.8, 5.9] & [2.0, 8.5]  &  4.1 & [2.8, 6.2] & [2.0, 8.9] \\
600 &  5.6 & [3.8, 8.1] & [2.7, 11.6] &  5.6 & [3.9, 8.4] & [2.8, 12.2]\\
\hline
\end{tabular}
}
\caption{Comparison of the median expected cross section ratio limits as a function 
of the Higgs mass, together with the 1/2-$\sigma$ uncertainty bands, without and with the 
lepton and jet energy shape uncertainties considered.}
\label{tab:mva_shapewithwithoutlepjes}
\end{center}
\end{table}

\begin{table}[!ht]
\begin{center}
{\normalsize
\begin{tabular}{|l|c|c|c|c|c|c|}
\hline
      &  \multicolumn{3}{c|}{alternative statistical uncertainty} &\multicolumn{3}{c|}{BDT shape analysis} \\
\hline
Mass  &  Median      &     68\% C.L. band &  95\% C.L. band &  Median	   &	 68\% C.L. band &  95\% C.L. band\\
      &  Expected    &                    &                 &  Expected    &			&		 \\
\hline
115 &  3.4 & [2.2, 5.3] & [1.5, 7.7]  &  3.3 & [2.1, 5.1] & [1.5, 7.6] \\
120 &  2.1 & [1.3, 3.2] & [0.9, 4.6]  &  2.0 & [1.3, 3.1] & [0.9, 4.5] \\
130 &  1.0 & [0.6, 1.5] & [0.4, 2.2]  &  0.9 & [0.6, 1.5] & [0.4, 2.2] \\
140 &  0.6 & [0.4, 0.9] & [0.3, 1.4]  &  0.6 & [0.4, 0.9] & [0.2, 1.3] \\
150 &  0.4 & [0.2, 0.6] & [0.2, 0.9]  &  0.4 & [0.2, 0.6] & [0.2, 0.8] \\
160 &  0.2 & [0.1, 0.3] & [0.1, 0.5]  &  0.2 & [0.1, 0.3] & [0.1, 0.5] \\
170 &  0.2 & [0.1, 0.4] & [0.1, 0.5]  &  0.2 & [0.1, 0.4] & [0.1, 0.6] \\
180 &  0.4 & [0.2, 0.6] & [0.1, 0.9]  &  0.4 & [0.2, 0.6] & [0.1, 0.9] \\
190 &  0.5 & [0.3, 0.9] & [0.2, 1.4]  &  0.5 & [0.3, 0.9] & [0.2, 1.3] \\
200 &  0.7 & [0.4, 1.2] & [0.3, 1.9]  &  0.7 & [0.4, 1.2] & [0.3, 1.9] \\
250 &  1.4 & [0.9, 2.2] & [0.6, 3.2]  &  1.4 & [0.9, 2.2] & [0.6, 3.2] \\
300 &  1.5 & [1.0, 2.4] & [0.7, 3.5]  &  1.5 & [1.0, 2.3] & [0.7, 3.5] \\
350 &  1.4 & [0.9, 2.2] & [0.6, 3.2]  &  1.4 & [0.9, 2.2] & [0.6, 3.1] \\
400 &  1.5 & [1.0, 2.3] & [0.7, 3.4]  &  1.5 & [1.0, 2.3] & [0.7, 3.3] \\
450 &  2.1 & [1.4, 3.2] & [1.0, 4.6]  &  2.0 & [1.4, 3.1] & [1.0, 4.5] \\
500 &  3.0 & [2.0, 4.5] & [1.4, 6.5]  &  3.0 & [2.0, 4.5] & [1.4, 6.5] \\
550 &  4.2 & [2.8, 6.4] & [2.0, 9.2]  &  4.1 & [2.8, 6.2] & [2.0, 8.9] \\
600 &  5.7 & [3.8, 8.5] & [2.7, 12.6] &  5.6 & [3.9, 8.4] & [2.8, 12.2]\\
\hline
\end{tabular}
}
\caption{Comparison of the median expected cross section ratio limits as a function 
of the Higgs mass, together with the 1/2-$\sigma$ uncertainty bands, an alternative 
approach of the statistical uncertainty with respect to the standard analysis}
\label{tab:mva_shapealtstat}
\end{center}
\end{table}
