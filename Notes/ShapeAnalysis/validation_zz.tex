To validate the procedures discussed in the previous section, detailed studies 
have been carried out using simulated events from the Spring11 CMSSW\_4\_1\_X production.
The basic object selections are taken from the Ref~\cite{hzz}-~\cite{hzzlppas}. 
We evaluate the effects on the expected limits projected to $\mathcal{L}~=~2.1~\pm~0.1~\ifb$. 
The Top and \WW{} background are estimated using data-driven techinques documented in 
Ref.~\cite{hzz}. 
%The data to simulation scale factors for the Top and \WW{} background are derived using 
%a dataset corresponding to $\mathcal{L}~=~2.1~\pm~0.05~\ifb$. 
%The Drell-Yan background are derived using the photon jet method based on 
%$\mathcal{L}~=~2.1~\pm~0.05~\ifb$, which is then scaled to $5~\ifb$ assuming 
%the same systematic uncertainty. 
In all the test, we run 10000 toymc to estimate the  effects on the median expected cross-section ration limits and the 1/2-$\sigma$ 
uncertainty bands with a statistical precision about 0.01. 


\subsubsection{Shape vs. Counting Analyses}

As in the $\hww$ analysis, we compare the improvement in the performance using the shape analysis 
with respect to the simple cut-and-cut analysis, shown in 
Table~\ref{tab:mva_mtshapevscuts_hzz}-\ref{tab:mvashape_mevsbdt_hzz}. 

In the case of the $H\to ZZ$ analysis, using shape based on a single variation $m_T$ yield 
comparable performance comparing to using the shape based on BDT or Matrix Element method. 
The shape analysis based on the Matrix Element outputs brings additional $X\%$ improvement 
compared to the shape analysis based on $m_T$ for the low higgs mass hypothesis $mH=250\GeVcc$. 
Compared to the shape analysis based on $m_T$, the shape analysis based on the BDT outputs 
brings additional $X\%$ improvement in the intermediate higgs mass hypotheses 
$ 350<=~mH~<=~400\GeVcc$ and additional $X\%$ improvement in the higher mass hypotheses
$mH~>~400\GeVcc$. 

In the following validations, we use the $m_T$ based shape analysis as an example 
to demonstrate the effects due to the shape systematics. 

%%%%%%%%%%%%%%%%%%%%%%%%%%%%%
\begin{table}[!ht]
\begin{center}
{\normalsize
\begin{tabular}{|l|c|c|c|c|c|c|}
\hline
      &  \multicolumn{3}{c|}{Cut-and-Count Based Analysis} &\multicolumn{3}{c|}{Shape Analysis using $m_T$} \\
\hline
Mass  &  Median      &     68\% C.L. band &  95\% C.L. band &  Median	   &	 68\% C.L. band &  95\% C.L. band\\
      &  Expected    &                    &                 &  Expected    &			&		 \\
\hline
250 & 2.10 & [1.46, 3.03] & [1.07, 4.19] & 1.87 & [1.32, 2.66] & [0.97, 3.67] \\
300 & 1.49 & [1.04, 2.14] & [0.77, 3.00] & 1.29 & [0.91, 1.83] & [0.68, 2.54] \\
350 & 1.04 & [0.75, 1.54] & [0.56, 2.18] & 0.95 & [0.69, 1.36] & [0.51, 1.88] \\
400 & 1.03 & [0.77, 1.55] & [0.56, 2.17] & 0.99 & [0.71, 1.42] & [0.53, 1.97] \\
500 & 1.93 & [1.38, 2.91] & [1.09, 4.07] & 1.87 & [1.35, 2.68] & [1.03, 3.75] \\
600 & 3.93 & [2.90, 5.88] & [2.17, 8.87] & 4.31 & [3.09, 6.19] & [2.42, 8.83] \\
\hline
\end{tabular}
}
\caption{Comparison of the median expected cross section ratio limits as a function 
of the Higgs mass, together with the 1/2-$\sigma$ uncertainty bands between the cut-and-count 
analysis and the shape analysis using the transverse higgs mass. In this comparison, we do not include any systematics due to 
the shape variation. }
\label{tab:mva_mtshapevscuts_hzz}
\end{center}
%\end{table}
%%%%%%%%%%%%%%%%%%%%%%%%%%%%%
%%%%%%%%%%%%%%%%%%%%%%%%%%%%%
%\begin{table}[!ht]
\begin{center}
{\normalsize
\begin{tabular}{|l|c|c|c|c|c|c|}
\hline
      &  \multicolumn{3}{c|}{Shape Analysis using Matrix Element} &\multicolumn{3}{c|}{Shape Analysis using BDT} \\
\hline
Mass  &  Median      &     68\% C.L. band &  95\% C.L. band &  Median	   &	 68\% C.L. band &  95\% C.L. band\\
      &  Expected    &                    &                 &  Expected    &			&		 \\
\hline
250 & 1.19 & [0.84, 1.69] & [0.63, 2.34] & 1.62 & [1.17, 2.35] & [0.87, 3.09] \\
300 & 0.89 & [0.63, 1.25] & [0.46, 1.75] & 0.80 & [0.57, 1.16] & [0.42, 1.56] \\
350 & 0.65 & [0.46, 0.93] & [0.34, 1.31] & 0.45 & [0.32, 0.64] & [0.24, 0.92] \\
400 & 0.64 & [0.46, 0.92] & [0.34, 1.27] & 0.43 & [0.31, 0.62] & [0.23, 0.87]\\
500 & 1.08 & [0.78, 1.52] & [0.58, 2.13] & 0.82 & [0.59, 1.14] & [0.46, 1.67] \\
600 & 2.16 & [1.56, 3.07] & [1.18, 4.31] & 1.88 & [1.37, 2.69] & [1.04, 3.74]\\
\hline
\end{tabular}
}
\caption{\fixme {\bf still use the 5/fb based on 41XMC} 
Comparison of the median expected cross section ratio limits as a function 
of the Higgs mass, together with the 1/2-$\sigma$ uncertainty bands between the cut-and-count 
analysis and the shape analysis using the transverse higgs mass. 
In this comparison, we do not include any systematics due to the shape variation. }
\label{tab:mvashape_mevsbdt_hzz}
\end{center}
\end{table}
%%%%%%%%%%%%%%%%%%%%%%%%%%%%%

\subsubsection{Inclusion of Shape Uncertainties}

In this section, we document the shape systematics effects to the median expected cross section 
ratio limits as a function of the Higgs mass, together with the 1/2-$\sigma$ uncertainty bands. 
The following systematic variations are considered,
%%%%%%%%%%%%%%%%%%%%%%%%%%%%%
\begin{itemize}
\item {Statistic uncertainties in the template}
\item {QCD scale variations to the Higgs process}
\item {Top shape variations}
\item {WW shape variations}
\item {WZ shape variations}
\item {ZZ shape variations}.
\end{itemize}
%%%%%%%%%%%%%%%%%%%%%%%%%%%%%
The effects due to the lepton efficiency and scale variations are neglible and we assign 
the normalization uncertainties. 

Table~\ref{tab:mva_mtshapewithwithout_hzz} shows the effect on the results by including such uncertainties. 
Including the systematics for shape variations, we see the performance degrade by about 2\% compared 
to the results obtained without shape systematics. 
The largest effect is on the $m_H=250\GeVcc$ hypothesis mainly due to the $\WZ$ shape variation.
Table~\ref{tab:mva_mtshapevscuts_withshapevar_hzz} compares the cut-based results and the one using 
shape analysis including all systematics due to shape variation. 
Overall we see on average 7-11\% gain in the performance 
in the intermediate mass range between 300 and 400 $\GeVcc$. The 
performance at the $mH=500\GeVcc$ is consistent with the cut-based approach. 
The performance at the $mH=600\GeVcc$ is worse than teh cut-based approach. This 
is mainly because of the loose pre-selections. 
To understand the effects of the individual source of the uncertainties, 
we also compare the results by adding each source progressively, shown in Table~\ref{tab:mva_mtshape_detail}. 
Among all the shape systematics, the statistical uncertainty on the template is the leading effect. 



%%%%%%%%%%%%%%%%%%%%%%%%%%%%%
\begin{table}[!ht]
\begin{center}
{\normalsize
\begin{tabular}{|l|c|c|c|c|c|c|}
\hline
      &  \multicolumn{3}{c|}{ without shape uncertainty} &\multicolumn{3}{c|}{ with shape uncertainty} \\
\hline
Mass  &  Median      &     68\% C.L. band &  95\% C.L. band &  Median	   &	 68\% C.L. band &  95\% C.L. band\\
      &  Expected    &                    &                 &  Expected    &			&		 \\
\hline
250 & 1.87 & [1.32, 2.66] & [0.97, 3.67] & 1.92 & [1.36, 2.75] & [1.01, 3.84] \\
300 & 1.29 & [0.91, 1.83] & [0.68, 2.54] & 1.32 & [0.94, 1.90] & [0.70, 2.63] \\
350 & 0.95 & [0.69, 1.36] & [0.51, 1.88] & 0.97 & [0.70, 1.40] & [0.52, 1.96] \\
400 & 0.99 & [0.71, 1.42] & [0.53, 1.97] & 1.00 & [0.72, 1.45] & [0.54, 2.03] \\
500 & 1.87 & [1.35, 2.68] & [1.03, 3.75] & 1.91 & [1.37, 2.76] & [1.05, 3.91] \\
600 & 4.31 & [3.09, 6.19] & [2.42, 8.83] & 4.41 & [3.15, 6.40] & [2.42, 9.09] \\

\hline
\end{tabular}
}
\caption{Comparison of the median expected cross section ratio limits together with the 1/2-$\sigma$ uncertainty bands of 
the shape analysis with and without accounting for the systematics due to the shape variation. 
The results are presented as a function of the Higgs mass. }
\label{tab:mva_mtshapewithwithout_hzz}
\end{center}
%\end{table}
%%%%%%%%%%%%%%%%%%%%%%%%%%%%%
%%%%%%%%%%%%%%%%%%%%%%%%%%%%%
%\begin{table}[!ht]
\begin{center}
{\normalsize
\begin{tabular}{|l|c|c|c|c|c|c|}
\hline
      &  \multicolumn{3}{c|}{Cut-and-Count Based Analysis} &\multicolumn{3}{c|}{Shape Analysis using $m_T$} \\
\hline
Mass  &  Median      &     68\% C.L. band &  95\% C.L. band &  Median	   &	 68\% C.L. band &  95\% C.L. band\\
      &  Expected    &                    &                 &  Expected    &			&		 \\
\hline
250 & 2.10 & [1.46, 3.03] & [1.07, 4.19] & 1.92 & [1.36, 2.75] & [1.01, 3.84] \\
300 & 1.49 & [1.04, 2.14] & [0.77, 3.00] & 1.32 & [0.94, 1.90] & [0.70, 2.63] \\
350 & 1.04 & [0.75, 1.54] & [0.56, 2.18] & 0.97 & [0.70, 1.40] & [0.52, 1.96] \\
400 & 1.03 & [0.77, 1.55] & [0.56, 2.17] & 1.00 & [0.72, 1.45] & [0.54, 2.03] \\
500 & 1.93 & [1.38, 2.91] & [1.09, 4.07] & 1.91 & [1.37, 2.76] & [1.05, 3.91] \\
600 & 3.93 & [2.90, 5.88] & [2.17, 8.87] & 4.41 & [3.15, 6.40] & [2.42, 9.09] \\

\hline
\end{tabular}
}
\caption{Comparison of the median expected cross section ratio limits as a function 
of the Higgs mass, together with the 1/2-$\sigma$ uncertainty bands between the cut-and-count 
analysis and the shape analysis using the transverse higgs mass. In this comparison, we include all systematics due to 
the shape variation. }
\label{tab:mva_mtshapevscuts_withshapevar_hzz}
\end{center}
\end{table}
%%%%%%%%%%%%%%%%%%%%%%%%%%%%%
%%%%%%%%%%%%%%%%%%%%%%%%%%%%%
\begin{table}[!ht]
\begin{center}
{\normalsize
\begin{tabular}{|l|c|cccccc|}
\hline
      &  Analysis    & adding          &  adding      &  adding      &  adding      & adding      & adding \\
mH  &  without     & template        &  $H\to ZZ$   &  Top         &  WW          & WZ          & ZZ \\
      &  shape syst. & stat. uncert.   &  QCD effect &  shape syst. &  shape syst. & shape syst. & shape syst. \\
\hline
250 & 1.87 & 1.92 & 1.93 & 1.93 & 1.93 & 1.91 & 1.92 \\
300 & 1.29 & 1.31 & 1.31 & 1.31 & 1.32 & 1.31 & 1.32 \\
350 & 0.95 & 0.96 & 0.97 & 0.97 & 0.98 & 0.97 & 0.97 \\
400 & 0.99 & 1.00 & 1.00 & 1.00 & 1.00 & 1.00 & 1.00 \\
500 & 1.87 & 1.91 & 1.89 & 1.90 & 1.91 & 1.93 & 1.91 \\
600 & 4.31 & 4.42 & 4.37 & 4.36 & 4.38 & 4.40 & 4.41 \\
\hline
\end{tabular}
}
\caption{Comparison of the median expected cross section ratio limits as a function 
of the Higgs mass between shape analysis without and with accouting for the 
shape variation systematics. The results on the various sources are added sequentially 
to study the impact of each source. }
\label{tab:mva_mtshape_detail}
\end{center}
\end{table}
%%%%%%%%%%%%%%%%%%%%%%%%%%%%%