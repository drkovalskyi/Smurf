\begin{table}[!ht]
\begin{center}
\begin{tabular}{|l|ccccccccccc|}
\hline
 Systematic Effect & ggH & ZH & WH & qqWW & ggWW & VV & Top & Zjets & Wjets & Wgamma & Ztt \\
\hline
Higgs Theory       & X   & X  & X  & -    & -    & -  & -   & -     & -     & -      & - \\
PDF                & X   & X  & X  & X    & X    & X  & -   & -     & -     & X      & - \\
Lepton efficiency  & X   & X  & X  & X    & X    & X  & -   & -     & -     & X      & - \\
Lepton resolution  & X   & X  & X  & X    & X    & X  & X   & -     & -     & X      & - \\
MET resolution     & X   & X  & X  & X    & X    & X  & X   & -     & -     & X      & - \\
Jet Energy Scale   & X   & X  & X  & X    & X    & X  & X   & X     & X     & X      & X \\
\WW{}              & -   & -  & -  & X    & X    & -  & -   & -     & -     & -      & - \\
Top                & -   & -  & -  & -    & -    & -  & X   & -     & -     & -      & - \\
\dyll{}            & -   & -  & -  & -    & -    & -  & -   & X     & -     & -      & - \\
\wjets{}           & -   & -  & -  & -    & -    & -  & -   & -     & X     & -      & - \\
$\W\gamma$         & -   & -  & -  & -    & -    & -  & -   & -     & -     & X      & - \\
\dytt{}            & -   & -  & -  & -    & -    & -  & -   & -     & -     & -      & X \\
\hline
\end{tabular}
\caption{Summary of different systematic effects and components that they affect.}
\end{center}
\end{table}

\subsubsection{Higgs Theoretical Uncertainties}
Procedure:
  \begin{itemize}
    \item Vary the QCD renormalization ($\mu_R$) and scale ($\mu_F$) independently
    \item Reweight the $gg\to$H component with different k-factor distributions
    \item Only shape, since the normalization factor has its own factor
    \item Taking ``Up'' as $\mu_R$ = 0.5mu and $\mu_F$=2.0mu and ``Down'' as $\mu_R$ = 2.0mu and $\mu_F$=0.5mu
  \end{itemize}

\subsubsection{PDF}
Procedure:
  \begin{itemize}
    \item Only normalization is considered (no shape variation)
    \item Change in normalization is $\sim$10\%
    \item Have followed the method from AN2011/055
  \end{itemize}

\subsubsection{Lepton efficiency scale factors}
Procedure:
  \begin{itemize}
    \item Apply +1sigma and -1sigma reweighting factor to the scale factor efficiencies using the efficiency tables paramtrized in eta and pt
    \item Produce ``Up'' and ``Down'' histograms with the +1/-1 sigma factors
  \end{itemize}

\subsubsection{Lepton energy-momentum resolution and scale}

Procedure:
  \begin{itemize}
    \item Smear the lepton momenta by a gaussian with width equal to the central value determination of the resolution + the uncertainty, and the central value resolution
    \item Produce new BDT outputs using smeared lepton variables
    \item Produce ``Up'' and ``Down'' histograms using the new BDT outputs
  \end{itemize}

\subsubsection{MET resolution}
Procedure:
  \begin{itemize}
    \item Smear $Met_x$ and $Met_y$ components for both PFMET and trackMET
      \begin{itemize}
      \item $Syst_PFMet_X/Y$ = Gaus(0.0,4.8), $Syst_trackMet_X/Y$ = Gaus(0.0,3.0)
      \end{itemize}    
    \item Produce new BDT outputs using smeared MET variables
    \item Produce ``Up'' and ``Down'' histograms using the new BDT outputs
  \end{itemize}

\subsubsection{Jet Energy Scale}
Procedure:
  \begin{itemize}
    \item Change $\pm5$\% the jet pt and see the effect in the analysis
  \end{itemize}

\subsubsection{\WW{}}
Procedure involves two set of histograms:
  \begin{itemize}
    \item compare Madgraph with MC@NLO for the Up template and the mirror of that ratio is the Down template
    \item make the ratio between MC@NLO and MC@NLO with QCD scales up and down. Multiply the ratio of them on each bin to the default Madgraph to produce the Up/Down templates
  \end{itemize}

\subsubsection{\wjets{}}
Procedure:
  \begin{itemize}
    \item Produce a new distribution by using different lepton fake rates
      \begin{itemize}
        \item muon fake rate using jet pt threshold at 30 GeV (instead of 15)
        \item electron fake rate using jet pt threshold at 50 GeV (instead of 35)
      \end{itemize}
   \item Propagate Wjets MC closure tests uncertainty
      \begin{itemize}
        \item Get a scale-factor histogram from the closure test and apply it
      \end{itemize}
  \end{itemize}

\subsubsection{Drell-Yan}
Procedure:
  \begin{itemize}
    \item Use NNLO reweighting as default
    \item Default histogram is produced with MC events with
      min(PFMET,trackMET): [20-40] GeV, normalization is taken from
      the signal region evaluation
      \begin{itemize}
        \item ``Up'' histogram is produced using the standard MC with
          min(PFMET,trackMET) > 40 GeV
        \item ``Down'' histogram is the mirror of the ``Up'' histogram
          with respect to the default one
      \end{itemize}
  \end{itemize}

\subsubsection{Top}
Procedure:
  \begin{itemize}
    \item Take variation between Powheg and Madgraph as systematic uncertainty, this should cover most of the differences in the MC modeling
  \end{itemize}

\subsubsection{$\W\gamma$}
Procedure:
  \begin{itemize}
    \item No shape variation, only normalization. Make sure that uncertainty is large enough
    \item The shape is take from MC removing the conversion rejection requirements
  \end{itemize}

\subsubsection{\dytt{}}
Procedure:
  \begin{itemize}
    \item 
  \end{itemize}
