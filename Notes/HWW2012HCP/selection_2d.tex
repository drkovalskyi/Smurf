In the approach described in the previous section, the shape of 
the BDT output is used to separate signal and background.
In this section we describe an alternate approach,
using a binned fit to the 2D distribution of two physical variables.
The analysis method is described in brief here,
and in detail detail in Ref.~\cite{2DNote}.

This method uses two physical variables,
which are also inputs to the BDT discriminant: \mll~and \mt.
They are defined as follows:

\begin{itemize}
\item the dilepton mass $\mll$;
\item transverse Higgs mass,
$\mt^{\ell\ell\met} = \sqrt{2\pt^{ll}\met(1-cos(\Delta\phi_{\ell\ell-\met}))}$ where
$\Delta\phi_{\ell\ell-\met}$ is the angle between dilepton
direction and \met in the transverse plane.
\end{itemize}

We aim to improve upon 
two aspects of the BDT approach:

\begin{itemize}
    \item The position assigned to each event 
in \mll~and \mt~ is simpler to interpret physically 
than the output of the BDT, which depends non-trivially 
on the values of the input variables.
    \item The value of each event in \mll~and \mt~
is independent of the Higgs boson mass hypothesis considered,
and the dependence of the preselection requirements is reduced.
\end{itemize}

The 2D templates for signal and background processes are constructed 
with the WW preselection. For the high mass hypotheses ($\mHi\ge300~\GeV$),
we apply $\ptlmax > 50~\GeV$ to increase S/B. The bin size is chosen to avoid empty 
bins in the background templates, given the the available simulation and control sample statistics.  
Table~\ref{tab:binning_range} summarizes the bin size and range of templates. 

\vspace{25pt}
\begin{table}[!htb]
\centering
\begin{tabular}{c | c | c }
\hline \hline
     Variable  & $\mHi< 300~\GeV$  & $\mHi\ge 300~\GeV$    \\
	\hline \hline
	\mt       & [80,280] 10 bins  & [80,380] 10 bins      \\
	\mll      & [0,200] 8 bins    & [0,450] 8 bins        \\
	\hline
	\end{tabular}
	\label{tab:binning_range}
	\caption{Summary of template parameters. For the high-\mHi~templates, 
			 overflow to \mll=600~\GeV~and~\mt=600~\GeV~is included
		  	 in the content of the last bin.}
\end{table}

We have applied the method to the $e\mu$ final state in the 0 and 1 jet bins, 
which are the most sensitive channels.  The performance of the 2D analysis
compared with the BDT analysis is discussed in more detail
in ~\cite{2DNote} and the current results are given 
in this document.  We have observed an increase in sensitivity
at low Higgs boson mass compared to the BDT analysis. 
This is found to be a result of the improved background
constraints provided by the low signal/high background regions
of the 2D plane~\cite{YY2DTalk}.

