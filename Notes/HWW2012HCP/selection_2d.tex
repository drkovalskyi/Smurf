One of the drawbacks of the BDT approach is the non-trivial dependence of the discriminator value 
on the input variables. This means that physical interpretation of the data in the BDT discriminator 
can be difficult. Additionally, for Higgs boson masses below 140 GeV, most of the additional sensitivity 
of the shape-analysis cames from the fit rather than the BDT itself. This has been shown by performing 
mll shape analyis with 2011 data which gives about 90 \% of the sensitivity of the BDT approach.

Thus, we explore an alternative approach that allows for a simpler physical interpretation of the 
observed data with a sensitivity that is better than the default BDT shape-analysis for low \mHi hypotheses. 
We have developed a 2-dimensional shape analysis using two independent variables: \mll and \mt.
They are defined as 

\begin{itemize}
\item the dilepton mass $\mll$;
\item transverse Higgs mass, 
$\mt^{\ell\ell\met} = \sqrt{2\pt^{ll}\met(1-cos(\Delta\phi_{\ell\ell-\met}))}$ where 
$\Delta\phi_{\ell\ell-\met}$ is the angle between dilepton
direction and \met in the transverse plane.
\end{itemize} 

The 2D templates are constructed with WW preselection. For the high mass hypotheses ($\mHi<\ge300\GeV$),
we apply $\ptlmax > 50~\GeV$ to enhance S/B. The bin size is chosen to avoid empty 
bins in the background templates, given the the available simulation and control sample statistics.  
Table~\ref{tab:binning_range} summarizes the bin size and range of templates. 

\begin{table}[!htb]
\centering
\begin{tabular}{c | c | c }
\hline \hline
     Variable  & $\mHi< 300~\GeV$  & $\mHi\ge 300~\GeV$    \\
	\hline \hline
	\mt       & [80,280] 10 bins  & [80,380] 10 bins      \\
	\mll      & [0,200] 8 bins    & [0,450] 8 bins        \\
	\hline
	\end{tabular}
	\label{tab:binning_range}
	\caption{Summary of template parameters. For the high-\mHi~templates, overflow to \mt/\mll=600~\GeV~ is included
			in the content of the last bin.}
\end{table}


We have applied the method to the $e\mu$ final state in 0 and 1 jet bins, which are the most sensitive channels.  
Details of the 2D analysis is described in \cite{2DNote}. 


