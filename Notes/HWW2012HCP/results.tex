\subsection{Statistical Methods}
We derive upper limits on the product of the Higgs boson production
cross section and the $\Hi \to \WW$ branching fraction,
$\sigma_{\rm{H}} \times $BR($\Hi \to \WW)$, with respect to the SM
expectation, i.e. $\sigma^{95\%}/\sigma^{SM}$. Two different
statistical methods are used to report results. The first method is
based on Bayesian inference~\cite{bayesian} and the second one, known
as $CL_{s}$, is the modified frequentist approach~\cite{cls1,cls2}.

The likelihood function is defined as:
\begin{eqnarray}
  L(\rm{data}|\mu,\theta)&=&\rm{Poisson}(\rm{data}|\mu\cdot s(\theta)+b(\theta))\cdot p(\tilde{\theta}|\theta) \nonumber\\
 &=&\prod_i\frac{(\mu s_i+b_i)^{n_i}}{n_i!}e^{-\mu s_i-b_i}\cdot p(\tilde{\theta}|\theta)
\label{eq:likelihood}
\end{eqnarray}
where $\mu$ is the signal strength modifier which is often reported in
the upper limit results as a ratio of the cross-section upper limit
over the standard model cross-section and $\theta$ represents a full
set of nuisance parameters that are used to incorporate systematic
uncertainties. 

The first method (Bayesian) is based on interpreting the likelihood
(Eq.~\ref{eq:likelihood}) as a probability distribution function with
a flat prior for the signal strength and a set of pdfs for nuisance
parameters, which are often approximated with the log-normal
distribution. Integrating over the nuisance parameters we find the
upper limit for the signal strength.

For $CL_{s}$ method the test statistic is defined as a likelihood
ratio:
\begin{equation}
\tilde{q_\mu}=-2\log\frac{L(\rm{data}|\mu,\hat\theta_\mu)}{L(\rm{data}|\hat\mu,\hat\theta)}
\end{equation}
where the numerator corresponds to the maximum likelihood for given
``data'' and $\mu$ profiling over the nuisance parameters and the
denominator corresponds to the maximum likelihood for given ``data''
profiling over the nuisance parameters and $\mu$. This test statistic
differs from the ones used at LEP (no profiling of systematic errors)
and at Tevatron (the denominator likelihood uses $\mu=0$ and only
systematic errors are profiled).

The results obtained using the two methods may differ but in most cases
they are very close. To perform the computation of the limits, the
software packages
\texttt{RooStats}~\cite{rootstat} and \texttt{LandS}~\cite{lands} have 
been used.

\subsection{Background Estimation}

The estimation of the backgrounds follows the strategies described in
Section~\ref{sec:backgrounds}. As mentioned at the begining of the 
document, we are totally/partially missing $\wgamma$, $\wgamma^{*}$ and $\WZ$
in simulation. Thus, Monte Carlo yields and data/MC scale factors 
are preliminary.

First we estimate the $\dyll$ at the WW selection level shown in Table~\ref{tab:dy_wwlevel}. 
As it was seen before the simulation significantly underestimates this type of
background. It is important to keep in mind that $\WZ$ and $\ZZ$ 
contributions in the $\Z$-peak region are sizable, so the method depends
on the Monte Carlo simulation of these processes. It is not a problem
since the uncertainties on these di-boson contributions in the Z-peak
region are small compared with the systematic uncertainties of the
R-value extraction and the statistical uncertainties on the number of
the events in Z-peak region.
As we do not have enough statistics in the MC at the higgs selection level, 
we estimate directly at the Higgs selection level for both the 
cut-based and shape-based analyses. 
The results are shown in Table~\ref{tab:dy}. 

The $\Wjets$ background contribution is summarized in Table~\ref{tab:fake_est}. 
The same sign closure test in the 0-jet bin finds 276 events in data while 
the background expectation is $310 \pm 10~(stat.)$.

The top background estimation is shown in
Table~\ref{tab:ttbar_est}. The scale factors are consistent with unity within 
the current large statistical uncertainties. 

With these results, we compare the yields after the $\WW$ preselection 
in data and MC with min-MET(Table~\ref{tab:wwselection_all_minmet}) and 
DY MVA(Table~\ref{tab:wwselection_all_dymva}). Higgs contribution at
\WW\ selection level is negligible for not excluded Higgs mass
hypotheses. For the signal extraction we estimate the \WW\ background
contribution in data looking at events with large di-lepton mass, i.e.
$m_{ll}>100$~\GeV{} (Table~\ref{tab:ww_est}). 
Figures~\ref{fig:ww_ptmax}-\ref{fig:ww_deltaphi} show a few key distributions at \WW\ selection level.

%%%%%%%%%%%%%%%%%%%%%%%%%%%%%%
\begin{table}
\begin{center}
\begin{tabular}{c c c c c c}
\hline
       nJets & $N_{in}$(data)        & $R_{out/in}$        & $N_{out}$(data)  & $N_{out}$ (MC) \\ 
0 & $417.2\pm50.4$ 		& $0.27\pm0.01\pm0.02$ & $114.1\pm14.8\pm8.5$ 	& $18.97\pm5.59$ 	\\
1 & $191.8\pm27.4$ 		& $0.22\pm0.01\pm0.07$ & $43.0\pm6.4\pm13.0$ 	& $13.54\pm4.66$  \\
2 & $1964.4\pm48.9$ 	& $0.26\pm0.01\pm0.03$ & $507.0\pm21.0\pm53.9$ & $260.93\pm19.94$  \\
\hline
\end{tabular}
\caption{The Drell-Yan estimation in the same flavor final state at WW preselection level, using the DYMVA in 
0 and 1 Jet bins and the pfMET at the 2-jet bins. }
\label{tab:dy_wwlevel}
\end{center}
\end{table}

%%%%%%%%%%%%%%%%%%%%%%%%%%%%%%
\begin{table}
\begin{center}
\begin{tabular}{c c c c c c}
\hline
\hline
\multicolumn{5}{c}{0-jet} \\
\hline
mass & $N_{in}$(data)        & $R_{out/in}$        & $N_{out}$(data)  & $N_{out}$ (MC) \\ 
\hline
115 \GeV &$ 78.9\pm11.5 $&$ 0.32\pm0.02\pm0.08 $&$ 24.9\pm3.9\pm6.4 $&$ 6.06\pm3.59 $\\
120 \GeV &$ 129.2\pm15.6 $&$ 0.32\pm0.02\pm0.08 $&$ 40.7\pm5.5\pm10.5 $&$ 10.31\pm4.30 $\\
125 \GeV &$ 83.0\pm12.4 $&$ 0.62\pm0.04\pm0.12 $&$ 51.7\pm8.5\pm10.0 $&$ 10.31\pm4.30 $\\
130 \GeV &$ 64.1\pm11.0 $&$ 0.87\pm0.06\pm0.15 $&$ 56.1\pm10.4\pm9.7 $&$ 10.31\pm4.30 $\\
135 \GeV &$ 59.5\pm11.1 $&$ 0.82\pm0.06\pm0.12 $&$ 48.9\pm9.8\pm7.4 $&$ 8.71\pm3.99 $\\
140 \GeV &$ 59.2\pm11.1 $&$ 0.78\pm0.06\pm0.09 $&$ 46.0\pm9.3\pm5.1 $&$ 8.71\pm3.99 $\\
145 \GeV &$ 59.2\pm11.1 $&$ 0.78\pm0.06\pm0.09 $&$ 46.0\pm9.3\pm5.1 $&$ 8.71\pm3.99 $\\
150 \GeV &$ 44.2\pm11.2 $&$ 0.30\pm0.04\pm0.19 $&$ 13.5\pm3.9\pm8.3 $&$ 1.07\pm1.07 $\\
160 \GeV &$ 12.8\pm6.8 $&$ 0.79\pm0.14\pm0.37 $&$ 10.1\pm5.6\pm4.8 $&$ 1.07\pm1.07 $\\
170 \GeV &$ 3.5\pm6.2 $&$ 0.68\pm0.13\pm0.59 $&$ 2.4\pm4.2\pm2.0 $&$ 1.07\pm1.07 $\\
180 \GeV &$ 3.5\pm7.4 $&$ 0.52\pm0.09\pm0.09 $&$ 1.8\pm3.8\pm0.3 $&$ 0.00\pm0.00 $\\
190 \GeV &$ 23.7\pm11.5 $&$ 0.28\pm0.04\pm0.05 $&$ 6.6\pm3.3\pm1.1 $&$ 1.36\pm1.36 $\\
200 \GeV &$ 34.2\pm15.8 $&$ 0.18\pm0.03\pm0.03 $&$ 6.2\pm3.0\pm0.9 $&$ 1.36\pm1.36 $\\
250 \GeV &$ 87.4\pm25.4 $&$ 0.05\pm0.01\pm0.01 $&$ 4.0\pm1.3\pm1.1 $&$ 4.63\pm2.68 $\\
300 \GeV &$ 32.3\pm19.3 $&$ 0.09\pm0.02\pm0.20 $&$ 3.0\pm1.9\pm6.3 $&$ 4.63\pm2.68 $\\
\vspace{-3mm}  \\
\hline
\hline
\multicolumn{5}{c}{1-jet} \\
\hline
mass & $N_{in}$(data)        & $R_{out/in}$        & $N_{out}$(data)  & $N_{out}$ (MC) \\ 
\hline
115 \GeV &$ 19.2\pm6.7 $&$ 0.16\pm0.01\pm0.03 $&$ 3.0\pm1.1\pm0.5  $&$ 0.00\pm0.00 $\\
120 \GeV &$ 42.8\pm9.3 $&$ 0.16\pm0.01\pm0.03 $&$ 6.7\pm1.5\pm1.2  $&$ 0.00\pm0.00 $\\
125 \GeV &$ 32.8\pm7.9 $&$ 0.23\pm0.01\pm0.04 $&$ 7.5\pm1.9\pm1.3  $&$ 0.00\pm0.00 $\\
130 \GeV &$ 30.5\pm7.3 $&$ 0.30\pm0.02\pm0.05 $&$ 9.1\pm2.3\pm1.6  $&$ 0.00\pm0.00 $\\
135 \GeV &$ 29.7\pm7.6 $&$ 0.28\pm0.02\pm0.04 $&$ 8.2\pm2.2\pm1.3  $&$ 0.00\pm0.00 $\\
140 \GeV &$ 27.3\pm7.6 $&$ 0.25\pm0.02\pm0.05 $&$ 6.9\pm2.0\pm1.3  $&$ 0.00\pm0.00 $\\
145 \GeV &$ 27.3\pm7.6 $&$ 0.25\pm0.02\pm0.05 $&$ 6.9\pm2.0\pm1.3  $&$ 0.00\pm0.00 $\\
150 \GeV &$ 35.6\pm9.2 $&$ 0.16\pm0.01\pm0.04 $&$ 5.9\pm1.6\pm1.4  $&$ 0.00\pm0.00 $\\
160 \GeV &$ 12.8\pm4.9 $&$ 0.37\pm0.04\pm0.14 $&$ 4.8\pm1.9\pm1.9  $&$ 0.00\pm0.00 $\\
170 \GeV &$ 13.9\pm5.3 $&$ 0.34\pm0.04\pm0.12 $&$ 4.8\pm1.9\pm1.7  $&$ 0.00\pm0.00 $\\
180 \GeV &$ 14.2\pm5.9 $&$ 0.28\pm0.03\pm0.09 $&$ 4.0\pm1.7\pm1.3  $&$ 0.00\pm0.00 $\\
190 \GeV &$ 44.0\pm10.4 $&$ 0.21\pm0.02\pm0.04 $&$ 9.1\pm2.3\pm1.9  $&$ 0.00\pm0.00 $\\
200 \GeV &$ 59.2\pm12.6 $&$ 0.16\pm0.01\pm0.03 $&$ 9.5\pm2.1\pm1.7  $&$ 0.00\pm0.00 $\\
250 \GeV &$ 71.0\pm16.2 $&$ 0.09\pm0.01\pm0.00 $&$ 6.2\pm1.5\pm0.2  $&$ 1.62\pm1.62 $\\
300 \GeV &$ 40.2\pm13.2 $&$ 0.09\pm0.01\pm0.02 $&$ 3.8\pm1.3\pm0.9  $&$ 3.12\pm2.21 $\\
\vspace{-3mm}  \\
\hline
\hline
\multicolumn{5}{c}{2-jet} \\
\hline
mass & $N_{in}$(data)        & $R_{out/in}$        & $N_{out}$(data)  & $N_{out}$ (MC) \\
\hline
115 \GeV &$ 6.93\pm2.65 $&$ 0.25\pm0.06\pm0.11 $&$ 1.72\pm0.79\pm0.77 $&$ 2.17\pm1.58 $	\\
120 \GeV &$ 12.88\pm3.61 $&$ 0.25\pm0.06\pm0.11 $&$ 3.20\pm1.21\pm1.42 $&$ 2.17\pm1.58$	\\
125 \GeV &$ 6.90\pm2.65 $&$ 0.35\pm0.09\pm0.16 $&$ 2.40\pm1.11\pm1.07 $&$ 2.17\pm1.58$	\\
130 \GeV &$ 5.94\pm2.45 $&$ 0.47\pm0.13\pm0.12 $&$ 2.78\pm1.37\pm0.70 $&$ 2.17\pm1.58$	\\
135 \GeV &$ 7.94\pm2.83 $&$ 0.45\pm0.12\pm0.24 $&$ 3.54\pm1.59\pm1.94 $&$ 2.17\pm1.58$	\\
140 \GeV &$ 7.94\pm2.83 $&$ 0.48\pm0.13\pm0.28 $&$ 3.80\pm1.72\pm2.21 $&$ 0.81\pm0.81$	\\
145 \GeV &$ 7.94\pm2.83 $&$ 0.48\pm0.13\pm0.28 $&$ 3.80\pm1.72\pm2.21 $&$ 0.81\pm0.81$	\\
150 \GeV &$ 11.84\pm3.75 $&$ 0.27\pm0.10\pm0.17 $&$ 3.22\pm1.54\pm2.04 $&$ 0.00\pm0.00$	\\
160 \GeV &$ 4.95\pm2.24 $&$ 0.76\pm0.34\pm0.62 $&$ 3.76\pm2.38\pm3.05 $&$ 0.00\pm0.00$	\\
170 \GeV &$ 6.94\pm2.65 $&$ 0.76\pm0.34\pm0.76 $&$ 5.26\pm3.08\pm5.26 $&$ 0.00\pm0.00$ \\
180 \GeV &$ 10.83\pm3.61 $&$ 0.45\pm0.18\pm0.45 $&$ 4.91\pm2.57\pm4.91 $&$ 0.00\pm0.00$ \\
190 \GeV &$ 18.68\pm4.80 $&$ 0.33\pm0.11\pm0.33 $&$ 6.09\pm2.57\pm6.09 $&$ 0.00\pm0.00$ \\
200 \GeV &$ 19.56\pm4.91 $&$ 0.22\pm0.07\pm0.22 $&$ 4.32\pm1.81\pm4.32 $&$ 0.00\pm0.00$ \\
250 \GeV &$ 34.14\pm6.25 $&$ 0.15\pm0.05\pm0.15 $&$ 5.19\pm2.01\pm5.19 $&$ 0.00\pm0.00$ \\
300 \GeV &$ 25.27\pm5.67 $&$ 0.09\pm0.06\pm0.09 $&$ 2.26\pm1.52\pm2.26 $&$ 0.00\pm0.00$ \\
\hline 
\hline
\end{tabular}
\caption{The Drell-Yan estimation in the same flavor final state, for the cut-based selections.}
\label{tab:dy}
\end{center}
\end{table}

%%%%%%%%%%%%%%%%%%%%%%%%%%%%%% 
\begin{table}[ht!]
\begin{center}
\begin{tabular}{c c c c c c} 
\hline
jet-bin &	 $\mu\mu$ &	 $e \mu$ &	 $\mu e$ &	 $ee$ &	 total \\ 
\hline
0 &  67.65 +/-  7.10   &  58.88 +/-  4.56     &  199.91 +/-  8.27  & 46.74 +/-  2.49  & 373.19 +/- 12.08 \\
1 &  44.31 +/-  6.16   &  61.31 +/-  5.38     &  178.08 +/-  8.25  & 18.07 +/-  1.75  & 301.78 +/- 11.75 \\ 
2 &  46.44 +/-  7.69   &  35.34 +/-  4.63     &  118.47 +/-  7.39  & 15.93 +/-  1.67  & 216.20 +/- 11.75 \\ 
\hline
\end{tabular}
\caption{Predictions of the fake-induced background contribution 
in the data-driven estimation after the $\WW$ preselection. 
The analyzed data correspond to $\intlumiEightTeV$, where the reported uncertainties are statistical only.}
\label{tab:fake_est}
\end{center}
\end{table}
%%%%%%%%%%%%%%%%%%%%%%%%%%%%%%
\begin{table}[ht!]
\begin{center}
\begin{tabular}{l c c c}
\hline
                                   Sample & 0-jet           & 1-jet           & 2-jet       \\
\hline
estimated top events in simulation  & 423.1 $\pm$   2.7 &  1312.0 $\pm$   9.0 &  31.7 $\pm$   1.3 \\
tagging efficiency     (\%)         & 49.4 $\pm$  4.3 & 65.1 $\pm$  0.6 & - \\ 
data events in control region       &  599 & 2997 & - \\ 
background events in control region & 191.4 $\pm$  28.7 &  189.8 $\pm$  38.0 & - \\ 
top estimation in data              &  417.9 $\pm$  85.7 &  1425.4 $\pm$  46.5 &   4.2 $\pm$   3.0 \\
data/simulation scale factor        &   0.99 $\pm$  0.20 &   1.09 $\pm$  0.04 &  1.02 $\pm$  0.25 \\
\hline
\end{tabular}
\caption{Monte Carlo to data scale factor for the top background contribution for $\intlumiEightTeV$. 
In the 1-jet bin, the scale factor is derived in a region that is slightly different from the signal region.}
\label{tab:ttbar_est}
\end{center}
\end{table}
%%%%%%%%%%%%%%%%%%%%%%%%%%%%%%

\begin{table}[ht!]
  \begin{center}
 {\small
  \begin{tabular} {|c|c|c|c|c|c|c|}
\hline
          &   data & all bkg. & $qq \to \WW$ & $gg \to \WW$ &  $\ttbar+tW$   & $\Wjets$    \\
  \hline
  \hline
%	0-jet	&   1594 & 1320.41 $\pm$ 15.49 &   853.83  $\pm$  6.17 & 51.96 $\pm$  1.13 &  160.48 $\pm$  5.41  & 151.12 $\pm$  6.69  \\	   
%	1-jet	&   1171 & 1159.81 $\pm$ 15.92 &   391.48  $\pm$  4.22 & 21.32 $\pm$  0.73 &  526.30 $\pm$  8.04  & 108.05 $\pm$  6.71  \\   
	0-jet	&   4416 & ?  &   2478.31 $\pm$ 10.20  & 160.86 $\pm$ 1.94 &  415.74 $\pm$ 7.77  & 371.04 $\pm$ 12.09  \\	   
 \hline
 \hline
  \end{tabular}
  \begin{tabular} {|c|c|c|c|c|}
\hline
       & $WZ$/$ZZ$ not included in the $\dyll$ & $\dyll+WZ+ZZ$ & $W+\gamma$ \\
  \hline
  \hline
%	0-jet 	&  23.36 $\pm$  0.42 & 51.82 $\pm$ 10.64 & 27.84 $\pm$  3.62 \\ 
%	1-jet 	&  26.39 $\pm$  0.45 & 66.10 $\pm$ 10.51 & 20.17 $\pm$  3.83 \\
	0-jet 	&  96.90 $\pm$ 1.26  & 56.38 $\pm$ 2.96 & 131.48 $\pm$ 10.44  \\ 
 \hline
 \hline
  \end{tabular}
  }
  \caption{\fixme Expected number of signal and background events from the data-driven methods for 
  an integrated luminosity of \intlumiEightTeV after applying the $\WW$ selection requirements. 
  Only statistical uncertainties on the processes are reported. $\WW$ yield is from MC.}
   \label{tab:wwselection_all_dymva}
  \end{center}
\end{table}
%%%%%%%%%%%%%%%%%%%%%%%%%%%%%%%%%%%%

\begin{figure}[!hbtp]
\centering
\subfigure[]{
\centering
\label{subfig:ww_ptmin_0j}
\includegraphics[width=.3\textwidth]{figures/hww_analysis16_0_ALL_incl_0j_pt2.pdf}
}
\subfigure[]{
\centering
\label{subfig:ww_ptmin_1j}
\includegraphics[width=.3\textwidth]{figures/hww_analysis16_0_ALL_incl_1j_pt2.pdf}
}
\subfigure[]{
\centering
\label{subfig:ww_ptmin_2j}
\includegraphics[width=.3\textwidth]{figures/hww_analysis16_0_ALL_incl_2j_pt2.pdf}
}
\caption{Trailing lepton $p_T$ distribution after WW selection for \intlumiEightTeV of data in the 0-jet \subref{subfig:ww_ptmin_0j}, 
1-jet \subref{subfig:ww_ptmin_1j} and 2-jet \subref{subfig:ww_ptmin_2j} bin analyses. 
MC is scaled to data-driven estimates for all processes.}
\label{fig:ww_ptmin}
\end{figure}

\begin{figure}[!hbtp]
\centering
\subfigure[]{
\centering
\label{subfig:ww_ptmax_0j}
\includegraphics[width=.3\textwidth]{figures/hww_analysis16_0_ALL_incl_0j_pt1.pdf}
}
\subfigure[]{
\centering
\label{subfig:ww_ptmax_1j}
\includegraphics[width=.3\textwidth]{figures/hww_analysis16_0_ALL_incl_1j_pt1.pdf}
}
\subfigure[]{
\centering
\label{subfig:ww_ptmax_2j}
\includegraphics[width=.3\textwidth]{figures/hww_analysis16_0_ALL_incl_2j_pt1.pdf}
}\\
\caption{Leading lepton $p_T$ distribution after WW selection for \intlumiEightTeV of data in the 0-jet \subref{subfig:ww_ptmax_0j}, 
1-jet \subref{subfig:ww_ptmax_1j} and 2-jet \subref{subfig:ww_ptmax_2j} bin analyses. 
MC is scaled to data-driven estimates for all processes.}
\label{fig:ww_ptmax}
\end{figure}

\begin{figure}[!hbtp]
\centering
\subfigure[]{
\centering
\label{subfig:ww_pmet_0j}
\includegraphics[width=.3\textwidth]{figures/hww_analysis16_0_ALL_incl_0j_met.pdf}
}
\subfigure[]{
\centering
\label{subfig:ww_pmet_1j}
\includegraphics[width=.3\textwidth]{figures/hww_analysis16_0_ALL_incl_1j_met.pdf}
}
\subfigure[]{
\centering
\label{subfig:ww_pmet_2j}
\includegraphics[width=.3\textwidth]{figures/hww_analysis16_0_ALL_incl_2j_met.pdf}
}\\
\caption{The $\met$ distributions distribution after WW selection for \intlumiEightTeV of data in the 0-jet \subref{subfig:ww_pmet_0j}, 
1-jet \subref{subfig:ww_pmet_1j} and 2-jet \subref{subfig:ww_pmet_2j} bin analyses. 
Note that for the 0 and 1 Jet bins, $min(\text{proj}_\text{trk-MET}, \text{proj}_\text{PFMET})$ are plotted, 
while for the 2-jet bin, PFMET is used. MC is scaled to data-driven estimates for all processes.}
\label{fig:ww_pmet}
\end{figure}

\begin{figure}[!hbtp]
\centering
\subfigure[]{
\centering
\label{subfig:ww_mt_0j}
\includegraphics[width=.3\textwidth]{figures/hww_analysis16_0_ALL_incl_0j_mt.pdf}
}
\subfigure[]{
\centering
\label{subfig:ww_mt_1j}
\includegraphics[width=.3\textwidth]{figures/hww_analysis16_0_ALL_incl_1j_mt.pdf}
}
\subfigure[]{
\centering
\label{subfig:ww_mt_2j}
\includegraphics[width=.3\textwidth]{figures/hww_analysis16_0_ALL_incl_2j_mt.pdf}
} \\
\caption{Transverse mass distribution after WW selection for \intlumiEightTeV of data in the 0-jet \subref{subfig:ww_mt_0j}, 
1-jet \subref{subfig:ww_mt_1j} and 2-jet \subref{subfig:ww_mt_2j} bin analyses. 
MC is scaled to data-driven estimates for all processes.}
\label{fig:ww_mt}
\end{figure}

\begin{figure}[!hbtp]
\centering
\subfigure[]{
\centering
\label{subfig:ww_dilmass_0j}
\includegraphics[width=.3\textwidth]{figures/hww_analysis16_0_ALL_incl_0j_mll.pdf}
}
\subfigure[]{
\centering
\label{subfig:ww_dilmass_1j}
\includegraphics[width=.3\textwidth]{figures/hww_analysis16_0_ALL_incl_1j_mll.pdf}
}
\subfigure[]{
\centering
\label{subfig:ww_dilmass_2j}
\includegraphics[width=.3\textwidth]{figures/hww_analysis16_0_ALL_incl_2j_mll.pdf}
} \\
\caption{Invariant dilepton mass distribution after WW selection for \intlumiEightTeV of data in the 0-jet \subref{subfig:ww_dilmass_0j}, 
1-jet \subref{subfig:ww_dilmass_1j} and 2-jet \subref{subfig:ww_dilmass_2j} bin analyses. 
MC is scaled to data-driven estimates for all processes.}
\label{fig:ww_dilmass}
\end{figure}

\begin{figure}[!hbtp]
\centering
\subfigure[]{
\centering
\label{subfig:ww_deltaphi_0j}
\includegraphics[width=.3\textwidth]{figures/hww_analysis16_0_ALL_incl_0j_dphi.pdf}
}
\subfigure[]{
\centering
\label{subfig:ww_deltaphi_1j}
\includegraphics[width=.3\textwidth]{figures/hww_analysis16_0_ALL_incl_1j_dphi.pdf}
}
\subfigure[]{
\centering
\label{subfig:ww_deltaphi_2j}
\includegraphics[width=.3\textwidth]{figures/hww_analysis16_0_ALL_incl_2j_dphi.pdf}
} \\
\caption{Dilepton $\Delta\phi$ distribution after WW selection for \intlumiEightTeV of data in the 0-jet \subref{subfig:ww_deltaphi_0j}, 
1-jet \subref{subfig:ww_deltaphi_1j} and 2-jet \subref{subfig:ww_deltaphi_2j} bin analyses. 
MC is scaled to data-driven estimates.}
\label{fig:ww_deltaphi}
\end{figure}

\begin{figure}[!hbtp]
\centering
\subfigure[]{
\centering
\label{subfig:ww_mjj_2j}
\includegraphics[width=.3\textwidth]{figures/hww_analysis16_0_ALLjj_incl_2j_mjj.pdf}
}
\subfigure[]{
\centering
\label{subfig:ww_detajj_2j}
\includegraphics[width=.3\textwidth]{figures/hww_analysis16_0_ALLjj_incl_2j_detajj.pdf}
}
\caption{Di-jet invariant mass \subref{subfig:ww_mjj_2j} and $\Delta\eta(j_1, j_2)$ \subref{subfig:ww_detajj_2j} distributions after the 
WW selection. MC is scaled to data-driven estimates for all processes.}
\label{fig:ww_2j}
\end{figure}

\begin{table}[ht!]
\begin{center}
\begin{tabular}{c | c | c } 
\hline
            & \multicolumn{1}{c|}{0-jet} & \multicolumn{1}{c}{1-jet} \\
mass [\GeV] & scale factor & scale factor \\
\hline
115 &  1.14  $\pm$  0.07  &  0.87  $\pm$  0.12 \\
120 &  1.14  $\pm$  0.07  &  0.87  $\pm$  0.12 \\
125 &  1.14  $\pm$  0.07  &  0.87  $\pm$  0.12 \\
130 &  1.14  $\pm$  0.07  &  0.87  $\pm$  0.12 \\
135 &  1.15  $\pm$  0.07  &  0.88  $\pm$  0.12 \\
140 &  1.14  $\pm$  0.07  &  0.87  $\pm$  0.12 \\
145 &  1.14  $\pm$  0.07  &  0.87  $\pm$  0.12 \\
150 &  1.11  $\pm$  0.07  &  0.87  $\pm$  0.12 \\
155 &  1.11  $\pm$  0.07  &  0.87  $\pm$  0.12 \\
160 &  1.10  $\pm$  0.07  &  0.87  $\pm$  0.12 \\
170 &  1.10  $\pm$  0.07  &  0.87  $\pm$  0.12 \\
180 &  1.10  $\pm$  0.07  &  0.87  $\pm$  0.12 \\
190 &  1.10  $\pm$  0.07  &  0.86  $\pm$  0.12 \\
200 &  1.10  $\pm$  0.07  &  0.86  $\pm$  0.12 \\
\hline
\end{tabular}
\caption{WW background estimation for $\intlumiEightTeV$.}
\label{tab:ww_est_cut}
\end{center}
\end{table}

\begin{table}[ht!]
\begin{center}
\begin{tabular}{c | c } 
\hline
\multicolumn{1}{c|}{0-jet} & \multicolumn{1}{c}{1-jet} \\
scale factor & scale factor \\
\hline
1.23  $\pm$  0.07  &  1.08  $\pm$  0.13 \\
\hline
\end{tabular}
\caption{WW background estimation for $\intlumiEightTeV$.}
\label{tab:ww_est_shape}
\end{center}
\end{table}
%%%%%%%%%%%%%%%%%%%%%%%%%%%%%%

\clearpage
\subsection{Final Results for the Higgs Search with \intlumiEightTeV{}}
\label{sec:search_results}

The expected and observed upper limits at 95\% C.L. for the cut based and
multivariate analyses are shown in Tables~\ref{tab:cutbase_uls}
and~\ref{tab:mvabase_uls}, respectively. The corresponding exclusion
limits are shown in Figure~\ref{fig:uls}. The detailed event yields 
for both analyses are summarized in Appendices.~\ref{app:appendix_cutresults} 
and~\ref{app:appendix_bdtresults}. 
The expected and observed upper limits at 95\% C.L. for the individual channels 
are summarized in Appendices~\ref{app:appendix_limits_bychannel}. 
The results of the shape analysis using the dilepton mass single variable are 
summarized in Appendix~\ref{app:appendix_mll_bdt2011}.
The results of the shape analysis based on the Matrix Element method 
are summarized in Appendix~\ref{app:appendix_me}. 


%%%%%%%%%%%%%%%%%
% plot
\begin{figure}[!hbtp]
\centering
\includegraphics[width=.75\textwidth]{figures/table_limits_nj_shape_of_log.pdf}
\caption{Expected upper limits for SM Higgs in $\intlumiEightTeV$ at 8 TeV in the $e\mu$ channel. 
BDT result is used for 0/1jet bin and cut-based result is used for VBF channel. }
\label{fig:uls_of_bdt01_cut2}
\end{figure}
% table
\begin{table}[!htbp]
\begin{center}
\begin{tabular}{c c c c c}
\hline
\vspace{-3mm} && \\
Higgs Mass & Observed  & Median expected & Expected range for 68\% & Expected range for 95\%   \\
\hline
\vspace{-3mm} && \\
110 & -1.00 & 3.31 & [2.39, 4.61] & [1.78, 6.18] \\
115 & -1.00 & 1.90 & [1.37, 2.64] & [1.02, 3.54] \\
120 & -1.00 & 1.06 & [0.76, 1.48] & [0.57, 1.98] \\
125 & -1.00 & 0.73 & [0.52, 1.01] & [0.39, 1.36] \\
130 & -1.00 & 0.52 & [0.37, 0.72] & [0.28, 0.96] \\
135 & -1.00 & 0.42 & [0.30, 0.58] & [0.22, 0.78] \\
140 & -1.00 & 0.32 & [0.23, 0.44] & [0.17, 0.60] \\
145 & -1.00 & 0.32 & [0.23, 0.44] & [0.17, 0.60] \\
150 & -1.00 & 0.20 & [0.15, 0.28] & [0.11, 0.38] \\
155 & -1.00 & 0.20 & [0.15, 0.28] & [0.11, 0.38] \\
160 & -1.00 & 0.14 & [0.10, 0.19] & [0.07, 0.26] \\
170 & -1.00 & 0.15 & [0.11, 0.21] & [0.08, 0.28] \\
180 & -1.00 & 0.18 & [0.13, 0.26] & [0.10, 0.34] \\
190 & -1.00 & 0.27 & [0.19, 0.37] & [0.14, 0.49] \\
200 & -1.00 & 0.33 & [0.24, 0.46] & [0.18, 0.62] \\
250 & -1.00 & 0.68 & [0.49, 0.95] & [0.36, 1.27] \\
300 & -1.00 & 0.77 & [0.55, 1.07] & [0.41, 1.43] \\
350 & -1.00 & 0.71 & [0.51, 0.99] & [0.38, 1.32] \\
400 & -1.00 & 0.74 & [0.53, 1.02] & [0.39, 1.37] \\
450 & -1.00 & 0.78 & [0.56, 1.09] & [0.42, 1.46] \\
500 & -1.00 & 0.95 & [0.68, 1.32] & [0.51, 1.77] \\
550 & -1.00 & 1.11 & [0.80, 1.54] & [0.59, 2.07] \\
600 & -1.00 & 1.48 & [1.07, 2.06] & [0.80, 2.77] \\
\hline
\end{tabular}
\caption{Expected upper limits for SM Higgs in $\intlumiEightTeV$ at 8 TeV in the $e\mu$ channel. 
BDT result is used for 0/1jet bin and cut-based result is used for VBF channel. }
\label{tab:uls_of_bdt01_cut2}
\end{center}
\end{table} 
%%%%%%%%%%

%%%%%%%%%%%%%%%%%
% plot
\begin{figure}[!hbtp]
\centering
\includegraphics[width=.75\textwidth]{figures/table_limits_nj_shape2d_of_log.pdf}
\caption{Expected upper limits for SM Higgs in $\intlumiEightTeV$ at 8 TeV in the $e\mu$ channel. 
2D result is used for 0/1jet bin and cut-based result is used for VBF channel. }
\label{fig:uls_of_2d01_cut2}
\end{figure}
% table
\begin{table}[!htbp]
\begin{center}
\begin{tabular}{c c c c c}
\hline
\vspace{-3mm} && \\
Higgs Mass & Observed  & Median expected & Expected range for 68\% & Expected range for 95\%   \\
\hline
\vspace{-3mm} && \\
\hline
110 & -1.00 & 3.26 & [2.35, 4.54] & [1.75, 6.09] \\
115 & -1.00 & 1.67 & [1.21, 2.33] & [0.90, 3.12] \\
120 & -1.00 & 0.97 & [0.70, 1.35] & [0.52, 1.81] \\
125 & -1.00 & 0.62 & [0.45, 0.87] & [0.33, 1.16] \\
130 & -1.00 & 0.44 & [0.32, 0.62] & [0.24, 0.83] \\
135 & -1.00 & 0.35 & [0.25, 0.49] & [0.19, 0.65] \\
140 & -1.00 & 0.29 & [0.21, 0.40] & [0.16, 0.54] \\
150 & -1.00 & 0.21 & [0.15, 0.30] & [0.11, 0.40] \\
160 & -1.00 & 0.16 & [0.11, 0.22] & [0.08, 0.29] \\
170 & -1.00 & 0.17 & [0.12, 0.23] & [0.09, 0.31] \\
180 & -1.00 & 0.20 & [0.15, 0.28] & [0.11, 0.38] \\
190 & -1.00 & 0.29 & [0.21, 0.40] & [0.15, 0.53] \\
200 & -1.00 & 0.36 & [0.26, 0.51] & [0.19, 0.68] \\
250 & -1.00 & 0.62 & [0.45, 0.86] & [0.33, 1.16] \\
300 & -1.00 & 0.74 & [0.53, 1.03] & [0.40, 1.38] \\
350 & -1.00 & 0.66 & [0.48, 0.92] & [0.36, 1.24] \\
400 & -1.00 & 0.70 & [0.51, 0.98] & [0.38, 1.31] \\
450 & -1.00 & 0.86 & [0.62, 1.19] & [0.46, 1.60] \\
500 & -1.00 & 1.11 & [0.80, 1.54] & [0.60, 2.07] \\
550 & -1.00 & 1.34 & [0.96, 1.86] & [0.72, 2.49] \\
600 & -1.00 & 1.62 & [1.17, 2.26] & [0.87, 3.03] \\
\end{tabular}
\caption{Expected upper limits for SM Higgs in $\intlumiEightTeV$ at 8 TeV in the $e\mu$ channel. 
2D result is used for 0/1jet bin and cut-based result is used for VBF channel. }
\label{tab:uls_of_2d01_cut2}
\end{center}
\end{table} 
%%%%%%%%%%


%%%%%%%%%%%%%%%%%
% plot
\begin{figure}[!hbtp]
\centering
\includegraphics[width=.75\textwidth]{figures/table_limits_2j_cut_log.pdf}
\caption{Expected upper limits for SM Higgs in $\intlumiEightTeV$ at 8 TeV in the VBF channel. 
The cut-based result is used. }
\label{fig:uls_2j_cut}
\end{figure}
% table
\begin{table}[!htbp]
\begin{center}
\begin{tabular}{c c c c c}
\hline
\vspace{-3mm} && \\
Higgs Mass & Observed  & Median expected & Expected range for 68\% & Expected range for 95\%   \\
\hline
110 & -1.00 & 7.95 & [5.73, 11.06] & [4.27, 14.83] \\
115 & -1.00 & 4.67 & [3.37, 6.50] & [2.51, 8.71] \\
120 & -1.00 & 2.60 & [1.88, 3.62] & [1.40, 4.86] \\
125 & -1.00 & 1.74 & [1.25, 2.42] & [0.93, 3.24] \\
130 & -1.00 & 1.19 & [0.85, 1.65] & [0.64, 2.21] \\
135 & -1.00 & 0.96 & [0.70, 1.34] & [0.52, 1.80] \\
140 & -1.00 & 0.82 & [0.59, 1.15] & [0.44, 1.54] \\
150 & -1.00 & 0.57 & [0.41, 0.80] & [0.31, 1.07] \\
160 & -1.00 & 0.41 & [0.30, 0.58] & [0.22, 0.77] \\
170 & -1.00 & 0.44 & [0.32, 0.61] & [0.24, 0.82] \\
180 & -1.00 & 0.49 & [0.35, 0.68] & [0.26, 0.91] \\
190 & -1.00 & 0.63 & [0.45, 0.88] & [0.34, 1.18] \\
200 & -1.00 & 0.81 & [0.58, 1.12] & [0.43, 1.50] \\
250 & -1.00 & 1.36 & [0.98, 1.89] & [0.73, 2.53] \\
300 & -1.00 & 1.57 & [1.13, 2.19] & [0.84, 2.93] \\
350 & -1.00 & 1.67 & [1.20, 2.33] & [0.90, 3.12] \\
400 & -1.00 & 1.78 & [1.28, 2.48] & [0.96, 3.32] \\
450 & -1.00 & 1.85 & [1.33, 2.57] & [0.99, 3.45] \\
500 & -1.00 & 2.15 & [1.55, 2.98] & [1.15, 4.00] \\
550 & -1.00 & 2.53 & [1.82, 3.52] & [1.36, 4.71] \\
600 & -1.00 & 2.95 & [2.13, 4.11] & [1.58, 5.50] \\
\vspace{-3mm} && \\
\hline
\end{tabular}
\caption{Expected upper limits for SM Higgs in $\intlumiEightTeV$ at 8 TeV in the VBF channel. 
The cut-based result is used. }
\label{tab:uls_2j_cut}
\end{center}
\end{table} 
%%%%%%%%%%


%%%%%%%%%%%%%%%%%
% plot
\begin{figure}[!hbtp]
\centering
\includegraphics[width=.75\textwidth]{figures/table_limits_nj_cut_log.pdf}
\caption{Expected upper limits for SM Higgs in $\intlumiEightTeV$ at 8 TeV in all final states combined. 
Cut-based result is used. }
\label{fig:uls_cut}
\end{figure}
% table
\begin{table}[!htbp]
\begin{center}
\begin{tabular}{c c c c c}
\hline
\vspace{-3mm} && \\
Higgs Mass & Observed  & Median expected & Expected range for 68\% & Expected range for 95\%   \\
\hline 
110 & -1.00 & 78.80 & [56.77, 109.64] & [42.28, 146.98] \fixme \\ 
115 & -1.00 & 2.34 & [1.69, 3.26] & [1.26, 4.37] \\
120 & -1.00 & 1.34 & [0.97, 1.87] & [0.72, 2.51] \\
125 & -1.00 & 0.93 & [0.67, 1.29] & [0.50, 1.73] \\
130 & -1.00 & 0.69 & [0.49, 0.95] & [0.37, 1.28] \\
135 & -1.00 & 0.51 & [0.37, 0.71] & [0.27, 0.95] \\
140 & -1.00 & 0.41 & [0.29, 0.57] & [0.22, 0.76] \\
145 & -1.00 & 0.36 & [0.26, 0.50] & [0.19, 0.67] \\
150 & -1.00 & 0.26 & [0.19, 0.37] & [0.14, 0.49] \\
155 & -1.00 & 0.26 & [0.19, 0.37] & [0.14, 0.49] \\
160 & -1.00 & 0.16 & [0.12, 0.23] & [0.09, 0.31] \\
170 & -1.00 & 0.17 & [0.12, 0.23] & [0.09, 0.31] \\
180 & -1.00 & 0.22 & [0.16, 0.30] & [0.12, 0.40] \\
190 & -1.00 & 0.33 & [0.24, 0.45] & [0.18, 0.61] \\
200 & -1.00 & 0.43 & [0.31, 0.60] & [0.23, 0.81] \\
250 & -1.00 & 0.82 & [0.59, 1.14] & [0.44, 1.53] \\
300 & -1.00 & 0.91 & [0.66, 1.27] & [0.49, 1.70] \\
350 & -1.00 & 0.85 & [0.61, 1.18] & [0.46, 1.58] \\
400 & -1.00 & 0.91 & [0.66, 1.27] & [0.49, 1.70] \\
450 & -1.00 & 1.01 & [0.73, 1.40] & [0.54, 1.88] \\
500 & -1.00 & 1.27 & [0.91, 1.76] & [0.68, 2.36] \\
550 & -1.00 & 1.50 & [1.08, 2.08] & [0.80, 2.79] \\
600 & -1.00 & 1.76 & [1.27, 2.45] & [0.94, 3.28] \\
\vspace{-3mm} && \\
\hline
\end{tabular}
\caption{Expected upper limits for SM Higgs in $\intlumiEightTeV$ at 8 TeV in all final states combined. 
Cut-based result is used. }
\label{tab:ulscut}
\end{center}
\end{table} 
%%%%%%%%%%

%%%%%%%%%%%%%%%%%
% plot
\begin{figure}[!hbtp]
\centering
\includegraphics[width=.75\textwidth]{figures/table_limits_nj_shape_of_cut_log.pdf}
\caption{Expected upper limits for SM Higgs in $\intlumiEightTeV$ at 8 TeV. 
BDT result is used for OF 0/1jet bin and cut-based result is used for VBF channel 
and in the SF final states. }
\label{fig:uls_bdt01_cut2_cutsf}
\end{figure}
% table
\begin{table}[!htbp]
\begin{center}
\begin{tabular}{c c c c c}
\hline
\vspace{-3mm} && \\
Higgs Mass & Observed  & Median expected & Expected range for 68\% & Expected range for 95\%   \\
\hline
110 & -1.00 & 3.00 & [2.16, 4.18] & [1.61, 5.60] \\
115 & -1.00 & 1.70 & [1.22, 2.36] & [0.91, 3.17] \\
120 & -1.00 & 0.94 & [0.68, 1.31] & [0.51, 1.76] \\
125 & -1.00 & 0.64 & [0.46, 0.89] & [0.34, 1.19] \\
130 & -1.00 & 0.46 & [0.33, 0.64] & [0.25, 0.86] \\
135 & -1.00 & 0.36 & [0.26, 0.50] & [0.19, 0.67] \\
140 & -1.00 & 0.28 & [0.20, 0.39] & [0.15, 0.53] \\
145 & -1.00 & 0.40 & [0.29, 0.56] & [0.22, 0.76] \\
150 & -1.00 & 0.18 & [0.13, 0.26] & [0.10, 0.34] \\
155 & -1.00 & 0.18 & [0.13, 0.26] & [0.10, 0.34] \\
160 & -1.00 & 0.13 & [0.09, 0.18] & [0.07, 0.24] \\
170 & -1.00 & 0.14 & [0.10, 0.19] & [0.07, 0.26] \\
180 & -1.00 & 0.17 & [0.12, 0.23] & [0.09, 0.31] \\
190 & -1.00 & 0.24 & [0.17, 0.33] & [0.13, 0.44] \\
200 & -1.00 & 0.30 & [0.22, 0.42] & [0.16, 0.56] \\
250 & -1.00 & 0.61 & [0.44, 0.85] & [0.33, 1.14] \\
300 & -1.00 & 0.66 & [0.47, 0.91] & [0.35, 1.23] \\
350 & -1.00 & 0.61 & [0.44, 0.85] & [0.33, 1.14] \\
400 & -1.00 & 0.64 & [0.46, 0.89] & [0.34, 1.20] \\
450 & -1.00 & 0.70 & [0.50, 0.97] & [0.37, 1.30] \\
500 & -1.00 & 0.85 & [0.61, 1.18] & [0.46, 1.59] \\
550 & -1.00 & 0.98 & [0.71, 1.37] & [0.53, 1.84] \\
600 & -1.00 & 1.24 & [0.89, 1.72] & [0.66, 2.31] \\
\vspace{-3mm} && \\
\hline
\end{tabular}
\caption{Expected upper limits for SM Higgs in $\intlumiEightTeV$ at 8 TeV. 
BDT result is used for OF 0/1jet bin and cut-based result is used for VBF channel 
and in the SF final states. }
\label{tab:uls_bdt01_cut2_cutsf}
\end{center}
\end{table} 
%%%%%%%%%%

%%%%%%%%%%%%%%%%%
% plot
\begin{figure}[!hbtp]
\centering
\includegraphics[width=.75\textwidth]{figures/table_limits_nj_shape2d_of_cut_log.pdf}
\caption{Expected upper limits for SM Higgs in $\intlumiEightTeV$ at 8 TeV. 
2D result is used for OF 0/1jet bin and cut-based result is used for VBF channel 
and in the SF final states. }
\label{fig:uls_2d01_cut2_cutsf}
\end{figure}
% table
\begin{table}[!htbp]
\begin{center}
\begin{tabular}{c c c c c}
\hline
\vspace{-3mm} && \\
Higgs Mass & Observed  & Median expected & Expected range for 68\% & Expected range for 95\%   \\
\hline
110 & -1.00 & 64.05 & [46.14, 89.12] & [34.37, 119.47] \fixme  \\
115 & -1.00 & 1.55 & [1.12, 2.15] & [0.83, 2.89] \\
120 & -1.00 & 0.89 & [0.64, 1.23] & [0.48, 1.65] \\
125 & -1.00 & 0.57 & [0.41, 0.80] & [0.31, 1.07] \\
130 & -1.00 & 0.41 & [0.29, 0.57] & [0.22, 0.76] \\
135 & -1.00 & 0.32 & [0.23, 0.45] & [0.17, 0.60] \\
140 & -1.00 & 0.27 & [0.19, 0.37] & [0.14, 0.50] \\
145 & -1.00 & 0.40 & [0.29, 0.56] & [0.22, 0.76] \\
150 & -1.00 & 0.19 & [0.14, 0.27] & [0.10, 0.36] \\
155 & -1.00 & 0.19 & [0.14, 0.27] & [0.10, 0.36] \\
160 & -1.00 & 0.14 & [0.10, 0.20] & [0.08, 0.26] \\
170 & -1.00 & 0.15 & [0.11, 0.21] & [0.08, 0.28] \\
180 & -1.00 & 0.18 & [0.13, 0.25] & [0.10, 0.34] \\
190 & -1.00 & 0.26 & [0.18, 0.36] & [0.14, 0.48] \\
200 & -1.00 & 0.33 & [0.24, 0.46] & [0.18, 0.61] \\
250 & -1.00 & 0.58 & [0.41, 0.80] & [0.31, 1.07] \\
300 & -1.00 & 0.65 & [0.47, 0.91] & [0.35, 1.22] \\
350 & -1.00 & 0.59 & [0.42, 0.82] & [0.32, 1.10] \\
400 & -1.00 & 0.62 & [0.45, 0.87] & [0.33, 1.16] \\
450 & -1.00 & 0.74 & [0.53, 1.03] & [0.40, 1.38] \\
500 & -1.00 & 0.95 & [0.69, 1.33] & [0.51, 1.78] \\
550 & -1.00 & 1.12 & [0.81, 1.56] & [0.60, 2.09] \\
600 & -1.00 & 1.30 & [0.94, 1.81] & [0.70, 2.42] \\
\vspace{-3mm} && \\
\hline
\end{tabular}
\caption{Expected upper limits for SM Higgs in $\intlumiEightTeV$ at 8 TeV. 
2D result is used for OF 0/1jet bin and cut-based result is used for VBF channel 
and in the SF final states. }
\label{tab:uls_2d01_cut2_cutsf}
\end{center}
\end{table} 
%%%%%%%%%%

We also calculate the expected significance.
Results are summarized in Table~\ref{tab:significance_8TeV}.

\begin{table}[!htbp]
\begin{center}
\begin{tabular}{c | c c c c c c }
\hline 
\vspace{-3mm} && \\
Higgs Mass(\GeV) & (1) & (2) & (3) & (4) & (5) & (6)  \\
\hline \hline
115 & 1.2  	& 1.2 	& 0.6 & 1.0 	& 1.3	& 1.3 	\\
125 & 2.7  	& 3.1  	& 1.4 & 2.3		& 3.1	& 3.4	\\
140 & 6.2  	& 6.7 	& 2.5 & 4.7 	& 7.2	& 7.4 	\\
160 & 16.6 	& 14.3  & 4.1 & 11.2	& 18.4	& 16.1	\\
200 & 5.8 	& 5.4  	& 2.2 & 4.4 	& 6.4	& 6.0	\\
400 & 2.8 	& 2.7 	& 1.0 & 2.1		& 3.2	& 3.1	\\
600 & 1.4  	& 1.2 	& 0.7 & 1.2		& 1.7	& 1.5	\\
\hline
\end{tabular}
\caption{Expected significance SM Higgs in $\intlumiEightTeV$ at 8 TeV. (1) is BDT 0/1j OF + cut 2j OF. (2) is 2D 0/1j OF+ cut 2j OF. (3) is VBF cut SF+OF. (4) is cut OF+SF. (5) is BDT 0/1j OF + cut 2j OF + cut SF. (6) is 2D 0/1j OF + cut 2j OF + cut SF.} 
\label{tab:significance_8TeV}
\end{center}
\end{table} 

%%%%%%%%%%%%%%%%%%%%%%%%%%%%%%




%%%%%%%%%%%
\clearpage 

\subsection{Final Results for the Higgs Search Combing 7 TeV and 8 TeV Data}
\label{sec:search_results_finalcomb}

In this section we document the Higgs search results combining the 7 \TeV\ and 8 \TeV\ data.  
For the 0 and 1 Jet bin final states, the 7 TeV analysis uses the shape based approach for all 
lepton flavor final states, while the 8 TeV analysis uses the shape based approach only 
in the $e\mu$ channel. 
The expected and observed upper limits at 95\% C.L. are shown in Figure~\ref{fig:uls_finalcomb_shape} 
and Table~\ref{tab:uls_finalcomb_shape}.  We also calculate the expected significance in Table~\ref{tab:significance_78TeV}. 

\begin{figure}[!hbtp]
\centering
%\includegraphics[width=.75\textwidth]{figures/ana_ICHEP2012_finalcomb-CLs-asymptotic_log.pdf}
\caption{Expected and observed upper limits for SM Higgs combining the $\intlumiSevenTeV$ data
at 7 TeV and the $\intlumiEightTeV$ at 8 TeV.
For the 0 and 1 Jet bin final states, the 7 TeV analysis uses the shape based approach for all
lepton flavor final states, while the 8 TeV analysis uses the shape based approach only
in the $e\mu$ channel.
}
\label{fig:uls_finalcomb_shape}
\end{figure}


Appendix~\ref{app:appendix_limits_combination7and8} documents the 
results combining the shape based approach in 7 TeV and the cut based approach in 8 TeV, as shown 
in the ICHEP 2012 conference. 

\begin{table}[!htbp]
\begin{center}
\begin{tabular}{c | c c c  }
\hline 
\vspace{-3mm} && \\
Higgs Mass(\GeV) & (1) & (2) & (3) \\
\hline \hline
115 &  x	& x 	& x \\
125 &  x 	& x  	& x \\
140 &  x 	& x 	& x \\
160 &  x	& x 	& x	\\
200 &  x	& x 	& x	\\
400 &  x	& x 	& x \\
600 &  x	& x		& x \\
\hline
\end{tabular}
\caption{Expected significance for SM Higgs combining the $\intlumiSevenTeV$ data
at 7 TeV and the $\intlumiEightTeV$ at 8 TeV. 
For the 0 and 1 Jet bin final states, the 7 TeV analysis uses the shape based approach for all
lepton flavor final states. For (1) cut-based result in 8 TeV is used. 
For (2) BDT 0/1j OF + cut 2j OF + cut SF in 8 TeV is used. For (3) 2D 0/1j OF + cut 2j OF + cut SF is used. 
}
\label{tab:significance_78TeV}
\end{center}
\end{table} 

