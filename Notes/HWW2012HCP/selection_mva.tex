To make maximal use of the event information we have performed a multivariate analysis 
using a multivariate classifier based on the Boosted Decision Tree (BDT) technique. 
The BDT is implemented using the TMVA~\cite{tmva} toolkit and has been 
successfully applied in high energy physics to increase the 
statistical significance of a signal extraction
It requires less training than other multivariate classifiers and 
it is insensitive to the inclusion of poorly discriminating input variables.

To improve the search sensitivity, we apply a loose cut in addition to the 
$\WW$ preselection, $80<\mt<280~\GeV$ and $\mll<200~\GeV$ for $\mHi<300~\GeV$ and
$80<\mt<600~\GeV$ and $\mll<600~\GeV$ for $\mHi\ge300~\GeV$. The difference 
from the ICHEP analysis is that we apply only two sets of cuts for low/high 
mass hypotheses intead of cuts depending on every mass point. This approach gives 
larger lever arm to constraint backgrounds and gives better performance.

In addition to the selection variables for the cut-based analysis, the multivariate signal extraction 
procedure uses the following ones: 
\begin{itemize}
\item $\Delta R_{\Lep\Lep}\equiv\sqrt{\deletall^2 + \delphill^2}$ between the leptons, 
with $\deletall$ the $\eta$ difference between the leptons, 
which has similar properties as $\delphill$
\item lepton flavors ($\mu\mu$, $ee$, $e\mu$ or $\mu e$ );
\item finally, for the 1-jet bin, the azimutal angles between the dilepton 
system and $\met$, and between the dilepton system and the 
highest $\pt$ jet, are included.
\end{itemize}

The training has been carried out separately in the 0-jet and 1-jet bins 
for different Higgs masses using the corresponding signal samples. We use a new 
training with respect to last year~\cite{HWW2011}, and the full shape of the 
classifier output is used as final discriminant variable. 
