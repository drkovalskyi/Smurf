
The missing transverse energy is used to reject background events
where there is no natural source of missing energy, like in Drell-Yan and
QCD events. In the $\dytt$ process there is a large difference in the masses of 
$\tau$ and $\Z$. The taus are produced with large boost and their decay products, including 
neutrinos, are aligned with the leptons. Therefore a transverse component 
of missing energy with respect to the leptons is a better measure of true 
missing energy in the event, not originating from $\tau$ decay. 
To reject such background events with a small opening angle
between \met\ and one of the leptons, we used the projected \met~\cite{HWW2011} for 
event selection, defined as:
\begin{equation}
\text{with } \Delta\phi_{min} =  min(\Delta\phi(\ell_1,\met),\Delta\phi(\ell_2,\met))\\
\end{equation}
\begin{equation}
= 
\begin{cases} \met & \text{if $\Delta\phi_{min}>\frac{\pi}{2}$,}
\\
\met\sin(\Delta\phi_{min}) & \text{if $\Delta\phi_{min}<\frac{\pi}{2}$}
\end{cases}
\end{equation}
where $\Delta\phi(\ell_i,\met)$ is the angle between \met\ and lepton
 $i$ in the transverse plane. 
 In the presence of high multiple-interactions (pile-up), the instrumental \met\ tail in 
$\dyll$ events increases significantly. 

In the analysis, two \met\ definitions are used: PFMet and trk-MET.
PFMet is constructed from all PFCandidates in the event~\cite{PFMET}; in order to correct for the $\phi$ modulation observed both in data and in MC, 
we apply a ``x/y shift'' correction~\cite{xyShiftMetCorr}. 
The ``trk-MET''~\cite{trkMET} algorithm is a novel \met\ definition we have developed in order to 
improve the signal over background performance of \met\ selections in the presence of pile-up; 
it is constructed from charged particles consistent with originating from the primary vertex. 
The event $\met$ trk-MET is defined as 
\begin{equation}
\text{trk-MET} \equiv -\overrightarrow{p_T}(l_1) - \overrightarrow{p_T}(l_2) - \sum_i{\overrightarrow{p_T}(i)}, \\
\label{eq:trkmet}
\end{equation}
where $\overrightarrow{p_T}(l_1)$ and $\overrightarrow{p_T}(l_2)$ are the transverse momentum vectors of the two 
leptons passing the lepton selections described in Section~\ref{sec:sel_muons} and Section~~\ref{sec:sel_electrons}, 
and $\overrightarrow{p_T}(i)$ represent the tranverse momentum vectors of the charged PFCandidates satisfying the following requirements:
%%%%%%%%%%%%%%%%%%%%%%%%%%
\begin{itemize}
\item the track matched to PFCandidate has $\Delta z < 0.1$~cm with respect to the signal primary vertex;
\item the track has $\Delta R > 0.1$ with respect to both leptons, to avoid double-counting of the leptons.
\end{itemize}
%%%%%%%%%%%%%%%%%%%%%%%%%%

Comparing to the projected PFMet, we observed that the projected trk-MET has 
a larger tail in $\dyll$ background events~\cite{trkMET}. 
However these two \met\ values are weakly-correlated in $\dyll$ backgrounds with no geninue $\met$, and 
strongly correlated for the signal processes with geninue $\met$. 
Therefore the signal over background ratio is improved if we select the events 
based on the mininum of these two projected $\met$ values, $\text{min-MET} \equiv min(\text{proj}_\text{trk-MET}, \text{proj}_\text{PFMET})$. 

A baseline selection requiring $\textrm{min-MET} > 20~ \GeV$ is applied in all final states and jet bins.
As discussed in Section~\ref{sec:sel_sf}, tighter requirements to suppress the \dyll\ background are applied in the same-flavor final states.
