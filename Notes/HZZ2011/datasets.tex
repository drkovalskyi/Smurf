%UPDATEME%
The datasets used for this analysis are summarized in 
Tables.~\ref{tab:DatasetsData} and~\ref{tab:DatasetsMC} for data and Monte 
Carlo, respectively. The total integrated luminosity is \intlumi. 
We used the official good run list~\cite{json}. For Monte Carlo simulation 
we use madgraph when possible, but different generators such as Pythia and Powheg are also used. 
%For \wz\ and \zz\ processes we use Pythia, since MadGraph samples are mixed with $\WW$ in
%a single $VV$ sample, which is difficult to use properly. 
For $gg \to \WW$ a dedicated generator is used. 


\begin{table}[!ht]
\begin{center}
\begin{tabular}{|c|c|}
\hline
 Dataset Description                   &   Dataset Name   \\
\hline
\hline
\multicolumn{2}{|c|}{$H \to \ZZ$ Signal Selection Samples} \\
\hline
Run2011A DiElectron PromptReco      &  /DoubleElectron/Run2011A-PromptReco-v4/AOD   \\
Run2011A DiMuon PromptReco          &  /DoubleMu/Run2011A-PromptReco-v4/AOD   \\
Run2011A MuEl PromptReco            &  /MuEG/Run2011A-PromptReco-v4/AOD   \\
Run2011A SingleMuon PromptReco      &  /SingleMu/Run2011A-PromptReco-v4/AOD   \\
Run2011A SingleElectron PromptReco  &  /SingleElectron/Run2011A-PromptReco-v4/AOD   \\
Run2011A DiElectron May10ReReco      &  /DoubleElectron/Run2011A-May10ReReco-v1/AOD \\
Run2011A DiMuon May10ReReco          &  /DoubleMu/Run2011A-May10ReReco-v1/AOD \\
Run2011A MuEl May10ReReco            &  /MuEG/Run2011A-May10ReReco-v1/AOD \\
Run2011A SingleMuon May10ReReco      &  /SingleMu/Run2011A-May10ReReco-v1/AOD \\
Run2011A SingleElectron May10ReReco  &  /SingleElectron/Run2011A-May10ReReco-v4/AOD   \\
\hline
\hline
\multicolumn{2}{|c|}{Photon+Jets Samples} \\
\hline
Run2011A Photon May10ReReco         & /Photon/Run2011A-May10ReReco-v1/AOD \\
Run2011A Photon PromptReco          & /Photon/Run2011A-PromptReco-v4/AOD \\
\hline
\multicolumn{2}{|c|}{Fake Rate Measurement Samples} \\
\hline
Run2011A Jet  May10ReReco           & /Jet/Run2011A-May10ReReco-v1/AOD \\
Run2011A Jet  PromptReco            & /Jet/Run2011A-PromptReco-v4/AOD	\\

\hline
\end{tabular}
\caption{Summary of data datasets used. In general we use the combination of the May10ReReco and the PromptReco-v4 datasets, 
selecting the part the official lumis}
\label{tab:DatasetsData}
\end{center}
\end{table}

\begin{table}[!ht]
\begin{center}
{\footnotesize
\begin{tabular}{|c|c|c|}
\hline
\multicolumn{3}{|c|}{With Pileup: Processed dataset name is always} \\
\multicolumn{3}{|c|}{Summer11-PU\_S4\_START42\_V11-v*/AODSIM} \\
\hline
 Dataset Description                     &   Primary Dataset Name   & cross-section (pb)\\
\hline
%qq $\rightarrow WW$                  	 &   /VVJetsTo4L\_TuneD6T\_7TeV-madgraph-tauola                        &  43.0  \\
%$\ttbar$                              	 &   /TTJets\_TuneZ2\_7TeV-madgraph-tauola                             &  157.5 \\
%$\singletops$                  	 	 &   /TToBLNu\_TuneZ2\_s-channel\_7TeV-madgraph                        &  1.4 \\
%$\singletopt$                  	 	 &   /TToBLNu\_TuneZ2\_t-channel\_7TeV-madgraph                        &  20.9 \\
%tW                                    	 &   /TToBLNu\_TuneZ2\_tW-channel\_7TeV-madgraph                       &  10.6 \\
%WZ                               	 &   /WZtoAnything\_TuneZ2\_7TeV-pythia6-tauola                        &  18.2 \\
%ZZ                               	 &   /ZZtoAnything\_TuneZ2\_7TeV-pythia6-tauola                        &  7.4\\
%W/Z+$\gamma$                       	 &   /PhotonVJets\_7TeV-madgraph                                       &  165.0 \\
qq $\rightarrow WW$                  	 &   /WWJetsTo2L2Nu\_TuneZ2\_7TeV-madgraph-tauola                        &  X\fixme \\
gg $\rightarrow WW \to 2l 2\nu$          &   /GluGluToWWTo4L\_TuneZ2\_7TeV-gg2ww-pythia6                       &  0.153\\
$\ttbar$                              	 &   /TTTo2L2Nu2B\_7TeV-powheg-pythia6/                                 &  157.5 \\
$\singletops$                  	 	 &   /T\_TuneZ2\_s-channel\_7TeV-powheg-tauola                         &  1.4 \\
$\singletopt$                  	 	 &   /T\_TuneZ2\_t-channel\_7TeV-powheg-tauola                         &  20.9 \\
tW                                    	 &   /TToBLNu\_TuneZ2\_tW-channel\_7TeV-madgraph                       &  10.6 \\
Z[20-inf] $\rightarrow ee$	  	 &   /DYToEE\_M-20\_CT10\_TuneZ2\_7TeV-powheg-pythia                   &  1666.0 \\
Z[20-inf] $\rightarrow \mu\mu$        	 &   /DYToMuMu\_M-20\_CT10\_TuneZ2\_7TeV-powheg-pythia                 &  1666.0 \\
Z[20-inf] $\rightarrow \tau\tau$  	 &   /DYToTauTau\_M-20\_CT10\_TuneZ2\_7TeV-powheg-pythia-tauola        &  1666.0 \\
Z[10-20]  $\rightarrow ee$	  	 &   /DYToEE\_M-10To20\_CT10\_TuneZ2\_7TeV-powheg-pythia               &  3892.9 \\
Z[10-20]  $\rightarrow \mu\mu$    	 &   /DYToMuMu\_M-10To20\_CT10\_TuneZ2\_7TeV-powheg-pythia             &  3892.9 \\
Z[10-20]  $\rightarrow \tau\tau$  	 &   /DYToTauTau\_M-10To20\_CT10\_TuneZ2\_7TeV-powheg-pythia-tauola    &  3892.9 \\
W $\rightarrow$ $\ell\nu$           	 &   /WJetsToLNu\_TuneZ2\_7TeV-madgraph-tauola                         &  31314.0 \\
WZ                               	 &   /WZJetsTo3LNu\_TuneZ2\_7TeV-madgraph-tauola                       &  X\fixme \\
ZZ                               	 &   ZZJetsTo2L2Nu\_TuneZ2\_7TeV-madgraph-tauola                       &  X\fixme \\
$\gamma$ + Jets                          &   /G\_Pt\_15to3000\_TuneZ2\_Flat\_7TeV\_pythia6                     & 1.5e+07    \\
$gg \to H \to ZZ \to 2\ell2\nu$          &   /GluGluToHToZZTo2L2Nu\_M-*\_7TeV-powheg-pythia6                   & vary \\
$qq \to H \to ZZ \to 2\ell2\nu$          &   /VBF\_ToHToZZTo2L2NU\_M-[250-600]\_7TeV-powheg-pythia6            & vary \\
$gg \to H \to WW \to 2\ell2\nu$          &   /GluGluToHToWWTo2L2Nu\_M-*\_7TeV-powheg-pythia6                   & vary \\
$gg \to H \to WW \to \ell\tau2\nu$       &   /GluGluToHToWWTo2L2Nu\_M-*\_7TeV-powheg-pythia6                   & vary \\
$gg \to H \to WW \to 2\tau2\nu$          &   /GluGluToHToWWTo2Tau2Nu\_M-*\_7TeV-powheg-pythia6                 & vary \\
\hline
\end{tabular}
}
\caption{Summary of Monte Carlo datasets used.. The cross sections for a SM Higgs boson
is taken from the LHC Higgs cross-section working group~\cite{LHCHiggsCrossSectionWorkingGroup:2011ti}}
\label{tab:DatasetsMC}
\end{center}
\end{table}

Due to details in the implementation of the Powheg calculation in the Higgs MC, 
the Higgs $\pt$ spectrum for $gg \to H$ in MC has a much harder
spectrum compared with the most precise spectrum calculated to NNLO
with resummation to NNLL order, as illustrated in Ref.~\cite{HWW2011AN}. 
Therefore we reweight the MC simulated events according to the 
Higgs $\pt$ to match the spectrum obtained in NNLO calculation. 
This is done on the event-by-event level for both $gg \to H \to ZZ$ and 
$gg \to H \to WW$ processes. 

To account for the difference in the $p_{T}$ spectrum of the Z between the leading order Monte Carlo prediction
and the NLO prediction, we apply an event-by-event k-factor parameterized in the $p_{T}$ of the Z, given by:

\begin{eqnarray}
  \mathrm{KFactor}(p_{T\mathrm{ Z}}) = ( 1.108 + 0.002429 \times p_{T\mathrm{ Z}} - 1.655e-06 \times p_{T\mathrm{ Z}}^{2} ),
\end{eqnarray}

for the $qq \rightarrow ZZ$ process, computed using MCFM \cite{HZZ2011EPS}. To account for the $gg \rightarrow ZZ$ 
process we add to the ZZ cross section a term 
$\sigma_{gg \rightarrow ZZ} = 0.12 \times \sigma_{qq \rightarrow ZZ \mathrm{ , LO}}$. 
The total cross section for the ZZ process of $X\pb$\fixme given in 
Table \ref{tab:DatasetsMC} includes this term.

 
