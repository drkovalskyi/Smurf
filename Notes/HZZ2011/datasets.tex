%UPDATEME%
The datasets used for this analysis are summarized in 
Tables~\ref{tab:DatasetsData} and~\ref{tab:DatasetsMC} for data and Monte 
Carlo, respectively. The total integrated luminosity is \intlumi. 
We used the official good run list~\cite{json}. For Monte Carlo simulation 
we use MadGraph when available, but different generators such as Pythia and Powheg are also used. 
%For \wz\ and \zz\ processes we use Pythia, since MadGraph samples are mixed with $\WW$ in
%a single $VV$ sample, which is difficult to use properly. 
%For $gg \to \WW$ a dedicated generator is used. 


\begin{table}[!ht]
\begin{center}
\begin{tabular}{|l|l|}
\hline
 Dataset Description                   &   Dataset Name   \\
\hline
\hline
\multicolumn{2}{|c|}{$H \to \ZZ$ Signal Selection Samples} \\
\hline
Run2011A DiElectron May10ReReco      &  /DoubleElectron/Run2011A-May10ReReco-v1/AOD \\
Run2011A DiMuon May10ReReco          &  /DoubleMu/Run2011A-May10ReReco-v1/AOD \\
Run2011A MuEl May10ReReco            &  /MuEG/Run2011A-May10ReReco-v1/AOD \\
Run2011A SingleMuon May10ReReco      &  /SingleMu/Run2011A-May10ReReco-v1/AOD \\
Run2011A SingleElectron May10ReReco  &  /SingleElectron/Run2011A-May10ReReco-v1/AOD   \\

Run2011A DiElectron PromptReco      &  /DoubleElectron/Run2011A-PromptReco-v4/AOD   \\
Run2011A DiMuon PromptReco          &  /DoubleMu/Run2011A-PromptReco-v4/AOD   \\
Run2011A MuEl PromptReco            &  /MuEG/Run2011A-PromptReco-v4/AOD   \\
Run2011A SingleMuon PromptReco      &  /SingleMu/Run2011A-PromptReco-v4/AOD   \\
Run2011A SingleElectron PromptReco  &  /SingleElectron/Run2011A-PromptReco-v4/AOD   \\

Run2011A DiElectron Aug05ReReco      &  /DoubleElectron/Run2011A-05Aug2011-v1/AOD \\
Run2011A DiMuon Aug05ReReco          &  /DoubleMu/Run2011A-05Aug2011-v1/AOD \\
Run2011A MuEl Aug05ReReco            &  /MuEG/Run2011A-05Aug2011-v1/AOD \\
Run2011A SingleMuon Aug05ReReco      &  /SingleMu/Run2011A-05Aug2011-v1/AOD \\
Run2011A SingleElectron Aug05ReReco  &  /SingleElectron/Run2011A-05Aug2011-v1/AOD   \\

Run2011A DiElectron PromptReco      &  /DoubleElectron/Run2011A-PromptReco-v6/AOD   \\
Run2011A DiMuon PromptReco          &  /DoubleMu/Run2011A-PromptReco-v6/AOD   \\
Run2011A MuEl PromptReco            &  /MuEG/Run2011A-PromptReco-v6/AOD   \\
Run2011A SingleMuon PromptReco      &  /SingleMu/Run2011A-PromptReco-v6/AOD   \\
Run2011A SingleElectron PromptReco  &  /SingleElectron/Run2011A-PromptReco-v6/AOD   \\

Run2011B DiElectron PromptReco      &  /DoubleElectron/Run2011B-PromptReco-v1/AOD   \\
Run2011B DiMuon PromptReco          &  /DoubleMu/Run2011B-PromptReco-v1/AOD   \\
Run2011B MuEl PromptReco            &  /MuEG/Run2011B-PromptReco-v1/AOD   \\
Run2011B SingleMuon PromptReco      &  /SingleMu/Run2011B-PromptReco-v1/AOD   \\
Run2011B SingleElectron PromptReco  &  /SingleElectron/Run2011B-PromptReco-v1/AOD   \\
\hline
\hline
\multicolumn{2}{|c|}{Photon+Jets Samples} \\
\hline
Run2011A Photon May10ReReco         & /Photon/Run2011A-May10ReReco-v1/AOD \\
Run2011A Photon PromptReco          & /Photon/Run2011A-PromptReco-v4/AOD \\
Run2011A Photon May10ReReco         & /Photon/Run2011A-05Aug2011-v1/AOD \\
Run2011A Photon PromptReco          & /Photon/Run2011A-PromptReco-v6/AOD \\
Run2011B Photon PromptReco          & /Photon/Run2011B-PromptReco-v1/AOD \\

\hline
\end{tabular}
\caption{Summary of data datasets used.}
\label{tab:DatasetsData}
\end{center}
\end{table}

\begin{table}[!ht]
\begin{center}
{\footnotesize
\begin{tabular}{|c|c|c|}
\hline
\multicolumn{3}{|c|}{With Pileup: Processed dataset name is always} \\
\multicolumn{3}{|c|}{Summer11-PU\_S4\_START42\_V11-v*/AODSIM} \\
\hline
 Dataset Description                     &   Primary Dataset Name   & cross-section (pb)\\
\hline
\multicolumn{3}{|c|}{Default Generators} \\
\hline
%qq $\rightarrow WW$                  	 &   /VVJetsTo4L\_TuneD6T\_7TeV-madgraph-tauola                        &  43.0  \\
%$\ttbar$                              	 &   /TTJets\_TuneZ2\_7TeV-madgraph-tauola                             &  157.5 \\
%$\singletops$                  	 	 &   /TToBLNu\_TuneZ2\_s-channel\_7TeV-madgraph                        &  1.4 \\
%$\singletopt$                  	 	 &   /TToBLNu\_TuneZ2\_t-channel\_7TeV-madgraph                        &  20.9 \\
%tW                                    	 &   /TToBLNu\_TuneZ2\_tW-channel\_7TeV-madgraph                       &  10.6 \\
%WZ                               	 &   /WZtoAnything\_TuneZ2\_7TeV-pythia6-tauola                        &  18.2 \\
%ZZ                               	 &   /ZZtoAnything\_TuneZ2\_7TeV-pythia6-tauola                        &  7.4\\
%W/Z+$\gamma$                       	 &   /PhotonVJets\_7TeV-madgraph                                       &  165.0 \\
qq $\rightarrow WW$                  	 &   /WWJetsTo2L2Nu\_TuneZ2\_7TeV-madgraph-tauola                      &  4.783 \\
%gg $\rightarrow WW \to 2l 2\nu$          &   /GluGluToWWTo4L\_TuneZ2\_7TeV-gg2ww-pythia6                       &  0.153\\
$\ttbar$                              	 &   /TTTo2L2Nu2B\_7TeV-powheg-pythia6/                                &  157.5 \\
$t$ ($s$-chan)                 	 	 &   /T\_TuneZ2\_s-channel\_7TeV-powheg-tauola                         &  3.19 \\
$\bar{t}$ ($s$-chan)                 	 &   /Tbar\_TuneZ2\_s-channel\_7TeV-powheg-tauola                      &  1.44 \\
$t$ ($t$-chan)             	 	 &   /T\_TuneZ2\_t-channel\_7TeV-powheg-tauola                         &  41.92 \\
$\bar{t}$ ($t$-chan)                 	 &   /Tbar\_TuneZ2\_t-channel\_7TeV-powheg-tauola                      &  22.65 \\
$tW$                                     &   /T\_TuneZ2\_tW-channel-DR\_7TeV-powheg-tauola                     &  7.87 \\
$\bar{t}W$                               &   /Tbar\_TuneZ2\_tW-channel-DR\_7TeV-powheg-tauola                  &  7.87 \\
%Z[20-inf] $\rightarrow ee$	  	 &   /DYToEE\_M-20\_CT10\_TuneZ2\_7TeV-powheg-pythia                   &  1666.0 \\
%Z[20-inf] $\rightarrow \mu\mu$        	 &   /DYToMuMu\_M-20\_CT10\_TuneZ2\_7TeV-powheg-pythia                 &  1666.0 \\
Z[50-inf] $\rightarrow \ell\ell$         &   /DYJetsToLL\_TuneZ2\_M-50\_7TeV-madgraph-tauola                   &  3048.0 \\
Z[20-inf] $\rightarrow \tau\tau$  	 &   /DYToTauTau\_M-20\_CT10\_TuneZ2\_7TeV-powheg-pythia-tauola        &  1666.0 \\
%Z[10-20]  $\rightarrow ee$	  	 &   /DYToEE\_M-10To20\_CT10\_TuneZ2\_7TeV-powheg-pythia               &  3892.9 \\
%Z[10-20]  $\rightarrow \mu\mu$    	 &   /DYToMuMu\_M-10To20\_CT10\_TuneZ2\_7TeV-powheg-pythia             &  3892.9 \\
%Z[10-20]  $\rightarrow \tau\tau$  	 &   /DYToTauTau\_M-10To20\_CT10\_TuneZ2\_7TeV-powheg-pythia-tauola    &  3892.9 \\
W $\rightarrow$ $\ell\nu$           	 &   /WJetsToLNu\_TuneZ2\_7TeV-madgraph-tauola                         &  31314.0 \\
WZ                               	 &   /WZJetsTo3LNu\_TuneZ2\_7TeV-madgraph-tauola                       &  0.857 \\
ZZ                               	 &   /ZZJetsTo2L2Nu\_TuneZ2\_7TeV-madgraph-tauola                      &  0.179 \\%0.299 \\
$\gamma$ + Jets                          &   /G\_Pt\_15to3000\_TuneZ2\_Flat\_7TeV\_pythia6                     & 1.5e+07    \\
$gg \to H(250) \to ZZ \to 2\ell2\nu$     &   /GluGluToHToZZTo2L2Nu\_M-250\_7TeV-powheg-pythia6                 & 0.03974 \\
$qq \to H(250) \to ZZ \to 2\ell2\nu$     &   /VBF\_ToHToZZTo2L2NU\_M-250\_7TeV-powheg-pythia6                  & 0.0051643 \\
$gg \to H(300) \to ZZ \to 2\ell2\nu$     &   /GluGluToHToZZTo2L2Nu\_M-300\_7TeV-powheg-pythia6                 & 0.0300396 \\
$qq \to H(300) \to ZZ \to 2\ell2\nu$     &   /VBF\_ToHToZZTo2L2N\_M-300\_7TeV-powheg-pythia6                  & 0.0037345 \\
$gg \to H(350) \to ZZ \to 2\ell2\nu$     &   /GluGluToHToZZTo2L2Nu\_M-350\_7TeV-powheg-pythia6                 & 0.0286009 \\
$qq \to H(350) \to ZZ \to 2\ell2\nu$     &   /VBF\_ToHToZZTo2L2NU\_M-350\_7TeV-powheg-pythia6                  & 0.0026443 \\
$gg \to H(400) \to ZZ \to 2\ell2\nu$     &   /GluGluToHToZZTo2L2Nu\_M-400\_7TeV-powheg-pythia6                 & 0.0220830 \\
$qq \to H(400) \to ZZ \to 2\ell2\nu$     &   /VBF\_ToHToZZTo2L2NU\_M-400\_7TeV-powheg-pythia6                  & 0.0017606 \\
$gg \to H(500) \to ZZ \to 2\ell2\nu$     &   /GluGluToHToZZTo2L2Nu\_M-500\_7TeV-powheg-pythia6                 & 0.0089522 \\
$qq \to H(500) \to ZZ \to 2\ell2\nu$     &   /VBF\_ToHToZZTo2L2NU\_M-500\_7TeV-powheg-pythia6                  & 0.0010014 \\
$gg \to H(600) \to ZZ \to 2\ell2\nu$     &   /GluGluToHToZZTo2L2Nu\_M-600\_7TeV-powheg-pythia6                 & 0.0035900 \\
$qq \to H(600) \to ZZ \to 2\ell2\nu$     &   /VBF\_ToHToZZTo2L2NU\_M-600\_7TeV-powheg-pythia6                  & 0.0006342 \\
%$gg \to H \to WW \to 2\ell2\nu$          &   /GluGluToHToWWTo2L2Nu\_M-*\_7TeV-powheg-pythia6                   & vary \\
%$gg \to H \to WW \to \ell\tau2\nu$       &   /GluGluToHToWWTo2L2Nu\_M-*\_7TeV-powheg-pythia6                   & vary \\
%$gg \to H \to WW \to 2\tau2\nu$          &   /GluGluToHToWWTo2Tau2Nu\_M-*\_7TeV-powheg-pythia6                 & vary \\
\hline
\multicolumn{3}{|c|}{Different Generators Considered}\\
\hline
qq $\rightarrow WW$                  	 &   /WWTo2L2Nu\_TuneZ2\_7TeV\_pythia6\_tauola                         &  4.783 \\
WZ                               	 &   /WZTo3LNu\_TuneZ2\_7TeV\_pythia\_tauola                           &  0.596 \\
ZZ                               	 &   /ZZ\_TuneZ2\_7TeV\_pythia6\_tauola/                               &  7.406 \\
\hline
\end{tabular}
}
\caption{Summary of Monte Carlo datasets used.. The cross sections for a SM Higgs boson
is taken from the LHC Higgs cross-section working group~\cite{LHCHiggsCrossSectionWorkingGroup:2011ti}}
\label{tab:DatasetsMC}
\end{center}
\end{table}

Due to details in the implementation of the Powheg calculation in the Higgs MC, 
the Higgs $\pt$ spectrum for $gg \to H$ in MC has a much harder
spectrum compared with the most precise spectrum calculated to NNLO
with resummation to NNLL order, as illustrated in Ref.~\cite{HWW2011AN}. 
Therefore we reweight the MC simulated events according to the 
Higgs $\pt$ to match the spectrum obtained in NNLO calculation. 
This is done on the event-by-event level. 

For the $\qq\rightarrow ZZ$ process, the cross section listed in 
Table \ref{tab:DatasetsMC} corresponds to the LO calculation. 
We account for the NLO effects by applying an event-by-event k-factor parameterized in the $p_{T}$ of the Z, given by:
\begin{eqnarray}
  \mathrm{KFactor}(p_{T\mathrm{ Z}}) = ( 1.108 + 0.002429 \times p_{T\mathrm{ Z}} - 1.655\times10^{-6} \times p_{T\mathrm{ Z}}^{2} ).
\end{eqnarray}
%for the $qq \rightarrow ZZ$ process, computed using MCFM \cite{HZZ2011EPS}. 
We do not have the MC for the $gg\rightarrow ZZ$ process. The contribution in the 
LO is about $12\%$ of the $qq\rightarrow ZZ$ process. We assign a theoretical 
uncertainty of $20\%$. 
%The $gg \rightarrow ZZ$ process we add to the ZZ cross section a term 
%$\sigma_{gg \rightarrow ZZ} = 0.12 \times \sigma_{qq \rightarrow ZZ \mathrm{ , LO}}$. 

%Accouting for the efficiency loss in the NLO terms, the contribution of the $gg\rightarrow ZZ$ reduces to 
%$0.12/1.277$. 

 
