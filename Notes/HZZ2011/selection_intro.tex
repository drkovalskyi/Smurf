The $H \to ZZ \to 2\ell2\nu$ final state consists of two oppositely 
charged same flavor isolated leptons in the Z peak 
and large missing energy from the two undetectable neutrinos. 
%This is the same final states as in the $H \to \WW$ analysis~\cite{HWW2010}~\cite{HWW2011AN}. 
The Higgs cross-section is several orders of magnitude lower than
the major background processes: irreducible dibson processes \wz{} and \zz{}, $\WW$ and \ttbar{}, 
and $\dyll$. We thus perform several steps to select and extract the Higgs boson signal from data:

\begin{enumerate}
    \item We select events that pass pre-defined lepton triggers.
    \item We then select those events with two oppositely charged 
    high $\pt$ isolated leptons in the same flavor ($ee$ and $\mu\mu$) final states requiring:
        \begin{itemize}    
            \item $\pt>20~\GeVc$ for both leptons;
            \item standard identification and isolation requirements on both leptons;
             \item the dilepton invariant mass within $15~\GeV/cc$ of the $\Z$.
        \end{itemize}    
      \item We apply a common $\ZZ$ preselection, which requires in brief: 
         \begin{itemize}
             \item large transverse missing energy due to the neutrinos (\met$>50$ GeV);
             \item inconsistent with a top decay;
             \item the dilepton $\pt$ greater than 55 GeV to match the $\pt$ thresold used in the $\gamma$+Jet 
	dataset for the Drell-Yan data-driven background estimates.
             \item large transverse mass between the dilepton and \met\ to reject mainly the \dytt\ background ($M_T>150$ GeV);
          \end{itemize}
    \item We perform \emph{Higgs mass dependent} event selections 
%, one cut-based and one using a multivariate technique 
described in detail in Section~\ref{sec:signal_selection} for the cut-based and shape based analyses separately.
\end{enumerate}

%In addition the events are split into three categories according to the 
%number of reconstructed jets. We perform analysis in the corresponding 0-jet, 1-jet and 2-jet bins. 
%The $\ZZ$ preselection steps are now described in detail below.
