 
To determine the efficiency of the dilepton triggers, we derive the efficiency of the 
requirements imposed on each leg separately. A discussion about how to modify the
tag and probe method to handle the cases where the trigger objects are fully correlated
is provided in Ref.~\cite{HWW2011AN}. The double muon triggers require this special treament, as 
well as the double electron triggers starting from run 170826.

The per leg efficiency for the double electron triggers and the double muon triggers are given in 
Tables~\ref{tab:eff_trigger_ee} and~\ref{tab:eff_trigger_mm}, respectively. The single muon trigger
efficiencies are given in Table~\ref{tab:eff_trigger_smu}.
The listed values represent the overall efficiencies averaged over the run range
of the dataset, absorbing changes in thresholds and seeding requirements over time.
\vspace{10pt}

\begin{table}[!ht]
\begin{center}
\begin{tabular}{c|c|c|c}
\hline
Measurement  & $0<|\eta|<1$  & $1<|\eta|<1.5$ & $1.5<|\eta|<2.5$  \\ 
\hline
$20 < p_T < 30$   & $0.9969 \pm 0.0003$ & $0.9983 \pm 0.0004$ & $0.9985 \pm 0.0003$  \\ \hline
$30 < p_T < 40$   & $0.9973 \pm 0.0002$ & $0.9990 \pm 0.0002$ & $0.9976 \pm 0.0002$  \\ \hline
$40 < p_T < 50$   & $0.9981 \pm 0.0001$ & $0.9992 \pm 0.0001$ & $0.9976 \pm 0.0002$  \\ \hline
$50 < p_T < 100$  & $0.9985 \pm 0.0002$ & $0.9994 \pm 0.0003$ & $0.9978 \pm 0.0004$  \\ \hline
$p_T > 100$       & $0.9988 \pm 0.0016$ & $0.9967 \pm 0.0043$ & $0.9980 \pm 0.0045$  \\ \hline 
\end{tabular}
\caption{Per leg double electron trigger efficiency as a function of $p_T$ and $|\eta|$. The
efficiency is averaged over the different run-range specific triggers for the entire dataset.}
\label{tab:eff_trigger_ee}
\end{center}
\end{table}
%
%
%
\begin{table}[!ht]
\begin{center}
\begin{tabular}{c|c|c|c|c}
\hline
Measurement  & $0<|\eta|<0.8$  & $0.8<|\eta|<1.2$  & $1.2<|\eta|<2.1$ & $2.1<|\eta|<2.4$  \\ \hline
$20 < p_T < 30$   & $0.9781 \pm 0.0007$ & $0.9636 \pm 0.0014$ & $0.9613 \pm 0.0010$ & $0.9163 \pm 0.0026$  \\ \hline
$30 < p_T < 40$   & $0.9767 \pm 0.0005$ & $0.9612 \pm 0.0009$ & $0.9600 \pm 0.0007$ & $0.9156 \pm 0.0020$  \\ \hline
$40 < p_T < 50$   & $0.9775 \pm 0.0004$ & $0.9622 \pm 0.0008$ & $0.9602 \pm 0.0006$ & $0.9145 \pm 0.0021$  \\ \hline
$50 < p_T < 100$  & $0.9779 \pm 0.0008$ & $0.9596 \pm 0.0015$ & $0.9577 \pm 0.0011$ & $0.9112 \pm 0.0038$  \\ \hline
$p_T > 100$       & $0.9801 \pm 0.0044$ & $0.9579 \pm 0.0095$ & $0.9499 \pm 0.0079$ & $0.9459 \pm 0.0407$  \\ \hline
\end{tabular}
\caption{Per leg double muon trigger efficiency as a function of $p_T$ and $|\eta|$. The
efficiency is averaged over the different run-range specific triggers for the entire dataset.}
\label{tab:eff_trigger_mm}
\end{center}
\end{table}
%
%
%
\begin{table}[!ht]
\begin{center}
\begin{tabular}{c|c|c|c|c}
\hline
Measurement  & $0<|\eta|<0.8$  & $0.8<|\eta|<1.2$  & $1.2<|\eta|<2.1$ & $2.1<|\eta|<2.4$  \\ \hline
$20 < p_T < 30$   & $0.3261 \pm 0.0009$ & $0.2992 \pm 0.0012$ & $0.3009 \pm 0.0009$ & $0.2172 \pm 0.0015$  \\ \hline
$30 < p_T < 40$   & $0.8800 \pm 0.0005$ & $0.8044 \pm 0.0011$ & $0.7995 \pm 0.0008$ & $0.2580 \pm 0.0010$  \\ \hline
$40 < p_T < 50$   & $0.8895 \pm 0.0005$ & $0.8176 \pm 0.0009$ & $0.8189 \pm 0.0007$ & $0.2634 \pm 0.0011$  \\ \hline
$50 < p_T < 100$  & $0.8841 \pm 0.0009$ & $0.8142 \pm 0.0016$ & $0.8170 \pm 0.0012$ & $0.2627 \pm 0.0021$  \\ \hline
$p_T > 100$       & $0.8601 \pm 0.0052$ & $0.7854 \pm 0.0094$ & $0.7911 \pm 0.0076$ & $0.2496 \pm 0.0210$  \\ \hline
\end{tabular}
\caption{Single muon trigger efficiency as a function of $p_T$ and $|\eta|$. The
efficiency is averaged over the different run-range specific triggers for the entire dataset.}
\label{tab:eff_trigger_smu}
\end{center}
\end{table}
