The background due to top quarks differ from the $H \to ZZ$ signals in the 
presence of one or two additional bottom quarks. 
As explained in Section~\ref{sec:sel_toptag}, we use a dedicated top tagging 
veto, which relies on identifying $b$-quarks from top decay to 
further suppress the top background. 
To estimate the remaining top background in the signal region, we 
can use opposite flavor subtraction method as described below. 

For processes where the final states involving $\mu\mu$, $e\mu$ and $ee$ occur at equal 
rates, we can estimate the background contamination in the signal 
region (same flavor di-leptons) by counting opposite flavor di-lepton events. 
This method can be used to estimate the top (including $\ttbar$ and $\tw$), 
$\WW$ and $\ztt$ as well as $\wz$ and $\zz$ when the selected same 
flavor leptons do not come from the same Z boson. 
Electrons and muons have different reconstuction and selection 
efficiencies so this difference must be taken into account 
when making a prediction for same flavor dilepton yields.

%By assessing the tagging efficiency and applying this to the number of
%tagged events, we can estimate the residual top background after the veto.
%Because details of the jet fragmentation cannot be reliably simulated at 
%low energy, the tagging efficiency should be estimated from data where possible.

 



