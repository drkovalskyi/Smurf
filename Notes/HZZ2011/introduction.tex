The standard model (SM) of particle physics successfully describes the majority of high-energy
experimental data~\cite{pdg}. One of the key remaining questions is the origin of the masses of
$\W$~and $\Z$~bosons.  In the SM in its simplest implementation it is attributed to the spontaneous
breaking of electroweak symmetry caused by a new scalar field~\cite{Higgs1, Higgs2, Higgs3}. The
existence of the associated field quantum, the Higgs boson, has yet to be experimentally confirmed.

The $\Hi\to\ZZ$ channel is particularly sensitive for Higgs boson searches in the high mass 
range ($>300\GeVcc$)~\cite{dittmar}. This document describes the search for the Higgs boson 
in the $\Hi\To\ZZ\To \Lep\Lep\Nu\Nu$ channel, for Higgs boson masses in the range of 
$250-600~\GeVcc$ using the 2011 data. Signal and background yields are extrapolated to 
$5~\ifb$ of collected data to get projected sensitivity.
    
The main analysis strategy is to select events with two opposite charged leptons within 15 $\GeVcc$ 
of the $Z$ mass window and large missing energy. The two leptons are required to be isolated electrons 
or muons, of high transverse momenta ($\pt > 20\GeVcc$). 
The main physics objects used in this analysis are common to the selections defined in the 
$\Hi\To\ZZ$ Lepton photon analysis in Ref~\cite{hzzlppas}.  
We then perform the higgs mass dependent analyses to enhance the sensitivity to the Higgs boson signal 
to cover a large Higgs boson mass range $250-600~\GeVcc$, using both cut-and-count method and 
the method based on the shape of event kinematics. 
The analysis is carried out in the inclusive jet bin final states. 
%30 \GeVc and we perform the analysis in the corresponding zero, one and two jet bins. 
%The selected events are categories according to the number of reconstructed jets with $\pt$ above 
%30 \GeVc and we perform the analysis in the corresponding zero, one and two jet bins. 
%$\Hi\to\WW$ analysis~\cite{HWW2011AN}, 
%with detailed studies addressing higher instantaneous luminosity regime in 2011.  



The note is structured as follows. A discussion about the data samples used in the analysis is
presented in Section~\ref{sec:datasets}.  The trigger selection, lepton selection, and other
preselection requirements are described in detail in Section~\ref{sec:selection}.  
%A summary of the expected event yields based on Monte Carlo is shown in Section~\ref{sec:yields_mc}. 
The strategy for signal extraction is discussed in Section~\ref{sec:signal_selection}, followed by a description of
the techniques to estimate the backgrounds in Section~\ref{sec:backgrounds}. The signal efficiency
estimation is presented in Section~\ref{sec:alleff}.  All sources of systematic uncertainty are
shown in Section~\ref{sec:systematics}.  
%The sensitivity projections for a $1~\ifb$ sample are given in Section~\ref{sec:results_mc}. 
The results from \intlumi of data from 2011 are given in Section~\ref{sec:dataresults}.  
Finally, the conclusions are presented in Section~\ref{sec:summary}.

%An analysis of the $\Hi\To\ZZ\To \Lep\Lep\Nu\Nu$ channel has been approved by CMS for the EPS conference in 
%2011~\cite{HZZ2011EPS, HZZ2011EPSPAS}.  In the appendices, we cross check these results as validation of
%the new results presented in this note.
