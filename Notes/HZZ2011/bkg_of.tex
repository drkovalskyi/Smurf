
For processes where the final states involving $\mu^+\mu^-$, $e^{\pm}\mu^{\mp}$ and $e^+e^-$ occur at equal 
rates, we can estimate the yields in the signal region (same flavor di-leptons) by counting 
opposite flavor di-lepton events. This method can be used to estimate backgrounds from top (including $\ttbar$ and $\tw$), 
$\WW$, and $\ztt$, as well as $\wz$ and $\zz$ when the selected same flavor leptons do not come from the same 
$Z$ boson. Since the non-resonant contributions from $\wz$ and $\zz$ are small, we estimate these backgrounds
fully from Monte Carlo simulation. Hence, we restrict the scope of the opposite flavor method to
evaluating the contamination from top, $\WW$ and $\ztt$.

The method begins with a control sample of opposite flavor di-lepton events. The selection criteria are 
identical to those used for signal extraction except for the requirement that the final state is $e\mu$. 
The opposite flavor events would then be selected with kinematics and event variables that mimic the conditions
in which we select same flavor events. The predicted same flavor yields are determined from the opposite flavor yields as follows:

\begin{equation}
N_{ee} + N_{\mu\mu} = \frac{1}{2}N_{e\mu}\left(\frac{\varepsilon_{e}}{\varepsilon_{\mu}} + \frac{\varepsilon_{\mu}}{\varepsilon_{e}}\right),
\end{equation}

where $\varepsilon_{e}$ and $\varepsilon_{\mu}$ are the selection and reconstruction efficiencies for electrons and muons
respectively. The efficiencies are derived from $Z\rightarrow\ell\ell$ simulation using Tag-and-Probe and binned into barrel and endcap in order
to account for the different kinematics of the various backgrounds. 

There are contributions to the control sample in data from processes other than top, $\WW$, and $\ztt$; these are $\wz$, $\zz$ and 
$W+$jets. The $\wz$ and $\zz$ events are adequately modeled in the Monte Carlo and the yields are expected to be small and are neglected here. The $W+$jets backgrounds are negligible as well (see Section~\ref{sec:bkg_fakes}).

\subsubsection{Closure test in Monte Carlo simulation}

\begin{itemize}
\item TODO:
\begin{itemize}
    \item repeat closure test with new MC and inclusive jet bin
\end{itemize}
\end{itemize}

To check on the validity of the flavor symmetry assumption of top, $\WW$, and $\ztt$, a closure test is performed using 
Monte Carlo simulation. The test is to make a direct prediction of the $ee$ and $\mu\mu$ background in the signal 
region using opposite flavor events. Limited by the size of the Monte Carlo samples, this closure test is done
at the preselection level. 

The results of the closure test in the 0-jet, 1-jet, and $\geq$2-jets bins are 
tabulated in Tables~\ref{tab:ofmcj0}, ~\ref{tab:ofmcj1},~\ref{tab:ofmcj2}, showing the counted opposite flavor yields, the prediction 
for same flavor yields after efficiency corrections, and the counted same flavor yields. These tables show that flavor symmetry is 
valid for $\WW$ and $t\bar{t}$ and furthermore, that the required efficiency correction is a small effect. The $\ztt$ sample has too few 
events to make a definitive statement. The $tW$ does not appear to be flavor symmetric and further investigations are needed to understand 
why. Overall, the opposite flavor prediction and the same flavor expectation for the total background are within the statistical uncertainties 
of the simulation samples.


%%%%%%%%%%%%%
\begin{table}[!ht]
\begin{center}
\begin{tabular}{c|c|c|c}
\hline
Process & $N_{e\mu}$ & Predicted $N_{ee/\mu\mu}$ & $N_{ee/\mu\mu}$ \\
\hline
$t\bar{t}$  & $1.375 \pm 0.435$  & $1.421 \pm 0.451$  & $1.375 \pm 0.435$ \\
$tW$        & $0.482 \pm 0.103$  & $0.489 \pm 0.104$  & $0.876 \pm 0.139$ \\
$\WW$       & $13.398 \pm 0.285$ & $13.632 \pm 0.291$ & $14.016 \pm 0.292$ \\
$\ztt$      & $0.838 \pm 0.838$  & $0.845 \pm 0.845$  & $0.000 \pm 0.000$ \\
\hline
Total       & $16.093 \pm 0.992$ & $16.387 \pm 1.006$ & $16.267 \pm 0.542$ \\
\hline
\end{tabular}
\caption{Results of the closure test in the 0-jet bin.}
\label{tab:ofmcj0}
\end{center}
\end{table}
%%%%%%%%%%%%%

%%%%%%%%%%%%%
\begin{table}[!ht]
\begin{center}
\begin{tabular}{c|c|c|c}
\hline
Process & $N_{e\mu}$ & Predicted $N_{ee/\mu\mu}$ & $N_{ee/\mu\mu}$ \\
\hline
$t\bar{t}$  & $3.850 \pm 0.727$ & $3.923 \pm 0.743$ & $4.125 \pm 0.753$ \\
$tW$        & $1.095 \pm 0.155$ & $1.116 \pm 0.158$ & $1.314 \pm 0.170$ \\
$\WW$       & $4.527 \pm 0.166$ & $4.617 \pm 0.170$ & $4.635 \pm 0.168$ \\
$\ztt$      & $0.838 \pm 0.838$ & $0.845 \pm 0.845$ & $0.000 \pm 0.000$ \\
\hline
Total       & $10.310 \pm 1.132$ & $10.501 \pm 1.149$ & $10.074 \pm 0.790$ \\
\hline
\end{tabular}
\caption{Results of the closure test in the 1-jet bin.}
\label{tab:ofmcj1}
\end{center}
\end{table}
%%%%%%%%%%%%%

%%%%%%%%%%%%%
\begin{table}[!ht]
\begin{center}
\begin{tabular}{c|c|c|c}
\hline
Process & $N_{e\mu}$ & Predicted $N_{ee/\mu\mu}$ & $N_{ee/\mu\mu}$ \\
\hline
$t\bar{t}$  & $7.974 \pm 1.047$ & $8.139 \pm 1.072$ & $7.149 \pm 0.991$ \\
$tW$        & $0.723 \pm 0.126$ & $0.732 \pm 0.128$ & $0.526 \pm 0.107$ \\
$\WW$       & $1.697 \pm 0.103$ & $1.721 \pm 0.105$ & $1.628 \pm 0.101$ \\
$\ztt$      & $0.000 \pm 0.000$ & $0.000 \pm 0.000$ & $0.000 \pm 0.000$ \\
\hline
Total       & $10.394 \pm 1.060$ & $10.592 \pm 1.085$ & $9.303 \pm 1.002$ \\
\hline
\end{tabular}
\caption{Results of the closure test in the $\geq$2-jets bin.}
\label{tab:ofmcj2}
\end{center}
\end{table}
%%%%%%%%%%%%%

\subsubsection{Application in $M_T$ Shape Analysis}

In the $M_T$ shape analysis, different procedures are applied to quantify
the uncertainties on the $WW$ and Top shapes. 
These are described in more detail in Section \ref{sec:anal_mt}.
To set the normalisation of the two processes together in a way
that is consistent with the cut-based analysis, we use the method 
described in the previous section to obtain simulation-to-data scale 
factors. These scale factors are extracted separtately for each
Higgs boson mass dependent shape selection, and assumed to be equal
in their application to the $WW$ and Top normalisations.
The magnitude of these scale factors and their uncertainties 
are given in Table \ref{tab:shape_sf_emu}.

\begin{table}[!ht]
\begin{center}
\small{
\begin{tabular}{c|c|c|c|c} 
\hline
Mass &   $WW$ MC           & Top MC            & Scaled Data ($e\mu$)   & SF \\
\hline \hline
\multicolumn{5}{c}{Muon Channel} \\ 
\hline \hline
250 & $12.74 \pm 0.49$ & $17.08 \pm 0.53$ & $30.34 \pm 4.10$  & $1.0173 \pm 0.1395$ \\ \hline 
300 & $5.57 \pm 0.32$ & $8.61 \pm 0.37$ & $18.93 \pm 3.25$  & $1.3352 \pm 0.2339$ \\ \hline 
350 & $5.61 \pm 0.33$ & $8.72 \pm 0.37$ & $18.93 \pm 3.25$  & $1.3211 \pm 0.2314$ \\ \hline 
400 & $5.61 \pm 0.33$ & $8.73 \pm 0.37$ & $18.93 \pm 3.25$  & $1.3204 \pm 0.2313$ \\ \hline 
500 & $5.61 \pm 0.33$ & $8.73 \pm 0.37$ & $18.93 \pm 3.25$  & $1.3204 \pm 0.2313$ \\ \hline 
600 & $5.61 \pm 0.33$ & $8.73 \pm 0.37$ & $18.93 \pm 3.25$  & $1.3204 \pm 0.2313$ \\ 
\hline \hline
\multicolumn{5}{c}{Electron Channel} \\ 
\hline \hline
250 & $10.13 \pm 0.44$ & $12.00 \pm 0.45$ & $24.38 \pm 3.30$  & $1.1014 \pm 0.1522$ \\ \hline
300 & $3.92 \pm 0.27$ & $6.18 \pm 0.34$ & $15.27 \pm 2.63$  & $1.5109 \pm 0.2677$ \\ \hline
350 & $4.00 \pm 0.28$ & $6.33 \pm 0.34$ & $15.27 \pm 2.63$  & $1.4790 \pm 0.2620$ \\ \hline
400 & $4.00 \pm 0.28$ & $6.34 \pm 0.34$ & $15.27 \pm 2.63$  & $1.4769 \pm 0.2616$ \\ \hline
500 & $4.02 \pm 0.28$ & $6.34 \pm 0.34$ & $15.27 \pm 2.63$  & $1.4740 \pm 0.2611$ \\ \hline
600 & $4.02 \pm 0.28$ & $6.34 \pm 0.34$ & $15.27 \pm 2.63$  & $1.4740 \pm 0.2611$ \\ \hline 
\end{tabular}
\caption{The number of WW and Top events estimated from data and MC, along with the data-to-simulation 
scale factors for shape analysis derived using the $e\mu$ estimate in data.
The $WW$ and Top MC are compared to the expected yield from $e\mu$ data scaled by the expected muon
and electron reconstruction and trigger efficiencies.}
\label{tab:shape_sf_emu}}
\end{center}
\end{table}




