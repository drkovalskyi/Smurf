\subsubsection{Method using opposite flavor di-leptons}

For processes where the final states involving $\mu^+\mu^-$, $e^{\pm}\mu^{\mp}$ and $e^+e^-$ occur at equal 
rates, we can estimate the yields in the signal region (same flavor di-leptons) by counting 
opposite flavor di-lepton events. This method can be used to estimate backgrounds from top (including $\ttbar$ and $\tw$), 
$\WW$, and $\ztt$, as well as $\wz$ and $\zz$ when the selected same flavor leptons do not come from the same 
$Z$ boson. Since the non-resonant contributions from $\wz$ and $\zz$ are small, we estimate these backgrounds
fully from Monte Carlo simulation. Hence, we restrict the scope of the opposite flavor method to
evaluating the contamination from top, $\WW$ and $\ztt$.

The method begins with a control sample of opposite flavor di-lepton events. Ideally, the selection criteria are 
identical to those used for signal extraction except for the requirement that the final state is $e\mu$. 
The opposite flavor events would then be selected with kinematics and event variables that mimic the conditions
in which we select same flavor events. However, due to the flavor-symmetric nature of the top, $\WW$, and $\ztt$, 
this means the control sample is roughly the same size as the background contribution to the signal region and 
is therefore statistically limited with only 187~$\ipb$. To make our background estimations a little more significant, 
we relax the selection requirement as follows:
\begin{enumerate}
\item $m_{e\mu} > 12\:\GeVcc$,
\item no $M_{T}$ requirement,
\item Min(PF-MET,track-MET) $>60\:\GeV$ for all jet multiplicity bins. 
\end{enumerate}
With this control sample definition we derive a scale factor from the yields in data and in simulation. This is then 
used to rescale the $ee$ and $\mu\mu$ Monte Carlo expectations to obtain the background estimate. 

There are contributions to the control sample in data from processes other than top, $\WW$, and $\ztt$; these are $\wz$, $\zz$ and 
$W+$jets. The $\wz$ and $\zz$ events are adequately modeled in the Monte Carlo and the yields are expected to be small. We subtract
the expected $\wz$ and $\zz$ yields from our data control sample. The $W+$jets background is to be estimated with the
fake rate method but for now we perform the subtraction on the data control sample with the Monte Carlo expectation.

When extracting the control region with Monte Carlo samples, an event by event correction is applied to
account for pileup and lepton efficiencies. The ratio of the data yield and the corrected simulations is taken as the scale
factor. 

\subsubsection{Closure test in Monte Carlo simulation}

As a check on the validity of the flavor symmetry assumption of top, $\WW$, and $\ztt$, a closure test is performed using 
Monte Carlo simulation. The test is to make a direct prediction of the $ee$ and $\mu\mu$ background in the signal 
region using opposite flavor events. Limited by the size of the Monte Carlo samples, this closure test is done
at the preselection level. In particular, the missing energy requirement is Min(PF-MET,track-MET) $>50\:\GeV$ and
$m_{e\mu}$ must lie within the $Z$ mass window. Electrons and muons have different reconstuction and selection 
efficiencies so this difference must be taken into account when making a prediction for same flavor di-lepton yields. The 
predicted same flavor yields is determined from the opposite flavor yields as follows:
\begin{equation}
N_{ee} + N_{\mu\mu} = \frac{1}{2}N_{e\mu}\left(\frac{\varepsilon_{e}}{\varepsilon_{\mu}} + \frac{\varepsilon_{\mu}}{\varepsilon_{e}}\right),
\end{equation}
where $\varepsilon_{e}$ and $\varepsilon_{\mu}$ are the selection and reconstruction efficiencies for electrons and muons
respectively.
