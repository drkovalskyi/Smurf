
To enhance the sensitivity to the Higgs boson signal, we apply additional 
$m_{\rm{H}}$ hypothesis dependent selections on the 
minimum $\met$ and Higgs transverse mass ($M_{T}$) observables, 
covering a large Higgs boson mass ($m_{\rm{H}}$) range. 

The transverse mass is defined by

\begin{equation}
M_{T}^{2} = (\sqrt{p_{T\mathrm{ ll}}^{2} + M_{ll}^{2}} + \sqrt{(\met)^{2} + M_{ll}^{2}})^{2} - (\vec{p_{T\mathrm{ ll}}} + \vec{\met})^{2}. \\
\label{eq:MTHZZ}
\end{equation}

The higgs selection cuts are optimized for the significance ($S/\sqrt{S+B}$)
separately in the 0-jet and 1-jet. The values of the cuts are summarized in
Tables \ref{tab:HiggsSelectionCutBased_0j} and \ref{tab:HiggsSelectionCutBased_1j} for the 0-jet and 1-jet bins. 
A one sided cut is applied on the minimum \met and a two sided cut is applied on the transverse mass. 

%%%%%%%%%%%%%
\begin{table}[!ht]
\begin{center}
\begin{tabular}{c|c|c|c}
\hline
Higgs Mass        & Min $\met$ Cut Value  & Min $\mt$ Cut Value   & Max $\mt$ Cut Value \\ 
\hline 
%200               & $> 50$ GeV            & $> 180$ GeV            & $< 220$ GeV          \\ \hline 
250               & $> 60$ GeV            & $> 220$ GeV            & $< 260$ GeV          \\ \hline 
300               & $> 70$ GeV            & $> 260$ GeV            & $< 320$ GeV          \\ \hline 
400               & $> 70$ GeV            & $> 260$ GeV            & $< 320$ GeV          \\ \hline 
\end{tabular}
\caption{Expected number of signal and background events for an 
  integrated luminosity of 1\ifb{} after applying the \zz\ 
  0-jet selection requirements. Monte Carlo statistical 
  uncertainties are included.}
\label{tab:HiggsSelectionCutBased_0j}
\end{center}
\end{table}
%%%%%%%%%%%%%

%%%%%%%%%%%%%
\begin{table}[!ht]
\begin{center}
\begin{tabular}{c|c|c|c}
\hline
Higgs Mass        & Min $\met$ Cut Value  & Min $\mt$ Cut Value   & Max $\mt$ Cut Value \\ 
\hline 
%200               & $> 60$ GeV            & $> 180$ GeV            & $< 200$ GeV          \\ \hline 
250               & $> 60$ GeV            & $> 180$ GeV            & $< 260$ GeV          \\ \hline 
300               & $> 100$ GeV           & $> 260$ GeV            & $< 320$ GeV          \\ \hline 
400               & $> 100$ GeV           & $> 300$ GeV            & $< 450$ GeV          \\ \hline 
\end{tabular}
\caption{Expected number of signal and background events for an 
  integrated luminosity of 1\ifb{} after applying the \zz\ 
  1-jet selection requirements. Monte Carlo statistical 
  uncertainties are included.}
\label{tab:HiggsSelectionCutBased_1j}
\end{center}
\end{table}
%%%%%%%%%%%%%

%%%%%%%%%%%%%
%\begin{table}[!ht]
%\begin{center}
%\begin{tabular}{c|c|c|c}
%\hline
%Higgs Mass        & Min $\met$ Cut Value  & Min $\mt$ Cut Value   & Max $\mt$ Cut Value \\ 
%\hline 
%%200               & $> 80$ GeV            & $> 180$ GeV            & $< 200$ GeV          \\ \hline 
%250               & $> 80$ GeV            & $> 180$ GeV            & $< 260$ GeV          \\ \hline 
%300               & $> 100$ GeV           & $> 240$ GeV            & $< 320$ GeV          \\ \hline 
%400               & $> 100$ GeV           & $> 300$ GeV            & $< 450$ GeV          \\ \hline 
%\end{tabular}
%\caption{Expected number of signal and background events for an 
%  integrated luminosity of 1\ifb{} after applying the \zz\ 
%  2-jet selection requirements. Monte Carlo statistical 
%  uncertainties are included.}
%\label{tab:HiggsSelectionCutBased_2j}
%\end{center}
%\end{table}
%%%%%%%%%%%%%

Due to the missing VBF production samples for the signal in our Spring11 set of 
Monte Carlo samples,  we did not include the yields for the VBF contribution. 
In Table \ref{tab:VBFSignalContribution} we summarize the expected VBF contribution to the signal yield
estimated from the Summer11 Monte Carlo samples.
We do not perform any scaling based on these expectations.
Because we could not make a meaningful choice of cuts for the 2-jet bin,
we temporarily dropped it from this analysis.

%%%%%%%%%%%%%
\begin{table}[!ht]
\begin{center}
\begin{tabular}{|c|c|c|}
\hline
Higgs Mass        & Relative VBF Contribution in 0-Jet Bin & Relative VBF Contribution in 1-Jet Bin \\ 
\hline 
250               & $1.8\%$                                & $11\%$                                 \\ 
300               & $1.9\%$                                & $9.8\%$                                \\ 
\hline 
\end{tabular}
\caption{Expected relative contribution from the VBF production process to the total
expected signal yield.}
\label{tab:VBFSignalContribution}
\end{center}
\end{table}
%%%%%%%%%%%%%


%% The quoted results in this section are scaled to 1 $\ifb$ of integrated luminosity. 
%% The background yields have been scaled taking into account the observed data 
%% corrections, discussed in Section~\ref{sec:backgrounds}, with the current data 
%% sample to give realistic estimations, while $gg \to H \to ZZ$ simulated 
%% events are reweighted to match the Higgs $\pt$ at NNLO, as explained in 
%% Section~\ref{sec:datasets}. 

%% Stuff to add in the MVA later
%To enhance the sensitivity to the Higgs boson signal, two different approaches 
%are performed. The first one is a cut-based approach where further requirements 
%on a few observables are applied, while the second one makes use of
%multivariate techniques. Both of them cover a large Higgs boson mass
%($m_{\rm{H}}$) range, and each is separately optimized for different
%$m_{\rm{H}}$ hypotheses. The first method is the simplest approach with smaller
%systematic uncertainties. The second one is
%more powerful, since it exploits the information present in the
%correlation among the variables. 

%Output of the multivariate discriminator has two different use
%cases. In the first case we use it as just one more variable to cut on
%in the cut-based analysis. In the second case we use the discriminator
%output distribution for the final signal extraction.

%All analyses are further split in the corresponding 0-jet, 1-jet and
%2-jet bins. In the 2-jet bin we use a simple cut-based approach for
%now due to the limited sensitivity and the limited number of events in
%simulation.

