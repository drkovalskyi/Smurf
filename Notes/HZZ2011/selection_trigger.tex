Triggering on Higgs boson decays in the dilepton final state increases 
in difficulty with increasing instantaenous luminosity.
Single lepton triggers can only be sustained with very tight identification and
isolation requirements and large transverse momentum thresholds.
This means that double lepton triggers are the only sustainable option. 
%to maintain sensitivity to a low mass Higgs boson, where the leptons transverse momentum
%can be small.
We use a suite of signal and control triggers designed in the $H \to \WW$ analysis~\cite{HWW2011AN}. 
These dilepton triggers have a high efficiency to collect Higgs boson events
and are sufficiently loose to collect control events to estimate the 
selection efficiencies and with adequate precision.
%We describe the features and motivations for the analysis triggers in Section~\ref{sec:mainTriggers},
%and additional triggers to collect control events in 
%Section~\ref{sec:utilityTriggers}.

%\subsubsection{Analysis Triggers}
%\label{sec:mainTriggers}

The main dielectron triggers, described in Table~\ref{tab:triggers_ee}, require two HLT electron
candidates with loose shower shape and calorimeter isolation requirements on both legs
and a match to a Level-1 seed for the leading leg.
To accomodate the offline selection of $E_{T}>20,20$~GeV for the leading and trailing
electrons, $E_{T}>17,8$~GeV is required at the HLT level.
The main dimuon triggers
require two HLT muon candidates with transverse momentum greater than $7$~$\GeVc$ and
a match to a Level-1 seed is required for both legs.
These are described in Table~\ref{tab:triggers_mm}.

%%%%%%%%%
\begin{table}[!ht]
  \caption{Analysis triggers for the $ee$ final state. 
The identification and isolation requirements are described in Ref.~\cite{HWW2011AN}.}
    \vspace{5pt}
   \label{tab:triggers_ee}
  \begin{center}
 {\small
  \begin{tabular} {l|l|l|c}
\hline
  Dataset & Trigger name & L1 seed & Description\\
  \hline \hline
  \multirow{2}{*}{DoubleElectron} & HLT\_Ele17\_CaloIdL\_CaloIsoVL\_&  L1\_SingleEG12  & $p_T>17,8~\GeVc$ \\
                                  & Ele8\_CaloIdL\_CaloIsoVL &                  & \\

  \multirow{2}{*}{DoubleElectron} & HLT\_Ele17\_CaloIdT\_TrkIdVL\_CaloIsoVL\_TrkIsoVL\_ &  L1\_SingleEG12  & $p_T>17,8~\GeVc$ \\
                                  & Ele8\_CaloIdT\_TrkIdVL\_CaloIsoVL\_TrkIsoVL &                  & \\
  \hline
  \end{tabular}
}
  \end{center}
\end{table}
%%%%%%%%%

%%%%%%%%%
\begin{table}[!ht]
  \caption{Analysis triggers for the $\mu\mu$ final state. Triggers marked (*) are also used for efficiency studies.}
    \vspace{5pt}
   \label{tab:triggers_mm}
  \begin{center}
 {\small
  \begin{tabular} {l|l|l|c}
\hline
  Dataset & Trigger name & L1 seed & Description\\
  \hline \hline
  DoubleMu & HLT\_DoubleMu6 & L1\_DoubleMu3  & $p_T>6,6~\GeVc$\\
  DoubleMu & HLT\_DoubleMu7 & L1\_DoubleMu3  & $p_T>7,7~\GeVc$ \\
  DoubleMu & HLT\_Mu13\_Mu8 & L1\_DoubleMu3  & $p_T>13,8~\GeVc$ \\
  DoubleMu & HLT\_Mu17\_Mu8 & L1\_DoubleMu3  & $p_T>17,8~\GeVc$ \\
  \hline
  \end{tabular}
}
  \end{center}
\end{table}
%%%%%%%%%

%%%%%
\begin{table}[!ht]
  \caption{Single lepton triggers to recover lost efficiency. These trigges are also used for efficiency studies.
The identification and isolation requirements for electrons are described in Ref.~\cite{HWW2011AN}. }
    \vspace{5pt}
   \label{tab:triggers_single}
  \begin{center}
 {\small
  \begin{tabular} {l|l|l|c}
\hline
  Dataset & Trigger name & L1 seed & Description\\
  \hline \hline
  SingleEle & HLT\_Ele27\_CaloIdVT\_CaloIsoT\_TrkIdT\_TrkIsoT & L1\_SingleEG15  & $p_T>27~\GeVc$ \\
  \hline \hline
  SingleMu & HLT\_IsoMu12   & L1\_SingleMu7  & $p_T>12~\GeVc$ \\
  SingleMu & HLT\_IsoMu17   & L1\_SingleMu10 & $p_T>17~\GeVc$ \\
  SingleMu & HLT\_Mu15      & L1\_SingleMu10 & $p_T>15~\GeVc$ \\
  \hline 
  \end{tabular}
}
  \end{center}
\end{table}
%%%%%


%\subsubsection{Utility Triggers}
%\label{sec:utilityTriggers}

%We now describe additional triggers that are introduced to collect control or
%calibration events not covered by the main analysis triggers.

To estimate the non-$Z$ dilepton backgrounds, where the selected dileptons are 
not from the $Z$ details, we need to use the events collected in the $e\mu$ final sates. 
In the electron-muon channel we use two complementary triggers, which require
both muon and electron HLT candidates.
These are summarised in Table~\ref{tab:triggers_em}.
Finally, to recover any residual inefficiency,
we also allow events that passed only the single electron
or single isolated muon triggers summarised in Table~\ref{tab:triggers_single}.

%%%%%
\begin{table}[!ht]
  \caption{Analysis triggers for the $e\mu$ final state.
The identification and isolation requirements for electrons are described in Ref.~\cite{HWW2011AN}.}
    \vspace{5pt}
   \label{tab:triggers_em}
  \begin{center}
 {\small
  \begin{tabular} {l|l|l|c}
\hline
  Dataset & Trigger name & L1 seed & Description\\
  \hline \hline
  MuEG & HLT\_Mu17\_Ele8\_CaloIdL & L1\_Mu3\_EG5 & $p_T>17,8~\GeVc$ \\
  MuEG & HLT\_Mu8\_Ele17\_CaloIdL & L1\_Mu3\_EG5 & $p_T>8,17~\GeVc$ \\
 \hline
  \end{tabular}
}
  \end{center}
\end{table}
%%%%%

Because the main dielectron analysis triggers make requirements on
both legs, events collected with those triggers cannot be used to measure
efficiencies without introducing unacceptable bias.
Thus, to measure the electron selection and trigger efficiency
we introduce two specialised tag and probe triggers designed to maximize
the number of useful \dyll~events, % for both low and high $p_{T}$ electrons,
while keeping the total trigger rate at a reasonable level. 
%The tag and probe method is described later in Section~\ref{sec:efficiency}.

%The first trigger probes low $p_T$ electrons and applies very tight identification 
%and isolation requirements on the tag leg to reduce the background rate.
%The second trigger probes higher $p_{T}$ electrons.
%Both triggers are described in Table~\ref{tab:triggers_util} and labeled "eff".

Table~\ref{tab:triggers_photon} shows the list of the triggers used to 
collect the $\gamma$ plus jet events to be used in the data-driven estimate of the 
$\dyll$ plus jets. 
%%%%%
\begin{table}[!ht]
  \caption{Photon triggers used for the $\met$ modelling of the $\dyll$ backgrounds. }
    \vspace{5pt}
   \label{tab:triggers_photon}
  \begin{center}
 {\small
  \begin{tabular} {l|l|c}
\hline
  Trigger name & pre-scale & rate \\
  \hline \hline
HLT\_Photon20\_CaloIdVL\_IsoL\_v5 & prescale 1790 & 0.5Hz \\
HLT\_Photon30\_CaloIdVL\_IsoL\_v6 & prescale 360 & 0.5Hz \\
HLT\_Photon30\_CaloIdVL\_v6 & prescale 720 & 0.7Hz \\
HLT\_Photon50\_CaloIdVL\_IsoL\_v5 & prescale 85 & 0.3Hz \\
HLT\_Photon50\_CaloIdVL\_v3 & prescale 180 & 0.4Hz \\
HLT\_Photon75\_CaloIdVL\_IsoL\_v6 & Prescale 7 & 0.8Hz \\
HLT\_Photon75\_CaloIdVL\_v6 & Prescale 30 & 0.5Hz \\
HLT\_Photon90\_CaloIdVL\_IsoL\_v3 & Unprescaled & 3Hz \\
HLT\_Photon90\_CaloIdVL\_v3 & prescaled 15 & 0.5Hz\\
  \hline 
  \end{tabular}
}
  \end{center}
\end{table}
%%%%%


Another set of specialised triggers are used to record events
enriched in fake electrons and muons for the measurement of jet induced backgrounds.
This is done using the fake rate method, which is described in detail in ref.~\cite{HWW2011AN}. 
