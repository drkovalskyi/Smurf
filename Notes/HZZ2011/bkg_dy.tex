Because the production cross-section is several orders of magnitude 
higher than the $H \to ZZ$ signal, the Drell-Yan background poses a significant 
challenge. In the leading order, the Drell-Yan process does not contain \met\. 
Therefore its contribution is signficantly reduced by the 
tight \met selection. The residual Drell-Yan contribution in the signal region 
is mainly from the events with fake \met due to the mis-measurement in 
either the jets recoiling against the $Z$ boson or the selected dileptons. 
The contribution of events with fake \met due to the lepton momentum mismeasurement 
is expected to be small since we are selecting the events within 15 GeV from the $Z$ boson mass. 
The background due to the jet mismeasurement becomes the largest source of Drell-Yan background. 

The estimation of residual Drell-Yan background in the signal region depends highly 
our understanding of the \met tails which is sensitive to many factors such as 
the jet energy corrections, number of multiple interaction. 
These effects in data are difficult to model precisely in MC. 
Therfore we need a data-driven method to estimate the \met distribution for the 
Drell-Yan background. In events without natural \met\, the fake \met due to the 
jet mismeasurement is not sensitive to the other activities in the event as long 
as those objects are well-measured. Therefore we can use a control sample in data 
with no natural \met\ to estimate the \met\ distribution in $\Zjets$ background. 
We choose to use the $\gamma$ plus jets for this measurement as the photon energy 
resolution of is well understood. 

The \met\ in $\Zjets$ is highly correlated with the dilepton $\pt$ the same way that the 
photon $\pt$ is directly corrleated with the \met. To account for the sample dependence, we 
reweight the photon $\pt$ distribution to match to the one measured in the $\Zjets$ sample. 


