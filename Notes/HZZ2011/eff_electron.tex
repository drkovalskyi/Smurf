
The electron selection efficiency can be factorised into two contributions,
the efficiency from the electron reconstruction and from the additional
analysis selections that are described in Section~\ref{sec:sel_electrons}.

The electron reconstruction efficiency is defined as the efficiency for a
supercluster to be matched to a reconstructed ECAL driven GSF electron.
%The data to simulation scale factor was measured for the W and Z cross-section
%analysis~\cite{VBTFCrossSectionNote}~\cite{ref:tagprobe_mit_w},
%and found to be consistent with $1.0$ with a total uncertainty of
%$1.3\%$ and $1.5\%$ for the barrel and endcap, respectively.
The data to simulation scale factor was measure by the Egamma POG binned in
$p_T$ and $\eta$~\cite{ref:egamma_eff_gsf}. From these studies, we take an overall scale factor of
$0.99$ with an uncertainty of $2.0\%$.

We thus measure the efficiency of our offline analysis selection 
with respect to a reconstructed ECAL driven GSF electron denominator. 
The resulting data to simulation scale factors are given in Table~\ref{tab:eff_ele_offline}.


\begin{table}[!ht]
\begin{center}
\begin{tabular}{c|c|c}
\hline
Measurement & Barrel ( $|\eta|<1.5$ )   & Endcap ( $|\eta|>1.5$ )  \\ 
\hline
$20 < p_T < 30$  & $0.9769 \pm 0.0015$ & $0.9955 \pm 0.0028$  \\ \hline
$30 < p_T < 40$  & $0.9773 \pm 0.0003$ & $0.9873 \pm 0.0005$  \\ \hline
$40 < p_T < 50$  & $0.9807 \pm 0.0005$ & $0.9863 \pm 0.0013$  \\ \hline
$50 < p_T < 100$ & $0.9759 \pm 0.0019$ & $0.9900 \pm 0.0018$  \\ \hline
$p_T > 100$      & $0.9779 \pm 0.0058$ & $0.9877 \pm 0.0266$  \\ \hline 
\end{tabular}
\caption{Offline selection scale factors for electrons.}
\label{tab:eff_ele_offline}
\end{center}
\end{table}

