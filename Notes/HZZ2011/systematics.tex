We have taken into account the following systematic uncertainties:

\begin{itemize}
\item {\it Luminosity:} We use the official CMS result, which is $4.5\%$.

\item {\it Lepton identification and trigger efficiencies:} 
We measure the efficiencies in data using the tag and probe method that is described
in detail in Section~\ref{sec:efficiency}. 
The estimated uncertainty is about $2\%$ per lepton leg.

\item {\it Momentum scale:} 
Due to several factors, the energy scale for electrons and the momentum 
scale for muons have relatively large uncertainties in the current data
processing. 
We assign a systematic uncertainty by varying the transverse momentum of the muons by $1\%$, 
and $2\%$ and $5\%$ for electrons in the barrel and the endcap, respectively. 
The contribution to the uncertainty on the dilepton efficiency is about $2\%$ for muons
and $5\%$ for electrons.

\item {\it Background estimation:} 
The methods to estimate the different backgrounds are explained in 
Section~\ref{sec:backgrounds}.
Here we summarize the systematic uncertainties associated with the methods used.
  \begin{itemize}
  \item the peaking component of $\WZ/ZZ$ Background: this background is estimated from simulation. 
We assign systmeatic uncertainties on their cross-sections as $8\%$ and $6\%$ respectively, 
augmentated by the luminosity normalisation uncertainty. 
  \item Top background: this background is estimated using the opposite flavor events in the $e\mu$ final states 
described in section~\ref{sec:bkg_of}. The systematic uncertainties in the cut-based 
analysis is reflected by the statistical uncertainties on the $e\mu$ event counting, modelled by 
gamma functions. The systematic uncertainties in the shape based analysis are around 20\% for most of the 
higgs hypotheses ( see Table~\ref{tab:shape_sf_emu}).  
%    The associated systematic uncertainties are $2\%$ and $5\%$ for the 0 and 1 jet bins respectively, evaluated 
%from the MC closure tests. The statistical 
%uncertainties are about $13\%$ and $16\%$ for the 0 and 1 jet bins respectively. 
  \item Drell-Yan background: The uncertainty arises from the photon reweighting method described in Section~\ref{sec:bkg_dy}. 
    The systematic uncertainty is evaluated from the MC closure test and a cross check at intermediate \met~in data to be 25\%.
%    and 20\% in the 0 and 1 jet bins respectively.
%  \item Other backgrounds: the background contribution from $\Wjets$ are estimated from the 
%        fake rate method, with a $36\%$ systematic uncertainty.
  \end{itemize}

\item {\it Pileup:} an incorrect modeling of the pileup in the Monte Carlo samples 
can bias the expected event yields. As described in section~\ref{sec:sel_pv}, 
the Monte Carlo events have been re-weighted on the basis of expected pile up events for a 
given instanenous luminosity. The re-weighting procedure affects only slightly the results of the analysis,
the event yields changing by $\sim1\%$. The latter is conservatively assumed as 
the corresponding systematic uncertainty. 

\item {\it Higgs cross section:} these uncertainties are taken from the Higgs cross
section working group~\cite{LHCHiggsCrossSectionWorkingGroup:2011ti}. The uncertainty 
on the $gg \to H$ production is about 15\% and it is one of the dominant effects. For
Higgs mass hypotheses of 200 $\GeVcc$ and above, we include a systematic uncertainty
for not accounting for the Higgs width when generating the Monte Carlo events. The 
formula for this uncertainty is,
%%%%%%%%
\begin{equation}
\Delta\sigma/\sigma = (150\%)\times\left(\frac{m_{H}}{1\TeVcc}\right)^3.
\end{equation}
%%%%%%%%


%\item {\it Jet counting and theoretical uncertainties:} 
%Since the analysis is categorized into jet bins, there are systematic uncertainties
%associated with events migrating from one jet bin to the next one due to 
%experimental and theoretical uncertainties. The experimental jet counting efficiency is 
%measured in data using $\dyll$ events. The effect of the statistical uncertainty 
%of the jet energy correction measurement on the extrapolation
%of the jet counting efficiency from $\dyll$ events to Higgs signal
%events is propagated and accounted for as the experimental 
%systematic uncertainty.

\item {\it Theoretical uncertainties:} 
We consider the theoretical uncertainties on the Parton Distribution Functions (PDFs), 
the uncertainty due to higher order corrections and the effect of the fragmentation and 
hadronization process. These uncertainties were taken from the Higgs cross
section working group~\cite{LHCHiggsCrossSectionWorkingGroup:2011ti}. 

%The overall uncertainties on the signal efficiency are 
%about $7\%$, $10\%$ and $20\%$ for the 0-jet, 1-jet and 2-jet bins, respectively.
%Correlations between different jet bins are taken into account when computing
%the upper limits. A detailed discussion of the method and procedures for estimating
%theoretical uncertainties on the signal yield can be found in Ref.~\cite{HWW2011AN}. 

\item {\it Monte Carlo statistics:} We also take into account the 
size of the simulated event samples. 
This contributes an uncertainty of about $2-3\%$ to the signal
efficiencies, but it is as large as $50\%$ for some background components on specific
Higgs mass points.

\item {\it Shape variations:} In the shape based analysis, we estimate the 
uncertainties due to the shape variations. The strategy is described in detail in Ref.~\cite{shapeananote}.
\end{itemize}



\subsection{Lepton Acceptance and Selection Efficiency }

The effect of the parton distribution function and the value of $\alpha_{s}$
 on the lepton acceptance and the efficiency of all the selection cuts are 
estimated using the prescription from the PDF4LHC recommendations \cite{PDF4LHC}. We 
propagate the uncertainty for each of the three PDF sets, MSTW2008, CT10, and
NNPDF, according to their own prescriptions, and then take the envelope
of the systematic uncertainties as the total PDF+$\alpha_{s}$  uncertainty. 

The effect of the missing higher order corrections are accounted for by
reweighting the Higgs $p_{T}$ spectrum to the one obtained from the
NNLO+NNLL calculation with the renormalization and factorization scales
varied by factors of $2$ and $1/2$. It is assumed that any change in the
lepton acceptance and the efficiency of other selection cuts are only
influenced via the Higgs $p_{T}$ spectrum. This effect is much smaller in 
magnitude than the effect on the jet bin fractions in any case, and 
essentially can be neglected.

\subsection{Shape Systematics}

Shape analysis relies on a good understanding of the signal and background distributions. 
As such, additional shape systematic uncertainties must be evaluated
on top of the background normalisation uncertainties. 
We now describe the procedures used to evaluate these uncertainties,
which are described in more detail in Section 4 of Reference \cite{shapeananote}.	

There are three different ways to account for systematic uncertainties for a
given source:

\begin{itemize}
  \item {\bf normalization uncertanty} - account only for the overall
    normalization assuming that the shape is perfectly known;
  \item {\bf uncorrelated normalization and shape uncertainties} - this is
    the most common case considered here;
  \item {\bf correlated normalization and shape uncertainties} - in cases 
    such as bin-by-bin statistical variation and lepton efficiencies,
    both the normalization and the shape must be treated as correlated.
\end{itemize}

Normalization uncertanties are the most straightforward to treat. They
are identical to those used in a cut-based analysis. To simplify the
analysis in some cases, such as background contributions with
large normalization uncertainties, it can be used as the only source of
systematic uncertainty ignoring the shape variation.

The current available official software framework does not support a full proper treatment of the statistical
shape uncertainties on each bin independently. 
Thus to account for most of the effect, we build ``up" (``down") variations as the
+1(-1)$\sigma$ statistical uncertainty on every bin. In this way, we
consider the uncertainty on every bin in a correlated way.

We assess the variation on the shape of each process by constructing alternate simulation
and data bounding shapes as appropriate or available for each process.
The sources of these alternate shapes are now described:

\begin{itemize}

    \item {{$H\rightarrow ZZ$ Signal}}:
    \begin{itemize}
        \item Central Shape: Powheg MC with Higgs boson $p_T$ dependent NLO k-Factor.
        \item Alternate Shapes: We include the shape uncertainty from the variation of 
the QCD scales, which are found the effect the expected limits at the percent level.
Shape variation based on the PDF uncertainty is expected to be contained within the
existing PDF normalisation uncertainties. Experimental effects from variation of the lepton
efficiencies and momentum scale are negligible.
    \end{itemize}

    \item {{$ZZ$}} (normalisation from MC): 
    \begin{itemize}
        \item Central Shape: Madgraph MC with dilepton $p_T$ dependent NLO k-Factor.
        \item Alternate Shapes: We compare the $p_T$ re-weighted Madgraph shape
with the other available Pythia MC sample, with equivalent $p_T$ reweighting. 
%shown in Figure \ref{fig:zzsyst_hzz}.
    \end{itemize}

    \item {{$WZ$}} (normalisation from MC):
    \begin{itemize}
        \item Central Shape: Madgraph MC.
        \item Alternate Shapes: We compare the Madgraph shape
with the other available Pythia MC sample.%, shown in Figure \ref{fig:wzsyst_hzz}.
We cross check the $WZ$ shape using data events with three leptons,
however this is not used as an alternate shape in the current analysis due to low statistics.
    \end{itemize}

    \item {{Top/WW}} (normalisation from $e\mu$ data):
    \begin{itemize}
        \item Central Shape: $e\mu$ events passing the selection for shape analysis. Since 
the $e\mu$ sample is statistically limited the $M_{T}$ is smoothened out. 
        \item Alternate Shapes: The alternate shape is taken by combining Top and WW events 
in Monte Carlo
%The alternate shapes are shown in Figure \ref{fig:topsyst_hzz}.
    \end{itemize}

    \item Z+jets (normalisation from $\gamma$+jets data):
    \begin{itemize}
        \item Central Shape: From $\gamma$+jets data.
        \item Alternate Shapes: Intrinsic differences between the $\gamma$+jets and Z+jets processes
are taken into account by the assigned normalisation uncertainty, which is 100\%.
    \end{itemize}

\vspace{15pt}

To understand the effects of the individual source of the uncertainties,
we compare the results by adding each source progressively, shown in Table~\ref{tab:mva_mtshape_detail}.
Among all the shape systematics, the statistical uncertainty on the template is the leading effect.
The degradation in the expected limit by adding shape systematics is found to be up to $6\%$.



%%%%%%%%%%%%%%%%%%%%%%%%%%%%%
\begin{table}[!ht]
\begin{center}
{\normalsize
\begin{tabular}{|l|c|ccccc|}
\hline
      &  Analysis    & adding          &  adding      &  adding      & adding      & adding \\
mH  &  without     & template        &  $H\to ZZ$   &  Top/WW             & WZ          & ZZ \\
      &  shape syst. & stat. uncert.   &  QCD effect &  shape syst. & shape syst. & shape syst. \\
\hline
250 & 1.38 & 1.38 & 1.38 & 1.45 & 1.38 & 1.38 \\   
300 & 0.92 & 0.92 & 0.92 & 0.92 & 0.93 & 0.93 \\ 
350 & 0.62 & 0.62 & 0.62 & 0.63 & 0.63 & 0.63 \\
400 & 0.63 & 0.63 & 0.63 & 0.63 & 0.63 & 0.63 \\
500 & 1.06 & 1.06 & 1.06 & 1.06 & 1.08 & 1.08 \\
600 & 2.20 & 2.21 & 2.21 & 2.20 & 2.23 & 2.23 \\
\hline
\end{tabular}
}
\caption{Comparison of the median expected cross section ratio limits as a function 
of the Higgs mass between shape analysis without and with accouting for the 
shape variation systematics. The results on the various sources are added sequentially 
to study the impact of each source. } 
\label{tab:mva_mtshape_detail}
\end{center}
\end{table}
%%%%%%%%%%%%%%%%%%%%%%%%%%%%%

\end{itemize}

Table~\ref{tab:limits_mtshape_uncert_5fb}-\ref{tab:limits_meshape_uncert_5fb} show the 
expected limits as a function of the higgs mass for the shape analyses based on 
the $M_T$ variable and matrix element output, comparing the effects of including the 
shape systematics. Overall, the effects due to the shape systematics are small. This 
is mainly because of the large overall normalization uncertainties present. 

%%%%%%%%%%%%%%%%%%%%%%%%%%%%%
\begin{table}[!htbp]
\begin{center}
{\normalsize
\begin{tabular}{|l|c|c|c|c|c|c|}
\hline
      &  \multicolumn{3}{c|}{ without shape uncertainty} &\multicolumn{3}{c|}{ with shape uncertainty} \\
\hline
Mass  &  Median      &     68\% C.L. band &  95\% C.L. band &  Median	   &	 68\% C.L. band &  95\% C.L. band\\
      &  Expected    &                    &                 &  Expected    &			&		 \\
\hline
250 & 1.38 & [1.00, 1.92] & [0.75, 2.55] & 1.38 & [1.00, 1.92] & [0.75, 2.56] \\
300 & 0.92 & [0.66, 1.28] & [0.50, 1.69] & 0.93 & [0.67, 1.29] & [0.51, 1.72] \\
350 & 0.62 & [0.45, 0.86] & [0.34, 1.15] & 0.63 & [0.45, 0.87] & [0.34, 1.16] \\
400 & 0.63 & [0.45, 0.87] & [0.34, 1.16] & 0.63 & [0.46, 0.88] & [0.34, 1.17] \\
500 & 1.06 & [0.76, 1.47] & [0.57, 1.95] & 1.08 & [0.78, 1.50] & [0.58, 1.99] \\
600 & 2.20 & [1.59, 3.06] & [1.19, 4.06] & 2.23 & [1.61, 3.10] & [1.21, 4.12] \\
\hline
\end{tabular}
}
\end{center}
\caption{The median expected cross section ratio limits as a function 
of the Higgs mass, together with the 1/2-$\sigma$ uncertainty bands obtained in the shape analysis based on $M_T$, 
corresponding to an integrated luminosity of \intlumi}
\label{tab:limits_mtshape_uncert_5fb}
\end{table}
%%%%%%%%%%%%%%%%%%%%%%%

%%%%%%%%%%%%%%%%%%%%%%%
\begin{table}[!htbp]
\begin{center}
{\normalsize
\begin{tabular}{|l|c|c|c|c|c|c|}
\hline
      &  \multicolumn{3}{c|}{ without shape uncertainty} &\multicolumn{3}{c|}{ with shape uncertainty} \\
\hline
Mass  &  Median      &     68\% C.L. band &  95\% C.L. band &  Median	   &	 68\% C.L. band &  95\% C.L. band\\
      &  Expected    &                    &                 &  Expected    &			&		 \\
\hline
250 & 1.33 & [0.96, 1.84] & [0.72, 2.44] & 1.31 & [0.95, 1.83] & [0.71, 2.43] \\
300 & 0.85 & [0.61, 1.18] & [0.46, 1.57] & 0.87 & [0.63, 1.21] & [0.47, 1.60] \\
350 & 0.64 & [0.46, 0.88] & [0.35, 1.18] & 0.64 & [0.46, 0.88] & [0.35, 1.17] \\
400 & 0.64 & [0.46, 0.89] & [0.35, 1.18] & 0.64 & [0.46, 0.89] & [0.35, 1.19] \\
500 & 1.13 & [0.82, 1.57] & [0.61, 2.09] & 1.13 & [0.82, 1.57] & [0.61, 2.09] \\
600 & 2.71 & [1.95, 3.76] & [1.47, 5.00] & 2.43 & [1.75, 3.37] & [1.32, 4.48] \\
\hline
\end{tabular}
}
\end{center}
\caption{The median expected cross section ratio limits as a function 
of the Higgs mass, together with the 1/2-$\sigma$ uncertainty bands obtained in the shape analysis based on matrix element output, 
corresponding to an integrated luminosity of \intlumi}
\label{tab:limits_meshape_uncert_5fb}
\end{table}
%%%%%%%%%%%%%%%%%%%%%%%




\subsection{Summary of Systematic Uncertainties}

All systematic uncertainties taken into account in this analysis
are summarized in Table~\ref{tab:systww}.
The total uncertainty depends on the Higgs mass, %and jet bin considered,
however is typically $20\%$ on the data-driven background estimation and about $6-35\%$ 
on the signal efficiency. These results assume an integrated luminosity of \intlumi. 
%The uncertainties due to the shape variation is discussed in Section 4 of Ref~\cite{shapeananote}. 

\begin{table}[!ht]
\begin{center}
\caption{\label{tab:systww} Summary of all systematic uncertainties (relative).}
\vspace{5pt}
{\footnotesize
\begin{tabular}{l|c|c|c|c|c|c}
\hline
Source  & $\hzz$ & $WZ$ & $ZZ$ & $\WW$ & $\ttbar$+$\tw$ & $\dyll$ \\
\hline
\hline
Luminosity                               & 4.5 & 4.5 & 4.5 & --- & --- & --- \\
Electron Trigger efficiencies            & 1.0 & 1.0 & 1.0 & --- & --- & --- \\
Muon efficiency                          & 2.0 & 2.0 & 2.0 & --- & --- & --- \\
Electron id efficiency                   & 2.0 & 2.0 & 2.0 & --- & --- & --- \\
Muon Momentum scale                      & 2.0 & 2.0 & 2.0 & --- & --- & --- \\
Electron Energy scale                    & 5.0 & 5.0 & 5.0 & --- & --- & --- \\
Higgs cross section                      & 5-15& --- & --- & --- & --- & --- \\
Higgs cross section due to wdith         & 2-32& --- & --- & --- & --- & --- \\
$WZ$ cross section                       & --- & 6.2 & --- & --- & --- & --- \\
$qq\rightarrow ZZ$/$gg\rightarrow ZZ$ cross section                       & --- & ---  & 8.2/20.0 & --- & --- & --- \\
$\WW$ and Top for cut-based analysis             & --- & --- & 14-58 & --- & --- & --- \\
$\WW$ and Top for $M_T$ shape-based analysis         & --- & --- & 10-13 & 10-13 & --- & --- \\
$\dyll$                                  & --- & --- & --- & --- & --- & 25 \\
Monte Carlo statistics                   & 2.0 & 2-7 & 1-3 & --- & --- & 14-30 \\
\hline
\end{tabular}
}
\end{center}
\end{table}
