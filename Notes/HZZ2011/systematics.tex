We have taken into account the following systematic uncertainties:

\begin{itemize}
\item {\it Luminosity:} We use the official CMS result, which is $6\%$.

\item {\it Lepton identification and trigger efficiencies:} 
We measure the efficiencies in data using the tag and probe method that is described
in detail in Section~\ref{sec:efficiency}. 
The estimated uncertainty is about $2\%$ per lepton leg.

\item {\it Momentum scale:} 
Due to several factors, the energy scale for electrons and the momentum 
scale for muons have relatively large uncertainties in the current data
processing. 
We assign a systematic uncertainty by varying the transverse momentum of the muons by $1\%$, 
and $2\%$ and $5\%$ for electrons in the barrel and the endcap, respectively. 
The contribution to the uncertainty on the dilepton efficiency is about $2\%$ for muons
and $5\%$ for electrons.

\item {\it Background estimation:} 
The methods to estimate the different backgrounds are explained in 
Section~\ref{sec:backgrounds}.
Here we summarize the systematic uncertainties associated with the methods used.
  \begin{itemize}
  \item the peaking component of $\WZ/ZZ$ Background: this background is estimated from simulation. 
We assign systmeatic uncertainties on their cross-sections as $8\%$ and $5\%$ respectively, 
augmentated by the luminosity normalisation uncertainty. 
  \item Top background: this background is estimated using the opposite flavor events in the $e\mu$ final states 
described in section~\ref{sec:bkg_of}. 
    The associated systematic uncertainties are $2\%$ and $5\%$ for the 0 and 1 jet bins respectively, evaluated 
from the MC closure tests. The statistical 
uncertainties are about $13\%$ and $16\%$ for the 0 and 1 jet bins respectively. 
  \item Drell-Yan background: The uncertainty arises from the photon reweighting method described in Section~\ref{sec:bkg_dy}. 
    The systematic uncertainty is evaluated from the MC closure test and a cross check at intermediate \met~in data to be 25\%
    and 20\% in the 0 and 1 jet bins respectively.
  \item Other backgrounds: the background contribution from $\Wjets$ are estimated from the 
        fake rate method, with a $36\%$ systematic uncertainty.
  \end{itemize}

\item {\it Pileup:} an incorrect modeling of the pileup in the Monte Carlo samples 
can bias the expected event yields. As described in section~\ref{sec:sel_pv}, 
the Monte Carlo events have been re-weighted on the basis of the number of reconstructed
primary vertices. The re-weighting procedure affects only slightly the results of the analysis,
the event yields changing by $\sim1\%$. The latter is conservatively assumed as 
the corresponding systematic uncertainty. 

\item {\it Higgs cross section:} these uncertainties are taken from the Higgs cross
section working group~\cite{LHCHiggsCrossSectionWorkingGroup:2011ti}. The uncertainty 
on the $gg \to H$ production is about 15\% and it is one of the dominant effects.

\item {\it Jet counting and theoretical uncertainties:} 
Since the analysis is categorized into jet bins, there are systematic uncertainties
associated with events migrating from one jet bin to the next one due to 
experimental and theoretical uncertainties. The experimental jet counting efficiency is 
measured in data using $\dyll$ events. The effect of the statistical uncertainty 
of the jet energy correction measurement on the extrapolation
of the jet counting efficiency from $\dyll$ events to Higgs signal
events is propagated and accounted for as the experimental 
systematic uncertainty.

We consider the theoretical uncertainties on the Parton Distribution Functions (PDFs), 
the uncertainty due to higher order corrections and the effect of the fragmentation and 
hadronization process. The overall uncertainties on the signal efficiency are 
about $7\%$, $10\%$ and $20\%$ for the 0-jet, 1-jet and 2-jet bins, respectively.
Correlations between different jet bins are taken into account when computing
the upper limits. A detailed discussion of the method and procedures for estimating
theoretical uncertainties on the signal yield can be found in Ref.~\cite{HWW2011AN}. 

\item {\it Monte Carlo statistics:} We also take into account the 
size of the simulated event samples. 
This contributes an uncertainty of about $2-3\%$ to the signal
efficiencies, but it is as large as $50\%$ for some background components on specific
Higgs mass points.
\end{itemize}


\subsubsection{Lepton Acceptance and Selection Efficiency }

The effect of the parton distribution function and the value of $\alpha_{s}$
 on the lepton acceptance and the efficiency of all the selection cuts are 
estimated using the prescription from the PDF4LHC recommendations \cite{PDF4LHC}. We 
propagate the uncertainty for each of the three PDF sets, MSTW2008, CT10, and
NNPDF, according to their own prescriptions, and then take the envelope
of the systematic uncertainties as the total PDF+$\alpha_{s}$  uncertainty. 

The effect of the missing higher order corrections are accounted for by
reweighting the Higgs $p_{T}$ spectrum to the one obtained from the
NNLO+NNLL calculation with the renormalization and factorization scales
varied by factors of $2$ and $1/2$. It is assumed that any change in the
lepton acceptance and the efficiency of other selection cuts are only
influenced via the Higgs $p_{T}$ spectrum. This effect is much smaller in 
magnitude than the effect on the jet bin fractions in any case, and 
essentially can be neglected.



\subsection{Summary of Systematic Uncertainties}
All systematic uncertainties taken into account in this analysis
are summarized in Table~\ref{tab:systww}.
The total uncertainty depends on the Higgs mass and jet bin considered,
however is typically $X\%$ on the background estimation and about $X\%$ 
on the signal efficiency. These results assume an integrated luminosity of $1~\ifb$.

\begin{table}[!ht]
\begin{center}
\caption{\label{tab:systww} Summary of all systematic uncertainties (relative).}
\vspace{5pt}
{\footnotesize
\begin{tabular}{l|c|c|c|c|c|c|c}
\hline
Source  & $\hzz$ & $WZ/ZZ$ & $\WW$ & $\ttbar$+$\tw$ & $\dyll$ & $\dytt$ & $\Wjets$ \\
\hline
\hline
Luminosity                               & 6.0 & 6.0 & --- & --- & --- & --- & --- \\
Trigger efficiencies                     & 1.5 & 1.5 & --- & --- & --- & --- & --- \\
Muon efficiency                          & 1.5 & 1.5 & --- & --- & --- & --- & --- \\
Electron id efficiency                   & 2.5 & 2.5 & --- & --- & --- & --- & --- \\
Muon Momentum scale                      & 2.0 & 2.0 & --- & --- & --- & --- & --- \\
Electron Energy scale                    & 5.0 & 5.0 & --- & --- & --- & --- & --- \\
Jet counting                             & 7-20& --- & --- & --- & --- & --- & --- \\
Higgs cross section                      & 5-15& --- & --- & --- & --- & --- & --- \\
$WZ/ZZ$ cross section                    & --- & 5.0/8.0 & --- & --- & --- & --- & --- \\
$\WW$                                    & --- & --- & 17 & --- & --- & --- & ---      \\
Top ($\ttbar$ + $\tw$)                   & --- & --- & --- & 17 & --- & --- & ---      \\
$\dyll$ 0 (1)-jet                        & --- & --- & --- & --- & 25(20)  & --- & ---     \\
$\dytt$                                  & --- & --- & --- & --- & --- & --- & ---     \\
$\Wjets$                                 & --- & --- & --- & --- & --- & --- & 36      \\
Monte Carlo statistics                   & 1.2 & 3.2/7.9 & 50 & 50 & 20-30 & --- & --- \\
\hline
\end{tabular}
}
\end{center}
\end{table}
