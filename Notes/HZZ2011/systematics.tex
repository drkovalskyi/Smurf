We have taken into account the following systematic uncertainties:

\begin{itemize}
\item {\it Luminosity:} We use the official CMS result, which is $6\%$.

\item {\it Lepton identification and trigger efficiencies:} 
We measure the efficiencies in data using the tag and probe method that is described
in detail in Section~\ref{sec:efficiency}. 
The estimated uncertainty is about $2\%$ per lepton leg.

\item {\it Momentum scale:} 
Due to several factors, the energy scale for electrons and the momentum 
scale for muons have relatively large uncertainties in the current data
processing. 
We assign a systematic uncertainty by varying the transverse momentum of the muons by $1\%$, 
and $2\%$ and $5\%$ for electrons in the barrel and the endcap, respectively. 
The contribution to the uncertainty on the dilepton efficiency is about $1.5\%$.

\item {\it $\met$ modeling:} We use a data-driven method to estimate the $\dyll$
background, which is affected by the $\met$ resolution. 
Events with neutrinos giving real $\met$ in the final state also have a small uncertainty. 
We assess this uncertainty on the event selection efficiency by varying the $\met$ in signal events
by an additional $10\%$. We find an uncertainty on the event selection efficiency of around {\fixme}X\%.

\item {\it Background estimation:} 
The methods to estimate the different backgrounds are explained in 
Section~\ref{sec:backgrounds}.
Here we summarize the systematic uncertainties associated with the methods used.
  \begin{itemize}
  \item the peaking component of $\WZ/ZZ$ Background: this background is estimated from simulation. 
We assign systmeatic uncertainties on their cross-sections as $X.X\%$ and $X.X\%$ respectively, 
augmentated by the luminosity normalisation uncertainty. 
  \item Top background: this background is estimated using the opposite flavor events in the $e\mu$ final states 
described in section~\ref{sec:bkg_of}. 
    The associated systematic uncertainties are $2\%$ and $5\%$ for the 0 and 1 jet bins respectively, evaluated 
from the MC closure tests. The statistical 
uncertainties are about $13\%$ and $16\%$ for the 0 and 1 jet bins respectively. 
  \item Drell-Yan background: The uncertainty arises from the photon reweighting method described in Section~\ref{sec:bkg_dy}. 
    The dominant systematics is taken from the MC closure test of $30\%$.
  \item Other backgrounds: the background contribution from $\Wjets$ are estimated from data-driven method to be negligible. 
  \end{itemize}

\item {\it Pileup:} an incorrect modeling of the pileup in the Monte Carlo samples 
can bias the expected event yields. As described in section~\ref{sec:sel_pv}, 
the Monte Carlo events have been re-weighted on the basis of the number of reconstructed
primary vertices. The re-weighting procedure affects only slightly the results of the analysis,
the event yields changing by $\sim1\%$. The latter is conservatively assumed as 
the corresponding systematic uncertainty. 

\item {\it Higgs cross section:} these uncertainties are taken from the Higgs cross
section working group~\cite{LHCHiggsCrossSectionWorkingGroup:2011ti}. The uncertainty 
on the $gg \to H$ production is about 15\% and it is one of the dominant effects.

\item {\it Jet counting and theoretical uncertainties:} 
we must make sure that the jet multiplicity is well reproduced by our 
simulation because we split our analysis in different jet bins. 
Experimental and theoretical uncertainties are both important.
The experimental jet counting efficiency is measured in data 
using $\dyll$ events. The effect of the statistical uncertainty 
of the jet energy correction measurement on the extrapolation
of the jet counting efficiency from $\dyll$ events to Higgs signal
events is propagated and accounted for as the experimental 
systematic uncertainty.

We consider the theoretical uncertainties on the Parton Distribution Functions (PDFs), 
the uncertainty due to higher order corrections and the effect of the fragmentation and 
hadronization process. The overall uncertainties on the signal efficiency are 
about $7\%$, $10\%$ and $20\%$ for the 0-jet, 1-jet and 2-jet bins, respectively.
Correlations between different jet bins are taken into account when computing
the upper limits. A detailed discussion of the method and procedures for estimating
theoretical uncertainties on the signal yield can be found in Ref.~\cite{HWW2011AN}. 


\item {\it Monte Carlo statistics:} We also take into account the 
size of the simulated event samples. 
This contributes an uncertainty of about $2-3\%$ to the signal
efficiencies, but it is as large as $100\%$ for some background components on specific
Higgs mass points.
\end{itemize}

\subsection{Theoretical Systematic Uncertainties on Signal Efficiency}
\label{sec:theorySystematicsSignal}
The theoretical systematic uncertainties on the signal yield is factorized into the
product of two components, assumed to be independent. The first component is the uncertainty
on the fraction of events categorized into the different jet bins and the effect
of migrations across jet bins. The second component is the uncertainty
on the lepton acceptance and the selection efficiency of all other cuts. The effect of
parton distribution function and the value of $\alpha_{s}$, and the effect of 
higher order corrections are considered for both components. For the jet categorization,
we consider, additionally, the effect of higher order log terms via the uncertainty in the 
parton shower model and the underlying event.

\subsubsection{Systematic Uncertainties on the Jet Bin Fractions}
\label{sec:HiggsJetBinFractionSystematics}
We consider the effect on the jet counting categorization without imposing any additional
selection cuts on the signal. A jet in the event will be counted if it has $p_{T} > 30$ GeV, 
following the procedure described in Ref.~\cite{HWW2011AN}. 
We define the jet bin fractions $f_{0}$, $f_{1}$, $f_{2}$
as the fraction of Higgs events with $0$, $1$, and $2$ or more counted jets. 
The uncertainty due to higher order corrections is evaluated by comparing the effect of
varying the renormalization scale and factorization scale on the jet bin fractions. 

To estimate these jet bin fractions, we combine the inclusive $gg \to H$ production cross section
computed at NNLO \cite{LHCHiggsCrossSectionWorkingGroup:2011ti} ($\sigma_{\geq 0}$) and the 
NLO calculations from MCFM \cite{MCFMHiggsProduction} for the inclusive production cross section 
of $gg \to H+1$jet ($\sigma_{\geq 1}$) and $gg \to H+2$jet ($\sigma_{\geq 2}$), 
through the following relations:
\begin{eqnarray}
\label{eqn:jetBinFractions}
  f_{0} = (\sigma_{\geq 0} - \sigma_{\geq 1} ) / \sigma_{\geq 0} \\
  f_{1} = (\sigma_{\geq 1} - \sigma_{\geq 2} ) / \sigma_{\geq 0} \\
  f_{2} = \sigma_{\geq 2} / \sigma_{\geq 0} \\
\end{eqnarray}


The systematic uncertainties on $\sigma_{\geq 0}$, $\sigma_{\geq 1}$, and $\sigma_{\geq 2}$, are referred 
to as $\kappa_{\geq 0}$, $\kappa_{\geq 1}$, and $\kappa_{\geq 2}$ respectively. 
The central values of the cross sections are computed  using the MSTW2008 NLO parton distribution functions and 
assuming a factorization and renormalization scale of $m_{H} / 2$. The uncertainty on the inclusive
production cross section, $\kappa_{\geq 0}$, is cited from the CERN Yellow Report 
\cite{LHCHiggsCrossSectionWorkingGroup:2011ti}.  The systematic uncertainties
$\kappa_{\geq 1}$, and $\kappa_{\geq 2}$ are estimated by performing the factorization scale ($\mu_{f}$) 
and renormalization scale ($\mu_{r}$) as detaied in Ref.~\cite{HWW2011AN}. 
The results for $\kappa_{\geq 0}$, $\kappa_{\geq 1}$, and $\kappa_{\geq 2}$ are summarized in Table 
\ref{tab:InclXS_ScaleVariation}.


\begin{table}[!htbp]
\begin{center}
\begin{tabular}{|c|c|c|c|}

\hline
Higgs Mass     &     $\kappa_{\geq 0}$        &   $\kappa_{\geq 1}$        &     $\kappa_{\geq 2}$       \\
\hline
 250 & $ 1.083$  & $ 1.208$  & $ 1.230$  \\
 300 & $ 1.082$  & $ 1.208$  & $ 1.205$  \\
% 350 & $ 1.090$  & $ 1.207$  & $ 1.209$  \\
 400 & $ 1.075$  & $ 1.195$  & $ 1.195$  \\
% 450 & $ 1.078$  & $ 1.194$  & $ 1.196$  \\
% 500 & $ 1.087$  & $ 1.188$  & $ 1.174$  \\
% 550 & $ 1.089$  & $ 1.191$  & $ 1.194$  \\
% 600 & $ 1.090$  & $ 1.187$  & $ 1.192$  \\
\hline
\end{tabular}
\caption{ $\kappa$ values for the systematic uncertainties due to missing higher order corrections
for the inclusive Higgs production cross section, the inclusive Higgs+1Jet production cross section, 
and inclusive Higgs+2Jet production cross section. }
\label{tab:InclXS_ScaleVariation}
\end{center}
\end{table}

The values of these $\kappa$'s reflecting the systematic uncertainties and their correlations are 
summarized in Table \ref{tab:JetBinFractionSystematics_ScaleVariation} for all Higgs masses considered.

\begin{table}[!htbp]
\begin{center}
\begin{tabular}{|c|c|c|c|c|c|}

\hline
Higgs Mass     & $\kappa^{0}_{\mathrm{QCDscale\_ggH}}$ & $\kappa^{0}_{\mathrm{QCDscale\_ggH1in}}$ & $\kappa^{1}_{\mathrm{QCDscale\_ggH1in}}$  & $\kappa^{1}_{\mathrm{QCDscale_ggH2in}}$  & $\kappa^{2}_{\mathrm{QCDscale\_ggH2in}}$   \\
\hline
 250 & $ 1.16$  & $ 0.86$  & $ 1.27$  & $ 0.96$  & $ 1.17$  \\
 300 & $ 1.17$  & $ 0.84$  & $ 1.27$  & $ 0.95$  & $ 1.20$  \\
% 350 & $ 1.20$  & $ 0.83$  & $ 1.27$  & $ 0.95$  & $ 1.21$  \\
 400 & $ 1.17$  & $ 0.82$  & $ 1.26$  & $ 0.95$  & $ 1.20$  \\
% 450 & $ 1.19$  & $ 0.81$  & $ 1.26$  & $ 0.95$  & $ 1.20$  \\
% 500 & $ 1.22$  & $ 0.80$  & $ 1.25$  & $ 0.95$  & $ 1.17$  \\
% 550 & $ 1.24$  & $ 0.78$  & $ 1.26$  & $ 0.95$  & $ 1.19$  \\
% 600 & $ 1.25$  & $ 0.78$  & $ 1.26$  & $ 0.94$  & $ 1.19$  \\
\hline

\end{tabular}
\caption{ Table of $\kappa$ values for the systematic uncertainties for the jet bin 
fractions due to missing higher order corrections for the total inclusive Higgs
cross section, the inclusive Higgs+1jet cross section, and the inclusive Higgs+2jet
cross section. }
\label{tab:JetBinFractionSystematics_ScaleVariation}
\end{center}
\end{table}

Finally, we estimate the systematic uncertainty due to higher order log terms by testing the default 
parton shower and hadronization model, Pythia, with the results obtained using an alternative
parton shower and hadronization model, Herwig. In the first case, Powheg is used for the matrix
element calculation, while in the second case MC@NLO is used. In order to isolate the effect of the 
parton shower and hadronization model, we reweight the Higgs $p_{T}$ distribution to the reference 
NNLO+NNLL resummed calculation, in both cases. Since the underlying event model in the two Monte Carlo 
are also different, this estimate includes also the effect of the underlying event model. 

%Figure \ref{fig:JetBinFraction_PartonShowerComparison} shows a comparison of the jet bin fractions 
%for these two Monte Carlo samples for a Higgs mass of $130$ GeV. 
The relative differences in the jet bin fractions, $f_{0}$, $f_{1}$, and $f_{2}$, between the two results are taken as a systematic
uncertainty and are summarized in Table~\ref{tab:JetBinFractionSystematics_PartonShower}.


\begin{table}[!htbp]
\begin{center}
\begin{tabular}{|c|c|c|c|}

\hline
               &   \multicolumn{3}{|c|}{ $\kappa$ values for systematic uncertainties on: } \\
\hline
Higgs Mass     &   $f_{0}$   &  $f_{1}$       &   $f_{2}$       \\
\hline
250 & $0.955$ & $1.058$ & $0.990$ \\
300 & $0.958$ & $1.061$ & $0.942$ \\
%350 & $0.964$ & $1.068$ & $0.889$ \\
400 & $0.966$ & $1.078$ & $0.856$ \\
%450 & $0.954$ & $1.092$ & $0.864$ \\
%500 & $0.946$ & $1.102$ & $0.868$ \\
%550 & $0.931$ & $1.117$ & $0.861$ \\
%600 & $0.920$ & $1.121$ & $0.872$ \\
\hline
\end{tabular}
\caption{Table of $\kappa$ values for the systematic uncertainty on the jet bin fractions
due to  uncertainties in the parton shower and hadronization model and uncertainties
on the model of the underlying event, as a function of the assumed Higgs mass.  }
\label{tab:JetBinFractionSystematics_PartonShower}
\end{center}
\end{table}



\subsubsection{Lepton Acceptance and Selection Efficiency }

The effect of the parton distribution function and the value of $\alpha_{s}$
 on the lepton acceptance and the efficiency of all the selection cuts are 
estimated using the prescription from the PDF4LHC recommendations \cite{PDF4LHC}. We 
propagate the uncertainty for each of the three PDF sets, MSTW2008, CT10, and
NNPDF, according to their own prescriptions, and then take the envelope
of the systematic uncertainties as the total PDF+$\alpha_{s}$  uncertainty. 

The effect of the missing higher order corrections are accounted for by
reweighting the Higgs $p_{T}$ spectrum to the one obtained from the
NNLO+NNLL calculation with the renormalization and factorization scales
varied by factors of $2$ and $1/2$. It is assumed that any change in the
lepton acceptance and the efficiency of other selection cuts are only
influenced via the Higgs $p_{T}$ spectrum. This effect is much smaller in 
magnitude than the effect on the jet bin fractions in any case, and 
essentially can be neglected.



\subsection{Summary of Systematic Uncertainties}
All systematic uncertainties taken into account in this analysis
are summarized in Table~\ref{tab:systww}.
The total uncertainty depends on the Higgs mass and jet bin considered,
however is typically $X\%$ on the background estimation and about $X\%$ 
on the signal efficiency. These results assume an integrated luminosity of $1~\ifb$.

\begin{table}[!ht]
\begin{center}
\caption{\label{tab:systww} Summary of all systematic uncertainties (relative).}
\vspace{5pt}
{\footnotesize
\begin{tabular}{l|c|c|c|c|c|c|c}
\hline
Source  & $\hzz$ & $WZ/ZZ$ & $\WW$ & $\ttbar$+$\tw$ & $\dyll$ & $\dytt$ & $\Wjets$ \\
\hline
\hline
Luminosity                               & 6.0 & 6.0 & --- & --- & --- & --- & --- \\
Trigger efficiencies                     & 1.5 & 1.5 & --- & --- & --- & --- & --- \\
Muon efficiency                          & 1.5 & 1.5 & --- & --- & --- & --- & --- \\
Electron id efficiency                   & 2.5 & 2.5 & --- & --- & --- & --- & --- \\
Momentum scale                           & x.x & x.x & --- & --- & --- & --- & --- \\
$\met$ resolution                        & 1.0 & 1.0 & --- & --- & --- & --- & --- \\
Jet counting                             & 7-20& --- & --- & --- & --- & --- & --- \\
Higgs cross section                      & 5-15& --- & --- & --- & --- & --- & --- \\
$WZ/ZZ$ cross section                    & --- & x.x & --- & --- & --- & --- & --- \\
$\WW$                                    & --- & --- & x.x & --- & --- & --- & --- \\
Top ($\ttbar$ + $\tw$)                   & --- & --- & --- & x.x & --- & --- & --- \\
$\dyll$                                  & --- & --- & --- & --- & 30  & --- & --- \\
$\dytt$                                  & --- & --- & --- & --- & --- & x.x &  \\
$\Wjets$                                 & --- & --- & --- & --- & --- & --- & x.x \\
Monte Carlo statistics                   & x.x & x.x & x.x & x.x & x.x & x.x & x.x \\
\hline
\end{tabular}
}
\end{center}
\end{table}
