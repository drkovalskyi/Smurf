
To compare consistently with the cut based analysis, we employ a scale factor
for the WW and Top simulation.  We do this by scaling the MC meeting the
shape based preselections to match the $e\mu$ estimate in data.
The resulting scale factors are tabulated in Table \ref{tab:shape_sf_emu}.
We then take the WW and Top shapes from the scaled simulation because the 
scaled $e\mu$ shape taken directly from data suffers from larger statistical
fluctuations than the simulation.

\begin{table}[!ht]
\begin{center}
\small{
\begin{tabular}{c|c|c|c|c}
Mass &   WW             & TOP              & $e\mu$ estimate   & SF \\ \hline
250  & $14.12 \pm 0.40$ & $19.88 \pm 0.43$ & $25.82 \pm 3.70$  & $0.76 \pm 0.11$ \\ \hline
300  & $5.38 \pm 0.25$  & $9.09 \pm 0.29$  & $14.24 \pm 2.73$  &  $0.98 \pm 0.19$ \\ \hline
350  & $5.44 \pm 0.25$  & $9.22 \pm 0.30$  & $14.24 \pm 2.73$  &  $0.97 \pm 0.19$ \\ \hline
400  & $5.44 \pm 0.25$  & $9.24 \pm 0.30$  & $14.24 \pm 2.73$  &  $0.97 \pm 0.19$ \\ \hline
500  & $5.44 \pm 0.25$  & $9.24 \pm 0.30$  & $14.24 \pm 2.73$  & $0.97 \pm 0.19$ \\ \hline
600  & $5.44 \pm 0.25$  & $9.24 \pm 0.30$  & $14.24 \pm 2.73$  & $0.97 \pm 0.19$ \\ \hline
\end{tabular}
\caption{Scale factors for WW and Top for shape analysis derived using the $e\mu$ estimate in data.}
\label{tab:shape_sf_emu}}
\end{center}
\end{table}


In the $M_T$-Shape based analysis, it is desirable to separate the $WW$ and $Top$ backgrounds,
to better separate the associated shape uncertainties.  In what follows we discuss how this can be done.
However, at the present time we do not use the separate scale factors because we find poor agreement
when cross checking the scale simulation with the expected yields using the $e\mu$ counting method.

\subsubsection{Top Background Estimate}

Top quarks decay via $t\rightarrow bW$ with almost certainty.  
We explout this to estimate the background by counting events where $b$-quarks were produced. 
If the counted events are dominated by $t\bar{t}$ events, 

The true number of $t\bar{t}$ events containing exactly one $b$-quark within the geometric acceptance 
of the detector and exactly two $b$-quarks within the same acceptance are denoted $N_{1}$ and $N_{2}$.  
These are related to the observed numbers of events, $n_{1}$ and $n_{2}$, 
through the efficiency with which a jet originating from a $b$-quark 
is reconstructed and tagged, $\varepsilon_{b}$, by 

\begin{eqnarray}
n_{1} = N_{1}\varepsilon_{b} + 2N_{2}\varepsilon_{b}(1-\varepsilon_{b}) \label{eqn:analysis_bTagEff1_n1},\\
n_{2} = N_{2}\varepsilon_{b}^{2} \label{eqn:analysis_bTagEff1_n2}.
\end{eqnarray}

To determine the value of $\varepsilon_{b}$ the unknown values of $N_{1}$ and $N_{2}$ 
can be eliminated by relating each to the true number of $t\bar{t}$ events, $N$, 
through the geometric acceptances, $A_1$ and $A_2$, for exactly one and exactly two $b$ quarks 
respectively in the case where the W bosons decay to the appropriate lepton flavor. 
This is expressed by

\begin{eqnarray}
N_{1} = N\cdot A_{1}, \\
N_{2} = N\cdot A_{2}.
\end{eqnarray}

Eliminating the unknowns, the value of $\varepsilon_{b}$ is related to the observed numbers of events through 

\begin{equation}
\label{eqn:analysis_bTagEff}
\varepsilon_{b} = \frac{(A_{1}/A_{2} + 2)}{(n_{1}/n_{2}+ 2)}.
\end{equation}

The values of $A_1$ and $A_2$ are computed using both $t\bar{t}$ and single top simulated events
in the appropriate proportion according to their cross sections. The values are found to be
$A_1=0.117$ and $A_2=0.876$ with respect to the following denominator 
(TODO: these are sort of arbitrary and need to be repeated - mainly that the $b$-quark 
$p_{T}$ should be increased to 30 GeV):

\begin{itemize}
    \item Reconstructed MET $> 55$ GeV
    \item Two reconstructed 20 GeV leptons
    \item At least one good primary vertex
    \item Dilepton pT $> 25$ GeV
    \item Generated $b$-quarks counted if pT$>5$ GeV and $|\eta|$ < 2.4
\end{itemize}

Then finally the number of top events expected after applying the Top veto is found
by scaling the $n_{1}$ sample by the expected efficiencies and acceptances:

\begin{equation}
\label{eqn:topVetoYieldEst}
N_{est} = \frac{n_{1}} {A_{1}\varepsilon + 2A_{2}\varepsilon(1-\varepsilon)} \times \frac{1}{A_{2}(1-\varepsilon)^2 + A_{1}\varepsilon(1-\varepsilon) + 1-(A_{1}+A_{2})}
\end{equation}

To derive a scale factor for the Top simulation, we compare the estimated number of Top events
passing the Top veto, $N_{est}$, with the simulation prediction.
Events in the $e\mu$ channel are used to avoid contamination from $Z+b\bar{b}$ events.
While these can be estimated using the photon+jets method described previously,
the systematic uncertainty attributable to their subtraction is large.
To reduce kinematic biasses, the invariant mass of the two leptons is 
required to be within the same Z mass window as the same flavor events 
in the rest of this analysis.
To set a systematic uncertainty on the prediction we apply the method
consistently on the Top simulation sample, comparing the predicted and
actual number of events that survive the Top veto.

\begin{table}[!ht]
\begin{center}
\small{
\begin{tabular}{c|c|c|c|c|c|c}
\hline
Sample & $\varepsilon$(Top MC) & $N_{est}^{MC}$ & $N_{true}^{MC}$ & $\varepsilon$(Data) & $N_{est}^{data}$ & SF \\ \hline
$MET>70$  &  $0.63 \pm 0.04$  &  $42.48 \pm 0.33$  &  $43.12 \pm 0.63$  &  $0.58 \pm 0.03$  &  $51.37 \pm 5.54$  &  $1.19 \pm 0.11 \pm 0.01$ \\ \hline
$MET>80$  &  $0.63 \pm 0.04$  &  $33.15 \pm 0.29$  &  $32.11 \pm 0.54$  &  $0.55 \pm 0.04$  &  $43.51 \pm 5.36$  &  $1.36 \pm 0.12 \pm 0.03$ \\ \hline
\end{tabular}
\caption{Top scale factors for MT shape analysis.}
\label{tab:top_sf}}
\end{center}
\end{table}

\subsubsection{WW Background Estimate}

We extract the scale factor for the $WW$ process by comparing the
yield at the preselection level, but requiring $e\mu$ events instead
of same flavor. 
To increase the statistical precision of the scale factor,
we relax the mass window to $>12$ GeV.
We subtract the expected Top yield using simulation, which is 
corrected for the appropriate Top scale factor.  
Other non-$WW$ backgrounds are subtracted from simulation.
The scale factors for the two shape analysis met selections
are given in Table \ref{tab:yield_of_0jet_wwsf}.

\begin{table}[!ht]
\begin{center}
\small{
\begin{tabular}{c|c|c|c|c|c|c|c}
\hline
Sample & Top & WW & ZZ & WZ & DYtt & DATA & SF \\ \hline
Met $> 70.00$ & $37.14 \pm 3.50$ & $67.58 \pm 0.89$ & $0.08 \pm 0.01$ & $1.34 \pm 0.05$ & $2.99 \pm 0.91$ & $130.00 \pm 11.40$ & $1.31 \pm 0.18$ \\ \hline
Met $> 80.00$ & $29.09 \pm 2.69$ & $34.24 \pm 0.63$ & $0.06 \pm 0.01$ & $0.71 \pm 0.04$ & $1.33 \pm 0.55$ & $72.00 \pm 8.49$ & $1.19 \pm 0.26$ \\ \hline
\end{tabular}
\caption{WW scale factors for MT shape analysis.}
\label{tab:yield_of_0jet_wwsf}}
\end{center}
\end{table}

\subsubsection{Cross check with OF Counting}

To verify the scale factors for WW and Top, we compare the predicted yields with the prediction
for the sum of the two processes derived using the opposite flavor counting method
as in the cut based analysis.

\begin{table}[!ht]
\begin{center}
\small{
\begin{tabular}{c|c|c|c}
sample  & mm    & ee     & TOTAL\\ \hline 
\hline
\multicolumn{4}{c}{Preselection (Met $>50$ and MT $>150$} \\
\hline
\hline
WWTop   & 104.58 $\pm$ 7.83 & 77.95 $\pm$ 5.83  & 182.53 $\pm$ 9.77 \\ \hline 
Top     & 51.51 $\pm$ 0.73  & 36.07 $\pm$ 0.62  & 87.57 $\pm$ 0.96 \\ \hline 
WW      & 54.59 $\pm$ 0.90  & 38.14 $\pm$ 0.76  & 92.73 $\pm$ 1.18 \\ \hline 
\hline
\multicolumn{4}{c}{Add Met $>70$} \\
\hline
\hline 
WWTop   & 45.91 $\pm$ 5.21  & 33.94 $\pm$ 3.85  & 79.85 $\pm$ 6.48 \\ \hline 
Top     & 32.99 $\pm$ 0.56  & 23.15 $\pm$ 0.50  & 56.14 $\pm$ 0.75 \\ \hline 
WW      & 23.65 $\pm$ 0.59  & 16.20 $\pm$ 0.49  & 39.84 $\pm$ 0.77 \\ \hline 
\hline
\multicolumn{4}{c}{Add $230<$MT$<300$} \\
\hline
\hline 
WWTop   & 26.55 $\pm$ 3.97  & 19.82 $\pm$ 2.96  & 46.38 $\pm$ 4.95 \\ \hline 
Top     & 19.85 $\pm$ 0.45  & 13.90 $\pm$ 0.40  & 33.76 $\pm$ 0.60 \\ \hline 
WW      & 16.83 $\pm$ 0.50  & 12.75 $\pm$ 0.44  & 29.58 $\pm$ 0.67 \\ \hline 
\hline
\multicolumn{4}{c}{Add $230<$MT$<300$ and Met $>70$} \\
\hline
\hline 
WWTop   & 16.78 $\pm$ 3.18  & 12.27 $\pm$ 2.32  & 29.04 $\pm$ 3.94 \\ \hline 
Top     & 15.59 $\pm$ 0.40  & 10.81 $\pm$ 0.35  & 26.40 $\pm$ 0.53 \\ \hline 
WW      & 11.06 $\pm$ 0.41  & 8.44 $\pm$ 0.36   & 19.49 $\pm$ 0.54 \\ \hline 
\end{tabular}
\caption{Validation of WW and Top Scale Factors.  Table will show WW, Top and WWTop from OF yields.}
\label{tab:yield_presel_ofsfval}}
\end{center}
\end{table}

We conclude that the separate WW and Top scale factors are suspicious and we do not use them.

