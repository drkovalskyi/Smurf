
In the $M_T$-Shape based analysis, we separate the $WW$ and $Top$ backgrounds,
in order to better quantify the relevant shape systematics.
We do so by deriving simulation-to-data scale factors separately for the two backgrounds.

\subsubsection{Top Background Estimate}

Top quarks decay via $t\rightarrow bW$ with almost certainty.  
We explout this to estimate the background by counting events where $b$-quarks were produced. 
If the counted events are dominated by $t\bar{t}$ events, 

The true number of $t\bar{t}$ events containing exactly one $b$-quark within the geometric acceptance 
of the detector and exactly two $b$-quarks within the same acceptance are denoted $N_{1}$ and $N_{2}$.  
These are related to the observed numbers of events, $n_{1}$ and $n_{2}$, 
through the efficiency with which a jet originating from a $b$-quark 
is reconstructed and tagged, $\varepsilon_{b}$, by 

\begin{eqnarray}
n_{1} = N_{1}\varepsilon_{b} + 2N_{2}\varepsilon_{b}(1-\varepsilon_{b}) \label{eqn:analysis_bTagEff1_n1},\\
n_{2} = N_{2}\varepsilon_{b}^{2} \label{eqn:analysis_bTagEff1_n2}.
\end{eqnarray}

To determine the value of $\varepsilon_{b}$ the unknown values of $N_{1}$ and $N_{2}$ 
can be eliminated by relating each to the true number of $t\bar{t}$ events, $N$, 
through the geometric acceptances, $A_1$ and $A_2$, for exactly one and exactly two $b$ quarks 
respectively in the case where the W bosons decay to the appropriate lepton flavor. 
This is expressed by

\begin{eqnarray}
N_{1} = N\cdot A_{1}, \\
N_{2} = N\cdot A_{2}.
\end{eqnarray}

Eliminating the unknowns, the value of $\varepsilon_{b}$ is related to the observed numbers of events through 

\begin{equation}
\label{eqn:analysis_bTagEff}
\varepsilon_{b} = \frac{(A_{1}/A_{2} + 2)}{(n_{1}/n_{2}+ 2)}.
\end{equation}

The values of $A_1$ and $A_2$ are computed using both $t\bar{t}$ and single top simulated events
in the appropriate proportion according to their cross sections. The values are found to be
$A_1=0.117$ and $A_2=0.876$ with respect to the following denominator 
(TODO: these are sort of arbitrary and need to be repeated):

\begin{itemize}
    \item Reconstructed MET $> 55$ GeV
    \item Two reconstructed 20 GeV leptons
    \item At least one good primary vertex
    \item Dilepton pT $> 25$ GeV
    \item Generated $b$-quarks counted if pT$>5$ GeV and $|\eta|$ < 2.4
\end{itemize}

Then finally the number of top events expected after applying the Top veto is found
by scaling the $n_{1}$ sample by the expected efficiencies and acceptances:

TODO: Insert equation here

We extract the scale factor between the number of events predicted in data at the 
preselection level and the number predicted by Top simulation.
The systematic uncertainty comes from applying the method consistently on
Top simulation (MC closure test).

$SF = 1.24 +/- 0.08 +/- 0.08 syst$

\subsubsection{WW Background Estimate}

We extract the scale factor for the $WW$ process by comparing the
yield at the preselection level, but requiring $e\mu$ events instead
of same flavor. 
In this case, to increase the statistical precision of the scale factor,
we relax the mass window to $>12$ GeV.
We subtract the expected Top yield using Top simulation, which is 
corrected for the scale factor for this process.  Other non-$WW$
backgrounds are subtracted from simulation.
We take the $WW$ scale factor from the $Met>50$ region
to be $1.29 \pm 0.13$.

\begin{table}[!ht]
\begin{center}
 \tiny{
\begin{tabular}{c|c|c|c|c|c|c|c}
\hline
Sample & Top & WW & ZZ & WZ & DYtt & DATA & SF \\ \hline
Met $> 20.00$ & $92.76 \pm 8.52$ & $330.71 \pm 1.97$ & $0.16 \pm 0.01$ & $6.88 \pm 0.12$ & $23.69 \pm 2.51$ & $576.00 \pm 24.00$ & $1.37 \pm 0.08$ \\ \hline
Met $> 30.00$ & $86.52 \pm 7.95$ & $293.46 \pm 1.85$ & $0.15 \pm 0.01$ & $6.05 \pm 0.11$ & $17.40 \pm 2.18$ & $513.00 \pm 22.65$ & $1.37 \pm 0.08$ \\ \hline
Met $> 40.00$ & $77.74 \pm 7.15$ & $238.56 \pm 1.67$ & $0.14 \pm 0.01$ & $4.74 \pm 0.10$ & $12.00 \pm 1.82$ & $400.00 \pm 20.00$ & $1.28 \pm 0.09$ \\ \hline
Met $> 50.00$ & $66.46 \pm 6.12$ & $175.62 \pm 1.43$ & $0.12 \pm 0.01$ & $3.41 \pm 0.08$ & $6.49 \pm 1.34$ & $316.00 \pm 17.78$ & $1.36 \pm 0.11$ \\ \hline
Met $> 60.00$ & $53.10 \pm 4.90$ & $118.13 \pm 1.17$ & $0.10 \pm 0.01$ & $2.28 \pm 0.07$ & $3.74 \pm 1.01$ & $217.00 \pm 14.73$ & $1.34 \pm 0.13$ \\ \hline
\end{tabular}
\caption{Zero jet bin WW Scale Factor}
\label{tab:yield_of_0jet_wwsf}}
\end{center}
\end{table}

\subsubsection{Preselection level cross-check}

To verify the scale factors for WW and Top, we compare the predicted yields with the prediction
for the sum of the two processes derived using the opposite flavor counting method
as in the cut based analysis.

\begin{table}[!ht]
\begin{center}
\begin{tabular}{c|c|c|c|c|c}
sample  & mm    & ee     & TOTAL\\ \hline 
WWTop(OF)   & 104.58 $\pm$ 7.83 & 77.95 $\pm$ 5.83  & 182.53 $\pm$ 9.77 \\ \hline 
WW$\times$SF  & 56.67 $\pm$ 0.94  & 39.60 $\pm$ 0.79  & 96.27 $\pm$ 1.23 \\ \hline 
Top$\times$SF & 53.67 $\pm$ 0.76  & 37.59 $\pm$ 0.65  & 91.25 $\pm$ 1.00 \\ \hline 
\end{tabular}
\caption{Validation of WW and Top Scale Factors}
\label{tab:yield_presel_ofsfval}
\end{center}
\end{table}

The Scale Factor method predicts 110.3 WW+Top events in the $mm$ channel 
compared to 104.6 events predicted by the OF method.
Likewise the Scale factor method predicts 77.2 events in the $ee$ channel
compared to 77.95 events predicted by the OF method.

\begin{itemize}
    \item TO DO:
    \begin{itemize}
        \item This can be used to set a systematic uncertainty on the SF.
        \item comparing the SF prediction and emu counting can be used as the cuts get tighter for the higher mH points
to determine the uncertainty on the scale factors extrapolating from the control region to tighter cuts.
    \end{itemize}
\end{itemize}

