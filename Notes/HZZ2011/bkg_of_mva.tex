
To compare consistently with the cut based analysis, we employ a scale factor
for the WW and Top simulation.  We do this by scaling the MC meeting the
shape based preselections to match the $e\mu$ estimate in data.
The resulting scale factors are tabulated in Table \ref{tab:shape_sf_emu}.
We then take the WW and Top shapes from the scaled simulation because the 
scaled $e\mu$ shape taken directly from data suffers from larger statistical
fluctuations than the simulation.

\begin{table}[!ht]
\begin{center}
\small{
\begin{tabular}{c|c|c|c|c}
Mass &   WW             & TOP              & $e\mu$ estimate   & SF \\ \hline
250  & $14.12 \pm 0.40$ & $19.88 \pm 0.43$ & $25.82 \pm 3.70$  & $0.76 \pm 0.11$ \\ \hline
300  & $5.38 \pm 0.25$  & $9.09 \pm 0.29$  & $14.24 \pm 2.73$  &  $0.98 \pm 0.19$ \\ \hline
350  & $5.44 \pm 0.25$  & $9.22 \pm 0.30$  & $14.24 \pm 2.73$  &  $0.97 \pm 0.19$ \\ \hline
400  & $5.44 \pm 0.25$  & $9.24 \pm 0.30$  & $14.24 \pm 2.73$  &  $0.97 \pm 0.19$ \\ \hline
500  & $5.44 \pm 0.25$  & $9.24 \pm 0.30$  & $14.24 \pm 2.73$  & $0.97 \pm 0.19$ \\ \hline
600  & $5.44 \pm 0.25$  & $9.24 \pm 0.30$  & $14.24 \pm 2.73$  & $0.97 \pm 0.19$ \\ \hline
\end{tabular}
\caption{Scale factors for WW and Top for shape analysis derived using the $e\mu$ estimate in data.}
\label{tab:shape_sf_emu}}
\end{center}
\end{table}

