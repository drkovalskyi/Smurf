To enhance the sensitivity to the Higgs boson signal, two different approaches 
are performed. The first one is a cut-based approach where further requirements 
on a few observables are applied, while the second one makes use of
multivariate techniques. Both of them cover a large Higgs boson mass
($m_{\rm{H}}$) range, and each is separately optimized for different
$m_{\rm{H}}$ hypotheses. The first method is the simplest approach with smaller
systematic uncertainties. The second one is
more powerful, since it exploits the information present in the
correlation among the variables. 
The cut-based analysis is described in this section, while the matrix element 
method based approach is documented in appendix~\ref{app:anal_me}. 



%% The quoted results in this section are scaled to 1 $\ifb$ of integrated luminosity. 
%% The background yields have been scaled taking into account the observed data 
%% corrections, discussed in Section~\ref{sec:backgrounds}, with the current data 
%% sample to give realistic estimations, while $gg \to H \to ZZ$ simulated 
%% events are reweighted to match the Higgs $\pt$ at NNLO, as explained in 
%% Section~\ref{sec:datasets}. 

%% Stuff to add in the MVA later
%To enhance the sensitivity to the Higgs boson signal, two different approaches 
%are performed. The first one is a cut-based approach where further requirements 
%on a few observables are applied, while the second one makes use of
%multivariate techniques. Both of them cover a large Higgs boson mass
%($m_{\rm{H}}$) range, and each is separately optimized for different
%$m_{\rm{H}}$ hypotheses. The first method is the simplest approach with smaller
%systematic uncertainties. The second one is
%more powerful, since it exploits the information present in the
%correlation among the variables. 

%Output of the multivariate discriminator has two different use
%cases. In the first case we use it as just one more variable to cut on
%in the cut-based analysis. In the second case we use the discriminator
%output distribution for the final signal extraction.

%All analyses are further split in the corresponding 0-jet, 1-jet and
%2-jet bins. In the 2-jet bin we use a simple cut-based approach for
%now due to the limited sensitivity and the limited number of events in
%simulation.

