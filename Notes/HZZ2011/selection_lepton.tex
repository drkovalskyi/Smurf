For the lepton selections, we use the criteria defined in Ref~\cite{HZZ2011EPS}. 
In this section, we briefly list the selections. 

We select electrons using a cut-based approach consistent with the electron 
selection criteria used for the measurement of the inclusive W and Z 
cross-section~\cite{VBTFCrossSectionNote}. 
The specific requirements to select good prompt electrons are the following:

\begin{itemize}
    \item Basic acceptance cuts are imposed,  $p_T>20$~GeV and $|\eta| < 2.5$
    \item Standard electron identification is applied to barrel (endcap) electrons
    \begin{itemize}
        \item The latteral shower shape, $\sigma_{i\eta i\eta} < 0.01~(0.03)$
        \item The track-cluster matching in the $\phi$-direction, $\Delta \phi_{\mathrm{in}} < 0.06~(0.03)$
        \item The track-cluster matching in the $\eta$-direction, $\Delta \eta_{\mathrm{in}} < 0.004~(0.007)$
        \item The relative hadronic activity, $H/E<0.04$~(0.15) 
        \item The transverse impact parater with respect to the primary vertex, $|d_0|<0.02$ cm
        \item The longitudinal impact parameter with respect to the primary vertex, $|d_z|<0.1$ cm
    \end{itemize}
\end{itemize}

The H/E requirement is looser in the endcap since the tighter values
introduce noticeable reconstruction efficiency drop with large pileup.

Tracker and calorimeter based isolation variations are 
used to reduce the fake muons originating from jets. 
we require $\rm{Iso}/\pt~<~0.1$, where 
$\rm{Iso} = \rm{Iso}_{\rm{Track}} + \rm{Iso}_{\rm{ECAL}} + \rm{Iso}_{\rm{HCAL}} - \rho_{\mathrm{FJ}}\cdot\pi\cdot 0.3^2$ 
are calculated within a cone of $0.3$ around the electron. 
We correct for the average energy density, $\rho_{\mathrm{FJ}}$, due to 
pileup using the Fast Jet procedure. Additional vetos have been applied to remove the tracks and ECAL crystals footprint.

Lastly to veto fake electrons from converted photons, 
%we look for a reconstructed conversion vertex where one of the two tracks 
%is compatible with the electron~\cite{ConversionNote}.
%The vertex fit probability is required to be $>10^{-6}$.
we require that there are no missing expected inner hits forming the electron 
track~\cite{ConversionNote},~\cite{NExpHits}. 


Muons in CMS are reconstructed as either $StandAloneMuons$ (track
in the muon detector with low momentum resolution), $GlobalMuons$
(outside-in approach seeded by a $StandAloneMuon$ with a global fit
using hits in the muon, silicon strip and pixel 
detectors) or $TrackerMuons$ (inside-out approach seeded by an offline 
silicon strip track, using the muon detector only for muon identification 
without refitting the track). Most good quality muons are reconstructed as 
all three types at the same time and the momentum resolution is dominated by the inner
tracker system up to about 200~$\GeVc$ in transverse momentum. 

The specific requirements to select good prompt isolated muons are the following:
\begin{itemize}
\item the muon must be found by both the global and tracker muon algorithms;
\item the global muon must have at least one good muon hit;
\item the tracker muon must have at least two matches to muon segments in 
      different muon stations;
\item more than 10 hits in the inner tracker;
\item at least one pixel hit;
\item $\chi^2/{\mathrm{ndof}} < 10$ on the global fit;
\item The transverse impact parater with respect to the primary vertex, $|d_0|<0.02$ cm
\item The longitudinal impact parater with respect to the primary vertex, $|d_z|<0.1$ cm
\item pseudorapidity $|\eta|$ must be smaller than 2.4;
\item relative \pt\ resolution is better than 10\%.
\end{itemize}

The muons are required to be isolated by relative isolation cuts 
$\rm{Iso}/\pt~<~0.15$ with the same procedure to correct for the 
pileup effects. Note that we also subtract the contributions within a 
small veto cone $\Delta R < 0.01$ around the muon. 

