We use a combination of data-driven methods and detailed Monte Carlo
simulation studies to estimate background contributions. From data we
can estimate the following backgrounds:  $\WW$, $\dyll$, top ($\ttbar$ and $\tw$), 
$\Wjets$. The background from the remaining processes ($\zz$ and $\wz$)
are taken from simulation. 

Background composition and yields depend on the final state and on
the Higgs boson mass hypothesis under study. In the 0-Jet final state, 
the diboson backgrounds coming from the non-resonant $\zz$, $\wz$ and $\ww$ processes dominate. 
In comparison, the contributions from the other backgrounds such as top 
and Drell-Yan become neglible. 
In the 1-Jet and 2-Jet final states, the largest background contribution from 
the $\Zjets$ process, while the non-resonant \zz\ and \wz\ backgrounds become the second largest source. 

For the backgrounds that can be estimated from data, 
we perform a data-driven background estimate in the signal region 
if the expected background contribution is sizable. 
If the expected contribution in the signal region is limited by statistics, 
we first estimate the background contribution with the $\zz$ preselection from data 
and then extrapolate this estimation to the signal region using MC. The particular
choice of which backgrounds are estimated in the first or second way depends on the
integrated luminosity of the data sample that we analyze.
For the peaking component of $\wz$ and $\zz$ backgrounds estimated using simulation, 
We assign systematic uncertainties on their cross-sections as $8\%$ and $5\%$ respectively, 
augmentated by the luminosity normalisation uncertainty. 

We now begin describing the background estimated using data-driven methods. 