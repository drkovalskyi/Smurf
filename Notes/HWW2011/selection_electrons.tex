We select electrons using a cut-based approach consistent with the electron 
selection criteria used for the measurement of the inclusive W and Z 
cross-section~\cite{VBTFCrossSectionNote}. 
Because we allow the trailing electron leg to have lower $p_T$ compared
to that measurement, we make additional requirements to further reduce
jet induced background.
The choice of electron selection is described in detail in Appendix~\ref{app:els}.
The specific requirements to select good prompt electrons are the following:

\begin{itemize}
    \item Basic fiduciality cuts are imposed,  $p_T>15$~GeV and $|\eta| < 2.5$
    \item Standard electron identification is applied to barrel (endcap) electrons when $p_T>20$~GeV
    \begin{itemize}
        \item The latteral shower shape, $\sigma_{i\eta i\eta} < 0.01~(0.03)$
        \item The track-cluster matching in the $\phi$-direction, $\Delta \phi_{\mathrm{in}} < 0.06~(0.03)$
        \item The track-cluster matching in the $\eta$-direction, $\Delta \eta_{\mathrm{in}} < 0.004~(0.007)$
        \item The relative hadronic activity, $H/E<0.04$~(N/A)
    \end{itemize}
    \item Standard electron identification is applied to barrel (endcap) electrons when $15<p_T<20$~GeV
    \begin{itemize}
        \item The lateral shower shape, $\sigma_{i\eta i\eta} < 0.01~(0.03)$
        \item The track-cluster matching in the $\phi$-direction, $\Delta \phi_{\mathrm{in}} < 0.03~(0.02)$
        \item The track-cluster matching in the $\eta$-direction, $\Delta \eta_{\mathrm{in}} < 0.004~(0.005)$
        \item The relative hadronic activity, $H/E<0.025$~(N/A)
        \item The track bremsstrahlung fraction, $\mathrm{fbrem}>0.15~\mathrm{OR}~(|\eta|<1.0~\mathrm{AND}~E/p_{IN}>0.95)$
    \end{itemize}
\end{itemize}

Isolation requirements are then imposed by computing the combined relative isolation.

\begin{itemize}
    \item $\Sigma (\rm{Iso}_{Track}$, $\rm{Iso}_{ECAL}$, $\rm{Iso}_{HCAL}) / p_T < 0.10$
\end{itemize}

When computing the sum, tracks and ECAL energy deposits that are consistent 
with the footprint of the electron are vetoed~\cite{ElIso}.
In the barrel region we subtract 1~GeV from the ECAL isolation sum to account for pedestal effects.
If the subtraction causes this sum to become negative, it is set to zero. 

The effect of the pileup contributions to the muon isolation selection efficiency 
is similar to that of muons (Sec.~\ref{sec:sel_muons}). 
The average efficiency loss for signal electrons with $p_{T} > 20~\GeVc$ 
was assessed to be approximately $2\%$ using Monte Carlo simulation of the decay
of a Higgs boson with a mass of $130~\GeVcc$. 
We currently do not perform any pileup corrections for isolation while 
we continue to investigate other options.

%The presence of pileup introduces additional energy into the isolation cone.
%This is expected to reduce the efficiency of the isolation cut as the ammount of 
%pileup increases.

%One specific procedure to correct for pileup contributions to the
%isolation sum is referred to as the fastjet correction method.
%This was investigated and found to over-correct the efficiency on background electrons,
%giving comparatively worse performance.
%Therefore we currently do not perform any pileup corrections for isolation while
%we continue to investigate other options.
%This is discussed in more detail in Appendix \ref{app:PUIso}.

In order to veto fake electrons from converted photons, 
we look for a reconstructed conversion vertex where one of the two tracks 
is compatible with the electron~\cite{ConversionNote}.
The vertex fit probability is required to be $>10^{-6}$.
We then require that there are no missing expected missing hts forming the electron track~\cite{ConversionNote},~\cite{NExpHits}. 
Finally to reduce fake electrons from non-prompt sources,
we require the transverse and longitudinal impact parameters with
respect to the primary vertex to be less than $0.02$ and $0.2$~cm respectively.

