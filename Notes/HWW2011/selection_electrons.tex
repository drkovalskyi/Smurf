We select electrons using a cut-based approach consistent with the electron 
selection criteria used for the measurement of the inclusive W and Z 
cross-section~\cite{VBTFCrossSectionNote}. In order to deal with large 
background rates at low electron momentum, we developed a few additional 
requirements. The choice of the electron selector type and the optimization procedure
are described in Appendix~\ref{app:els}.

Electrons are required to have transverse momentum larger than $15$ GeV and $|\eta| < 2.5$. 
Electron identification is based on selection cuts on shower shape ($\sigma_{i\eta i\eta}$), 
track to cluster matching ($\Delta \phi_{\mathrm{in}}$ and $\Delta \eta_{\mathrm{in}}$), and the amount 
of relative hadronic activity (H/E). 
For electrons with $p_T<$20 GeV, the selection is tightened by adding a cut on the fraction of momentum 
loss due to radiation (fbrem) and on the ratio between the SuperCluster energy and the track momentum ($E/p$).
The electron is required to be isolated by imposing a requirement on the combined relative isolation 
($\rm{Iso}_{Track}$, $\rm{Iso}_{ECAL}$, $\rm{Iso}_{HCAL}$ in a $\Delta$R $< 0.3$ cone / $p_{T}$), 
where $\rm{Iso}_{ECAL}$ is calculated using ECAL crystals and additional vetos have been 
applied to remove tracks and ECAL crystals from the isolation sums to better account for 
the electron footprint~\cite{ElIso}. In the barrel region ($|\eta| < 1.479$) we subtract 1~GeV of 
ECAL energy deposition if it is more than 1~GeV to account for the noise pedestal. 
Due to the presence of pileup, which introduces additional energy into 
the isolation cone, the efficiency of the isolation cut is expected
to decrease with increasing amount of pileup. We currently do not perform
any pileup corrections. A more detailed discussion can be found in Appendix
\ref{app:PUIso}.

In order to veto fake electrons from converted photons, we look for a reconstructed conversion vertex where 
one of the two track is compatible with the electron~\cite{ConversionNote}, 
and require that there are no missing expected hits forming the electron track~\cite{ConversionNote},~\cite{NExpHits}. 
Finally we impose cuts on the transverse and longitudinal impact parameters with
respect to the primary vertex to reduce fake electrons from non-prompt
sources. Cut values are summarized in Tab.~\ref{tab:electronSelection}.

\begin{table}[!ht]
\begin{center}
\begin{tabular}{|c|c|c|}
 \hline
 \multicolumn{3}{|c|}{Identification $p_T>$20 GeV} \\
\hline
 Cut Variable           &   Cut Value (Barrel)                   & Cut Value (Endcap)    \\
\hline
 $\sigma_{i\eta i\eta}$      &   $<0.01$                              & $<0.03$               \\ 
 $\Delta\phi_{\mathrm{in}}$  &   $<0.06$                              & $<0.03$               \\ 
 $\Delta\eta_{\mathrm{in}}$  &   $<0.004$                             & $<0.007$               \\ 
 H/E                         &  $<0.04$                          &   -          \\ 
 \hline
 \hline
 \multicolumn{3}{|c|}{Identification 15$<p_T<$20 GeV} \\
\hline
 Cut Variable           &   Cut Value (Barrel)                   & Cut Value (Endcap)    \\
\hline
 $\sigma_{i\eta i\eta}$      &   $<0.01$                              & $<0.03$               \\ 
 $\Delta\phi_{\mathrm{in}}$  &   $<0.03$                              & $<0.02$               \\ 
 $\Delta\eta_{\mathrm{in}}$  &   $<0.004$                             & $<0.005$               \\ 
 H/E                       &  $<0.025$                          &   -          \\ \hline
 Additional cut           &  \multicolumn{2}{|c|}{$fbrem>0.15~OR~(|\eta|<1~AND~E/p>0.95)$} \\ 
 \hline
 \hline
 \multicolumn{3}{|c|}{Isolation and Impact Parameter} \\
\hline
 Combined relative isolation &  $<0.1$                           &  $<0.1$              \\
 transverse impact parameter $|d_{0}|$  &  $<0.02$ cm   & $<0.02$ cm    \\
 longitudinal impact parameter $|d_{z}|$  &  $<0.2$ cm   & $<0.2$ cm    \\
 \hline
 \hline
 \multicolumn{3}{|c|}{Conversion Rejection} \\
 \hline
 Missing hits in inner pixel layers  &   $=0$      &  $=0$           \\ 
 $N_{hits}$ before vertex       &  $=0$    &   $=0$      \\  
 Vertex fit probability       &  $>10^{-6}$    &   $>10^{-6}$      \\  
 Transverse vertex distance from PV       &  $>2.0$ cm    &   $>2.0$ cm      \\  
 \hline

\hline
\end{tabular}
\caption{Summary of the electron selection requirements. \label{tab:electronSelection}}

\end{center}
\end{table}
