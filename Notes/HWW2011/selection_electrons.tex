We select electrons using a cut-based approach consistent with the electron 
selection criteria used for the measurement of the inclusive W and Z 
cross-section~\cite{VBTFCrossSectionNote}. 
Because we allow the trailing electron leg to have lower $p_T$ compared
to that measurement, we make additional requirements to further reduce
jet induced background.
The choice of electron selection is described in detail in Appendix~\ref{app:els}.
The specific requirements to select good prompt electrons are the following:

\begin{itemize}
    \item Basic acceptance cuts are imposed,  $p_T>10$~GeV and $|\eta| < 2.5$
    \item Standard electron identification is applied to barrel (endcap) electrons when $p_T>20$~GeV
    \begin{itemize}
        \item The latteral shower shape, $\sigma_{i\eta i\eta} < 0.01~(0.03)$
        \item The track-cluster matching in the $\phi$-direction, $\Delta \phi_{\mathrm{in}} < 0.06~(0.03)$
        \item The track-cluster matching in the $\eta$-direction, $\Delta \eta_{\mathrm{in}} < 0.004~(0.007)$
        \item The relative hadronic activity, $H/E<0.04$~(0.1)
    \end{itemize}
    \item Standard electron identification is applied to barrel (endcap) electrons when $10<p_T<20$~GeV
    \begin{itemize}
        \item The lateral shower shape, $\sigma_{i\eta i\eta} < 0.01~(0.03)$
        \item The track-cluster matching in the $\phi$-direction, $\Delta \phi_{\mathrm{in}} < 0.03~(0.02)$
        \item The track-cluster matching in the $\eta$-direction, $\Delta \eta_{\mathrm{in}} < 0.004~(0.005)$
        \item The relative hadronic activity, $H/E<0.025$~(0.1)
        \item The track bremsstrahlung fraction, $\mathrm{fbrem}>0.15~\mathrm{OR}~(|\eta|<1.0~\mathrm{AND}~E/p_{IN}>0.95)$
    \end{itemize}
\end{itemize}

The H/E requirement is looser in the endcap since the tighter values
introduce noticible reconstruction efficiency drop with large pileup.

Isolation requirements are then imposed by computing the particle flow isolation,
defined as the scalar sum of the \pt\ of the particle flow candidates satisfying 
the following requirements:

\begin{itemize}
\item $\Delta R~<~0.4$ to the electron in the $\eta \times \phi$ plane,
\item for neutral hadron PF candidates, require that it is outside the footprint veto region of $\Delta R~<~0.07$,
\item for photon and electron PF candidates, require that it is outside the footprint veto region of $|\Delta\eta|<0.025$,
\item $|d_{z}(\mathrm{PF~Candidate}) - d_{z}(\mathrm{muon})| < 0.1$~cm, if the PF candidate is charged,
\item \pt $>1.0$ GeV, if the PF candidate is classified as a neutral hadron or a photon.
\end{itemize}

We require $\frac{\rm{Iso}_{PF}}{\pt}~<~0.13~(0.09)$ for electrons in the barrel (endcap). 
Further details of the choice of the isolation requirement is documented in Appendix \ref{app:pfIsoStudy}.

In order to veto fake electrons from converted photons, 
we look for a reconstructed conversion vertex where one of the two tracks 
is compatible with the electron~\cite{ConversionNote}.
The vertex fit probability is required to be $>10^{-6}$.
We then require that there are no missing expected missing hits forming the electron track~\cite{ConversionNote},~\cite{NExpHits}. 
Finally to reduce fake electrons from non-prompt sources,
we require the transverse and longitudinal impact parameters with
respect to the primary vertex to be less than $0.02$ and $0.1$~cm respectively.
