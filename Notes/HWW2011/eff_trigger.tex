 
To determine the efficiency of the dilepton triggers, 
we derive the efficiency of the requirements imposed on each leg separately.
This requires a modification to the tag and probe method described above in some cases.
If the trigger objects are saved by the HLT before the requirement that there be two valid objects then
we can check each leg independently of the other using the usual tag and probe method.
If the trigger objects are saved after the requirement that there are two valid objects, then there is 
a 100\% correlation between the decision we can probe on each lepton.
This means that we must pick exactly one tag candidate for each event a priori, which we do 
randomly. 
If the randomly selected tag candidate meets the tight requirements then we are free to 
probe the other lepton.

%\subsubsection{Electron triggers}
%%%%%% double
The per electron efficiency for the single electron triggers
is given in Table \ref{tab:eff_trigger_e}
and the per electron efficiency for the double electron triggers
is given in Table \ref{tab:eff_trigger_ee}. 
The double electron trigger requires the higher $p_T$ leg to be seeded at Level-1.
The efficiency of the seeded leg with respect to a denominator object above the
trigger $p_T$ threshold was found to be $0.993^{+0.001}_{-0.001}$ in the barrel and
$0.994^{+0.002}_{-0.003}$ in the endcap, which is compatible with the unseeded leg.
The per muon efficiency for
HLT\_IsoMu17 OR HLT\_Mu24
is given in Table \ref{tab:eff_trigger_m}
and the per muon efficiency for HLT\_DoubleMu7, which requires both legs to be seeded
at Level-1 is
given in Table \ref{tab:eff_trigger_mm}. The listed values
represent the overall efficiencies averaged over the run range
of the dataset, absorbing changes in thresholds and seeding requirements
over time.
\vspace{10pt}
\begin{table}[!ht]
\begin{center}
\begin{tabular}{c|c|c}
\hline
Measurement  & $0.00<\eta<1.479$  & $1.479<\eta<2.50$  \\ 
\hline
$  10<p_T<  25$ & 0.00 $\pm$ 0.00  & 0.00 $\pm$ 0.00  \\ \hline 
$  25<p_T<  30$ & 0.17 $\pm$ 0.00  & 0.06 $\pm$ 0.00  \\ \hline 
$  30<p_T<  35$ & 0.65 $\pm$ 0.00  & 0.52 $\pm$ 0.00  \\ \hline 
$  35<p_T<  50$ & 0.98 $\pm$ 0.00  & 0.96 $\pm$ 0.00  \\ \hline 
$  50<p_T<  55$ & 0.98 $\pm$ 0.00  & 0.97 $\pm$ 0.00  \\ \hline 
$  p_T>     55$ & 0.98 $\pm$ 0.00  & 0.98 $\pm$ 0.00  \\ \hline 
\end{tabular}
\caption{Overall single electron efficiency as a function of $p_T$ and $|\eta|$.}
\label{tab:eff_trigger_e}
\end{center}
\end{table}
%
%
%
\begin{table}[!ht]
\begin{center}
\begin{tabular}{c|c|c}
\hline
Measurement  & $0.00<\eta<1.479$  & $1.479<\eta<2.50$  \\ 
\hline
$  10<p_T<  15$ & 0.99 $\pm$ 0.00  & 0.99 $\pm$ 0.01  \\ \hline 
$  15<p_T<  20$ & 1.00 $\pm$ 0.00  & 1.00 $\pm$ 0.00  \\ \hline 
$  p_T>     20$ & 1.00 $\pm$ 0.00  & 1.00 $\pm$ 0.00  \\ \hline 
\end{tabular}
\caption{Overall double electron trigger efficiency as a function of $p_T$ and $|\eta|$.}
\label{tab:eff_trigger_ee}
\end{center}
\end{table}
%
%
%
\begin{table}[!ht]
\begin{center}
\begin{tabular}{c|c|c|c|c}
\hline
Measurement  & $0.00<\eta<0.80$  & $0.80<\eta<1.479$  & $1.479<\eta<2.10$  & $2.10<\eta<2.40$  \\ 
\hline
$  10<p_T<  14$ & 0.00 $\pm$ 0.00  & 0.00 $\pm$ 0.00  & 0.00 $\pm$ 0.00  & 0.01 $\pm$ 0.01  \\ \hline 
$  14<p_T<  16$ & 0.05 $\pm$ 0.01  & 0.03 $\pm$ 0.01  & 0.03 $\pm$ 0.01  & 0.02 $\pm$ 0.01  \\ \hline 
$  16<p_T<  18$ & 0.54 $\pm$ 0.02  & 0.52 $\pm$ 0.01  & 0.54 $\pm$ 0.02  & 0.45 $\pm$ 0.02  \\ \hline 
$  18<p_T<  23$ & 0.92 $\pm$ 0.00  & 0.87 $\pm$ 0.00  & 0.84 $\pm$ 0.01  & 0.69 $\pm$ 0.01  \\ \hline 
$  23<p_T<  25$ & 0.91 $\pm$ 0.00  & 0.87 $\pm$ 0.01  & 0.85 $\pm$ 0.01  & 0.71 $\pm$ 0.01  \\ \hline 
$  25<p_T<  29$ & 0.92 $\pm$ 0.00  & 0.86 $\pm$ 0.00  & 0.85 $\pm$ 0.00  & 0.72 $\pm$ 0.01  \\ \hline 
$  29<p_T<  31$ & 0.94 $\pm$ 0.00  & 0.89 $\pm$ 0.00  & 0.85 $\pm$ 0.01  & 0.75 $\pm$ 0.01  \\ \hline 
$  p_T>     31$ & 0.96 $\pm$ 0.00  & 0.92 $\pm$ 0.00  & 0.88 $\pm$ 0.00  & 0.76 $\pm$ 0.00  \\ \hline 
\end{tabular}
\caption{Overall single muon trigger efficiency as a function of $p_T$ and $|\eta|$.}
\label{tab:eff_trigger_m}
\end{center}
\end{table}

%
%
%
\begin{table}[!ht]
\begin{center}
\begin{tabular}{c|c|c|c}
\hline
Measurement  & $0.00<|\eta|<0.80$  & $0.80<|\eta|<1.20$  & $1.20<|\eta|<2.40$  \\ \hline
$  10<p_T<  15$ & 0.94 $\pm$ 0.02  & 0.92 $\pm$ 0.02  & 0.94 $\pm$ 0.01  \\ \hline 
$  15<p_T<  20$ & 0.97 $\pm$ 0.01  & 0.96 $\pm$ 0.01  & 0.96 $\pm$ 0.00  \\ \hline 
$  p_T>     20$ & 0.97 $\pm$ 0.00  & 0.95 $\pm$ 0.00  & 0.95 $\pm$ 0.00  \\ \hline 
\end{tabular}
\caption{Overall double muon trigger efficiency as a function of $p_T$ and $|\eta|$.}
\label{tab:eff_trigger_mm}
\end{center}
\end{table}

In the case of the $e\mu$ triggers, the efficiency of the muon leg can be
taken from Table \ref{tab:eff_trigger_mm} because the 
requirements other than the $p_T$ threshold are the similar.
In the case of the electron leg, the requirements applied are looser
to those of the double electron triggers.
The efficiency of the electron leg was cross checked using $t\bar{t}$ 
events in the $e\mu$ final state using the tag
and probe method described previously.
In this case the tag was required to be a muon passing the muon part of the
trigger, and the probe was an electron passing the offline selection.
To select a well defined $t\bar{t}$ event sample, the MET was required to be
above $20~GeV$.
The efficiency was consistent with the other double electron triggers,
thus the electron leg can be taken from 
Table \ref{tab:eff_trigger_ee}.

Having measured the per lepton trigger efficiencies 
and for the double and single trigger,
we compute the efficiency for dilepton events to be selected.
We do this by taking into account the two ways an event can be selected: 
the double trigger can pass or the double trigger can fail because one leg is bad
but the good leg can pass the single trigger.
If both legs are bad in the double trigger they will also both be bad in the single trigger
because the requirements of the single trigger are tighter than any single leg of the double trigger.
Thus taking into account combinatorics, the event efficiency $\varepsilon_{\ell\ell'}(p_T,\:\eta,\:p'_T,\:\eta')$
is given in Equation \ref{eqn:evteff}, where $\varepsilon_{S}(p_T,\:\eta)$ is the single 
lepton trigger efficiency and
$\varepsilon_{D}(p_T,\:\eta)$ is the per leg efficiency of the double trigger.

\begin{eqnarray}
\label{eqn:evteff}
\varepsilon_{\ell\ell'}(p_T,\:\eta,\:p'_T,\:\eta') & = & \varepsilon_{D}(p_T,\:\eta) \varepsilon_{D}(p_T',\:\eta') \nonumber\\
               &   & +~\varepsilon_{S}(p_T',\:\eta')(1-\varepsilon_{D}(p_T,\:\eta)) \nonumber\\
               &   & +~\varepsilon_{S}(p_T,\:\eta)(1-\varepsilon_{D}(p_T',\:\eta'))
\end{eqnarray}

The procedure of Equation \ref{eqn:evteff} is applied to simulated Higgs boson decays to obtain an event-by-event weight factor. We find a 
trigger efficiency with respect to the offline selection of $98\%$ for a Higgs boson mass of $115\:\GeVcc$ and this increases with larger
mass hypotheses to higher than $99\%$ for masses above $160\:\GeVcc$.

%Since the triggers involve different lepton flavour 
%combinations, an overall dilepton efficiency is computed for each final state ($ee$, $e\mu$, and $\mu\mu$) 
%by the weighted average of efficiencies from the various trigger paths. For a pair of selected leptons of flavour 
%$\ell$ and $\ell'$ with kinematics $(p_T,\:\eta)$ and $(p'_T,\:\eta')$, 
%
%\begin{equation}
%\varepsilon_{\ell\ell'}(p_T,\:\eta,\:p'_T,\:\eta') = \sum_i f_i\cdot\varepsilon_i(p_T,\:\eta,\:\ell,\:p'_T,\:\eta',\:\ell'),
%\end{equation}
%
%where the summation index $i$ runs over all the trigger paths used to obtain the signal sample and $f_i$
%is the weight of the contribution to the overall efficiency from each trigger.
%
%The determination of $f_i$ comes from the fraction of selected events in data which came 
%through that trigger path. For example, if $10\%$ of the selected $\mu\mu$ events were triggered 
%by HLT\_IsoMu17, then the overall efficiency for $\mu\mu$ events will include the efficiency 
%of HLT\_IsoMu17 weighted by $f_i=0.1$. There is no ambiguity in the assignment of a selected event to
%a trigger since the trigger requirements are applied in a mutually exclusive way: the selection is
%performed by checking each trigger sequentially and requiring that an event passing a trigger
%must not have also passed another trigger previously in the sequence.
%
%For each trigger path, the determination of $\varepsilon_i$ from the measured single leg efficiencies depends
%on whether the trigger is a single or a double lepton trigger. For a single lepton trigger,
%
%\begin{equation}
%\varepsilon_{single}(p_T,\:\eta,\:\ell',\:p'_T,\:\eta',\:\ell') = 
%1 - \left(1-\varepsilon_{leg}(p_T,\:\eta)\right)\left(1-\varepsilon_{leg}(p'_T,\:\eta')\right),
%\end{equation}
%
%whereas for a double lepton trigger,
%
%\begin{equation}
%\varepsilon_{double}(p_T,\:\eta,\:\ell',\:p'_T,\:\eta',\:\ell') = \varepsilon_{leg}(p_T,\:\eta)\cdot\varepsilon_{leg}(p'_T,\:\eta').
%\end{equation}
%
%If there is a mismatch of lepton flavour with the trigger (e.g. $\ell=e$ and $i=$HLT\_IsoMu17) then $\varepsilon_{leg}=0$.


