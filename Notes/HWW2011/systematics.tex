This analysis does not have any clear mass peak anywhere, and hence is to a 
large extend a counting experiment where it is important to understand the 
signal efficiency and the estimation from the backgrounds. 

\subsection{Signal Efficiencies and Background Evaluation}
To compute the signal efficiency, several factors have to be taken into 
account, such as the lepton identification and isolation, jet veto and 
kinematic requirements, in addition to the acceptance of having both leptons in 
the fiducial region. We use simulation to predict those efficiencies, but 
applying the propersimulation to data corrections. For the lepton selection 
and jet veto efficiencies, we use the $\dyll$ events as a control sample to 
study the difference between data and simulation, and to derive data to 
simulation scale factors, as explained in Section~\ref{sec:efficiency}. These 
scale factors are later used to either correct the signal efficiency in 
simulation or to provide the systematic uncertainties. 

The estimation of the different background components is summarized in
Section~\ref{sec:backgrounds}, where a combination of data-driven methods and 
detailed simulated events is used.

\subsection{Other Uncertainties}
One relevant uncertainty for the analysis comes from the luminosity 
measurement, with an uncertainty of 4\%.

Due to several factors, the energy scale for electrons and the momentum 
scale for muons have relatively large uncertainties for the current data 
reconstruction version. Hence, we need to assign a systematic uncertainty by 
varying the transverse momentum of the muons by 1\%, 
and 2\% and 4\% for electrons in the barrel and the endcap, respectively. 
The contribution to the uncertainty on the signal efficiency is about 1.3\% 
only.

Monte Carlo generators produce events with momentum fraction and energy taken 
from parton distribution functions (PDFs), which are empirical functions. They 
are obtained from the result of a QCD fit using data collected from many 
different experiments, and therefore are subject to uncertainties coming from 
these analyses. These uncertainties propagate from the global analysis 
into the predictions for normalizations, selection efficiencies and 
distribution shapes. In order to assign systematic uncertainties we follow the 
strategy defined by the CMS Generator Group described in~\cite{XS}, which is 
consistent with the latest PDF4LHC recommendations~\cite{PDF4LHC}. The 68\% 
C.L.of the positive and negative uncertainties obtained with 
CTEQ66~\cite{Nadolsky:2008zw}, MSTW2008NLO~\cite{Martin:2009iq} and 
NNPDF2.0~\cite{Ball:2010de} sets are considered, adopting the specific 
recommended recipes in each case. The final assigned systematic uncertainties 
corresponds to half of the maximum difference observed between positive and 
negative variations for any combination of the three sets. The maximum 
difference corresponds to a positive variation from one set minus a negative 
variation from a different set, since central values from different sets are 
typically of the size of the uncertainties within a set. Uncertainties due 
to $\alpha_s$ are also considered. An uncertainty of about 3\% has been 
estimated.

\subsection{Summary of Systematic Uncertainties}
The summary of all systematic uncertainties taken into account 
is shown in Tab.~\ref{tab:systww}. The total 
uncertainty depends on the Higgs mass case and jet bin, but it is about 30\% on the 
background estimation, while it is about 20\% on the signal efficiency. These are 
preliminary numbers, more detailed results will be included soon.

\begin{table}[!ht]
\begin{center}
{\tiny
\begin{tabular}{|l|c|c|c|c|c|c|c|c|}
\hline
            &       \multicolumn{8}{|c|}{Relative Uncertainty (\%)} \\
Source      &                            $H \to \WW$ & $qq \to \WW$ & $gg \to \WW$ & $VV$ non-$\Z$ resonant & top & $\dyll$ & $\Wjets$ & $V(W/Z)+\gamma$    \\              
\hline
\hline
Luminosity                               &   4 & --- & --- &   4 & --- & --- & --- &    4  \\
Trigger efficiencies                     & 1.5 & 1.5 & 1.5 & 1.5 & --- & --- & --- &  1.5  \\
Muon efficiency                          & 1.5 & 1.5 & 1.5 & 1.5 & --- & --- & --- &  1.5  \\
Electron id efficiency                   & 2.5 & 2.5 & 2.5 & 2.5 & --- & --- & --- &  2.5  \\
Pomentum scale                           & 1.5 & 1.5 & 1.5 & 1.5 & --- & --- & --- &  1.5  \\
Pile-up                                  & 0.5 & 0.5 & 0.5 & 0.5 & --- & --- & --- &  0.5  \\
$\met$ resolution                        & 1.0 & 1.0 & 1.0 & 1.0 & 1.0 & 3.0 & --- &  1.0  \\
Jet counting                             & 7-20& --- & 5.4 & 5.4 & --- & --- & --- &  5.4  \\  
PDF uncertainties                        & 3.0 & 2.6 & --- &   2 & --- & --- & --- &  5.4  \\    
NLO effects                              & 2.0 & 1.1 & --- & 3.5 & --- & --- & --- &   10  \\  
Fakes                                    & --- & --- & --- & --- & --- & --- &  50 &  ---  \\
Higgscross-section                       & 15  & --- & --- & --- & --- & --- & --- &  ---  \\
$WZ/ZZ$ cross-section                    & --- & --- & --- & 3.0 & --- & --- & --- &  ---  \\
$qq \to WW$ normalization                & --- &  20 & --- & --- & --- & --- & --- &  ---  \\
$gg \to WW$ normalization                & --- & --- &  50 & --- & --- & --- & --- &  ---  \\
$tX$ normalization                       & --- & --- & --- & --- &  20 & --- & --- &  ---  \\
$DY$ normalization                       & --- & --- & --- & --- & --- & 100 & --- &  ---  \\
Statistics                               &   1 &   1 &   1 &   4 &   6 &  20 &  20 &   10  \\
\hline
\end{tabular}
}
\caption{\label{tab:systww} Summary of all systematic uncertainties.}
\end{center}
\end{table}
