This analysis does not have any clear mass peak anywhere, hence is to a 
large extent a counting experiment.
Therefore it is important to understand the 
signal efficiency and the background predictions.
We have taken into account the following systematic uncertainties:

\begin{itemize}
\item {\it Luminosity:} We use the official CMS result, which is $4\%$.

\item {\it Lepton identification and trigger efficiencies:} 
We measure the efficiencies in data using the tag and probe method that is described
in detail in Section~\ref{sec:efficiency}. 
The estimated uncertainty is about $2\%$ per lepton leg.

\item {\it Momentum scale:} 
Due to several factors, the energy scale for electrons and the momentum 
scale for muons have relatively large uncertainties in the current data
processing. 
We assign a systematic uncertainty by varying the transverse momentum of the muons by $1\%$, 
and $2\%$ and $5\%$ for electrons in the barrel and the endcap, respectively. 
The contribution to the uncertainty on the dilepton efficiency is about $1.5\%$.

\item {\it $\met$ modeling:} We use a data-driven method to estimate the $\dyll$
background, which is affected by the $\met$ resolution. 
Events with neutrinos giving real $\met$ in the final state also have a small uncertainty. 
We assess this uncertainty on the event selection efficiency by varying the $\met$ in signal events
by an additional $10\%$. We find an uncertainty on the event selection efficiency of around 2\%.

\item {\it Background estimation:} 
The methods to estimate the different backgrounds are explained in 
Section~\ref{sec:backgrounds}.
Here we summarize the systematic uncertainties associated with the methods used.
  \begin{itemize}
  \item $\WW$ Background: the relative uncertainty on this background is about 
    $30\%$ for $\sim 200 \ipb$, and about $15\%$ for $1\ifb$ (see Table~\ref{tab:wwEstimationSByields}). 
    There is an additional uncertainty on the $gg \to WW$ contribution, which is around $50\%$.
  \item Jet induced backgrounds, $\Wjets$ and $QCD$: the associated systematic
    uncertainty is 35\%.
  \item Top background: this background is estimated using $b$-tagged events and
    the $b$-tagging efficiency, which is measured in control regions in data.
    The associated systematic uncertainties are below $5\%$, 
    while the statistical component is about $40\%$ for $\sim 200 \ipb$ and decreases to $20\%$ with $1 \ifb$.
  \item Drell-Yan background: The uncertainty arises from the limited knowledge of
    events with large $\met$ tails. 
    We conservatively quantify such uncertainty from the variation of the ratio $R_{out/in}$
    (Eq.\ref{eq:dyest}) as a function of the $\met$ cut (plots in Fig. \ref{fig:routin_met}),
    leading to an estimate of $50\%$. 
    As very few $\dyll$ events are selected, this has anyhow a small affect on the final analysis results ($\sim1\%$).
  \item Other backgrounds: The sub-dominant backgrounds are estimated from simulation 
    with appropriate systematic uncertainties on their cross section.
    We take $3\%$ for $\WZ$ and $\ZZ$ events and $10\%$ for $W+\gamma$ events.
    These uncertainties must be augmented by the luminosity normalization uncertainty.
  \end{itemize}

\item {\it Pileup:} an incorrect modeling of the pileup in the Monte Carlo samples 
can bias the expected event yields. As described in Appendix \ref{app:vertex_reweight},
the Monte Carlo events have been re-weighted on the basis of the number of reconstructed
primary vertices. The re-weighting procedure affects only slightly the results of the analysis,
the event yields changing by $\sim1\%$. The latter is conservatively assumed as 
the corresponding systematic uncertainty. 

\item {\it Higgs cross section:} these uncertainties are taken from the Higgs cross
section working group~\cite{LHCHiggsCrossSectionWorkingGroup:2011ti}. The uncertainty 
on the $gg \to H$ production is about 15\% and it is one of the dominant effects.

\item {\it Jet counting and theoretical uncertainties:} 
we must make sure that the jet multiplicity is well reproduced by our 
simulation because we split our analysis in different jet bins. 
Experimental and theoretical uncertainties are both important.
The experimental jet counting efficiency is measured in data 
using $\dyll$ events. The effect of the statistical uncertainty 
of the jet energy correction measurement on the extrapolation
of the jet counting efficiency from $\dyll$ events to Higgs signal
events is propagated and accounted for as the experimental 
systematic uncertainty.
% with a precision already better than $1\%$ at this moment. 
%The uncertainty due to the difference between the jet $\pt$ spectrum in $\dyll$ 
%and Higgs events is about $1\%$.
We consider the theoretical uncertainties on the Parton Distribution Functions (PDFs), 
the uncertainty due to higher order corrections and the effect of the fragmentation and 
hadronization process. The overall uncertainties on the signal efficiency are 
about $7\%$, $10\%$ and $20\%$ for the 0-jet, 1-jet and 2-jet bins, respectively.
Correlations between different jet bins are taken into account when computing
the upper limits. A detailed discussion of the method and procedures for estimating
theoretical uncertainties on the signal yield can be found in 
Section \ref{sec:theorySystematicsSignal} below.

\item {\it Monte Carlo statistics:} We also take into account the 
size of the simulated event samples. 
This contributes an uncertainty of about $2-3\%$ to the signal
efficiencies, but it is as large as $100\%$ for some background components on specific
Higgs mass points.
\end{itemize}

\subsection{Theoretical Systematic Uncertainties on Signal Efficiency}
\label{sec:theorySystematicsSignal}
The theoretical systematic uncertainties on the signal yield is factorized into the
product of two components, assumed to be independent. The first component is the uncertainty
on the fraction of events categorized into the different jet bins and the effect
of migrations across jet bins. The second component is the uncertainty
on the lepton acceptance and the selection efficiency of all other cuts. The effect of
parton distribution function and the value of $\alpha_{s}$, and the effect of 
higher order corrections are considered for both components. For the jet categorization,
we consider, additionally, the effect of higher order log terms via the uncertainty in the 
parton shower model and the underlying event.

\subsubsection{Systematic Uncertainties on the Jet Bin Fractions}
We consider the effect on the jet counting categorization without imposing any additional
selection cuts on the signal. A jet in the event will be counted if it has $p_{T} > 30$ GeV. 
We define the jet bin fractions $f_{0}$, $f_{1}$, $f_{2}$
as the fraction of Higgs events with $0$, $1$, and $2$ or more counted jets. 

The uncertainty due to higher order corrections is evaluated by comparing the effect of
varying the renormalization scale and factorization scale on the jet bin fractions. To 
estimate these jet bin fractions, we combine the inclusive $gg \to H$ production cross section
computed at NNLO \cite{} ($\sigma_{\geq 0}$) and the NLO calculations from MCFM \cite{} for 
the inclusive production cross section of $gg \to H+1$jet ($\sigma_{\geq 1}$) and 
$gg \to H+2$jet ($\sigma_{\geq 2}$), through the following relations:

\begin{eqnarray}
\label{eqn:jetBinFractions}
  f_{0} = (\sigma_{\geq 0} - \sigma_{\geq 1} ) / \sigma_{\geq 0} \\
  f_{1} = (\sigma_{\geq 1} - \sigma_{\geq 2} ) / \sigma_{\geq 0} \\
  f_{2} = \sigma_{\geq 2} / \sigma_{\geq 0} \\
\end{eqnarray}

The relative uncertainties on $\sigma_{\geq 0}$, $\sigma_{\geq 1}$, and $\sigma_{\geq 2}$, are defined as 
$\kappa_{\geq 0}$, $\kappa_{\geq 1}$, and $\kappa_{\geq 2}$ respectively. The uncertainty on the inclusive
production cross section, $\kappa_{\geq 0}$, is cited from the CERN Yellow Report \cite{}. 
$\kappa_{\geq 1}$, and $\kappa_{\geq 2}$ are estimated by performing the following four variations on
the factorization scale ($\mu_{f}$) and renormalization scale ($\mu_{r}$):

\begin{itemize}
\item: $\mu_{f} = m_{H} \times 2$, $\mu_{r} = m_{H} \times 2$,
\item: $\mu_{f} = m_{H} / 2$, $\mu_{r} = m_{H} / 2$,
\item: $\mu_{f} = m_{H} \times 2$, $\mu_{r} = m_{H} / 2$,
\item: $\mu_{f} = m_{H} / 2$, $\mu_{r} = m_{H} \times 2$,

\end{itemize}
using MCFM and evaluating the largest positive and largest negative difference from the central
value, where $\mu_{f} = m_{H}$, $\mu_{r} = m_{H}$, for $\sigma_{\geq 1}$ and $\sigma_{\geq 2}$. The results
for $\kappa_{\geq 0}$, $\kappa_{\geq 1}$, and $\kappa_{\geq 2}$ are summarized in Table 
\ref{tab:InclXS_ScaleVariation}.




\begin{table}[!htbp]
\begin{center}
\begin{tabular}{|c|c|c|c|}

\hline
Higgs Mass     &     $\kappa_{\geq 0}$        &   $\kappa_{\geq 1}$        &     $\kappa_{\geq 2}$       \\
\hline
 115 & $ 1.106$  & $ 1.252$  & $ 1.233$  \\
 120 & $ 1.104$  & $ 1.256$  & $ 1.241$  \\
 130 & $ 1.100$  & $ 1.245$  & $ 1.233$  \\
 140 & $ 1.096$  & $ 1.249$  & $ 1.268$  \\
 150 & $ 1.095$  & $ 1.242$  & $ 1.230$  \\
 160 & $ 1.095$  & $ 1.240$  & $ 1.263$  \\
 170 & $ 1.090$  & $ 1.241$  & $ 1.258$  \\
 180 & $ 1.089$  & $ 1.242$  & $ 1.253$  \\
 190 & $ 1.087$  & $ 1.236$  & $ 1.263$  \\
 200 & $ 1.087$  & $ 1.241$  & $ 1.258$  \\
 210 & $ 1.085$  & $ 1.242$  & $ 1.275$  \\
 220 & $ 1.085$  & $ 1.238$  & $ 1.303$  \\
 230 & $ 1.084$  & $ 1.238$  & $ 1.259$  \\
 240 & $ 1.083$  & $ 1.240$  & $ 1.282$  \\
 250 & $ 1.082$  & $ 1.237$  & $ 1.281$  \\
 300 & $ 1.090$  & $ 1.232$  & $ 1.285$  \\
 350 & $ 1.075$  & $ 1.232$  & $ 1.304$  \\
 400 & $ 1.078$  & $ 1.231$  & $ 1.274$  \\
 450 & $ 1.087$  & $ 1.229$  & $ 1.261$  \\
 500 & $ 1.089$  & $ 1.230$  & $ 1.271$  \\
 550 & $ 1.090$  & $ 1.227$  & $ 1.298$  \\
 600 & $ 1.090$  & $ 1.227$  & $ 1.298$  \\
\hline
\end{tabular}
\caption{  }
\label{tab:InclXS_ScaleVariation}
\end{center}
\end{table}




Using the expressions given in Table \ref{tab:JetBinFractionCorrelatedSystematicsFormulas}, we calculate 
the systematic uncertainties expressed in log-normal form on the jet bin fractions $f_{0}$, $f_{1}$, 
and $f_{2}$. There are three systematic uncertainties for each jet bin fraction due to the higher order
corrections on the three inclusive cross section calculations $\sigma_{\geq 0}$, $\sigma_{\geq 1}$, 
and $\sigma_{\geq 2}$, for a total of $9$ systematic uncertainties, of which only $5$ are non-zero.
The values of these $\kappa$'s reflecting the systematic uncertainties and their correlations are 
summarized in Table \ref{tab:JetBinFractionSystematics_ScaleVariation} for all Higgs masses considered.


\begin{table}[!htbp]
\begin{center}
\begin{tabular}{|c|c|c|c|}

\hline
Nuisance Parameter & $\kappa$'s for $0$ jet bin                                                       & $\kappa$'s for $1$ jet bin                                                      & $\kappa$'s for $2$ jet bin                       \\
\hline
QCDscale\_ggH       & $\kappa^{0}_{\mathrm{QCDscale\_ggH}} = (\kappa_{\geq 0})^{\frac{1}{f_{0}}}$                & $1.0$                                                                           & $1.0$                                            \\
\hline
QCDscale\_ggH1in    & $\kappa^{0}_{\mathrm{QCDscale\_ggH1in}} = (\kappa_{\geq 1})^{- \frac{f_{1}+f_{2}}{f_{0}}}$ & $\kappa^{1}_{\mathrm{QCDscale\_ggH1in}} = (\kappa_{\geq 1})^{\frac{f_{1}+f_{2}}{f_{1}}}$  & $1.0$                                            \\
\hline
QCDscale\_ggH2in    & $1.0$                                                                            & $\kappa^{1}_{\mathrm{QCDscale\_ggH2in}} = (\kappa_{\geq 2})^{- \frac{f_{2}}{f_{1}}} $     & $\kappa^{2}_{\mathrm{QCDscale\_ggH2in}} = \kappa_{\geq 2}$ \\
\hline

\end{tabular}
\caption{ }
\label{tab:JetBinFractionCorrelatedSystematicsFormulas}
\end{center}
\end{table}


\begin{table}[!htbp]
\begin{center}
\begin{tabular}{|c|c|c|c|c|c|}

\hline
Higgs Mass     & $\kappa^{0}_{\mathrm{QCDscale\_ggH}}$ & $\kappa^{0}_{\mathrm{QCDscale\_ggH1in}}$ & $\kappa^{1}_{\mathrm{QCDscale\_ggH1in}}$  & $\kappa^{1}_{\mathrm{QCDscale_ggH2in}}$  & $\kappa^{2}_{\mathrm{QCDscale\_ggH2in}}$   \\
\hline
115 & $ 1.15$  & $ 0.92$  & $ 1.38$  & $ 0.92$  & $ 1.23$  \\
 120 & $ 1.15$  & $ 0.92$  & $ 1.38$  & $ 0.91$  & $ 1.24$  \\
 130 & $ 1.14$  & $ 0.92$  & $ 1.37$  & $ 0.91$  & $ 1.23$  \\
 140 & $ 1.14$  & $ 0.91$  & $ 1.38$  & $ 0.90$  & $ 1.27$  \\
 150 & $ 1.14$  & $ 0.91$  & $ 1.39$  & $ 0.90$  & $ 1.23$  \\
 160 & $ 1.14$  & $ 0.91$  & $ 1.38$  & $ 0.89$  & $ 1.26$  \\
 170 & $ 1.14$  & $ 0.90$  & $ 1.38$  & $ 0.89$  & $ 1.26$  \\
 180 & $ 1.14$  & $ 0.89$  & $ 1.39$  & $ 0.89$  & $ 1.25$  \\
 190 & $ 1.14$  & $ 0.89$  & $ 1.38$  & $ 0.88$  & $ 1.26$  \\
 200 & $ 1.14$  & $ 0.88$  & $ 1.39$  & $ 0.88$  & $ 1.26$  \\
 210 & $ 1.14$  & $ 0.88$  & $ 1.40$  & $ 0.87$  & $ 1.27$  \\
 220 & $ 1.14$  & $ 0.88$  & $ 1.40$  & $ 0.86$  & $ 1.30$  \\
 230 & $ 1.14$  & $ 0.87$  & $ 1.40$  & $ 0.88$  & $ 1.26$  \\
 240 & $ 1.14$  & $ 0.87$  & $ 1.41$  & $ 0.86$  & $ 1.28$  \\
 250 & $ 1.15$  & $ 0.86$  & $ 1.42$  & $ 0.85$  & $ 1.28$  \\
 300 & $ 1.17$  & $ 0.85$  & $ 1.43$  & $ 0.84$  & $ 1.29$  \\
 350 & $ 1.13$  & $ 0.86$  & $ 1.43$  & $ 0.83$  & $ 1.30$  \\
 400 & $ 1.13$  & $ 0.89$  & $ 1.46$  & $ 0.82$  & $ 1.27$  \\
 450 & $ 1.12$  & $ 0.92$  & $ 1.47$  & $ 0.82$  & $ 1.26$  \\
 500 & $ 1.12$  & $ 0.94$  & $ 1.48$  & $ 0.81$  & $ 1.27$  \\
 550 & $ 1.11$  & $ 0.96$  & $ 1.47$  & $ 0.79$  & $ 1.30$  \\
 600 & $ 1.11$  & $ 0.96$  & $ 1.47$  & $ 0.79$  & $ 1.30$  \\
\hline

\end{tabular}
\caption{  }
\label{tab:JetBinFractionSystematics_ScaleVariation}
\end{center}
\end{table}


%% The uncertainties due to the parton distribution functions and $\alpha_{s}$ on the jet bin fractions 
%% are evaluated using the prescription from the PDF4LHC Recommendations \cite{}. As for the systematic
%% uncertainties due to missing higher order corrections, we evaluate the effect of the PDF uncertainties 
%% on the value of $\sigma_{\geq 0}$, $\sigma_{\geq 1}$, and $\sigma_{\geq 2}$ and propagate them to 
%% the jet bin fractions using the formulas from Equations \ref{eqn:jetBinFractions}.

Finally, we estimate the systematic uncertainty due to higher order log terms by testing the default 
parton shower and hadronization model, Pythia, with the results obtained using an alternative
parton shower and hadronization model, Herwig++. The same matrix element calculation, Powheg,
is used in both cases. In order to isolate the effect of the parton shower and hadronization 
model, we reweight the Higgs $p_{T}$ distribution to the reference NNLO+NNLL resummed calculation,
in both cases. Since the underlying event model in the two Monte Carlo are also different, this
estimate includes also the effect of the underlying event model. 

Figure \ref{} shows a comparison of the jet bin fractions for these two Monte Carlo samples
for a Higgs mass of $130$ GeV. The relative differences in the jet bin fractions,
$f_{0}$, $f_{1}$, and $f_{2}$, between the two results are taken as a systematic
uncertainty and are summarized in Table \ref{}.

\begin{table}[!htbp]
\begin{center}
\begin{tabular}{|c|c|c|c|}

\hline
               &   \multicolumn{3}{|c|}{ Systematic Uncertainties on: } \\
\hline
Higgs Mass     &   $f_{0}$   &  $f_{1}$       &   $f_{2}$       \\
\hline


\end{tabular}
\caption{  }
\label{tab:JetBinFractionSystematics_PartonShower}
\end{center}
\end{table}



\subsubsection{Lepton Acceptance and Selection Efficiency }






\subsection{Summary of Systematic Uncertainties}
All systematic uncertainties taken into account in this analysis
are summarized in Table~\ref{tab:systww}.
The total uncertainty depends on the Higgs mass and jet bin considered,
however is typically 35\% on the background estimation and about 20\% 
on the signal efficiency. These results assume an integrated luminosity of $1~\ifb$.

\begin{table}[!ht]
\begin{center}
\caption{\label{tab:systww} Summary of all systematic uncertainties (relative).}
\vspace{5pt}
{\footnotesize
\begin{tabular}{l|c|c|c|c|c|c|c|c}
\hline
%&       \multicolumn{8}{|c|}{Relative Uncertainty (\%)} \\
%Source      &                            $H \to \WW$ & $qq \to \WW$ & $gg \to \WW$ & $VV$ non-$\Z$ resonant & top & $\dyll$ & $\Wjets$ & $V(W/Z)+\gamma$    \\              
\multirow{2}{*}{Source} & $H \to \WW$ & $qq \to$ & $gg \to$  & non-$\Z$ resonant & top & DY & $\Wjets$ & $V(W/Z)+\gamma$    \\
                        &           & $\WW$    & $\WW$       & $VV$              &     &         &          &                     \\
\hline

\hline
Luminosity                               &   4 & --- & --- &   4 & --- & --- & --- &    4  \\
Trigger efficiencies                     & 1.5 & 1.5 & 1.5 & 1.5 & --- & --- & --- &  1.5  \\
Muon efficiency                          & 1.5 & 1.5 & 1.5 & 1.5 & --- & --- & --- &  1.5  \\
Electron id efficiency                   & 2.5 & 2.5 & 2.5 & 2.5 & --- & --- & --- &  2.5  \\
Momentum scale                           & 1.5 & 1.5 & 1.5 & 1.5 & --- & --- & --- &  1.5  \\
$\met$ resolution                        & 1.0 & 1.0 & 1.0 & 1.0 & 1.0 & 3.0 & --- &  1.0  \\
Jet counting                             & 7-20& --- & 5.4 & 5.4 & --- & --- & --- &  5.4  \\  
Higgs cross section                      & 5-15& --- & --- & --- & --- & --- & --- &  ---  \\
$WZ/ZZ$ cross section                    & --- & --- & --- & 3.0 & --- & --- & --- &  ---  \\
$qq \to WW$ norm.                        & --- &  30 & --- & --- & --- & --- & --- &  ---  \\
$gg \to WW$ norm.                        & --- & --- &  50 & --- & --- & --- & --- &  ---  \\
$\Wjets$ norm.                           & --- & --- & --- & --- & --- & --- &  35 &  ---  \\
top  norm.                               & --- & --- & --- & --- &  40 & --- & --- &  ---  \\
$\dyll$ norm.                            & --- & --- & --- & --- & --- &  50 & --- &  ---  \\
$WZ/ZZ$ cross section                    & --- & --- & --- & 3.0 & --- & --- & --- &  ---  \\
Monte Carlo statistics                   &   1 &   1 &   1 &   4 &   6 &  20 &  20 &   10  \\
\hline
\end{tabular}
}
\end{center}
\end{table}
