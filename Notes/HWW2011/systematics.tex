This analysis does not have any clear mass peak anywhere, hence is to a 
large extent a counting experiment.
Therefore it is important to understand the 
signal efficiency and the background predictions.
We have taken into account the following systematic uncertainties:

\begin{itemize}
\item {\it Luminosity:} We use the official CMS result, which is $4\%$.

\item {\it Lepton identification and trigger efficiencies:} 
We measure the efficiencies in data using the tag and probe method that is described
in detail in Section~\ref{sec:efficiency}. 
The estimated uncertainty is about $2\%$ per lepton leg.

\item {\it Momentum scale:} 
Due to several factors, the energy scale for electrons and the momentum 
scale for muons have relatively large uncertainties in the current data
processing. 
We assign a systematic uncertainty by varying the transverse momentum of the muons by $1\%$, 
and $2\%$ and $5\%$ for electrons in the barrel and the endcap, respectively. 
The contribution to the uncertainty on the dilepton efficiency is about $1.5\%$.

\item {\it $\met$ modeling:} We use a data-driven method to estimate the $\dyll$
background, which is affected by the $\met$ resolution. 
Events with neutrinos giving real $\met$ in the final state also have a small uncertainty. 
We assess this uncertainty on the event selection efficiency by varying the $\met$ in signal events
by an additional $10\%$.
We find an uncertainty on the event selection efficiency of around 1\%.

\item {\it Background estimation:} 
The methods to estimate the different backgrounds are explained in 
Section~\ref{sec:backgrounds}.
Here we summarize the systematic uncertainties associated with the methods used.
  \begin{itemize}
  \item $\WW$ Background: the relative uncertainty on this background is about 
    $50\%$ for $\sim 50 \ipb$, and about $15\%$ for $1\ifb$ (see Table~\ref{tab:wwEstimationSByields}). 
    There is an additional uncertainty on the $gg \to WW$ contribution, which is around $50\%$.
  \item Jet induced backgrounds, $\Wjets$ and $QCD$: the associated systematic
    uncertainty is 50\%.
    This is dominated by the uncertainty from the closure tests on simulated events. 
    With 50-100 $\ipb$ the statistical component of the uncertainty is also sizable. 
  \item Top background: this background is estimated using $b$-tagged events and
    the $b$-tagging efficiency, which is measured in control regions in data.
    The associated systematic uncertainties are below $5\%$, 
    while the statistical component is about $50\%$ for $50 \ipb$ and decreases to $20\%$ with $1 \ifb$.
  \item Drell-Yan background: The uncertainty arises from the limited knowledge of
    events with large $\met$ tails. 
    We conservatively quantify such uncertainty from the variation of the ratio $R_{out/in}$
    (Eq.\ref{eq:dyest}) as a function of the $\met$ cut (plots in Fig. \ref{fig:routin_met}),
    leading to an estimate of $100\%$. 
    As very few $\dyll$ events are selected, this has anyhow a small affect on the final analysis results ($\sim1\%$).
  \item Other backgrounds: The sub-dominant backgrounds are estimated from simulation 
    with appropriate systematic uncertainties on their cross section.
    We take $3\%$ for $\WZ$ and $\ZZ$ events and $10\%$ for $W+\gamma$ events.
    These uncertainties must be augmented by the luminosity normalization uncertainty.
  \end{itemize}

\item {\it Pileup:} an incorrect modeling of the pileup in the Monte Carlo samples 
can bias the expected event yields. As described in Appendix \ref{app:vertex_reweight},
the Monte Carlo events have been re-weighted on the basis of the number of reconstructed
primary vertices. The re-weighting procedure affects only slightly the results of the analysis,
the event yields changing by $\sim1\%$. The latter is conservatively assumed as 
the corresponding systematic uncertainty. 

\item {\it Higgs cross section:} these uncertainties are taken from the Higgs cross
section working group~\cite{LHCHiggsCrossSectionWorkingGroup:2011ti}. The uncertainty 
on the $gg \to H$ production is about 15\% and it is one of the dominant effects.

\item {\it Jet counting and theoretical uncertainties:} 
we must make sure that the jet multiplicity is well reproduced by our 
simulation because we split our analysis in different jet bins. 
Experimental and theoretical uncertainties are both important.
The experimental jet counting efficiency is measured in data 
using $\dyll$ events, where an additional uncertainty is added to
account for the difference between the jet $\pt$ spectrum in $\dyll$ and Higgs events.
% with a precision already better than $1\%$ at this moment. 
%The uncertainty due to the difference between the jet $\pt$ spectrum in $\dyll$ 
%and Higgs events is about $1\%$.
The theoretical uncertainties 
We consider the theoretical uncertainties on the Parton Distribution Functions (PDFs), 
the uncertainty due to higher order corrections and the effect of the fragmentation and 
hadronization process. The overall uncertainties on the signal efficiency are 
about $7\%$, $10\%$ and $20\%$ for the 0-jet, 1-jet and 2-jet bins, respectively.
Correlations between different jet bins are taken into account when computing
the upper limits.

\item {\it Monte Carlo statistics:} We also take into account the 
size of the simulated event samples. 
This contributes an uncertainty of about $2-3\%$ to the signal
efficiencies, but it is as large as $100\%$ for some background components on specific
Higgs mass points.
\end{itemize}
 


\subsection{Summary of Systematic Uncertainties}
All systematic uncertainties taken into account in this analysis
are summarized in Table~\ref{tab:systww}.
The total uncertainty depends on the Higgs mass and jet bin considered,
however is typically 35\% on the background estimation and about 20\% 
on the signal efficiency. These results assume an integrated luminosity of $1~\ifb$.

\begin{table}[!ht]
\begin{center}
\caption{\label{tab:systww} Summary of all systematic uncertainties (relative).}
\vspace{5pt}
{\footnotesize
\begin{tabular}{l|c|c|c|c|c|c|c|c}
\hline
%&       \multicolumn{8}{|c|}{Relative Uncertainty (\%)} \\
%Source      &                            $H \to \WW$ & $qq \to \WW$ & $gg \to \WW$ & $VV$ non-$\Z$ resonant & top & $\dyll$ & $\Wjets$ & $V(W/Z)+\gamma$    \\              
\multirow{2}{*}{Source} & $H \to \WW$ & $qq \to$ & $gg \to$  & non-$\Z$ resonant & top & DY & $\Wjets$ & $V(W/Z)+\gamma$    \\
                        &           & $\WW$    & $\WW$       & $VV$              &     &         &          &                     \\
\hline
\hline
Luminosity                               &   4 & --- & --- &   4 & --- & --- & --- &    4  \\
Trigger efficiencies                     & 1.5 & 1.5 & 1.5 & 1.5 & --- & --- & --- &  1.5  \\
Muon efficiency                          & 1.5 & 1.5 & 1.5 & 1.5 & --- & --- & --- &  1.5  \\
Electron id efficiency                   & 2.5 & 2.5 & 2.5 & 2.5 & --- & --- & --- &  2.5  \\
Momentum scale                           & 1.5 & 1.5 & 1.5 & 1.5 & --- & --- & --- &  1.5  \\
$\met$ resolution                        & 1.0 & 1.0 & 1.0 & 1.0 & 1.0 & 3.0 & --- &  1.0  \\
Jet counting                             & 7-20& --- & 5.4 & 5.4 & --- & --- & --- &  5.4  \\  
Higgs cross section                      & 5-15& --- & --- & --- & --- & --- & --- &  ---  \\
$WZ/ZZ$ cross section                    & --- & --- & --- & 3.0 & --- & --- & --- &  ---  \\
$qq \to WW$ norm.                        & --- &  20 & --- & --- & --- & --- & --- &  ---  \\
$gg \to WW$ norm.                        & --- & --- &  50 & --- & --- & --- & --- &  ---  \\
$\Wjets$ norm.                           & --- & --- & --- & --- & --- & --- &  50 &  ---  \\
top  norm.                               & --- & --- & --- & --- &  20 & --- & --- &  ---  \\
$\dyll$ norm.                            & --- & --- & --- & --- & --- & 100 & --- &  ---  \\
$WZ/ZZ$ cross section                    & --- & --- & --- & 3.0 & --- & --- & --- &  ---  \\
Monte Carlo statistics                   &   1 &   1 &   1 &   4 &   6 &  20 &  20 &   10  \\
\hline
\end{tabular}
}
\end{center}
\end{table}
