On top of the $\ww$ preselection, we apply extra selections based on the following variables to 
suppress mainly the nonresonant $\ww$ background:
\begin{itemize}
\item transverse momenta of the harder (\ptlmax ) 
and the softer (\ptlmin) leptons;
\item the dilepton mass $\mll$;
\item the azimuthal angle difference $\delphill$ between the two selected leptons;
\item transverse Higgs mass, 
$m_{T}^{\ell\ell\met} = \sqrt{2\pt^{ll}\met(1-cos(\Delta\phi_{\ell\ell-\met}))}$ where 
$\Delta\phi_{\ell\ell-\met}$ is the angle between dilepton
direction and \met\ in the transverse plane.
\end{itemize}
%% Makes no sense. Mll is definitely better.
%% Among these variables, $\delphill$ provides the best discriminating 
%% power between the Higgs signal and the majority of the backgrounds for 
%% $\hww$ production in the low mass range~\cite{HWW2010}. Leptons originating from 
%% $\hww$ decays tend to have a relatively small opening angle, while those from 
%% backgrounds are preferentially emitted back-to-back (see Figure~\ref{fig:dPhi_jets0}). 

The specific selection requirements vary depending on the $m_H$ hypothesis as detailed in 
Table~\ref{tab:cutanalysis0j}. 
Compared to the cut based approach in 2010 analysis~\cite{HWW2010}, the selections 
are re-tuned to improve the search sensitivity. 
Significant improvement is achieved in the search for low mass Higgs ($m_H<=160\GeVcc$) 
as we lowered the $\pt$ threshold of the trailing leptons compared to the 2010 analysis. 
For the searches of high mass Higgs ($m_H > 160 \GeVcc$) we use the same 
selections as in the 2010 analysis with only one additional cut on the transverse Higgs mass. 
Table~\ref{tab:cutanalysis_perf} compares the performance of a few selections.
The comparison is based on $R$, defined as the ratio of the mean expected upper limit 
cross-section and the SM value, for 1~\ifb\ of integrated luminosity. The expected number 
of signal and background events for an integrated luminosity of 1\ifb{} after 
applying the full selection requirements are shown in 
Table~\ref{tab:cutbase_yields_0j}. The $gg \to H \to WW$ 
simulated events are reweighted to match the Higgs $\pt$ at NNLO, as explained 
in Section~\ref{sec:datasets}; all data-driven corrections are not applied.

\begin{table}[!ht]
  \begin{center}
 {\small
  \begin{tabular} {|c|c|c|c|c|c|c|}
  \hline
\mHi [GeV] & $\ptlmax$ [$\GeV$] & $\ptlmin$ [$\GeV$] & $\mll$ [$\GeV$] & $\delphill$ [dg.] & $m_{T}^{\ell\ell\met}$ [GeV  \\  \hline
           &   $>$               &   $>$               &   $<$             &  $<$          &    [,]                       \\  \hline

    115 & 20  &  10 & 40  & 115 & [70,110]\\
    120 & 20  &  10 & 40  & 115 & [70,120]\\
    130 & 25  &  10 & 45  & 90  & [75,125]\\
    140 & 25  &  15 & 45  & 90  & [80,130]\\
    150 & 27  &  25 & 50  & 90  & [80,150]\\
    160 & 30  &  25 & 50  & 60  & [90,160]\\
    170 & 34  &  25 & 50  & 60  & [110,170]\\
    180 & 36  &  25 & 60  & 70  & [120,180]\\
    190 & 38  &  25 & 80  & 90  & [120,190]\\
    200 & 40  &  25 & 90  & 100 & [120,200]\\
    250 & 55  &  25 & 150 & 140 & [120,250]\\
    300 & 70  &  25 & 200 & 175 & [120,300]\\
    350 & 80  &  25 & 250 & 175 & [120,350]\\
    400 & 90  &  25 & 300 & 175 & [120,400]\\
    450 & 110 &  25 & 350 & 175 & [120,450]\\
    500 & 120 &  25 & 400 & 175 & [120,500]\\
    550 & 130 &  25 & 450 & 175 & [120,550]\\
    600 & 140 &  25 & 500 & 175 & [120,600]\\
  \hline
  \end{tabular}
  }
  \caption{Final event selection requirements for the cut-based analysis in the zero-jet bin. }
   \label{tab:cutanalysis0j}
  \end{center}
\end{table}


\begin{table}[!ht]
  \begin{center}
 {\small
  \begin{tabular} {|c|c|c|c|}
  \hline
  Mass   &  R(20/20) & R(20/10) & R(new) \\
  \hline
  H$_{120}$ & 7.6 & 5.0 & 3.4 \\
  H$_{130}$ & 2.7 & 2.3 & 1.7 \\
  H$_{140}$ & 1.4 & 1.4 & 1.2 \\
  H$_{150}$ & 0.88 & 0.93 & 0.80 \\
  H$_{160}$ & 0.44 & 0.60 & 0.43 \\
  \hline
  \end{tabular}
  }
  \caption{Cut based analysis performance for different tunes. R is a ratio of the mean
   expected upper limit cross-section and the SM value.  R(20/20) corresponds 
  to 2010 analysis with a minimum lepton \pt\ of 20 GeV. R(20/10) corresponds to the same 
  set of kinemaic cuts, but the trailing lepton \pt\ is lowered to 10 GeV for Higgs mass hypothesis 
  of 160 GeV and lower. R(new) corresponds to a new set of cuts used in this analysis. Only 
  basic systematic uncertainties are considered.}
   \label{tab:cutanalysis_perf}
  \end{center}
\end{table}

\begin{table}[!ht]
  \begin{center}
 {\normalsize
  \begin{tabular} {|c|c|c|c|c|c|}
\hline
  mass    & SM $H\to WW$ & all bkg. & $qq \to \WW$ & $gg \to \WW$ & non-resonant $WZ+ZZ$ \\
  \hline
  \hline
115 &    5.6 $\pm$   0.2 &  55.4	 $\pm$    6.0  &  32.3 $\pm$   0.4 &   1.4 $\pm$   0.0 &   0.9 $\pm$   0.1 \\
120 &   10.3 $\pm$   0.3 &  65.5	 $\pm$    6.0  &  41.0 $\pm$   0.5 &   1.9 $\pm$   0.1 &   1.1 $\pm$   0.1 \\
130 &   20.0 $\pm$   0.6 &  67.8	 $\pm$    5.4  &  45.8 $\pm$   0.5 &   2.3 $\pm$   0.1 &   1.2 $\pm$   0.1 \\
140 &   26.0 $\pm$   0.8 &  57.9	 $\pm$    4.5  &  40.8 $\pm$   0.5 &   2.2 $\pm$   0.1 &   0.9 $\pm$   0.1 \\
150 &   25.7 $\pm$   0.8 &  34.5	 $\pm$    2.3  &  26.8 $\pm$   0.4 &   2.0 $\pm$   0.1 &   0.7 $\pm$   0.1 \\
160 &   37.3 $\pm$   1.1 &  24.1	 $\pm$    2.1  &  18.0 $\pm$   0.3 &   1.8 $\pm$   0.1 &   0.5 $\pm$   0.1 \\
170 &   29.8 $\pm$   0.9 &  17.3	 $\pm$    0.5  &  13.6 $\pm$   0.3 &   1.8 $\pm$   0.0 &   0.4 $\pm$   0.1 \\
180 &   21.7 $\pm$   0.6 &  20.7	 $\pm$    0.6  &  15.7 $\pm$   0.3 &   2.2 $\pm$   0.1 &   0.4 $\pm$   0.1 \\
190 &   18.0 $\pm$   0.5 &  30.7	 $\pm$    0.7  &  23.9 $\pm$   0.4 &   3.0 $\pm$   0.1 &   0.6 $\pm$   0.1 \\
200 &   13.6 $\pm$   0.4 &  32.4	 $\pm$    0.7  &  24.9 $\pm$   0.4 &   3.1 $\pm$   0.1 &   0.6 $\pm$   0.1 \\
250 &    6.3 $\pm$   0.2 &  31.3	 $\pm$    0.9  &  21.3 $\pm$   0.4 &   2.2 $\pm$   0.1 &   0.6 $\pm$   0.1 \\
300 &    5.4 $\pm$   0.2 &  27.8	 $\pm$    0.8  &  19.1 $\pm$   0.3 &   1.7 $\pm$   0.0 &   0.5 $\pm$   0.1 \\
350 &    5.9 $\pm$   0.2 &  25.4	 $\pm$    0.8  &  17.1 $\pm$   0.3 &   1.4 $\pm$   0.0 &   0.5 $\pm$   0.1 \\
400 &    4.8 $\pm$   0.1 &  21.6	 $\pm$    0.8  &  14.2 $\pm$   0.3 &   1.2 $\pm$   0.0 &   0.4 $\pm$   0.1 \\
450 &    2.8 $\pm$   0.1 &  13.9	 $\pm$    0.7  &   8.4 $\pm$   0.2 &   0.7 $\pm$   0.0 &   0.3 $\pm$   0.0 \\
500 &    1.8 $\pm$   0.1 &  11.1	 $\pm$    0.6  &   6.7 $\pm$   0.2 &   0.6 $\pm$   0.0 &   0.2 $\pm$   0.0 \\
550 &    1.1 $\pm$   0.0 &   8.7	 $\pm$    0.6  &   5.4 $\pm$   0.2 &   0.5 $\pm$   0.0 &   0.2 $\pm$   0.0 \\
600 &    0.7 $\pm$   0.0 &   6.9	 $\pm$    0.5  &   4.4 $\pm$   0.2 &   0.4 $\pm$   0.0 &   0.1 $\pm$   0.0 \\
 \hline
  \end{tabular}
  }
 {\normalsize
  \begin{tabular} {|c|c|c|c|}
\hline
  mass    & $\ttbar+tW$ & $\dyll+WZ+ZZ$ & $\Wjets$\\
  \hline
  \hline
115 &  1.8 $\pm$	0.4 &	2.4 $\pm$   1.2 &  16.6 $\pm$   5.9  \\
120 &  2.3 $\pm$	0.5 &	2.5 $\pm$   1.2 &  16.6 $\pm$   5.9  \\
130 &  2.5 $\pm$	0.5 &	3.5 $\pm$   1.5 &  12.5 $\pm$   5.1  \\
140 &  2.3 $\pm$	0.4 &	3.3 $\pm$   1.5 &   8.3 $\pm$   4.2  \\
150 &  1.8 $\pm$	0.4 &	1.1 $\pm$   0.8 &   2.1 $\pm$   2.1  \\
160 &  1.5 $\pm$	0.4 &	0.2 $\pm$   0.0 &   2.1 $\pm$   2.1  \\
170 &  1.3 $\pm$	0.3 &	0.2 $\pm$   0.0 &   0.0 $\pm$   0.0  \\
180 &  2.3 $\pm$	0.5 &	0.2 $\pm$   0.0 &   0.0 $\pm$   0.0  \\
190 &  2.9 $\pm$	0.5 &	0.4 $\pm$   0.0 &   0.0 $\pm$   0.0  \\
200 &  3.3 $\pm$	0.6 &	0.5 $\pm$   0.0 &   0.0 $\pm$   0.0  \\
250 &  6.4 $\pm$	0.8 &	0.8 $\pm$   0.1 &   0.0 $\pm$   0.0  \\
300 &  5.7 $\pm$	0.7 &	0.8 $\pm$   0.1 &   0.0 $\pm$   0.0  \\
350 &  5.8 $\pm$	0.7 &	0.7 $\pm$   0.1 &   0.0 $\pm$   0.0  \\
400 &  5.3 $\pm$	0.7 &	0.5 $\pm$   0.0 &   0.0 $\pm$   0.0  \\
450 &  4.2 $\pm$	0.7 &	0.3 $\pm$   0.0 &   0.0 $\pm$   0.0  \\
500 &  3.3 $\pm$	0.6 &	0.2 $\pm$   0.0 &   0.0 $\pm$   0.0  \\
550 &  2.6 $\pm$	0.5 &	0.2 $\pm$   0.0 &   0.0 $\pm$   0.0  \\
600 &  1.9 $\pm$	0.4 &	0.1 $\pm$   0.0 &   0.0 $\pm$   0.0  \\
 \hline
  \end{tabular}
  }
  \caption{Expected number of signal and background events for an 
  integrated luminosity of 1\ifb{} after 
  applying the full cut-based 0-jet selection requirements. Monte Carlo 
  statistical uncertainties are included.}
   \label{tab:cutbase_yields_0j}
  \end{center}
\end{table}
