In this section we explain the definition of the main objects used in the analysis, 
together with the minimal preselection requirements common to all different approaches. 

The Higgs cross-section is several orders of magnitude lower than
major background processes, such as \ttbar{}, \wjets{} and
Drell-Yan. In addition, the Higgs signal has to be separated from the 
irreducible non-resonant $\WW$ background continuum. The fully leptonic 
signature in the detector is two isolated leptons and large missing energy 
from the two undetected neutrinos. The event selection is split in several
steps:

\begin{enumerate}
\item events must be selected by one of the lepton triggers;
\item events are required to have at least two opposite charged, isolated 
and fully identified leptons; $e$ or $\mu$, with 
$\pt>20~\GeVc$ for the leading lepton and $\pt>10~\GeVc$ for the 
trailing lepton. If the lepton is an electron, we raise the requirement to 15
$\GeVc$;
\item a full $\WW$ common preselection is applied with no restrictions in the number 
of reconstructed jets;
\item the final step is a \emph{Higgs mass dependent} event selection. We adopt two 
approaches, one is a cut-based selection and the other one is a multivariate analysis.
\end{enumerate}
