In this section we define the common objects and preselection requirements 
used by all of the approaches considered in this analysis.

The fully leptonic final state consists of two isolated leptons
and large missing energy from the two undetectable neutrinos.
This is the same final state as the non-resonant $\WW$ background.
The Higgs cross-section is several orders of magnitude lower than
the major reducible background processes: \ttbar{}, \wjets{} and Drell-Yan. 
We thus perform several steps to select and extract the Higgs boson signal from data:

\begin{enumerate}
    \item We select events that pass pre-defined lepton triggers
    \item We then select those events with two oppositely charged high $\pt$ isolated leptons ($ee$, $\mu\mu$, $e\mu$) requiring
        \begin{itemize}    
            \item $\pt>20~\GeVc$ for the leading lepton 
            \item $\pt>15(10)~\GeVc$ for the trailing lepton if it is an electron (muon) 
            \item standard identification and isolation requirements on both leptons
        \end{itemize}    
    \item We apply a common $\WW$ preselection with no restriction on the number of reconstructed jets
    \item Finally, we perform two \emph{Higgs mass dependent} event selections, one cut-based and one using a multivariate technique 
described in 
detail in Section~\ref{sec:signal_selection}. 
\end{enumerate}

The $\WW$ preselection steps are now described in detail below.

