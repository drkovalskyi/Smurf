The basic idea for the estimation of the nonresonant $WW$ contribution in the $\hww$ signal region is 
to infer it from data using the dilepton mass distribution:
the dilepton mass defines a control region where we can measure the $WW$ normalization, and then scale
the contribution to the signal region.

It turns out that this approach can be applied for $m_H \leq 200~\GeVcc$ only.
In fact, MC studies show that the di-lepton mass distribution in Higgs samples has a cut-off value at $m_H-50~\GeVcc$;
thus, for large $m_H$ it is possible to define an Higgs-depleted region only in a mass range populated by too 
few $WW$ events. 

Therefore, for $m_H > 200~\GeVcc$ the $WW$ contribution is estimated from simulated events.

\subsubsection{Estimation Procedure in Low Mass Range}

For low Higgs boson mass values ($m_{\rm{H}} \leq 200~\GeVcc$) events with $m_{\ell\ell} > 100~\GeVcc$ are used
to define a control region where the Higgs contribution is $<3\%$.

The procedure is as follows:
\begin{itemize}
\item the yield in the control region is measured after the lepton and jet selections 
(i.e. full selection except $m_{l,l}$, $m_T$ and $\Delta\phi_{ll}$ cuts) so that most of the systematics uncertainties 
cancel out (e.g. jet veto, lepton and trigger efficiencies); 
\item the contamination from other backgrounds (mainly $t\bar t$ and $W$+jets, for a total of $\sim$ 20\%) 
is subtracted using the corresponding data-driven techniques;
\item the obtained yield is scaled to the signal region using the control-to-signal region ratio from MC;
\item finally, the MC efficiency for $m_T$ and $\Delta\phi_{ll}$ cut is accounted to obtain the $WW$ contribution in the 
signal region after all cuts.
\end{itemize}

In order to consider this procedure as reliable we verify that 
the control-to-signal region ratio is stable using different generators and 
we check that the efficiency of the $m_T$ and $\Delta\phi_{ll}$ cuts in the control region in data and MC is consistent.

%Systematics are evaluated repeating the procedure varying the usual suspects.

%
%\subsection{Estimation in high mass range}
%We take it from MC.



%The nonresonant $qq \to \WW$ contribution in the $\hww$ signal region is 
%estimated from data using the dilepton mass distribution. For a given Higgs 
%boson mass, the region with a small contribution from Higgs boson decays is 
%selected and simulation is used to extrapolate this background into the signal 
%region. For low Higgs boson mass values ($m_{\rm{H}} < 200~\GeVcc$) events 
%with $m_{\ell\ell} > 100~\GeVcc$ are used, while for $m_{\rm{H}} > 200~\GeVcc$ 
%events with $m_{\ell\ell} < 100~\GeVcc$ are used. The statistical uncertainty 
%on the estimate of the nonresonant $\WW$ background with the current data 
%sample is approximately 50\%. For the 1- and 2- jet bin cases we use the results
%from the 0-jet bin, and then extrapolate to each jet bin.
%
%The $gg \to \WW$ background contribution has to be taken from simulated events 
%since we do not have enough sensitivity in the data to measure it. We assign a 
%50\% uncertainty to the overall normalization~\cite{ggWWError}. This is 
%obtained by studying the change in the cross-section when varying the parton 
%distribution functions (PDFs), QCD renormalization and scales.
