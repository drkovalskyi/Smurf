The basic idea for the estimation of the nonresonant $WW$ contribution in the $\hww$ signal region is 
to infer it from data using the dilepton mass distribution:
the dilepton mass defines a control region where we can measure the $WW$ normalization, and then scale
the contribution to the signal region.

It turns out that this approach can be applied for $m_H \leq 200~\GeVcc$ only.
In fact, MC studies show that the di-lepton mass distribution in Higgs samples has a cut-off value at $m_H-50~\GeVcc$ 
(Fig.~\ref{fig:higgsMllCutoff});
thus, for large $m_H$ it is possible to define an Higgs-depleted region only in a mass range populated by too 
few $WW$ events. 

Therefore, for $m_H > 200~\GeVcc$ the $WW$ contribution is estimated from simulated events.

\begin{figure}[!hbtp]
\centering
\includegraphics[width=.45\textwidth]{figures/higgsMllCutoff.png}
\caption{Difference between generated Higgs mass and di-lepton reconstructed invariant mass after WW cuts.
Distributions are normalized to unity.}
\label{fig:higgsMllCutoff}
\end{figure}

\subsubsection{Estimation in Low Mass Range}

For low Higgs boson mass values ($m_{\rm{H}} \leq 200~\GeVcc$) events with $m_{\ell\ell} > 100~\GeVcc$ are used
to define a control region where the Higgs contribution is $<3\%$.

The procedure is as follows:
\begin{itemize}
\item the yield in the control region is measured after the lepton and jet selections 
(i.e. full selection except $m_{l,l}$, $m_T$ and $\Delta\phi_{ll}$ cuts) so that most of the systematics uncertainties 
cancel out (e.g. jet veto, lepton and trigger efficiencies); 
\item the contamination from other backgrounds (mainly $t\bar t$ and $W$+jets, for a total of $\sim$ 25\%; see Table~\ref{tab:wwEstimationSByields}) 
is subtracted using the corresponding data-driven techniques;
\item the obtained yield is scaled to the signal region using the control-to-signal region ratio from MC;
\item finally, the MC efficiency for $m_T$ and $\Delta\phi_{ll}$ cut is accounted to obtain the $WW$ contribution in the 
signal region after all cuts.
\end{itemize}

In order to consider this procedure as reliable we check that the efficiency of the $m_T$ and $\Delta\phi_{ll}$ 
cuts in the control region in data and MC is consistent (to be done as soon as we have enough data) and we verify that the control-to-signal 
region ratio $R_{C/S}$ is stable using different generators (Table~\ref{tab:wwEstimationMC}). 
In the 1-jet bin case, we observe a significant discrepancy in the $R_{C/S}$ value between MadGraph and PYTHIA.
Such difference is very likely to be due to the worse description of the jet kinematics in the PYTHIA sample;
therefore, we consider the MadGraph value as reliable.

\begin{table}[!htbp]
\begin{center}
\begin{tabular}{|c|c|c|c|c|c|} \hline
 sample               &       MuMu &       ElMu &       MuEl &       ElEl &        All \\ \hline\hline
   $qq\rightarrow WW$ &      21.75 &      34.06 &      35.81 &      16.29 &     107.91 \\ 
   $gg\rightarrow WW$ &       0.87 &       1.20 &       1.25 &       0.69 &       4.00 \\ 
  $t\bar t$           &       2.73 &       3.82 &       3.82 &       1.77 &      12.14 \\ 
     tW               &       0.95 &       1.49 &       1.23 &       1.10 &       4.79 \\ 
 W+jets               &       0.00 &       2.08 &       2.08 &       6.24 &      10.40 \\ 
 others               &       2.07 &       0.66 &       0.96 &       1.72 &       5.41 \\ \hline
  total               &      28.36 &      43.31 &      45.16 &      27.81 &     144.64 \\ \hline
\end{tabular}
\caption{Expected yields in the control region from MC (0-jet bin); the assumed luminosity is $1~fb^{-1}$.}
\label{tab:wwEstimationSByields}
\end{center}
\end{table}

\begin{table}[!htbp]
\begin{center}
\begin{tabular}{|c|c|c|} \hline
\multicolumn{3}{|c|}{0-jet bin} \\ \hline
quantity                           &             MadGraph &              PYTHIA  \\ \hline
$R_{C/S}$                           &      0.294$\pm$ 0.005&      0.271$\pm$ 0.011\\
$\epsilon_{m_T}$(mass region)       &      0.877$\pm$ 0.018&      0.877$\pm$ 0.046\\
$\epsilon_{\Delta\phi}$(mass region) &      0.657$\pm$ 0.015&      0.645$\pm$ 0.040\\
$\epsilon_{m_T}$(side band)         &      0.544$\pm$ 0.007&      0.544$\pm$ 0.017\\
$\epsilon_{\Delta\phi}$(side band)   &      0.051$\pm$ 0.002&      0.036$\pm$ 0.005\\ \hline \hline
\multicolumn{3}{|c|}{1-jet bin} \\ \hline
quantity                           &             MadGraph &              PYTHIA  \\ \hline
$R_{C/S}$                           &    0.320$\pm$ 0.008 &      0.221$\pm$ 0.018 \\
$\epsilon_{m_T}$(mass region)       &    0.759$\pm$ 0.026 &      0.737$\pm$ 0.086 \\
$\epsilon_{\Delta\phi}$(mass region) &   0.814$\pm$ 0.031 &      0.775$\pm$ 0.103  \\
$\epsilon_{m_T}$(side band)         &    0.521$\pm$ 0.011 &      0.513$\pm$ 0.031 \\
$\epsilon_{\Delta\phi}$(side band)   &   0.105$\pm$ 0.006 &      0.086$\pm$ 0.015  \\ \hline
\end{tabular}
\caption{Control-to-signal region ratio and cut efficiencies using MadGraph $qq\rightarrow WW$ and PYTHIA $gg\rightarrow WW$
vs PYTHIA inclusive $WW$. Results for $m_H=160~\GeVcc$ analysis in the 0- and 1-jet bins. 
Uncertainties are statistical and account for the MC sample luminosity. }
\label{tab:wwEstimationMC}
\end{center}
\end{table}

As a closure test, we apply the above procedure on simulated data, using MadGraph $qq\rightarrow WW$ and PYTHIA $gg\rightarrow WW$
as data and PYTHIA inclusive $WW$ as MC (errors are statistical only and are computed for a luminosity of $1~fb^{-1}$). 
For the moment we consider only the main contamination source in the side band region, top background; 
the expected yield for $1~fb^{-1}$ is 128.8 events (111.9 for $WW$ and 16.9 for $t\bar t$ plus $tW$).
The top contribution is evaluated by measuring the top tagging efficiency in a top-enriched sample (see dedicated section),
with a result of 10.5$\pm$4.1 events. 
Therefore, we estimate 118.4$\pm$12.1 $WW$ events in the side band region and, after applying
the control-to-signal region ratio and the cut efficiencies, 18.2$\pm$2.3 events in the $m_H=160~\GeVcc$ signal region 
(consistent with the expected value of 18.9$\pm$4.4).

The results for all considered Higgs masses in the 0- and 1-jet bins are reported in Table~\ref{tab:wwEstimationRes0j}.

\begin{table}[!htbp]
\begin{center}
\begin{tabular}{|c|c|c|} \hline
\multicolumn{3}{|c|}{0-jet bin} \\ \hline
$m_H~[\GeVcc]$ & WW estimation ($1~fb^{-1}$) & WW expected ($1~fb^{-1}$)  \\ \hline
120 & 41.2 $\pm$ 4.9 & 43.4 $\pm$ 6.6 \\
130 & 45.4 $\pm$ 5.2 & 47.4 $\pm$ 6.9 \\
140 & 41.7 $\pm$ 4.9 & 42.7 $\pm$ 6.5 \\
150 & 28.3 $\pm$ 3.7 & 28.4 $\pm$ 5.3 \\
160 & 18.2 $\pm$ 2.5 & 18.9 $\pm$ 4.4 \\
200 & 11.7 $\pm$ 1.6 & 12.2 $\pm$ 3.5 \\ \hline \hline
\multicolumn{3}{|c|}{1-jet bin} \\ \hline
$m_H~[\GeVcc]$ & WW estimation ($1~fb^{-1}$) & WW expected ($1~fb^{-1}$)  \\ \hline
120 & 6.2 $\pm$ 4.6 & 9.6 $\pm$ 3.1 \\
130 & 6.8 $\pm$ 5.1 & 10.7$\pm$ 3.3 \\
140 & 6.3 $\pm$ 4.7 & 9.7 $\pm$ 3.1 \\
150 & 4.6 $\pm$ 3.8 & 8.9 $\pm$ 3.0 \\
160 & 3.6 $\pm$ 3.0 & 7.4 $\pm$ 2.7 \\
200 & 4.4 $\pm$ 3.7 & 8.8 $\pm$ 3.0 \\
 \hline
\end{tabular}
\caption{Result of $WW$ estimation for different Higgs mass analyses in the 0- and 1-jet bins. }
\label{tab:wwEstimationRes0j}
\end{center}
\end{table}


%Systematics are evaluated repeating the procedure varying the usual suspects.

%
%\subsection{Estimation in high mass range}
%We take it from MC.



%The nonresonant $qq \to \WW$ contribution in the $\hww$ signal region is 
%estimated from data using the dilepton mass distribution. For a given Higgs 
%boson mass, the region with a small contribution from Higgs boson decays is 
%selected and simulation is used to extrapolate this background into the signal 
%region. For low Higgs boson mass values ($m_{\rm{H}} < 200~\GeVcc$) events 
%with $m_{\ell\ell} > 100~\GeVcc$ are used, while for $m_{\rm{H}} > 200~\GeVcc$ 
%events with $m_{\ell\ell} < 100~\GeVcc$ are used. The statistical uncertainty 
%on the estimate of the nonresonant $\WW$ background with the current data 
%sample is approximately 50\%. For the 1- and 2- jet bin cases we use the results
%from the 0-jet bin, and then extrapolate to each jet bin.
%
%The $gg \to \WW$ background contribution has to be taken from simulated events 
%since we do not have enough sensitivity in the data to measure it. We assign a 
%50\% uncertainty to the overall normalization~\cite{ggWWError}. This is 
%obtained by studying the change in the cross-section when varying the parton 
%distribution functions (PDFs), QCD renormalization and scales.
