The nonresonant $WW$ contribution in the signal region can be estimated from data 
using the dilepton invariant mass distribution. For a given Higgs boson mass, 
a $\ww$-dominated control region with a small contribution from Higgs decays 
can be selected. The $\ww$ contribution estimated in the control region 
is then extrapolated into the signal region using simulation. 

Figure~\ref{fig:higgsMllCutoff} shows the dilepton invariant mass distributions for 
$\ww$ and $\hww$ with $m_{H} = 130, 200$, and 300~$\GeVcc$. 
For Higgs masses above 200~$\GeVcc$ there is significant overlap in the 
dilepton invariant mass distribution with the nonresonant $\ww$ contribution. 
The control region can not be defined efficiently in that case. Therefore for 
searches for a high mass Higgs ($m_H>200\GeVcc$) we estimate the nonresonant 
$\ww$ contributions from simulation. 

%The basic idea for the estimation of the nonresonant $WW$ contribution in the $\hww$ signal region is 
%to infer it from data using the dilepton mass distribution:
%the dilepton mass defines a control region where we can measure the $WW$ normalization, and then scale
%the contribution to the signal region.

%It turns out that this approach can be applied for $m_H \leq 200~\GeVcc$ only.
%In fact, MC studies show that the di-lepton mass distribution in Higgs samples has a cut-off value at $m_H-50~\GeVcc$ 
%(Fig.~\ref{fig:higgsMllCutoff});
%thus, for large $m_H$ it is possible to define an Higgs-depleted region only in a mass range populated by too 
%few $WW$ events. 

%Therefore, for $m_H > 200~\GeVcc$ the $WW$ contribution is estimated from simulated events.

%%%%%%%%%%%%%%%%%%%%%%%%%%%%%%%%%%%%%%%%%%%
\begin{figure}[!hbtp]
\centering
\includegraphics[width=.5\textwidth]{figures/higgsMllCutoff.png}
\caption{The dilepton invariant mass distributions for $\hww$ decays with 
$m_{H} = 130, 200, 300\GeVcc$ in simulation. 
The distributions are normalized to unity.}
\label{fig:higgsMllCutoff}
\end{figure}
%%%%%%%%%%%%%%%%%%%%%%%%%%%%%%%%%%%%%%%%%%%

%\subsubsection{Estimation in Low Mass Range}

For searches for low mass Higgs bosons with $m_{H}<200\GeVcc$, we define the $\ww$ control region as the events 
with $m_{\ell\ell} > 100~\GeVcc$ that pass the full event selection (see Section~\ref{sec:signal_selection}) 
except for the requirements on $m_T$ and $\Delta\phi$. The procedure to obtain the $\ww$ contribution 
in the signal region is as follows. 
%we define the $\ww$ dominated control region dominated by $\ww$ decays as 
%For low Higgs boson mass values ($m_{\rm{H}} \leq 200~\GeVcc$) events with $m_{\ell\ell} > 100~\GeVcc$ are used
%to define a control region where the Higgs contribution is $<3\%$.
\begin{itemize}
\item We first measure the yields in the control region in data; 
%applying full event selections except for the cuts on 
%$m_{ll}$, 
%$m_T$ and $\Delta\phi_{ll}$% so that most of the systematics uncertainties 
%cancel out (e.g. jet veto, lepton and trigger efficiencies); 
\item We then subtract the contamination from other backgrounds such as $t\bar t$ and 
$W$+jets (see Table~\ref{tab:wwEstimationSByields}) which can be estimated using 
corresponding data-driven techniques;
\item The resulting yield is subsequently extrapolated to the signal region, using 
the control-to-signal region ratio estimated from the dilepton invariant mass 
spectrum in simulation;
\item Finally, we multiply the yields in the signal region by the 
$m_T$ and $\Delta\phi_{ll}$ cut efficiency from simulation to get the 
$WW$ contribution in the signal region after all cuts.
\end{itemize}

We can validate this procedure by comparing the $m_T$ and $\Delta\phi_{ll}$ cut efficiency 
in the control region between data and simulation with enough statistics. 
Table~\ref{tab:wwEstimationMC} shows the ratio of the $\ww$ contribution in control and 
signal regions referred to as $R_{C/S}$ comparing different MC generators. 
In the 1-jet bin case, we observe a significant discrepancy in the $R_{C/S}$ value between MadGraph and PYTHIA.
Such difference is likely due to the worse description of jet kinematics in the PYTHIA sample. We will use 
samples produced with other generators like MC@NLO when we have them available.

\begin{table}[!htbp]
\begin{center}
\begin{tabular}{|c|c|c|c|c|c|} \hline
 Sample               &       MuMu &       ElMu &       MuEl &       ElEl &        All \\ \hline\hline
   $qq\rightarrow WW$ &      21.75 &      34.06 &      35.81 &      16.29 &     107.91 \\ 
   $gg\rightarrow WW$ &       0.87 &       1.20 &       1.25 &       0.69 &       4.00 \\ 
  $t\bar t$           &       2.73 &       3.82 &       3.82 &       1.77 &      12.14 \\ 
     tW               &       0.95 &       1.49 &       1.23 &       1.10 &       4.79 \\ 
 W+jets               &       0.00 &       2.08 &       2.08 &       6.24 &      10.40 \\ 
 others               &       2.07 &       0.66 &       0.96 &       1.72 &       5.41 \\ \hline
  total               &      28.36 &      43.31 &      45.16 &      27.81 &     144.64 \\ \hline
\end{tabular}
\caption{Expected yields in the control region from MC (0-jet bin) normalized to an integrated luminosity of $1~fb^{-1}$.}
\label{tab:wwEstimationSByields}
\end{center}
\end{table}

\begin{table}[!htbp]
\begin{center}
\begin{tabular}{|c|c|c|} \hline
\multicolumn{3}{|c|}{0-jet bin} \\ \hline
Quantity                                       &            MadGraph   &   PYTHIA           \\ 
\hline
$R_{C/S}$                                      &      0.294$\pm$ 0.005 &   0.271$\pm$ 0.011 \\
$\epsilon_{m_T}$(signal mass region)           &      0.877$\pm$ 0.018 &   0.877$\pm$ 0.046 \\
$\epsilon_{\Delta\phi}$(signal mass region)    &      0.657$\pm$ 0.015 &   0.645$\pm$ 0.040 \\
$\epsilon_{m_T}$(sideband mass region)         &      0.544$\pm$ 0.007 &   0.544$\pm$ 0.017 \\
$\epsilon_{\Delta\phi}$(sideband mass region)  &      0.051$\pm$ 0.002 &   0.036$\pm$ 0.005 \\ 
\hline \hline
\multicolumn{3}{|c|}{1-jet bin} \\ \hline
quantity                             &    MadGraph         &    PYTHIA           \\ 
\hline
$R_{C/S}$                            &    0.320$\pm$ 0.008 &    0.221$\pm$ 0.018 \\
$\epsilon_{m_T}$(mass region)        &    0.759$\pm$ 0.026 &    0.737$\pm$ 0.086 \\
$\epsilon_{\Delta\phi}$(mass region) &    0.814$\pm$ 0.031 &    0.775$\pm$ 0.103 \\
$\epsilon_{m_T}$(side band)          &    0.521$\pm$ 0.011 &    0.513$\pm$ 0.031 \\
$\epsilon_{\Delta\phi}$(side band)   &    0.105$\pm$ 0.006 &    0.086$\pm$ 0.015 \\ 
\hline
\end{tabular}
\caption{Control-to-signal region ratio and cut efficiencies using MadGraph $qq\rightarrow WW$ and PYTHIA $gg\rightarrow WW$
vs PYTHIA inclusive $WW$. Results for $m_H=160~\GeVcc$ analysis in the 0- and 1-jet bins. 
Uncertainties are statistical only and account for the MC sample luminosity. }
\label{tab:wwEstimationMC}
\end{center}
\end{table}

We perform a cross-check of this procedure on simulation events. In this test we mix the events from 
$\WW$ (including $qq\rightarrow WW$ and $gg\rightarrow WW$), $t\bar t$ and $tW$ according to their SM cross-sections 
normalized to an integrated luminosity of $1~\text{fb}^{-1}$ to approximate the data composition. 
The $gg\rightarrow WW$ events are taken from PYTHIA while the rest are taken from Madgraph. 
%For the moment we consider only the main contamination source in the side band region, top background; 
The expected yields of $\ww$ and top backgrounds in the control region % relaxing the $m_{T}$ and $\Delta\phi$ cuts  
are 111.9 and 16.9 respectively (see Table~\ref{tab:wwEstimationSByields}). 
Based on this mixed simulation data, we apply the $\ww$ background estimation method described above. 
The top contribution is estimated as 10.5$\pm$4.1 events with the data-driven method (Section~\ref{sec:bkg_top}) 
using the top tagging efficiency in a top-enriched sample. 
The $R_{C/S}$ ratio and selection efficiencies of $m_T$ and $\Delta\phi$ cuts evaluated in the inclusive PYTHIA MC 
are then 
used to estimate the $\ww$ contribution in the Higgs signal region. 
%Therefore, we estimate 118.4$\pm$12.1 $WW$ events in the side band region and, after applying
%the control-to-signal region ratio and the cut efficiencies, 18.2$\pm$2.3 events in the $m_H=160~\GeVcc$ signal region 
%(consistent with the expected value of 18.9$\pm$0.3).
Results for all considered Higgs masses in the 0- and 1-jet bins are reported in Table~\ref{tab:wwEstimationRes}.
The agreement in the 0-jet bin is very good, while in the 1-jet bins there are $\sim1\sigma$ discrepancies due to the underestimated 
PYTHIA $R_{C/S}$ value.

\begin{table}[!htbp]
\begin{center}
\begin{tabular}{|c|c|c|} \hline
\multicolumn{3}{|c|}{0-jet bin} \\ \hline
$m_H~[\GeVcc]$ & WW estimation ($1~fb^{-1}$) & WW expected ($1~fb^{-1}$)  \\ \hline
120 & 41.2 $\pm$ 4.9 & 43.4 $\pm$ 0.5 \\
130 & 45.4 $\pm$ 5.2 & 47.4 $\pm$ 0.5 \\
140 & 41.7 $\pm$ 4.9 & 42.7 $\pm$ 0.5 \\
150 & 28.3 $\pm$ 3.7 & 28.4 $\pm$ 0.4 \\
160 & 18.2 $\pm$ 2.5 & 18.9 $\pm$ 0.3 \\
200 & 11.7 $\pm$ 1.6 & 12.2 $\pm$ 0.3 \\ \hline \hline
\multicolumn{3}{|c|}{1-jet bin} \\ \hline
$m_H~[\GeVcc]$ & WW estimation ($1~fb^{-1}$) & WW expected ($1~fb^{-1}$)  \\ \hline
120 & 6.2 $\pm$ 4.6 & 9.6 $\pm$ 0.2 \\
130 & 6.8 $\pm$ 5.1 & 10.7$\pm$ 0.2 \\
140 & 6.3 $\pm$ 4.7 & 9.7 $\pm$ 0.2 \\
150 & 4.6 $\pm$ 3.8 & 8.9 $\pm$ 0.2 \\
160 & 3.6 $\pm$ 3.0 & 7.4 $\pm$ 0.2 \\
200 & 4.4 $\pm$ 3.7 & 8.8 $\pm$ 0.2 \\
 \hline
\end{tabular}
\caption{Closure test result of $WW$ estimation for different Higgs mass analyses in the 0- and 1-jet bins.  
Errors are statistical only; in the WW estimation column they are computed for a luminosity of $1~fb^{-1}$, 
while in the WW expected column they correspond to the MC sample luminosity.}
\label{tab:wwEstimationRes}
\end{center}
\end{table}

%\Fixme : We need to add procedure for estimating systematics: 
% lepton scales, jet energy scale, theory uncertainties...
%Systematics are evaluated repeating the procedure varying the usual suspects.


%
%\subsection{Estimation in high mass range}
%We take it from MC.



%The nonresonant $qq \to \WW$ contribution in the $\hww$ signal region is 
%estimated from data using the dilepton mass distribution. For a given Higgs 
%boson mass, the region with a small contribution from Higgs boson decays is 
%selected and simulation is used to extrapolate this background into the signal 
%region. For low Higgs boson mass values ($m_{\rm{H}} < 200~\GeVcc$) events 
%with $m_{\ell\ell} > 100~\GeVcc$ are used, while for $m_{\rm{H}} > 200~\GeVcc$ 
%events with $m_{\ell\ell} < 100~\GeVcc$ are used. The statistical uncertainty 
%on the estimate of the nonresonant $\WW$ background with the current data 
%sample is approximately 50\%. For the 1- and 2- jet bin cases we use the results
%from the 0-jet bin, and then extrapolate to each jet bin.
%
%The $gg \to \WW$ background contribution has to be taken from simulated events 
%since we do not have enough sensitivity in the data to measure it. We assign a 
%50\% uncertainty to the overall normalization~\cite{ggWWError}. This is 
%obtained by studying the change in the cross-section when varying the parton 
%distribution functions (PDFs), QCD renormalization and scales.

\subsubsection{Theoretical Systematic Uncertainties} 
Since we extrapolate the WW background yield from the 0-jet bin to the 1-jet bin using Monte Carlo
simulation, there is a systematic uncertainty associated with theoretical uncertainties in
the prediction for the 1-jet bin yield. This uncertainty is primarily dominated by missing higher
order corrections in the prediction for the WW+1jet cross section. To evaluate this systematic 
uncertainty we estimate the uncertainty due to missing higher order corrections on the
total inclusive WW production cross section ($\sigma^{\mathrm{WW}}_{\geq 0}$) and the inclusive 
WW+1 or more jets cross section ($\sigma^{\mathrm{WW}}_{\geq 1}$), and propagating these uncertainties 
to the WW+1 jet cross section. The procedure is described in greater detail for the Higgs signal
in Section \ref{sec:HiggsJetBinFractionSystematics}. To evaluate the uncertainty on 
$\sigma^{\mathrm{WW}}_{\geq 0}$, and $\sigma^{\mathrm{WW}}_{\geq 1}$ due to missing higher 
order corrections, we use MCFM \cite{MCFMVVProduction} to compute the inclusive cross sections, 
varying the renormalization and factorization scales. From these calculations we obtain 
systematic uncertainties on $\sigma^{\mathrm{WW}}_{\geq 0}$, and 
$\sigma^{\mathrm{WW}}_{\geq 1}$ of $\pm 3.4\%$ and $^{+15.2}_{-14.4} \%$ respectively. These are translated into
log normal representation as $\kappa^{\mathrm{WW}}_{\geq 0} = 1.034$ and $\kappa^{\mathrm{WW}}_{\geq 1} = 1.16$.

To propagate these uncertainties into uncertainties on the WW+1jet cross section, $\sigma^{\mathrm{WW}}_{1}$,
we evaluate the fraction of events in the 0-jet ($f_{0}$), 1-jet($f_{1}$), and 2-jet($f_{2}$) 
bins using the MCFM calculation and the Madgraph Monte Carlo simulation. The 0-jet fraction can be 
evaluated from the MCFM calculation via the relation 
$f_{0} = (\sigma^{\mathrm{WW}}_{\geq 0} - \sigma^{\mathrm{WW}}_{\geq 1}) / \sigma^{\mathrm{WW}}_{\geq 0}$.
Since we do not have a calculation of WW + 2 or more jets at next to leading order, the same calculation 
cannot be done for the  1jet fraction. Therefore, we use the Monte Carlo prediction and obtain 
$f_{1} = 0.19$ and $f_{2} = 0.05$. To propagate the effect of these uncertainties to the 
normalization of the 0-jet and 1-jet bins, we use the relations:

\begin{eqnarray}
\label{eqn:WWJetBinFractions}
\kappa^{\mathrm{0-jet}}_{\mathrm{QCDscale\_WW}} = (\kappa^{\mathrm{WW}}_{\geq 0})^{\frac{1}{f_{0}}},                 \\
\kappa^{\mathrm{0-jet}}_{\mathrm{QCDscale\_WW1in}} = (\kappa^{\mathrm{WW}}_{\geq 1})^{- \frac{f_{1}+f_{2}}{f_{0}}},  \\
\kappa^{\mathrm{1-jet}}_{\mathrm{QCDscale\_WW1in}} = (\kappa^{\mathrm{WW}}_{\geq 1})^{\frac{f_{1}+f_{2}}{f_{1}}},    \\
\end{eqnarray}

where $\kappa^{\mathrm{0-jet}}_{\mathrm{QCDscale\_WW}}$, $\kappa^{\mathrm{0-jet}}_{\mathrm{QCDscale\_WW1in}}$,
$\kappa^{\mathrm{1-jet}}_{\mathrm{QCDscale\_WW1in}}$, are the systematic uncertainties for the WW bkg normalization in 
the 0-jet bin due to missing higher order corrections in the inclusive WW cross section calculation,
for the WW background normalization in the 0-jet bin due to missing higher order corrections in the inclusive WW + 1 or 
more jet cross section calculation, and for the WW background normalization in the 1-jet bin due to missing higher 
order corrections in the inclusive WW + 1 or more jet cross section calculation, respectively. 
$\kappa^{\mathrm{0-jet}}_{\mathrm{QCDscale\_WW}} = 1.045$ and $\kappa^{\mathrm{0-jet}}_{\mathrm{QCDscale\_WW1in}} = 0.954$
are much smaller than the total systematic uncertainty from the data-driven estimate for the 0-jet bin and
therefore can be ignored. $\kappa^{\mathrm{1-jet}}_{\mathrm{QCDscale\_WW1in}}$ is $1.206$, corresponding to 
a $21\%$ systematic uncertainty on the normalization of the WW background in the 1-jet bin, and is propagated in the
analysis.

