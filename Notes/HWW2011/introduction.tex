The standard model (SM) of particle physics successfully describes
the majority of high-energy experimental data~\cite{pdg}. One of the key
remaining questions is the origin of the masses of $\W$~and $\Z$~bosons.
In the SM, it is attributed to the spontaneous breaking of electroweak
symmetry caused by a new scalar field~\cite{Higgs1, Higgs2, Higgs3}. The
existence of the associated field quantum,
the Higgs boson, has yet to be experimentally confirmed.
The $\WW$ channel is particularly sensitive for the Higgs boson
searches in the intermediate mass range (120 -- 200 $\GeVcc$).

This document describes the search of Higgs boson events with CMS  
at a center of mass energy of 7 TeV in the channel 
$\Hi\To\WW \To \Lep\Lprime\Nu\Nubar$, for Higgs boson masses in 
the range of 120-600 $\GeVcc$ using the 2011 data. Signal and background 
yields are extrapolate to 1 $\ifb$ of collected data to obtain 
sensible projections.
    
The main analysis strategy is to select events with two opposite charged leptons, 
large missing energy and little jet activity. The two leptons are required to be 
isolated electrons or muons, of moderate-high $\pt$ and with a small opening angle 
in the transverse plane. The 2010 $H \to \WW$ analysis~\cite{HWW2010} is the starting point, 
although several modifications and improvements have been added, partially to cope with 
the more difficult conditions due to the larger instantaneous luminosity regime, and 
partially to gain in sensitivity. In addition, the 1-jet and 2-jet bins are
included in the analysis.

The note is structured as follows. A discussion about the data samples
used in the analysis is presented in
Section~\ref{sec:datasets}. Trigger selection, lepton selection, and
other preselection requirements are described in detail in
Section~\ref{sec:selection}. The final signal extraction is discussed
in Section~\ref{sec:signal_selection}. A summary of the event data
yields and the expectations from Monte Carlo is shown in
Section~\ref{sec:yields}, followed by the estimation of the background
in Section~\ref{sec:backgrounds}. The signal efficiency estimation is
presented in Section~\ref{sec:efficiency}. Results with all sources of
systematic uncertainties are shown in
Section~\ref{sec:results}. Finally, the conclusions are included in
Section~\ref{sec:summary}.
