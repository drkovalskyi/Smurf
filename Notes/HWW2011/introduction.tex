The standard model (SM) of particle physics successfully describes the majority of high-energy
experimental data~\cite{pdg}. One of the key remaining questions is the origin of the masses of
$\W$~and $\Z$~bosons.  In the SM in its simplest implementation it is attributed to the spontaneous
breaking of electroweak symmetry caused by a new scalar field~\cite{Higgs1, Higgs2, Higgs3}. The
existence of the associated field quantum, the Higgs boson, has yet to be experimentally confirmed.

The $\Hi\to\WW$ channel is particularly sensitive for Higgs boson searches in the intermediate mass
range ($120-200~\GeVcc$)~\cite{dittmar}. This document describes the search for the Higgs boson 
in the $\Hi\To\WW \To \Lep\Lprime\Nu\Nubar$ channel, for Higgs boson masses in the range of 
$115-600~\GeVcc$ using the early 2011 data. Signal and background yields are extrapolated to 
$1~\ifb$ of collected data to get projected sensitivity.
    
The main analysis strategy is to select events with two opposite charged leptons, large missing
energy and little jet activity. The two leptons are required to be isolated electrons or muons, of
moderately high transverse momenta ($\pt$) and with a small opening angle in the transverse
plane. The 2010 $\Hi\to\WW$ analysis~\cite{HWW2010} is the starting point.  However, several
modifications and improvements have been added, both to cope with the more difficult conditions due
to the higher instantaneous luminosity regime in 2011, and also to extend the sensitivity.  While
the original analysis used exclusively events with zero reconstructed jets, this analysis also uses
events with one or two additional jets.

The note is structured as follows. A discussion about the data samples used in the analysis is
presented in Section~\ref{sec:datasets}.  The trigger selection, lepton selection, and other
preselection requirements are described in detail in Section~\ref{sec:selection}.  A summary of the
expected event yields based on Monte Carlo is shown in Section~\ref{sec:yields}.  The strategy for
signal extraction is discussed in Section~\ref{sec:signal_selection}, followed by a description of
the techniques to estimate the backgrounds in Section~\ref{sec:backgrounds}. The signal efficiency
estimation is presented in Section~\ref{sec:efficiency}.  All sources of systematic uncertainty are
shown in Section~\ref{sec:systematics}.  The sensitivity projections for a $1~\ifb$ sample are given
in Section~\ref{sec:results}, and the results from \intlumi of data from 2011 are given in
Section~\ref{sec:dataresults}.  Finally, the conclusions are presented in Section~\ref{sec:summary}.
