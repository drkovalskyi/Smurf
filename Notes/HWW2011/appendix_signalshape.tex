
In this section we study the uncertainties on the shape of the MVA output for  
$H \to WW$ signal events. The signal prediction is generated using the POWHEG 
Monte Carlo generator, where the Higgs $p_{T}$ spectrum has been reweighted 
to the one predicted by the HQT program, which computes the resummed differential
cross section for the Higgs $p_{T}$ at next-to-next-to-leading order. We study
the uncertainties on the shape of the MVA output from the two primary 
sources of uncertainty: higher order corrections and parton distribution
function uncertainties.

First to obtain some intuition about the particular phase space that the
MVA is preferentially selecting, we compare the distributions of the 
MVA input observables for the signal sample with no selection requirements
and the signal sample where the MVA predict a high score for the signal.
The comparisons are shown in Figure \ref{fig:MVAHighScorePhaseSpaceRegion}
for the six MVA input observables where a cut of the MVA output greater 
than $0.8$ has been made for the MVA trained with the $M_{H} = 130$ GeV
signal hypothesis. We observe that the MVA selects the region of
low $\Delta\phi$, low $\Delta$R, low dilepton mass, and transverse 
mass closer to the higgs mass hypothesis. 

%%%%%%%%%%%%%%%%%%%%%%%%%%%%%%%%%%%
\begin{figure}[!htbp]
\begin{center}
\subfigure[PtMax]{\includegraphics[width=0.49\textwidth]{figures/MVAHighScorePhaseSpace_MVA130_PtMax.pdf}}
\subfigure[PtMin]{\includegraphics[width=0.49\textwidth]{figures/MVAHighScorePhaseSpace_MVA130_PtMin.pdf}}  
\\
\subfigure[$M_{\mathrm{ll}}$]{\includegraphics[width=0.49\textwidth]{figures/MVAHighScorePhaseSpace_MVA130_DileptonMass.pdf}}
\subfigure[$M_{\mathrm{T Higgs}}$]{\includegraphics[width=0.49\textwidth]{figures/MVAHighScorePhaseSpace_MVA130_MTHiggs.pdf}}  
\\
\subfigure[$\Delta\phi_{\mathrm{ll}}$]{\includegraphics[width=0.49\textwidth]{figures/MVAHighScorePhaseSpace_MVA130_DeltaPhi.pdf}}
\subfigure[$\Delta\mathrm{R}_{\mathrm{ll}}$]{\includegraphics[width=0.49\textwidth]{figures/MVAHighScorePhaseSpace_MVA130_DeltaR.pdf}}
\caption{Comparisons of the distribution of the MVA input observables for the
inclusive sample of signal events and the sample of signal events where a
high score of the MVA is required (MVA $>0.8$).
}
\label{fig:MVAHighScorePhaseSpaceRegion}
\end{center}
\end{figure}
%%%%%%%%%%%%%%%%%%%%%%%%%%%%%%%%%%%

We factorize the effect of the higher order corrections into two
pieces: the effect of higher order corrections on the Higgs $p_{T}$ spectrum
and the effect of higher order corrections on everything else. To study the
effect on the Higgs $p_{T}$ spectrum we produce $p_{T}$ spectra using HQT
with the renormalization and factorization scales varied up and down by factors 
of $2$ and $1/2$, shown in Figure \ref{fig:signalshape_PtSpectrumScaleVariation_HiggsPt}.

%%%%%%%%%%%%%%%%%%%%%%%%%%%%%%%%%%%
\begin{figure}[!htbp]
\begin{center}
\includegraphics[width=0.49\textwidth]{figures/ShapeSystematics_HWW_HiggsPt_HQTScaleVariation.pdf}
\caption{Comparison of the resummed NNLO predicted Higgs $p_{T}$ spectrum with variations of
the renormalization and factorization scales.
}
\label{fig:signalshape_PtSpectrumScaleVariation_HiggsPt}
\end{center}
\end{figure}
%%%%%%%%%%%%%%%%%%%%%%%%%%%%%%%%%%%


These scale varied spectra give an indication of the
effect of the terms beyond NNLO in the perturbative expansion on the $p_{T}$ 
spectrum. Then we take the POWHEG Monte Carlo sample, reweight to the scale 
varied $p_{T}$ spectra, and compare the differences in the prediction of 
the distributions of the MVA input observables and the MVA output. In Figure
\ref{fig:signalshape_PtSpectrumScaleVariation_InputObservables} we show the
comparison for the four most important MVA input observables between the
Monte Carlo prediction reweighted to the scale varied $p_{T}$ spectra. We
observe that the effect of the scale variation is extremely small and only
visible as we approach the very high $p_{T}$ region of phase space. In Figure
\ref{fig:signalshape_PtSpectrumScaleVariation_MVAOutput} we show the comparison
of the MVA output distributions for the MVA trained with the $130$, $160$,
and $400$ GeV Higgs mass hypotheses. We again observe almost zero effect of 
the scale variation on the MVA output distributions over the bulk of the signal
populated region. The differences are at the $0.1\%$ level.



%%%%%%%%%%%%%%%%%%%%%%%%%%%%%%%%%%%
\begin{figure}[!htbp]
\begin{center}
\subfigure[PtMin]{\includegraphics[width=0.49\textwidth]{figures/ShapeSystematics_HWW_PtMax_HQTScaleVariation.pdf}}
\subfigure[PtMax]{\includegraphics[width=0.49\textwidth]{figures/ShapeSystematics_HWW_PtMin_HQTScaleVariation.pdf}} \\
\subfigure[$M_{\mathrm{T Higgs}}$]{\includegraphics[width=0.49\textwidth]{figures/ShapeSystematics_HWW_MTHiggs_HQTScaleVariation.pdf}}
\subfigure[$\Delta\phi_{\mathrm{ll}}$]{\includegraphics[width=0.49\textwidth]{figures/ShapeSystematics_HWW_DeltaPhi_HQTScaleVariation.pdf}}
\caption{Comparison of the four most discriminant MVA input variables between the prediction of the MC
reweighted to the scale varied and default $p_{T}$ spectra. 
}
\label{fig:signalshape_PtSpectrumScaleVariation_InputObservables}
\end{center}
\end{figure}
%%%%%%%%%%%%%%%%%%%%%%%%%%%%%%%%%%%



%%%%%%%%%%%%%%%%%%%%%%%%%%%%%%%%%%%
\begin{figure}[!htbp]
\begin{center}
\subfigure[$M_{H}=130$GeV MVA]{\includegraphics[width=0.49\textwidth]{figures/ShapeSystematics_HWW_MVA130_HQTScaleVariation.pdf}}
\subfigure[$M_{H}=160$GeV MVA]{\includegraphics[width=0.49\textwidth]{figures/ShapeSystematics_HWW_MVA160_HQTScaleVariation.pdf}}
\subfigure[$M_{H}=400$GeV MVA]{\includegraphics[width=0.49\textwidth]{figures/ShapeSystematics_HWW_MVA400_HQTScaleVariation.pdf}}
\caption{Comparison of the MVA outputs for the MVA trained on different Higgs mass hypotheses 
between the prediction of the MC reweighted to the scale varied and default $p_{T}$ spectra. 
}
\label{fig:signalshape_PtSpectrumScaleVariation_MVAOutput}
\end{center}
\end{figure}
%%%%%%%%%%%%%%%%%%%%%%%%%%%%%%%%%%%




To study the effect of higher order corrections on everything else, we produce
POWHEG Monte Carlo with the renormalization and factorization scales varied 
up and down by factors of $2$ and $1/2$, and reweight both scale varied samples
to the nominal NNLO Higgs $p_{T}$ spectrum from HQT. These predictions give 
an indication of the effect of terms beyond NLO in the perturbative expansion,
assuming that the Higgs $p_{T}$ spectrum has been fixed to the true spectrum. 
The study is performed only at generator level since the MVA input observables
are all leptonic observables which are not expected to be significantly 
affected by the simulation and reconstruction. The comparison is shown in 
Figure \ref{fig:signalshape_PtSpectrumScaleVariation_MVAOutput}
which indicates no significant differences between the scale varied and the default 
scale prediction beyond statistical uncertainties. Furthermore, the differences
between the scale varied predictions and the default scale predictions show
random behavior consistent with statistical fluctuations. 


%%%%%%%%%%%%%%%%%%%%%%%%%%%%%%%%%%%
\begin{figure}[!htbp]
\begin{center}
\includegraphics[width=0.49\textwidth]{figures/ShapeSystematics_HWW_MVA130_PowhegScaleVariationFixedPtSpectrum.pdf}
\caption{Comparison of the MVA output trained with the $M_{H}=130$ GeV Higgs mass hypothesis 
between the prediction of the MC reweighted to the scale varied and default $p_{T}$ spectra. 
}
\label{fig:signalshape_PtSpectrumScaleVariation_MVAOutput}
\end{center}
\end{figure}
%%%%%%%%%%%%%%%%%%%%%%%%%%%%%%%%%%%

\subsubsection{Parton Distribution Function}

Parton distribution functions (PDFS) are obtained from global fits 
to experimental data from deep in-elastic scattering, Drell-Yan, and jet processes.
Hence, the parameters describing the proton structure have associated  
uncertainties which are presented in the form of error PDFs packaaged with the PDF set.
The recommendations of the PDF4LHC group are followed for 
cross section uncertainty on the signal samples from the PDF. 
The predictions from CT10, MSTW2008, and NNPDF2.0 PDF sets are used. 
These uncertainties, including both the associated intrinsic uncertainties,
and the uncertainty introduced through the choice of $\alpha_S$ used in each PDF set,
are summarised at the following location 
\footnote{https://twiki.cern.ch/twiki/bin/view/LHCPhysics/CERNYellowReportPageAt7TeV\#gluon\_gluon\_Fusion\_Process}.

These uncertainties are presented as overall normalisation uncertainties.
Because we perform a shape analysis as a function of an MVA output variable,
it is necessary to verify that the this overall normalisation uncertainty
does not underestimate the uncertainty at any point in the shape of the
MVA output variable.  We checked this hypothesis by following the standard
procedure described in CMS-AN2011-055, used for the W/Z cross section measurement.
Rather than running Monte Carlo production for different PDFs,
an event by event re-weighting procedure is done by scaling one PDF to another,
and then calculating the  effect on the cross section as a function of MVA output variable.

The effect of variation of the intrinsic and $\alpha_S$ uncertainties is shown in
Figures \ref{fig:appendix_signalshape_ct10} and \ref{fig:appendix_signalshape_mstw}
for the CT10 and MSTW PDF sets respectively.
The NNPDF uncertainty is shown in \ref{fig:appendix_signalshape_mstw}.
In this case, the uncertainty from sampling a large number of replica NNPDF sets produce with
different central values of $\alpha_S$ in the correct proportions covers both the
intrinsic and $\alpha_S$ uncertainties simultaneously.
These are shown for the $m_H=115$~GeV Higgs boson mass Monte-Carlo sample.

\begin{figure}[!hbtp]
\begin{center}
\label{fig:appendix_signalshape_ct10}
\subfigure[a]{\includegraphics[width=.45\textwidth]{figures/nn_hww115_ww_CT10_alphaS.pdf}}
\subfigure[b]{\includegraphics[width=.45\textwidth]{figures/nn_hww115_ww_CT10_envelope.pdf}}
\caption{The intrinsic and $\alpha_S$ variation of the CT10 PDF set (a) and the combined intrinsic and $\alpha_S$ variation (b). B
oth are shown as a function of the MVA output for the $m_H=115$~GeV Higgs boson mass dependent analysis. The uncertainty is shown
relative to the central value of the CT10 PDF set.}
\end{center}
\end{figure}

\begin{figure}[!hbtp]\begin{center}\label{fig:appendix_signalshape_mstw}\subfigure[a]{\includegraphics[width=.45\textwidth]{figures/nn_hww115_ww_MSTW_alphaS.pdf}}
\subfigure[b]{\includegraphics[width=.45\textwidth]{figures/nn_hww115_ww_MSTW_envelope.pdf}}
\caption{The intrinsic and $\alpha_S$ variation of the MSTW PDF set (a) and the combined intrinsic and $\alpha_S$ variation (b). B
oth are shown as a function of the MVA output for the $m_H=115$~GeV Higgs boson mass dependent analysis. The uncertainty is shown 
relative to the central value of the MSTW PDF set.}
\end{center}
\end{figure}

\begin{figure}[!hbtp]
\begin{center}
\label{fig:appendix_signalshape_nnpdf}
\includegraphics[width=.45\textwidth]{figures/nn_hww115_ww_NNPDF_envelope.pdf}
\caption{The envelope of the intrinsic and $\alpha_S$ variation of the NNPDF PDF set shown as a function of the MVA output for the
 $m_H=115$~GeV Higgs boson mass dependent analysis. The uncertainty is shown relative to the central value of the NNPDF PDF set.}
\end{center}
\end{figure}

The variations of each PDF set are then combined to construct an overall envelope to
cover the largest possible expected variation in the signal sample cross section
as a function of the MVA output variable in Figure \ref{fig:appendix_signalshape_pdfenvelope}.
We find that this is consistent with the overall scale uncertainty quoted at the
YellowReport page of (+8.1, -6.9)\%.
Thus, we consider the overall normalisation uncertainty to adequately contain possible variations
in the uncertainty as a function of the MVA output variable.

\begin{figure}[!hbtp]\begin{center}
\label{fig:appendix_signalshape_pdfenvelope}
\includegraphics[width=.45\textwidth]{figures/nn_hww115_ww_envelope.pdf}
\caption{The overall envelope of the PDF uncertainty shwon as a function of the MVA output for the $m_H=115$~GeV Higgs boson mass
dependent analysis. The upper and lower bounds of the uncertainty are shown relative to the central value of the PDF set used to g
enerate the signal sample, taking into account changes to the central value from the different PDF sets considered.}
\end{center}
\end{figure}

