With higher instantaneous luminosities, triggering on Higgs decays in
the di-lepton final state becomes very challenging. The single lepton
triggers can only be sustained with very tight identification and
isolation cuts and fairly large momentum thresholds. This leaves only
the double lepton triggers as a viable option for low mass Higgs
scenarios. We design the dilepton triggers to be sufficiently loose to
allow for some background events to be collected so that the fake
lepton background contribution in the signal region can be estimated
from data with adequate precision. The full names of the triggers used
in the analysis are summarized in Table~\ref{tab:triggers}.  We
describe the features and motivations for these triggers in Section
\ref{sec:mainTriggers} below.


\begin{table}[!ht]
  \begin{center}
 {\small
  \begin{tabular} {|l|l|l|c|p{1.0in}|} 
\hline
  Dataset & Trigger name & L1 seed & Purpose\\
  \hline
  SingleEle & HLT\_Ele27\_CaloIdVT\_CaloIsoT\_TrkIdT\_TrkIsoT\_v1 & L1\_SingleEG15  & $ee$, $e\mu$ \\
  SingleEle & HLT\_Ele27\_CaloIdVT\_CaloIsoT\_TrkIdT\_TrkIsoT\_v2 & L1\_SingleEG15  & $ee$, $e\mu$ \\
  \hline
  SingleMu & HLT\_IsoMu12\_v1   & L1\_SingleMu7  & $\mu\mu$, $e\mu$, efficiency \\
  SingleMu & HLT\_IsoMu17\_v5   & L1\_SingleMu10 & $\mu\mu$, $e\mu$, efficiency \\
  SingleMu & HLT\_Mu15\_v2      & L1\_SingleMu10 & $\mu\mu$, $e\mu$, efficiency \\
  \hline
  DoubleMu & HLT\_DoubleMu6\_v1 & L1\_DoubleMu3  & $\mu\mu$, efficiency\\
  DoubleMu & HLT\_DoubleMu7\_v1 & L1\_DoubleMu3  & $\mu\mu$, efficiency \\
  \hline
  DoubleElectron & HLT\_Ele17\_SC8\_M30\_v1\myfootnotemark &  L1\_SingleEG12  & efficiency\\ 
  DoubleElectron & HLT\_Ele17\_SC8\_M30\_v2 &  L1\_SingleEG12  & efficiency\\ 
  DoubleElectron & HLT\_Ele17\_Ele8\_Loose\_v1\myfootnotemark &  L1\_SingleEG12  & $ee$\\ 
  DoubleElectron & HLT\_Ele17\_Ele8\_Loose\_v2 &  L1\_SingleEG12  & $ee$\\ 
  DoubleElectron & HLT\_Ele17\_Ele8\_Tight\_v2\myfootnotemark &  L1\_SingleEG12  & $ee$\\ 
  DoubleElectron & HLT\_Ele32\_CaloIdL\_CaloIsoVL\_SC17\_v1 & L1\_SingleEG20 & $ee$, efficiency\\
  DoubleElectron & HLT\_Ele32\_CaloIdL\_CaloIsoVL\_SC17\_v2 & L1\_SingleEG20 & $ee$, efficiency\\
  \hline
  MuEG & HLT\_Mu17\_Ele8\_CaloIdL\_v1 & L1\_Mu3\_EG5 & $e\mu$ \\
  MuEG & HLT\_Mu17\_Ele8\_CaloIdL\_v2 & L1\_Mu3\_EG5 & $e\mu$ \\
  MuEG & HLT\_Mu8\_Ele17\_CaloIdL\_v1 & L1\_Mu3\_EG5 & $e\mu$ \\
  MuEG & HLT\_Mu8\_Ele17\_CaloIdL\_v2 & L1\_Mu3\_EG5 & $e\mu$ \\
 \hline
  \end{tabular}
}
  \caption{Triggers to be used in the analysis}
   \label{tab:triggers}
  \end{center}
\end{table}
\myfootnotetext{HLT\_Ele17\_CaloIdVT\_CaloIsoVT\_TrkIdT\_TrkIsoVT\_SC8\_Mass30}
\myfootnotetext{HLT\_Ele17\_CaloIdL\_CaloIsoVL\_Ele8\_CaloIdL\_CaloIsoVL}
\myfootnotetext{HLT\_Ele17\_CaloIdT\_TrkIdVL\_CaloIsoVL\_TrkIsoVL\_Ele8\_CaloIdT\_TrkIdVL\_CaloIsoVL\_TrkIsoVL} 
 

\subsubsection{Main Analysis Triggers}
\label{sec:mainTriggers}

The double electron trigger, {\bf HLT\_Ele17\_CaloIdL\_CaloIsoVL\_Ele8\_CaloIdL\_CaloIsoVL} 
requires two high level trigger electron 
candidates with fairly loose shower shape and calorimeter isolation
requirements on both legs. This is the most challenging channel in terms
of the total rate, due to larger fake electron background rates. To 
accomodate the offline $E_{T}$ selection cuts at $20$ GeV and $10$ GeV
for the leading and trailing leptons, respectively, we require $E_{T}$
cuts of $17$ and $8$ GeV at the HLT level. The leading electron is seeded by the
e/gamma Level-1 non-isolated trigger, L1\_SingleEG12, while the trailing electron
is left unseeded. The cuts imposed on the trigger electrons can be found in 
Table \ref{tab:HLTElectronCuts}, by matching the labels in the trigger name. 
It is important to keep in mind that variables used for online and offline 
selections not always match. Appendix~\ref{app:online_vs_offline} provides a detail
comparison of online vs offline selection requirements. 

For instantaneous luminosities above 1E33, additional requirements are
needed on the track to cluster matching and track isolation to reduce
the background rates. The cuts imposed on the electron candidates for
the tighter version of the dielectron trigger \\
{\bf \small
HLT\_Ele17\_CaloIdT\_TrkIdVL\_CaloIsoVL\_TrkIsoVL\_Ele8\_CaloIdT\_TrkIdVL\_CaloIsoVL\_TrkIsoVL
} can be found in Table \ref{tab:HLTElectronCuts}.


\begin{table}[htb]
 \caption{The trigger electron requirements. Values in parentheses corresponds with endcaps when different than in barrel. L=Loose, VL=Very loose.}
 \label{tab:HLTElectronCuts}
 \centering
 \begin{tabular}{|l||c|}
   \hline
   name                       &  criterion \\
   \hline \hline
   \multirow{2}{*}{CaloId\_L} & $\mathrm{H/E < 0.15 (0.10) }$ \\
                               & $\sigma_{\eta\eta}\mathrm{< 0.014\;(0.035)}$ \\
    \hline
    \multirow{2}{*}{TrkId\_VL} & $|\Delta\eta|\mathrm{< 0.01\; (0.01)}$ \\
                               & $\Delta\phi\mathrm{< 0.15\;(0.10)}$  \\
    \hline
    \multirow{2}{*}{CaloIso\_VL} & $\mathrm{ECalIso/E_T <0.2\;(0.2)}$ \\
                                 & $\mathrm{HCalIso/E_T <0.2\;(0.2)}$ \\    
    \hline
    TrkIso\_VL                   & $\mathrm{TrkIso/E_T <0.2\;(0.2)}$ \\

   \hline
 \end{tabular}
\end{table}

 
The double muon trigger, {\bf HLT\_DoubleMu7},  requires two high level trigger 
muon candidates with transverse momentum greater than $7$ GeV. It is seeded 
by the DoubleMu3 Level-1 trigger, with looser quality criteria than the
single muon Level-1 seeds. 
 
In the electron muon channel, we use two complementary triggers, 
{\bf HLT\_Mu17\_Ele8\_CaloIdL} and {\bf HLT\_Mu8\_Ele17\_CaloIdL} requiring
a muon HLT candidate and an electron HLT candidate. The electron
candidate is required to pass the ``Loose'' CaloId requirement as 
summarized in Table \ref{tab:HLTElectronCuts}. These triggers are seeded by 
the Level-1 MuOpen\_EG5 trigger, with minimal requirements on the muon candidate.

Finally, to further recover trigger inefficiencies, we also allow events that 
passed only the single electron 
({\bf HLT\_Ele27\_CaloIdVT\_CaloIsoT\_TrkIdT\_TrkIsoT\_v1}) or single 
isolated muon ({\bf HLT\_IsoMu17\_v5}) triggers. The requirements on the 
electron candidate in the single electron trigger are summarized in 
Table \ref{tab:HLTTightElectronCuts}. 

\begin{table}[htb]
 \caption{The trigger electron requirements for the single electron trigger. 
Values in parentheses corresponds with endcaps when different than in 
barrel. T=Tight, VT=Very tight.}
 \label{tab:HLTTightElectronCuts}
 \centering
 \begin{tabular}{|l||c|}
   \hline
   name                       &  criterion \\
   \hline \hline
   \multirow{2}{*}{CaloId\_VT} & $\mathrm{H/E < 0.05 (0.05) }$ \\
                               & $\sigma_{\eta\eta}\mathrm{< 0.011\;(0.031)}$ \\
    \hline
    \multirow{2}{*}{TrkId\_T} & $|\Delta\eta|\mathrm{< 0.008\; (0.008)}$ \\
                               & $\Delta\phi\mathrm{< 0.07\;(0.05)}$  \\
    \hline
    \multirow{2}{*}{CaloIso\_T} & $\mathrm{ECalIso/E_T <0.15\;(0.075)}$ \\
                                 & $\mathrm{HCalIso/E_T <0.15\;(0.075)}$ \\    
    \hline
    TrkIso\_T                   & $\mathrm{TrkIso/E_T <0.15\;(0.075)}$ \\

   \hline
 \end{tabular}
\end{table}


\subsubsection{Utility Triggers}
\label{sec:utilityTriggers}

To measure the electron selection efficiency and the electron trigger
efficiency we use specialized tag and probe triggers designed to maximize
the number of useful \dyll~events for both low and high $p_{T}$ electrons,
while keeping the total trigger rate to a reasonable level. We use 
two different triggers, one specifically for low $p_{T}$ electrons,
{\bf HLT\_Ele17\_CaloIdVT\_CaloIsoVT\_TrkIdT\_TrkIsoVT\_SC8\_Mass30}, 
where very tight identification and isolation cuts are imposed on the 
tag leg to reduce the background rate, and another specifically for
higher $p_{T}$ electrons, {\bf HLT\_Ele32\_CaloIdL\_CaloIsoVL\_SC17}.
The cuts on the tight leg for the low $p_{T}$ trigger is summarized in 
Table \ref{tab:HLTVeryTightElectronCuts}. The cuts on the tight leg
for the high $p_{T}$ triggers can be found in 
Table \ref{tab:HLTElectronCuts}. 

\begin{table}[htb]
 \caption{The trigger electron requirements for the tag leg of the double electron 
tag and probe trigger. Values in parentheses corresponds with endcaps when 
different than in barrel. T=Tight, VT=Very tight.}
 \label{tab:HLTVeryTightElectronCuts}
 \centering
 \begin{tabular}{|l||c|}
   \hline
   name                       &  criterion \\
   \hline \hline
   \multirow{2}{*}{CaloId\_VT} & $\mathrm{H/E < 0.05 (0.05) }$ \\
                               & $\sigma_{\eta\eta}\mathrm{< 0.011\;(0.031)}$ \\
    \hline
    \multirow{2}{*}{TrkId\_T} & $|\Delta\eta|\mathrm{< 0.008\; (0.008)}$ \\
                               & $\Delta\phi\mathrm{< 0.07\;(0.05)}$  \\
    \hline
    \multirow{2}{*}{CaloIso\_VT} & $\mathrm{ECalIso/E_T <0.05\;(0.05)}$ \\
                                 & $\mathrm{HCalIso/E_T <0.05\;(0.05)}$ \\    
    \hline
    TrkIso\_VT                   & $\mathrm{TrkIso/E_T <0.05\;(0.05)}$ \\

   \hline
 \end{tabular}
\end{table}


Another set of specialized triggers are used for recording events
enriched in fake electrons and muons used to measure the lepton fake 
rates. There are two triggers each for electrons and muons with two 
different $p_{T}$ thresholds, in order to collect a sufficiently 
large sample for all $p_{T}$ bins. In order to collect a sample of
events with a fake lepton and a recoiling high $E_{T}$ jet for 
fake rate systematics studies, there are two additional 
triggers for an electron or muon with $p_{T} > 8$~GeV and a jet
with corrected $E_{T}>40$ GeV. Finally to study fake lepton composition
systematics, we use a photon plus electron and a photon plus muon
trigger, where very tight cuts are imposed on the photon to ensure purity. 
The exact cuts imposed on the photon are summarized in 
Table \ref{tab:PhotonPlusLeptonTriggerCuts}.
The full set of triggers and their Level-1 seeds are summarized in 
Table \ref{tab:HWWFakeRateL1Seeds}. 
%% These triggers are all prescaled, 
%% if necessary, to yield a rate of at  least $0.5$Hz each. This rate is 
%% sufficient to yield a sample of roughly $10^{6}$ events for every 4 
%% weeks of data taking, enough to make a fake rate measurement with 
%% small statistical uncertainties. 


\begin{table}[htb]
  \caption{Photon identification criteria. Values in parentheses corresponds 
with endcaps when different than in barrel. VT=Very tight, T=Tight.}
  \label{tab:PhotonPlusLeptonTriggerCuts}
  \centering
  \begin{tabular}{|l||c|}
    \hline
    name                        &  criterion \\
    \hline \hline
    \multirow{2}{*}{CaloId\_VT} & $\mathrm{H/E < 0.05 }$ \\
                                & $\sigma_{\eta\eta}\mathrm{< 0.011\;(0.01)}$ \\
    \hline
    \multirow{3}{*}{Iso\_T}     & $\mathrm{ECalIso} < 5.0 + 0.012*E_{T} $ \\
                                & $\mathrm{HCalIso} < 3.0 + 0.005*E_{T} $ \\
                                & $\mathrm{TrkIso}  < 3.0 + 0.002*E_{T} $ \\
    \hline
  \end{tabular}
\end{table}


\begin{table}[htb]
  \caption{Summary of all lepton fake rate triggers and their Level-1 seeds.}
  \label{tab:HWWFakeRateL1Seeds}
  \centering
  \begin{tabular}{|l||c|}
    \hline
    HLT Path                                  &  L1 Seed       \\
    \hline \hline
    HLT\_Ele8\_CaloIdL\_CaloIsoVL             & L1\_SingleEG5  \\
    HLT\_Ele17\_CaloIdL\_CaloIsoVL            & L1\_SingleEG12 \\
    HLT\_Ele8\_CaloIdL\_CaloIsoVL\_Jet40      & L1\_EG5\_Jet36\_deltaPhi  \\
    HLT\_Photon20\_CaloIdVT\_IsoT\_Ele8\_CaloIdL\_CaloIsoVL & L1\_SingleEG12 \\
    \hline \hline
    HLT\_Mu8                                  &  L1\_SingleMu3  \\
    HLT\_Mu15                                 &  L1\_SingleMu10 \\
    HLT\_Mu8\_Jet40                           &  L1\_Mu3\_Jet20   \\
    HLT\_Mu8\_Photon20\_CaloIdVT\_IsoT        &  L1\_Mu3\_EG5   \\
    \hline
  \end{tabular}
\end{table}

