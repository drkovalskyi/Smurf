Triggering on Higgs boson decays in the dilepton final state increases 
in difficulty with increasing instantaenous luminosity.
Single lepton triggers can only be sustained with very tight identification and
isolation requirements and large transverse momentum thresholds.
This means that double lepton triggers are the only viable option to maintain
sensitivity to the low mass Higgs boson, where the leptons transverse momentum
can be small.

We designed a suite of signal and control triggers appropriate for this analysis
These dilepton triggers have a high efficiency to collect Higgs boson events
and are sufficiently loose to collect control events to estimate
fake lepton backgrounds and selection efficiencies with adequate precision.
The full names of the triggers used are summarised in Table~\ref{tab:triggers}.  
We describe the features and motivations for these triggers in Section \ref{sec:mainTriggers}.

\begin{table}[!ht]
  \caption{Triggers to be used in the analysis}
    \vspace{5pt}
   \label{tab:triggers}
  \begin{center}
 {\small
  \begin{tabular} {l|l|l|c} 
\hline
  Dataset & Trigger name & L1 seed & Purpose\\
  \hline \hline
  SingleEle & HLT\_Ele27\_CaloIdVT\_CaloIsoT\_TrkIdT\_TrkIsoT\_v1 & L1\_SingleEG15  & $ee$, $e\mu$ \\
  SingleEle & HLT\_Ele27\_CaloIdVT\_CaloIsoT\_TrkIdT\_TrkIsoT\_v2 & L1\_SingleEG15  & $ee$, $e\mu$ \\
  \hline \hline
  SingleMu & HLT\_IsoMu12\_v1   & L1\_SingleMu7  & $\mu\mu$, $e\mu$, efficiency \\
  SingleMu & HLT\_IsoMu17\_v5   & L1\_SingleMu10 & $\mu\mu$, $e\mu$, efficiency \\
  SingleMu & HLT\_Mu15\_v2      & L1\_SingleMu10 & $\mu\mu$, $e\mu$, efficiency \\
  \hline \hline
  DoubleMu & HLT\_DoubleMu6\_v1 & L1\_DoubleMu3  & $\mu\mu$, efficiency\\
  DoubleMu & HLT\_DoubleMu7\_v1 & L1\_DoubleMu3  & $\mu\mu$, efficiency \\
  \hline \hline

  \multirow{2}{*}{DoubleElectron} & HLT\_Ele17\_CaloIdVT\_CaloIsoVT\_TrkIdT\_TrkIsoVT\_ &  L1\_oingleEG12  & efficiency \\
                                  & SC8\_Mass30\_v1 &                  & \\    
  \multirow{2}{*}{DoubleElectron} & HLT\_Ele17\_CaloIdVT\_CaloIsoVT\_TrkIdT\_TrkIsoVT\_ &  L1\_oingleEG12  & efficiency \\
                                  & SC8\_Mass30\_v2 &                  & \\ 

    \hline \hline

  \multirow{2}{*}{DoubleElectron} & HLT\_Ele17\_CaloIdL\_CaloIsoVL\_&  L1\_oingleEG12  & $ee$ \\
                                  & Ele8\_CaloIdL\_CaloIsoVL\_v1 &                  & \\
  \multirow{2}{*}{DoubleElectron} & HLT\_Ele17\_CaloIdL\_CaloIsoVL\_&  L1\_oingleEG12  & $ee$ \\
                                  & Ele8\_CaloIdL\_CaloIsoVL\_v2 &                  & \\

  \multirow{2}{*}{DoubleElectron} & HLT\_Ele17\_CaloIdT\_TrkIdVL\_CaloIsoVL\_TrkIsoVL\_ &  L1\_oingleEG12  & $ee$ \\
                                  & Ele8\_CaloIdT\_TrkIdVL\_CaloIsoVL\_TrkIsoVL\_v1 &                  & \\
  \multirow{2}{*}{DoubleElectron} & HLT\_Ele17\_CaloIdT\_TrkIdVL\_CaloIsoVL\_TrkIsoVL\_ &  L1\_oingleEG12  & $ee$ \\
                                  & Ele8\_CaloIdT\_TrkIdVL\_CaloIsoVL\_TrkIsoVL\_v2 &                  & \\

  DoubleElectron & HLT\_Ele32\_CaloIdL\_CaloIsoVL\_SC17\_v1 & L1\_SingleEG20 & $ee$, efficiency\\
  DoubleElectron & HLT\_Ele32\_CaloIdL\_CaloIsoVL\_SC17\_v2 & L1\_SingleEG20 & $ee$, efficiency\\
  \hline \hline
  MuEG & HLT\_Mu17\_Ele8\_CaloIdL\_v1 & L1\_Mu3\_EG5 & $e\mu$ \\
  MuEG & HLT\_Mu17\_Ele8\_CaloIdL\_v2 & L1\_Mu3\_EG5 & $e\mu$ \\
  MuEG & HLT\_Mu8\_Ele17\_CaloIdL\_v1 & L1\_Mu3\_EG5 & $e\mu$ \\
  MuEG & HLT\_Mu8\_Ele17\_CaloIdL\_v2 & L1\_Mu3\_EG5 & $e\mu$ \\
 \hline
  \end{tabular}
}
  \end{center}
\end{table}

\subsubsection{Main Analysis Triggers}
\label{sec:mainTriggers}

Controlling the total trigger rate is most challenging in
the dielectron channel, due to large fake electron backgrounds rates.

The main dielectron trigger is {\bf HLT\_Ele17\_CaloIdL\_CaloIsoVL\_Ele8\_CaloIdL\_CaloIsoVL},
which requires two high level trigger electron 
candidates with loose shower shape and calorimeter isolation requirements on both legs. 
To accomodate the offline selection of $E_{T}>20,10$~GeV for the leading and trailing
electrons, this trigger requires $E_{T}>17,8$~GeV at the HLT level.
The leading electron is seeded by {L1\_SingleEG12} at Level-1, 
while the trailing electron has no seeding requirement. 
The cuts imposed at the HLT level are summarised in Table \ref{tab:HLTElectronCuts}
according to the mnemonics used in the trigger names.

Additional requirements must be added to the track to cluster matching 
and track isolation to control the total trigger rate at instantaneous luminosities above $1\times10^{33}$. 
The cuts imposed on the electron candidates for the tighter version of the dielectron trigger\\
{\bf HLT\_Ele17\_CaloIdT\_TrkIdVL\_CaloIsoVL\_TrkIsoVL\_Ele8\_CaloIdT\_TrkIdVL\_CaloIsoVL\_TrkIsoVL}\\
are described in Table \ref{tab:HLTElectronCuts}.

Because the HLT uses simplified algorithms compared to the offline selections
the variables used online and offline do not always correspond exactly.
A detailed comparison between the online and offline selection requirements is given in
Appendix~\ref{app:online_vs_offline}.

\begin{table}[htb]
 \caption{Summary of requirements applied in the electron and muon triggers used for this analysis. 
Values in parentheses corresponds with endcaps when different than in barrel. L=Loose, VL=Very loose, T=Tight, VT=Very Tight.}
    \vspace{5pt}
 \label{tab:HLTElectronCuts}
 \centering
 \begin{tabular}{l|c}
   \hline
   name                       &  criterion \\
   \hline \hline
   \multirow{2}{*}{CaloId\_L} & $\mathrm{H/E < 0.15 (0.10) }$ \\
                               & $\sigma_{\eta\eta}\mathrm{< 0.014\;(0.035)}$ \\
    \hline
   \multirow{2}{*}{CaloId\_VT} & $\mathrm{H/E < 0.05 (0.05) }$ \\
                               & $\sigma_{\eta\eta}\mathrm{< 0.011\;(0.031)}$ \\
    \hline \hline
    \multirow{2}{*}{TrkId\_VL} & $|\Delta\eta|\mathrm{< 0.01\; (0.01)}$ \\
                               & $\Delta\phi\mathrm{< 0.15\;(0.10)}$  \\
    \hline
    \multirow{2}{*}{TrkId\_T} & $|\Delta\eta|\mathrm{< 0.008\; (0.008)}$ \\
                               & $\Delta\phi\mathrm{< 0.07\;(0.05)}$  \\
    \hline \hline
    \multirow{2}{*}{CaloIso\_VL} & $\mathrm{ECalIso/E_T <0.2\;(0.2)}$ \\
                                 & $\mathrm{HCalIso/E_T <0.2\;(0.2)}$ \\    
    \hline
    \multirow{2}{*}{CaloIso\_T} & $\mathrm{ECalIso/E_T <0.15\;(0.075)}$ \\
                                 & $\mathrm{HCalIso/E_T <0.15\;(0.075)}$ \\
    \hline
    \multirow{2}{*}{CaloIso\_VT} & $\mathrm{ECalIso/E_T <0.05\;(0.05)}$ \\
                                 & $\mathrm{HCalIso/E_T <0.05\;(0.05)}$ \\
    \hline \hline
    TrkIso\_VL                   & $\mathrm{TrkIso/E_T <0.2\;(0.2)}$ \\
    \hline
    TrkIso\_T                   & $\mathrm{TrkIso/E_T <0.15\;(0.075)}$ \\
   \hline
    TrkIso\_VT                   & $\mathrm{TrkIso/E_T <0.05\;(0.05)}$ \\
    \hline
 \end{tabular}
\end{table}

 
The main dimuon trigger is {\bf HLT\_DoubleMu7}, which
requires two HLT muon candidates with transverse momentum greater than $7$~GeV. 
It is seeded by the {L1\_DoubleMu3} at Level-1, 
which has looser quality criteria than the single muon Level-1 seeds.
 
In the electron muon channel, we use two complementary triggers: 
{\bf HLT\_Mu17\_Ele8\_CaloIdL} and {\bf HLT\_Mu8\_Ele17\_CaloIdL}, which require
both muon and electron HLT candidates.
The electron candidate must pass the CaloIdL requirement
summarised in Table \ref{tab:HLTElectronCuts}.
Both triggers are seeded by {MuOpen\_EG5} at Level-1, which has 
with minimal requirements on the muon candidate.

Finally, to recover any residual inefficiency, 
we also allow events that passed only the single electron 
({\bf HLT\_Ele27\_CaloIdVT\_CaloIsoT\_TrkIdT\_TrkIsoT\_v1}) or single 
isolated muon ({\bf HLT\_IsoMu17\_v5}) triggers. 
The requirements of the single electron trigger are summarised in Table \ref{tab:HLTElectronCuts}. 

%\begin{table}[htb]
% \caption{The trigger electron requirements for the single electron trigger. 
%Values in parentheses corresponds with endcaps when different than in 
%barrel. T=Tight, VT=Very tight.}
% \label{tab:HLTTightElectronCuts}
% \centering
% \begin{tabular}{|l||c|}
%   \hline
%   name                       &  criterion \\
%   \hline \hline
%   \multirow{2}{*}{CaloId\_VT} & $\mathrm{H/E < 0.05 (0.05) }$ \\
%                               & $\sigma_{\eta\eta}\mathrm{< 0.011\;(0.031)}$ \\
%    \hline
%    \multirow{2}{*}{TrkId\_T} & $|\Delta\eta|\mathrm{< 0.008\; (0.008)}$ \\
%                               & $\Delta\phi\mathrm{< 0.07\;(0.05)}$  \\
%    \hline
%    \multirow{2}{*}{CaloIso\_T} & $\mathrm{ECalIso/E_T <0.15\;(0.075)}$ \\
%                                 & $\mathrm{HCalIso/E_T <0.15\;(0.075)}$ \\    
%    \hline
%    TrkIso\_T                   & $\mathrm{TrkIso/E_T <0.15\;(0.075)}$ \\
%
%   \hline
% \end{tabular}
%\end{table}

\subsubsection{Utility Triggers}
\label{sec:utilityTriggers}

The main dimuon analysis triggers are suitable to collect an event sample
for efficiency measurements.
Because the main dielectron analysis triggers make requirements on
both legs, events collected with those triggers cannot be used to measure
efficiencies without introducing unacceptable bias.

Thus, to measure the electron selection and trigger efficiency
we introduce two specialised tag and probe triggers designed to maximize
the number of useful \dyll~events for both low and high $p_{T}$ electrons,
while keeping the total trigger rate at a reasonable level. 
The tag and probe method is described later in Section \ref{sec:efficiency}.

The first trigger, {\bf HLT\_Ele17\_CaloIdVT\_CaloIsoVT\_TrkIdT\_TrkIsoVT\_SC8\_Mass30},
which is used to probe low $p_T$ electrons
applies very tight identification and isolation requirements on the tag leg to reduce the background rate.
The second trigger, \\{\bf HLT\_Ele17\_CaloIdVT\_CaloIsoVT\_TrkIdT\_TrkIsoVT\_SC8\_Mass30},\\
is used to probe higher $p_{T}$ electrons.
The requirements of both triggers are summarised in Table \ref{tab:HLTElectronCuts}.

%\begin{itemize}
%    \item {\bf HLT\_Ele17\_CaloIdVT\_CaloIsoVT\_TrkIdT\_TrkIsoVT\_SC8\_Mass30}
%    \item {\bf HLT\_Ele32\_CaloIdL\_CaloIsoVL\_SC17
%\end{itemize}

%\begin{table}[htb]
% \caption{The trigger electron requirements for the tag leg of the double electron 
%tag and probe trigger. Values in parentheses corresponds with endcaps when 
%different than in barrel. T=Tight, VT=Very tight.}
% \label{tab:HLTVeryTightElectronCuts}
% \centering
% \begin{tabular}{|l||c|}
%   \hline
%   name                       &  criterion \\
%   \hline \hline
%   \multirow{2}{*}{CaloId\_VT} & $\mathrm{H/E < 0.05 (0.05) }$ \\
%                               & $\sigma_{\eta\eta}\mathrm{< 0.011\;(0.031)}$ \\
%    \hline
%    \multirow{2}{*}{TrkId\_T} & $|\Delta\eta|\mathrm{< 0.008\; (0.008)}$ \\
%                               & $\Delta\phi\mathrm{< 0.07\;(0.05)}$  \\
%    \hline
%    \multirow{2}{*}{CaloIso\_VT} & $\mathrm{ECalIso/E_T <0.05\;(0.05)}$ \\
%                                 & $\mathrm{HCalIso/E_T <0.05\;(0.05)}$ \\    
%    \hline
%    TrkIso\_VT                   & $\mathrm{TrkIso/E_T <0.05\;(0.05)}$ \\
%
%   \hline
% \end{tabular}
%\end{table}

Another set of specialised triggers are used to record events
enriched in fake electrons and muons for the measurement of jet induced backgrounds.
This is done using the fake rate method, which is described in detail in
Section \ref{sec:bkg_fakes}.

We introduce four triggers to for each of the electron and muon fake rate measurements.
Because these triggers are prescaled, the first two impose different $p_T$ thresholds 
to collect a sufficient sample over a large $p_T$ range.
The third trigger requires an additional jet with corrected $E_{T}>40$~GeV
to perform systematic studies on the fake rate measurement.
The fourth trigger is used to collect $\mathrm{\gamma+jet}$ events to
make a second independent measurement of the fake rate.
This trigger imposes tight cuts on the photon to ensure purity.
These cuts are summarised in Table \ref{tab:PhotonPlusLeptonTriggerCuts}.

\begin{table}[htb]
 \caption{Summary of requirements applied in the photon triggers used for this analysis. 
Values in parentheses corresponds with endcaps when different than in barrel. T=Tight, VT=Very Tight.}
    \vspace{5pt}
  \label{tab:PhotonPlusLeptonTriggerCuts}
  \centering
  \begin{tabular}{l||c}
    \hline
    name                        &  criterion \\
    \hline \hline
    \multirow{2}{*}{CaloId\_VT} & $\mathrm{H/E < 0.05 }$ \\
                                & $\sigma_{\eta\eta}\mathrm{< 0.011\;(0.01)}$ \\
    \hline \hline
    \multirow{3}{*}{Iso\_T}     & $\mathrm{ECalIso} < 5.0 + 0.012*E_{T} $ \\
                                & $\mathrm{HCalIso} < 3.0 + 0.005*E_{T} $ \\
                                & $\mathrm{TrkIso}  < 3.0 + 0.002*E_{T} $ \\
    \hline
  \end{tabular}
\end{table}

The full set of triggers and their Level-1 seeds are summarised in 
Table \ref{tab:HWWFakeRateL1Seeds}. 
%% These triggers are all prescaled, 
%% if necessary, to yield a rate of at  least $0.5$Hz each. This rate is 
%% sufficient to yield a sample of roughly $10^{6}$ events for every 4 
%% weeks of data taking, enough to make a fake rate measurement with 
%% small statistical uncertainties. 

\begin{table}[htb]
  \caption{Summary of all lepton fake rate triggers and their Level-1 seeds.}
    \vspace{5pt}
  \label{tab:HWWFakeRateL1Seeds}
  \centering
  \begin{tabular}{l||c}
    \hline
    HLT Path                                  &  L1 Seed       \\
    \hline \hline
    HLT\_Ele8\_CaloIdL\_CaloIsoVL             & L1\_SingleEG5  \\
    HLT\_Ele17\_CaloIdL\_CaloIsoVL            & L1\_SingleEG12 \\
    HLT\_Ele8\_CaloIdL\_CaloIsoVL\_Jet40      & L1\_EG5\_Jet36\_deltaPhi  \\
    HLT\_Photon20\_CaloIdVT\_IsoT\_Ele8\_CaloIdL\_CaloIsoVL & L1\_SingleEG12 \\
    \hline \hline
    HLT\_Mu8                                  &  L1\_SingleMu3  \\
    HLT\_Mu15                                 &  L1\_SingleMu10 \\
    HLT\_Mu8\_Jet40                           &  L1\_Mu3\_Jet20   \\
    HLT\_Mu8\_Photon20\_CaloIdVT\_IsoT        &  L1\_Mu3\_EG5   \\
    \hline
  \end{tabular}
\end{table}

