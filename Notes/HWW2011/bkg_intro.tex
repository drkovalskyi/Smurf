We use a combination of data-driven methods and detailed Monte Carlo
simulation studies to estimate background contributions.  From data we
can estimate the following backgrounds: $\Wjets$, $\dyll$, $\WZ$ and
$\ZZ$ (for events where both leptons come from a $\dyll$), top
background and \WW{}. The rest of processes are taken from simulation.

Background composition and yeilds depends on the final state and on
the Higgs boson mass hypothesis under study. In the 0-Jet final state, 
the non-resonant \WW{} background dominates, while \wjets\ background contribution 
becomes sizable in the low Higgs mass hypotheses (see Table~\ref{tab:wwselection0}). 
In the 1-Jet and 2-Jet final states, the largest background contribution comes from 
top decays, while the non-resonant \ww\ background contribution is the second largest. 

For the backgrounds that can be estimated from data, 
we perform a data-driven background estimate in the signal region 
if the expected background contribution is sizable in the signal region. 
If the expected contribution in the signal region is limited by statistics, 
we first estimate the background contribution with the $WW$ preselection from data 
and then extrapolate this estimation to the signal region using MC. 

%For the backgrounds that are expected to have a sizable number of events after
%final Higgs cuts we perform a data driven background estimation for
%the final selection. If the expected number of background events is
%small (just a few event), we estimate the background at the \WW\
%selection level and scale it down for a final selection using the
%Monte Carlo predicted scale factors.
