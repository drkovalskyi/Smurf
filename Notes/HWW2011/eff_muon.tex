
The muon selection efficiency and the resulting data to simulation
scale factors are estimated using a similar method to the electron efficiency.
Since the muon reconstruction was proved to be well modeled in the simulation~\cite{VBTFCrossSectionNote}
we test only the analysis selection part.

We measure the muon selection efficiency with respect to a reconstructed Global muon
denominator in the
following four detector regions according to $\eta$: barrel ($|\eta|<0.8$),
overlap region between DT and CSC ($0.8<|\eta|<1.2$), endcaps ($1.2<|\eta|<2.1$) and
most forward region ($2.1<|\eta|<2.4$).
The resulting data to simulation scale factors are given in Table \ref{tab:eff_mu_offline}.

\begin{table}[!ht]
\begin{center}
\begin{tabular}{c|c|c|c|c} 
\hline
              & Barrel ($|\eta|<0.8$) & DT ($0.8<|\eta|<1.2$) &  CSC ($1.2<|\eta|<2.1$) & Forward  ($2.1<|\eta|<2.4$)  \\ 
\hline
$10<p_{T}<15$ & XXX$\pm$YYY & XXX$\pm$YYY    & XXX$\pm$YYY & XXX$\pm$YYY             \\ \hline
$15<p_{T}<20$ & XXX$\pm$YYY & XXX$\pm$YYY    & XXX$\pm$YYY & XXX$\pm$YYY        \\ \hline
$p_T>20$   & XXX$\pm$YYY & XXX$\pm$YYY & XXX$\pm$YYY & XXX$\pm$YYY\\ \hline
\end{tabular}
\caption{Muon selection efficiency for the barrel and endcap.
\label{tab:eff_mu_offline}}
\end{center}
\end{table}
