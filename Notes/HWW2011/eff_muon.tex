
The muon selection efficiency and the resulting data to simulation
scale factors are estimated using a similar method to the electron efficiency.
Since the muon reconstruction was proved to be well modeled in the simulation~\cite{VBTFCrossSectionNote}
we test only the analysis selection part.

We measure the muon selection efficiency with respect to a reconstructed Global muon
denominator.
% in the
%following four detector regions according to $\eta$: barrel ($|\eta|<0.8$),
%overlap region between DT and CSC ($0.8<|\eta|<1.2$), endcaps ($1.2<|\eta|<2.1$) and
%most forward region ($2.1<|\eta|<2.4$).
The resulting data to simulation scale factors are given in Table \ref{tab:eff_mu_offline}.

\begin{table}[!ht]
\begin{center}
\begin{tabular}{c|c|c}
\hline
Measurement & Barrel ( $|\eta|<1.479$ )   & Endcap ( $|\eta|>1.479$ )  \\ 
\hline
<<<<<<< eff_muon.tex
$  10<p_T<  15$ & 0.93 $\pm$ 0.02  & 0.95 $\pm$ 0.02  \\ \hline 
$  15<p_T<  20$ & 0.96 $\pm$ 0.01  & 0.93 $\pm$ 0.01  \\ \hline 
$  p_T>     20$ & 1.00 $\pm$ 0.00  & 0.98 $\pm$ 0.00  \\ \hline
=======
$  10<p_T<  15$ & 0.93 $\pm$ 0.02  & 0.95 $\pm$ 0.02  \\ \hline 
$  15<p_T<  20$ & 0.96 $\pm$ 0.01  & 0.93 $\pm$ 0.01  \\ \hline 
$  p_T>     20$ & 1.00 $\pm$ 0.00  & 0.98 $\pm$ 0.00  \\ \hline 
>>>>>>> 1.8
\end{tabular}
\caption{Offline selection scale factors for muons.}
\label{tab:eff_mu_offline}
\end{center}
\end{table}

