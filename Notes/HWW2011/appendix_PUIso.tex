This section presents the effect of pileup events on the isolation of the leptons 
produced in the primary interaction. The presence of additional inelastic
interactions is expected to add some additional energy to the isolation cone
of the leptons from the primary interaction, thus decreasing the efficiency
of the isolation cut on signal events. This decrease in efficiency can be 
studied as a function of the number of reconstructed primary vertices, serving
as a reasonable measure of the number of additional pileup interaction.

A number of techniques exist to correct for this inefficiency due to pileup,
which are still under investigation at the moment. For the current analysis, 
we do not perform any pileup correction for isolation and quantify the loss
in this section.

\begin{figure}[!htbp]
\begin{center}
\subfigure[Data Barrel ]{\includegraphics[width=0.45\textwidth]{figures/ElectronIsolationEfficiency_Barrel_vs_NVertices_DataTagAndProbe.pdf}}
\subfigure[Data Endcap]{\includegraphics[width=0.45\textwidth]{figures/ElectronIsolationEfficiency_Endcap_vs_NVertices_DataTagAndProbe.pdf}}
\subfigure[MC Barrel ]{\includegraphics[width=0.45\textwidth]{figures/ElectronIsolationEfficiency_Barrel_vs_NVertices_DataTagAndProbe.pdf}}
\subfigure[MC Endcap]{\includegraphics[width=0.45\textwidth]{figures/ElectronIsolationEfficiency_Endcap_vs_NVertices_DataTagAndProbe.pdf}}
\caption{Isolation efficiency vs number of reconstructed primary vertices for electrons, comparing the 
results from the tag and probe selection on 2011 data with the Z Monte Carlo simulation.}
\label{fig:eleIsoEff_TagAndProbe_vs_NVertices}
\end{center}
\end{figure}

\begin{figure}[!htbp]
\begin{center}
\subfigure[Data Barrel ]{\includegraphics[width=0.45\textwidth]{figures/MuonIsolationEfficiency_PUStudy_vs_NVertices_DataTagAndProbe.pdf}}
\subfigure[Data Endcap]{\includegraphics[width=0.45\textwidth]{figures/MuonIsolationEfficiency_PUStudy_vs_NVertices_MCTagAndProbe.pdf}}
\caption{Isolation efficiency vs number of reconstructed primary vertices for muons, comparing the 
results from the tag and probe selection on 2011 data with the Z Monte Carlo simulation.}
\label{fig:muIsoEff_TagAndProbe_vs_NVertices}
\end{center}
\end{figure}

Figs. \ref{fig:eleIsoEff_TagAndProbe_vs_NVertices} and \ref{fig:muIsoEff_TagAndProbe_vs_NVertices} 
show the isolation efficiency for electrons and muons respectively, measured using the
tag and probe selection, for 2011 data and Z Monte Carlo simulation. Agreement in these
efficiencies allow us to use the signal Monte Carlo to infer the efficiency loss under 
high pileup environment. The lepton isolation efficiency for HWW signal events are shown
in Fig \ref{fig:HWW130IsoEff_vs_NVertices}, showing a loss of efficiency of 8\% for 
electrons and 2\% for muons, between events with one reconstructed primary vertex
and 10 reconstructed primary vertices. The efficiency loss is higgs mass dependent due to
the dependence on the $p_{T}$ of the lepton. Fig \ref{fig:HWW130IsoEff_vs_NVertices_ptbins}
shows this efficiency loss for leptons with $p_{T}$ between $10$ and $20$ GeV compared
to leptons with $p_{T}$ between $35$ and $50$ GeV. The efficiency loss is almost $30\%$
for low $p_{T}$ electrons, while it is almost negligible for high $p_{T}$ leptons.

\begin{figure}[!htbp]
\begin{center}
\subfigure[HWW130 Electrons]{\includegraphics[width=0.45\textwidth]{figures/ElectronIsolationEfficiency_BkgData_vs_NVertices_HWW130_Pt20.pdf}}
\subfigure[HWW130 Muons]{\includegraphics[width=0.45\textwidth]{figures/MuonIsolationEfficiency_BkgData_vs_NVertices_HWW130_Pt20.pdf}}
\caption{Isolation efficiency vs number of reconstructed primary vertices for electrons and muons
in the HWW ($m_{H} = 130$) Monte Carlo simulation.}
\label{fig:HWW130IsoEff_vs_NVertices}
\end{center}
\end{figure}

\begin{figure}[!htbp]
\begin{center}
\subfigure[HWW130 Electrons Pt 10-20 GeV]{\includegraphics[width=0.45\textwidth]{figures/ElectronIsolationEfficiency_BkgData_vs_NVertices_HWW130_Pt10To20.pdf}}
\subfigure[HWW130 Muons Pt 10-20 GeV]{\includegraphics[width=0.45\textwidth]{figures/MuonIsolationEfficiency_BkgData_vs_NVertices_HWW130_Pt10To20.pdf}}
\subfigure[HWW130 Electrons Pt 35-50 GeV]{\includegraphics[width=0.45\textwidth]{figures/ElectronIsolationEfficiency_BkgData_vs_NVertices_HWW130_Pt35To50.pdf}}
\subfigure[HWW130 Muons Pt 35-50 GeV]{\includegraphics[width=0.45\textwidth]{figures/MuonIsolationEfficiency_BkgData_vs_NVertices_HWW130_Pt35To50.pdf}}
\caption{Isolation efficiency vs number of reconstructed primary vertices for electrons and muons
in the HWW ($m_{H} = 130$) Monte Carlo simulation, for different lepton $p_{T}$ bins.}
\label{fig:HWW130IsoEff_vs_NVertices_ptbins}
\end{center}
\end{figure}

This is one aspect of the analysis that will be improved in the future, either
employing appropriate pileup corrections to recover these inefficiencies, or 
parameterizing and re-tuning the cuts as a function of some observable correlated
to pileup. 


