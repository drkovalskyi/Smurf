Primary vertices are reconstructed using the so-called Deterministic Annealing (DA) 
clustering of tracks.  Reconstructed primary vertices are required to have a
$z$ position within 24~cm of the nominal detector center and a radial position within 
2~cm of the beamspot.  There must also be greater than four degrees of freedom in
the fitted vertex.  From the set of primary vertices in the event passing these
selection cuts, the vertex with the largest summed squared-$\pt$ of the associated
tracks is chosen as the event primary vertex.  Reconstructed leptons will be required 
to have small impact parameters with respect to this vertex.

Due to the fast evolution of the LHC machine, with a rapid rise in the instantaneous
luminosity, the data taking conditions have changed rapidly. 
In particular it is difficult to exactly reproduce the number of overlapping 
events (i.e. pileup) between data and simulation, and thus there will be differences
in the number of reconstructed primary vertices.
We can correct this disagreement by reweighting the simulation to
match the distribution in data. Details about the reweighting procedure are
reported in Appendix~\ref{app:vertex_reweight}.

