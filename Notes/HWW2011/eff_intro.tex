 
We used the tag and probe method on \dyll~events to provide an unbiased, high-purity, 
lepton sample with which to measure both online and offline selection efficiencies.
This method, which is now described, 
has been used successfully in previous CMS analyses \cite{ref:tagprobe_mit_w}\cite{ref:tagprobe_snt_top}.

\subsection{Method}
We used a single lepton triggered sample available from the Express Stream, 
from which we selected a subset of di-lepton events.
Ultimately we will repeat this using a double lepton triggered sample from the Prompt Reco,
where we select events using one of our tag and probe triggers which are very tight on one lepton,
but loose on the other.

At least one of the leptons, the {\it tag}, was required to pass the full selection criteria criteria 
whilst the other lepton, the {\it probe}, was required to pass a set of identification criteria leaving 
it unbiassed with respect to the criterion under study. 
By requiring that the tag was able to have passed the single lepton trigger on which the events were acquired, 
we reduced the bias due to the trigger on the probe.
Because the analysis uses the same mass window to reduce the \dyll~contribution, 
the tag and probe sample represents an independent control sample.
The tight criteria imposed on the tag coupled with the invariant mass requirement were sufficient to ensure high purity.

To extract the efficiency of offline selection and single trigger efficiency on a per lepton basis, 
we first construct all possible tag-probe pairs in every event.
Because more than one lepton can meet the tag criteria it is possible to use the same event more than once, to find the efficiency

\begin{eqnarray}
\label{eqn:tagAndProbeEfficiencyEqn}
\varepsilon = \frac{2TT + TP}{2TT + TP + TF}.
\end{eqnarray}

Where the event categories are defined as:

\begin{itemize}
	\item 2TT: Both leptons passed the tight criteria, including the trigger. This means that either lepton could be used as a probe, 
	so such events were counted twice.
	\item TP: The probe passed the selection criterion but did not pass the tight criteria.
	\item TF: The probe failed the selection criterion.
\end{itemize}

For the categories to be mutually exclusive it is important to note that the probe definition, 
and the selection criteria studied are always a subset of the tag criteria.
Classifying events and looping over all possible tag-probe combinations are thus equivalent.
To take into account the possibility of non \dyll~background events, we use a fitting technique
to extract the signal component for the passing and failing categories. 
This method and its associated systematics are discussed in detail in Reference \cite{ref:tagprobe_mit_w}.
Because the kinematics differ between \dyll~events and those on which we apply the measured efficiencies,
we split the efficiency measurements into the barrel and endcap regions.
We picked this division because it covers the largest observed variation in efficiency.

To produce overall data-MC scale factors to apply in the analysis, we factorise the efficiency measurements
into two steps such that $\varepsilon_{total} = \varepsilon_{offline} \times \varepsilon_{trigger}$.
The offline efficiency $\varepsilon_{offline} = \varepsilon_{offline}^{l1} \times \varepsilon_{offline}^{l2}$
is the product of the efficiencies of the two leptons and is discussed in more detail in Sections \ref{sec:eff_electron}
and \ref{sec:eff_muon} for electrons and muons respectively.
The trigger efficiency is measured with respect to the offline selection and
 is discussed in more detail in Section \ref{sec:eff_trigger}.

