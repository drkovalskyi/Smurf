%To split the analysis in different jet bins, we count events 
%containing jets with $\pt > ~30~\GeV$ within $|\eta|<5.0$. 

Jets are reconstructed using calorimeter and tracker information using a particle flow 
algorithm~\cite{jetpas}. The anti-${\rm k_T}$ clustering algorithm~\cite{antikt} 
with ${\rm R=0.5}$ is used. We apply the standard jet energy 
corrections~\cite{jes} to the reconstructed jets, where the L1 Fast Jets 
corrections are included. The latter corrections are rather important since 
they help in flatening the reconstruction efficiency as a function of the 
number of overlapping events.
To exclude electrons and muons from the jet sample, these 
jets are required to be separated from the selected leptons in $\Delta R$ 
by at least $\Delta R^{\mathrm{jet-lepton}}>0.3$.

In this analysis we use high $p_T$ jets to define the analysis jet bin
and low $p_T$ jets to do the top events veto.
We define:
\begin{itemize}
\item {\it counted jet}: a reconstructed jets with $\pt > ~30~\GeV$ within $|\eta|<5.0$;
\item {\it low $p_T$ jet}: a reconstructed jets with $7~ <\pt < ~30~\GeV$ within $|\eta|<5.0$
\end{itemize}

We analyze the events separately based on the number of counted jets
in the event.

%In the 0-Jet bin, the performance of the jet veto 
%is validated on data using Drell-Yan events, 
%as will be explained in Sec.~\ref{sec:backgrounds}. 
