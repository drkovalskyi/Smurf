%UPDATEME%
The datasets used for this analysis are summarized in Tables.~\ref{tab:DatasetsData} 
and~\ref{tab:DatasetsMC} for data and Monte Carlo, respectively. The total integrated
luminosity is 49 $\pm$ 2 $\ipb$. 
We use madgraph when possible, but different generators such as Pythia and Powheg 
are also used. 
%For $gg \to \WW$ a dedicated generator is used. For \wz\ and \zz\
%processes we use Pythia, since MadGraph samples are mixed with $\WW$ in
%a single $VV$ sample, which is difficult to use properly.

%The choice of the Monte Carlo samples depends on the sample
%availability, but in general we tried to be consistent and use a
%single generator - MadGraph. In the case of Drell-Yan, MadGraph samples
%are not adequate to cover the full mass spectrum. The main sample has a 50 $\GeVcc$ 
%minimum dilepton mass requirement, while the other one, covering
%the low mass region, has an additional requirement on extra jet
%activity. 
%We use madgraph when possible, but different generators are used for some samples
%For $gg \to \WW$ a dedicated generator is used. For \wz\ and \zz\
%processes we use Pythia, since MadGraph samples are mixed with $\WW$ in
%a single $VV$ sample, which is difficult to use properly.

%UPDATEME%
\begin{table}[!ht]
\begin{center}
\begin{tabular}{|c|c|}
\hline
 Dataset Description                   &   Dataset Name   \\
\hline
\hline
\multicolumn{2}{|c|}{$H \to \WW$ Signal Selection Samples} \\
\hline
Run2011A MuEl PromptReco            &  /MuEG/Run2011A-PromptReco-v1/AOD   \\
Run2011A DiMuon PromptReco          &  /DoubleMu/Run2011A-PromptReco-v1/AOD   \\
Run2011A SingleMuon PromptReco      &  /SingleMu/Run2011A-PromptReco-v1/AOD   \\
Run2011A DiElectron PromptReco      &  /DoubleElectron/Run2011A-PromptReco-v1/AOD   \\
%Run2011A SingleElectron PromptReco  &  /SingleElectron/Run2011A-PromptReco-v1/AOD   \\
\hline
\hline
\multicolumn{2}{|c|}{Fake Rate Measurement Samples} \\
\hline
Run2010A Jet  PromptReco            & /Jet/Run2011A-PromptReco-v1/AOD	\\
Run2010B Photon PromptReco          & /Photon/Run2011A-PromptReco-v1/AOD \\
\hline
\end{tabular}
\caption{Summary of data datasets used.\label{tab:DatasetsData}}
\end{center}
\end{table}

\begin{table}[!ht]
\begin{center}
{\footnotesize
\begin{tabular}{|c|c|c|}
\hline
\multicolumn{3}{|c|}{With Pileup: Processed dataset name is always} \\
\multicolumn{3}{|c|}{/Spring11-PU\_S1\_START311\_V1G1-v*/AODSIM} \\
\hline
 Dataset Description                     &   Primary Dataset Name   & cross-section (pb)\\
\hline
qq $\rightarrow WW$                  	 &   /VVJetsTo4L\_TuneD6T\_7TeV-madgraph-tauola                        &  43.0  \\
gg $\rightarrow WW \to 2l 2\nu$          &   /GluGluToWWTo4L\_TuneZ2\_7TeV-gg2ww-pythia6                       &   0.153\\
$\ttbar$                              	 &   /TTJets\_TuneZ2\_7TeV-madgraph-tauola                             & 157.5 \\
$\singletops$                  	 	 &   /TToBLNu\_TuneZ2\_s-channel\_7TeV-madgraph                        &  1.4 \\
$\singletopt$                  	 	 &   /TToBLNu\_TuneZ2\_t-channel\_7TeV-madgraph                        &  20.9 \\
tW                                    	 &   /TToBLNu\_TuneZ2\_tW-channel\_7TeV-madgraph                       &  10.6 \\
Z[20-inf] $\rightarrow ee$	  	 &   /DYToEE\_M-20\_CT10\_TuneZ2\_7TeV-powheg-pythia                   &  1666.0 \\
Z[20-inf] $\rightarrow \mu\mu$        	 &   /DYToMuMu\_M-20\_CT10\_TuneZ2\_7TeV-powheg-pythia                 &  1666.0 \\	       
Z[20-inf] $\rightarrow \tau\tau$  	 &   /DYToTauTau\_M-20\_CT10\_TuneZ2\_7TeV-powheg-pythia-tauola        &  1666.0 \\
Z[10-20]  $\rightarrow ee$	  	 &   /DYToEE\_M-10To20\_CT10\_TuneZ2\_7TeV-powheg-pythia               &  3892.9 \\
Z[10-20]  $\rightarrow \mu\mu$    	 &   /DYToMuMu\_M-10To20\_CT10\_TuneZ2\_7TeV-powheg-pythia             &  3892.9 \\
Z[10-20]  $\rightarrow \tau\tau$  	 &   /DYToTauTau\_M-10To20\_CT10\_TuneZ2\_7TeV-powheg-pythia-tauola    &  3892.9 \\
W/Z+$\gamma$                       	 &   /PhotonVJets\_7TeV-madgraph                                       &  165.0 \\
W $\rightarrow$ $\ell\nu$           	 &   /WJetsToLNu\_TuneZ2\_7TeV-madgraph-tauola                         &  31314.0 \\
WZ                               	 &   /WZtoAnything\_TuneZ2\_7TeV-pythia6-tauola                        &  18.2 \\
ZZ                               	 &   /ZZtoAnything\_TuneZ2\_7TeV-pythia6-tauola                        &   5.9\\
$gg \to H \to WW \to 2\ell2\nu$          &   /GluGluToHToWWTo2L2Nu\_M-*\_7TeV-powheg-pythia6                   & vary \\
$gg \to H \to WW \to \ell\tau2\nu$       &   /GluGluToHToWWTo2L2Nu\_M-*\_7TeV-powheg-pythia6                   & vary \\
$gg \to H \to WW \to 2\tau2\nu$          &   /GluGluToHToWWTo2Tau2Nu\_M-*\_7TeV-powheg-pythia6                 & vary \\
$qqH,~H \to WW \to 2\ell2\nu$            &   /VBF\_HToWWTo2L2Nu\_M-*\_7TeV-powheg-pythia6                      & vary \\
$qqH,~ H \to WW \to \ell\tau2\nu$	 &   /VBF\_HToWWTo2Tau2Nu\_M-*\_7TeV-powheg-pythia6                    & vary \\
$qqH,~H \to WW \to 2\tau2\nu$	         &   /VBF\_HToWWToLNuTauNu\_M-*\_7TeV-powheg-pythia6                   & vary \\
$WH/ZH/\ttbar H,~H\to WW$                &   /WH\_ZH\_TTH\_HToWW\_M-*\_7TeV-pythia6                            & vary \\
\hline
\hline
\end{tabular}
}
\caption{Summary of Monte Carlo datasets used.\label{tab:DatasetsMC}}
\end{center}
\end{table}

Due to details in the implementation of the Powheg calculation, the
resulting Higgs $\pt$ spectrum for $gg \to H$ has a much harder
spectrum compared with the most precise spectrum calculated to NNLO
with resummation to NNLL order, as illustrated in
Figure~\ref{fig:h160ww_pthiggs}(a). Therefore, the proper procedure is
to apply an event-by-event rewighting to the Powheg simulated
events. For the time being we correct the $gg \to H \to \WW$ jet bin
efficiency computed from the Powheg Monte Carlo sample, by a scale
factor which is approximately identical for all Higgs masses. The
scale factors applied to each jet bin in the Powheg simulation are
shown in Figure~\ref{fig:h160ww_pthiggs}(b).

\begin{figure}[!htbp]
\begin{center}
   \subfigure[]{\includegraphics[width=0.49\textwidth]{figures/h160ww_pthiggs.pdf}}
   \subfigure[]{\includegraphics[width=0.49\textwidth]{figures/h160ww_njets_kfactor_ratio.pdf}} 
\caption{(a) Higgs transverse momentum spectrum as predicted by Powheg and the NNLO+NNLL calculation; (b) 
scale factors applied to each jet bin in the Powheg simulation.}
\label{fig:h160ww_pthiggs}
\end{center}
\end{figure}
