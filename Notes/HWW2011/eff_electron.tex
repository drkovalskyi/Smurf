
The electron selection efficiency can be factorised into two contributions,
the efficiency from the electron reconstruction and from the additional
analysis selections that are described in Section~\ref{sec:sel_electrons}.

The electron reconstruction efficiency is defined as the efficiency for a
supercluster to be matched to a reconstructed ECAL driven GSF electron.
The data to simulation scale factor was measured for the W and Z cross-section
analysis~\cite{VBTFCrossSectionNote}~\cite{ref:tagprobe_mit_w},
and found to be consistent with $1.0$ with a total uncertainty of
$1.3\%$ and $1.5\%$ for the barrel and endcap, respectively.

We thus measure the efficiency of our offline analysis selection 
with respect to a reconstructed ECAL driven GSF electron denominator. 
The resulting data to simulation scale factors are given in Table~\ref{tab:eff_ele_offline}.

\begin{table}[!ht]
\begin{center}
\begin{tabular}{c|c|c}
\hline
Measurement & Barrel ( $|\eta|<1.5$ )   & Endcap ( $|\eta|>1.5$ )  \\ 
\hline
$  10<p_T<  15$ & 0.87 $\pm$ 0.05  & 0.78 $\pm$ 0.08  \\ \hline 
$  15<p_T<  20$ & 0.94 $\pm$ 0.02  & 0.89 $\pm$ 0.04  \\ \hline 
$  20<p_T<  inf$ & 0.99 $\pm$ 0.00  & 0.95 $\pm$ 0.00  \\ \hline 
\end{tabular}
\caption{Offline selection scale factors for electrons.}
\label{tab:eff_ele_offline}
\end{center}
\end{table}

