
The electron selection efficiency can be factorised into two contributions,
the efficiency from the electron reconstruction and from the additional
analysis selections that are described in Section~\ref{sec:sel_electrons}.

The electron reconstruction efficiency is defined as the efficiency for a
supercluster to be matched to a reconstructed ECAL driven GSF electron.
The data to simulation scale factor was measured for the W and Z cross-section
analysis~\cite{VBTFCrossSectionNote}~\cite{ref:tagprobe_mit_w},
and found to be consistent with $1.0$ with a total uncertainty of
$1.3\%$ and $1.5\%$ for the barrel and endcap, respectively.

We thus measure the efficiency of our offline analysis selection 
with respect to a reconstructed ECAL driven GSF electron denominator. 
The resulting data to simulation scale factors are given in Table~\ref{tab:eff_ele_offline}.

\begin{table}[!ht]
\begin{center}
\begin{tabular}{c|c|c} 
\hline
              & Barrel ( $|\eta|<1.5$ )  & Endcap ( $|\eta|>1.5$ )  \\ 
\hline
$10<p_{T}<15$ & XXX$\pm$YYY & XXX$\pm$YYY                 \\ \hline
$15<p_{T}<20$ & XXX$\pm$YYY & XXX$\pm$YYY            \\ \hline
$p_T>20$   & XXX$\pm$YYY & XXX$\pm$YYY \\ \hline
\end{tabular}
\caption{Electron selection efficiency for the barrel and endcap.
\label{tab:eff_ele_offline}}
\end{center}
\end{table}
