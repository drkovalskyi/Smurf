\subsection{Electron Fake Rate}
This section presents in more detail the studies on the electron fake rate measurement.

\subsubsection{Fakeable Object}
We perform the fake rate measurement using a number of different electron fakeable 
object definitions defined in Sec.~\ref{sec:fakerateDenominatorObjectDef}, 
each giving different performance in terms of systematic uncertainties. 
One of the two main systematic uncertainties associated with
the fake rate method is the uncertainty on the $p_{T}$
spectrum of the jet or parton from which the fake electron originates. 
Extrapolations in isolation, such as the V1 or V3 fakeable object, is susceptible
to these uncertainties. The V2 and V4 fakeable objects are already fully or 
partially isolated and therefore yield smaller systematic uncertainties due
to the jet $p_{T}$ spectrum. These options are intended to explore
the performance differences in the trade-off between jet $p_{T}$ spectrum systematic
uncertainties and statistical uncertainties in the fake rate extrapolation.


\subsubsection{Calibration Sample Selection}
\label{sec:ElectronFakeRate_CalibrationSampleSelection}
The electron fake rate is measured in a QCD-dominated event sample
selected by events triggered by {\bf Ele8\_CaloIdL\_CaloIsoVL } or 
{\bf Ele17\_CaloIdL\_CaloIsoVL }. We impose a few additional requirements in order
to reduce contamination due to events containing electrons from $W$ and $Z$ decays:

\begin{itemize}
  \item $Z$ veto: reject the event if there are more than one reconstructed electrons,
  \item $W$ veto: reject the event if PF-MET $> 20\:\GeV$.
\end{itemize}

In order to study the dependance of the fake rate on the $p_{T}$ of the jet from which
the fake electron originates, we impose requirements on the $p_{T}$ of the leading jet 
in the event. The jet is required to have $\Delta$ R $ > 1.0$ 
to the fake electron candidate. The jets used in this study have the L1FastJet, L2Relative, and
L3Absolute jet energy corrections applied. 

\subsubsection{Fake Rates}
The electron fake rates measured in the 2011A data requiring the leading jet $p_{T}$ to be 
larger than $35$ GeV for the V2 and V4 fakeable object definitions described above are shown 
in Figures \ref{fig:ele_fr_V2_jet35}, and \ref{fig:ele_fr_V4_jet35}, as a function of the 
$p_{T}$, $\eta$, and $\phi$ of the electron, comparing the result measured from data to the
results from the W+Jet Monte Carlo simulation and the cross section weighted combination 
of QCD Monte Carlo samples with Pt-hat $15-30$ and $30-50$.


\begin{figure}[!htbp]
\begin{center}
\includegraphics[width=0.45\textwidth]{figures/ElectronFakeRate_DenominatorV2_ptThreshold35_Pt.pdf}
\includegraphics[width=0.45\textwidth]{figures/ElectronFakeRate_DenominatorV2_ptThreshold35_Eta.pdf}
\includegraphics[width=0.45\textwidth]{figures/ElectronFakeRate_DenominatorV2_ptThreshold35_Phi.pdf}
\caption{Fake rates for V2 definition as a function of $p_T$,$\eta$, and $\phi$.}
\label{fig:ele_fr_V2_jet35}
\end{center}
\end{figure}


\begin{figure}[!htbp]
\begin{center}
\includegraphics[width=0.45\textwidth]{figures/ElectronFakeRate_DenominatorV4_ptThreshold35_Pt.pdf}
\includegraphics[width=0.45\textwidth]{figures/ElectronFakeRate_DenominatorV4_ptThreshold35_Eta.pdf}
\includegraphics[width=0.45\textwidth]{figures/ElectronFakeRate_DenominatorV4_ptThreshold35_Phi.pdf}
\caption{Fake rates for V4 definition as a function of $p_T$,$\eta$, and $\phi$.}
\label{fig:ele_fr_V4_jet35}
\end{center}
\end{figure}

Due to the fact that the Monte Carlo simulation does not have an implementation of the trigger used
to collect the fake electron candidates, it is expected that the Monte Carlo fake rate will differ
from the data fake rates. The V2 fake rate is expected to give better agreement between data and 
Monte Carlo simulation, since the denominator is already fully isolated. For $p_{T} > 20$ GeV, 
the fake rate from QCD Monte Carlo actually gives a reasonable description of the data. The difference
between the QCD Monte Carlo and the W+Jet Monte Carlo fake rate reflects the systematic uncertainties
due to different fake composition.

The fake rates used to extract the background prediction is summarized in Table \ref{tab:ele_fr_V4_jet35},
where we have used a combination of the {\bf Ele8\_CaloIdL\_CaloIsoVL}, {\bf Ele17\_CaloIdL\_CaloIsoVL}, 
{\bf Ele8\_CaloIdL\_CaloIsoVL\_Jet40} triggers, with a $p_{T}$ threshold on the leading jet in
the event of $35$ GeV. Figure \ref{fig:ele_fr_triggerDependence} shows that the fake rate measurement 
is insensitive to the particular trigger being used for this event selection, and therefore we combined 
them to minimize the statistical uncertainties of the fake rate measurement. The choice of the $p_{T}$
threshold on the leading jet is further elaborated in Section \ref{sec:FakeElectronBkgJetSpectrumSystematics}.



\begin{table}[!htbp]
\begin{center}
\begin{tabular}{|c|c|c|c|c|c|}

\hline
                         & $0<\eta<0.5$            & $0.5<\eta<1$            & $1<\eta<1.5$            & $1.5<\eta<2$            & $2<\eta<2.5$              \\
\hline
    $10 < p_{T} <= 15$ &        $0.129^{+0.023}_{-0.020}$ &        $0.109^{+0.019}_{-0.016}$ &        $0.068^{+0.013}_{-0.011}$ &        $0.024^{+0.011}_{-0.008}$ &        $0.050^{+0.013}_{-0.011}$  \\ 
 \hline
    $15 < p_{T} <= 20$ &        $0.079^{+0.019}_{-0.016}$ &        $0.079^{+0.016}_{-0.014}$ &        $0.055^{+0.012}_{-0.010}$ &        $0.036^{+0.012}_{-0.009}$ &        $0.056^{+0.013}_{-0.011}$  \\ 
 \hline
    $20 < p_{T} <= 25$ &        $0.088^{+0.018}_{-0.016}$ &        $0.118^{+0.020}_{-0.017}$ &        $0.113^{+0.017}_{-0.015}$ &        $0.104^{+0.016}_{-0.014}$ &        $0.100^{+0.015}_{-0.013}$  \\ 
 \hline
    $25 < p_{T} <= 30$ &        $0.070^{+0.020}_{-0.016}$ &        $0.132^{+0.024}_{-0.021}$ &        $0.092^{+0.017}_{-0.015}$ &        $0.069^{+0.016}_{-0.013}$ &        $0.092^{+0.016}_{-0.014}$  \\ 
 \hline
    $30 < p_{T} <= 35$ &        $0.126^{+0.030}_{-0.025}$ &        $0.083^{+0.024}_{-0.019}$ &        $0.123^{+0.024}_{-0.021}$ &        $0.070^{+0.019}_{-0.016}$ &        $0.118^{+0.021}_{-0.018}$  \\ 
 \hline




\end{tabular}
\caption{Electron fake rate in $\eta$-$p_T$ for the V4 fakeable object definition. Uncertainties are statistical only.
A combination of the {\bf Ele8\_CaloIdL\_CaloIsoVL}, {\bf Ele17\_CaloIdL\_CaloIsoVL}, 
{\bf Ele8\_CaloIdL\_CaloIsoVL\_Jet40} triggers are used, with a $p_{T}$ threshold on the leading jet in
the event of $35$ GeV. }
\label{tab:ele_fr_V4_jet35}
\end{center}
\end{table}

\begin{figure}[!htbp]
\begin{center}
\includegraphics[width=0.45\textwidth]{figures/ElectronFakeRate_DenominatorV4_TriggerDependence.pdf}
\caption{Comparison of the fake rates for the V4 fakeable object definition as a function of $p_T$
for a variety of trigger selections.}
\label{fig:ele_fr_triggerDependence}
\end{center}
\end{figure}


\subsubsection{Jet $p_{T}$ Spectrum Systematics}
\label{sec:FakeElectronBkgJetSpectrumSystematics}

The isolation efficiency for fake electrons are affected by any difference of the spectrum of the
jet that a fakeable object lies inside between the fake rate measurement sample and the 
application sample. To study this systematic uncertainty, we compare the $p_{T}$ spectrum 
of the jet that a particular fakeable object lies inside for the W+Jet Monte Carlo sample and the
data sample in which we measure the fake rates. This spectrum is representative of the $p_{T}$
spectrum of a particular parton which produces the fake electron after fragmentation and 
hadronization. Figure \ref{fig:ele_fr_jetspectrum20To50} shows this comparison between the W+Jet spectrum and the
spectrum obtained from the fake rate measurement sample in data for three different 
thresholds on the leading jet in the event. Ideally one wants to match the data spectrum to the
W+Jet spectrum to obtain an unbiased fake rate determination, however none of our data-based
samples matches completely. The sample obtained with a threshold on the leading jet of
$35$ GeV matches the W+Jet spectrum the best, and it taken for the central value of the 
fake rate measurement. The systematic uncertainty is estimated by using fake rates from the sample 
defined by thresholds of $20$ and $50$ GeV, since they are observed to cover the 
correct W+Jet spectrum. Figure \ref{fig:ele_fr_jetspectrumNarrowerWindow} shows analogous 
comparisons of the jet spectrum with narrower windows on the leading jet $p_{T}$ threshold,
providing some indication on the possibility for more aggressive systematic uncertainty 
estimates. Table \ref{tab:ele_fr_JetSpectrumSystematics} summarizes the systematic
uncertainties due to the jet spectrum. From these results, we estimate a systematic uncertainty 
of $28\%$, taking the largest difference between the $20$GeV threshold result and the $50$ GeV 
threshold result after the WW selection.

\begin{figure}[!htbp]
\begin{center}
\includegraphics[width=0.75\textwidth]{figures/LeptonJetPt_ElectronV4_20To50.pdf}
\caption{The $p_{T}$ spectrum of the jet that an electron fakeable object lies inside for the 
W+Jet Monte Carlo sample and the data fake rate measurement sample, requiring a threshold on the $p_{T}$ of 
the leading jet in the event between $20$ and $50$ GeV.}
\label{fig:ele_fr_jetspectrum20To50}
\end{center}
\end{figure}

\begin{figure}[!htbp]
\begin{center}
\subfigure[$p_{T}$ Threshold : $25-45$ GeV]{\includegraphics[width=0.45\textwidth]{figures/LeptonJetPt_ElectronV4_25To45.pdf}}
\subfigure[$p_{T}$ Threshold : $30-40$ GeV]{\includegraphics[width=0.45\textwidth]{figures/LeptonJetPt_ElectronV4_30To40.pdf}}
\caption{The $p_{T}$ spectrum of the jet that an electron fakeable object lies inside for the 
W+Jet Monte Carlo sample and the data fake rate measurement sample, for a narrower window of thresholds
on the $p_{T}$ of the leading jet in the event.}
\label{fig:ele_fr_jetspectrumNarrowerWindow}
\end{center}
\end{figure}


\begin{table}[!htbp]
\begin{center}
\begin{tabular}{|l|c|c|}
\hline
                        & \multicolumn{2}{|c|}{ $\%$ Change in Bkg Estimate} \\
\hline
Jet Threshold           & After WW Selection  & After HWW130 Selection \\
\hline
%Jet15                  &  $+31\%$     & $+31\%$    \\
Jet20                   &  $+28\%$     & $+33\%$    \\
Jet25                   &  $+20\%$     & $+23\%$    \\
Jet30                   &  $+6\%$      & $+1\%$     \\
Jet35 (Central Value)   &  $0\%$       & $0\%$      \\
Jet40                   &  $-2\%$      & $-2\%$     \\
Jet45                   &  $-15\%$     & $-14\%$    \\
Jet50                   &  $-22\%$     & $-25\%$    \\
\hline

\hline
\end{tabular}
\caption{Relative change in the fake electron background estimate using different thresholds on the leading jet to compute the fake rate. }
\label{tab:ele_fr_JetSpectrumSystematics}
\end{center}
\end{table}



%% \subsubsection{Run dependence}

%% - fake rates can be affected by run and detector conditions.
%% - we demonstrate how much run dependence there is


 \subsubsection{Sample Extrapolation Systematics and Closure Test}
\label{sec:FakeElectronBkgClosureTest}
 To quantify the systematic uncertainties in extrapolating from the QCD dominated fake rate
calibration sample to the W+Jet dominated application sample, we perform a closure test using 
the Monte Carlo simulation. Fake rates measured from a cross-section weighted combination of 
the QCD Monte Carlo samples, generated with pt-hat $15-30$ GeV and $30-50$ GeV, are applied on
a W+Jet Monte Carlo sample. To measure the fake rate, the selection documented in Section
\ref{sec:ElectronFakeRate_CalibrationSampleSelection} is used as in data, with a $p_{T}$ 
threshold of $35$ GeV on the leading jet in the event. These predictions of yields and 
distributions are compared with the yields and distributions obtained from the 
simulation-based result after the WW selection. The sample is normalized to the 
total number of Monte Carlo events in the W+Jet sample, corresponding roughly to $3$ \ifb.
To ensure that we compare only the relevent components of the fake electron background,
we remove any events containing fake muons or W+$\gamma$ events with a generator level photon 
produced with $p_{T} > 10$ GeV. In the subsequent results, only the statistical uncertainty 
from the limited sample size is shown. The statistical uncertainty from the limited size of 
the fake rate measurement sample are not propagated. 

Figure \ref{fig:FakeElectronClosureTest_FakeElePt} shows the comparison of the $p_{T}$
distribution of the fake electron predicted by the fake rate method and the full simulation,
after the dilepton selection and the WW pre-selection. The difference that is observed 
reflects the systematic uncertainty of the fake rate extrapolation between the different
event samples. 

\begin{figure}[!htbp]
\begin{center}
\subfigure[After dilepton selection]{\includegraphics[width=0.45\textwidth]{figures/FakeElectronClosureTest_FakeElePt_PreSelection.pdf}}
\subfigure[After WW selection]{\includegraphics[width=0.45\textwidth]{figures/FakeElectronClosureTest_FakeElePt.pdf}}
\caption{The $p_{T}$ distribution for the fake electron, after dilepton selection and after
WW selection, is compared between the fake rate method prediction and the simulation prediction. }
\label{fig:FakeElectronClosureTest_FakeElePt}
\end{center}
\end{figure}

Table \ref{tab:FakeElectronClosureTest_Yields} summarizes the comparison between the
predicted yields and the yields from the Monte Carlo simulation, at various stages
of the Higgs selection. At all stages of the selection, the relative difference
between the predicted yield and the simulated yield is less than about $20\%$. 
At the final higgs selection level, the difference in the total yield is about
$30\%$, but suffers from statistical uncertainties due to the limited Monte Carlo 
sample size. 

\begin{table}[!htbp]
\begin{center}
\begin{tabular}{|l|c|c|c|}
\hline
\multicolumn{4}{|c|}{After dilepton selection} \\
\hline
Final State     & Predicted Yield   & Simulation Yield & Fractional Difference\\
\hline
ee              &  $188 \pm 6$      & $203 \pm 14$     & $-7\%$\\
e $\mu$         &  $250 \pm 8$      & $203 \pm 14$     & $+23\%$\\
total           &  $437 \pm 10$     & $406 \pm 20$     & $+8\%$\\
\hline
\multicolumn{4}{|c|}{After dilepton selection and $\met > 20$ GeV, $M_{\mathrm{ll}} > 12$ GeV cuts} \\
\hline
Final State     & Predicted Yield   & Simulation Yield & Fractional Difference\\
\hline
ee              &  $147 \pm 6$      & $164 \pm 13$     & $-10\%$\\
e $\mu$         &  $202 \pm 7$     & $169 \pm 13$     & $+20\%$\\
total           &  $344 \pm 9$     & $333 \pm 18$     & $+3\%$\\
\hline
\multicolumn{4}{|c|}{After WW selection} \\
\hline
Final State     & Predicted Yield   & Simulation Yield & Fractional Difference\\
\hline
ee              &  $40.5  \pm 2.9$  & $33  \pm  5.7$   & $+23\%$  \\
e $\mu$         &  $119.9 \pm 5.3$  & $105 \pm 10.2$   & $+14\%$  \\
total           &  $160.4 \pm 6.1$  & $138 \pm 11.7$   & $+16\%$  \\
\hline
\multicolumn{4}{|c|}{After HWW 130 selection} \\
\hline
Final State     & Predicted Yield   & Simulation Yield & Fractional Difference\\
\hline
ee              &  $4.8  \pm 0.9$   & $8  \pm 2.8$     & $-40\%$\\
e $\mu$         &  $16.4  \pm 1.9$  & $11 \pm 3.3$     & $+50\%$\\
total           &  $21.2 \pm 2.1$   & $19 \pm 4.4$     & $+12\%$\\
\hline
\end{tabular}
\caption{Comparison of fake electron background yields at various stages of the 
analysis selection between the fake rate method prediction and the simulation
prediction. }
\label{tab:FakeElectronClosureTest_Yields}
\end{center}
\end{table}

Figure \ref{fig:FakeElectronClosureTest_LeptonPt} shows the comparison of the $p_{T}$
of the leading and trailing leptons after the WW selection between the prediction
from the fake rate method and from the simulation. Analogous comparisons for 
the missing transverse energy, the $\Delta\phi$ between the two leptons, 
the dilepton mass, and the transverse mass of the Higgs are shown in 
Figures \ref{fig:FakeElectronClosureTest_MetAndDeltaPhi} and 
\ref{fig:FakeElectronClosureTest_MetAndDeltaPhi}. While there are statistically 
significant systematic differences in the leading lepton $p_{T}$ and dilepton
mass distributions for example, there are no major systematic differences 
observed in any of these distributions beyond the $20-30\%$ level. Finally, 
Figure \ref{fig:FakeElectronClosureTest_CutFlow} shows the comparison of the 
cut flow for the $ee$ and $e\mu$ final states between
the prediction from the fake rate method and the simulation. Again, no major 
discrepancies beyond the $20-30\%$ level are observed. 


\begin{figure}[!htbp]
\begin{center}
\subfigure[Leading lepton $p_{T}$]{\includegraphics[width=0.45\textwidth]{figures/FakeElectronClosureTest_PtMax.pdf}}
\subfigure[Trailing lepton $p_{T}$]{\includegraphics[width=0.45\textwidth]{figures/FakeElectronClosureTest_PtMin.pdf}}
\caption{A comparison of the $p_{T}$ distribution of the leading and trailing lepton
between the fake rate method prediction and the simulation prediction. }
\label{fig:FakeElectronClosureTest_LeptonPt}
\end{center}
\end{figure}

\begin{figure}[!htbp]
\begin{center}
\subfigure[$\met$]{\includegraphics[width=0.45\textwidth]{figures/FakeElectronClosureTest_Met.pdf}}
\subfigure[$\Delta\phi$]{\includegraphics[width=0.45\textwidth]{figures/FakeElectronClosureTest_DeltaPhi.pdf}}
\caption{A comparison of the distribution of the missing transverse energy, 
and the $\Delta\phi$ between the two leptons predicted using the fake rate method and the
full simulation.}
\label{fig:FakeElectronClosureTest_MetAndDeltaPhi}
\end{center}
\end{figure}

\begin{figure}[!htbp]
\begin{center}
\subfigure[Dilepton Mass]{\includegraphics[width=0.45\textwidth]{figures/FakeElectronClosureTest_DileptonMass.pdf}}
\subfigure[Higgs $M_{T}$]{\includegraphics[width=0.45\textwidth]{figures/FakeElectronClosureTest_MtHiggs.pdf}}
\caption{A comparison of the distribution of the dilepton mass and Higgs transverse mass 
predicted using the fake rate method and the full simulation.}
\label{fig:FakeElectronClosureTest_DileptonMassAndMtHiggs}
\end{center}
\end{figure}


\begin{figure}[!htbp]
\begin{center}
\subfigure[ee]{\includegraphics[width=0.45\textwidth]{figures/FakeElectronClosureTest_CutFlowEE.pdf}}
\subfigure[e$\mu$]{\includegraphics[width=0.45\textwidth]{figures/FakeElectronClosureTest_CutFlowEMu.pdf}}
\caption{A comparison of the cut flow of the W+Jet background for the ee and e$\mu$ final states
between the fake rate method prediction and the full simulation prediction. }
\label{fig:FakeElectronClosureTest_CutFlow}
\end{center}
\end{figure}

To summarize, ignoring the yield comparison after the final Higgs selection cuts which 
suffer from large statistical uncertainties, we estimate a systematic uncertainty of 
$23\%$ for the fake electron background prediction due to the sample extrapolation 
coming from the largest differences observed in the yields of the closure test from 
Table \ref{tab:FakeElectronClosureTest_Yields}. 

%% \subsubsection{Data prediction}

%% - describe how we propagate uncertainties
%%   - assume statistical uncertainties are uncorrelated per bin
%%   - assume systematics are 100\% correlated
