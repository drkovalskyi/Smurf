Muons in CMS are reconstructed as either $StandAloneMuons$ (track
in the muon detector with low momentum resolution), $GlobalMuons$
(outside-in approach seeded by a $StandAloneMuon$ with a global fit
using hits in the muon, silicon strip and pixel 
detectors) and $TrackerMuons$ (inside-out approach seeded by an offline 
silicon strip track, using the muon detector only for muon identification 
without refitting the track). Most good quality muons are reconstructed as 
all three types at the same time and resolution is dominated by the inner
tracker system up to about 200~GeV in transverse momentum. Details about the
optimization at low $\pt$ are given in Appendix~\ref{app:mus}. The specific
requirements to select good prompt isolated muons are the following:
\begin{itemize}
\item the muon must be found by both the global and tracker muon algorithms;
\item the global muon must have at least one good muon hit;
\item the tracker muon must have at least two matches to muon segments in 
      different muon stations;
\item more than 10 hits in the inner tracker;
\item at least one pixel hit;
\item $\chi^2/{\mathrm{ndof}} < 10$ on a global fit;
\item impact parameter in the transverse plane $|d_{0}| < 0.02~(0.01)$~cm for
      muons with $\pt$ greater (smaller) than 20 $\GeVc$,
      calculated with respect to the primary vertex;
\item longitudinal impact parameter $|d_{z}| <0.2$~cm,
      calculated with respect to the primary vertex;
\item $|\eta|$ angle must be smaller than 2.4;
\item relative \pt\ resolution is better than 10\%.
\end{itemize}

Furthermore, tracker- and calorimeter- based isolation variables are
used to reduce the contamination from fake muons originating from
jets. We have chosen to use a combination of tracker, ECAL and HCAL
based isolation:

\begin{itemize}
\item $\rm{Iso}_{Track}$: it is defined as the scalar sum of the \pt\ of the 
    tracks in the $\eta \times \phi$ plane in a 0.01$~<~\Delta R~<~$0.3 cone 
    around the muon;

\item $\rm{Iso}_{ECAL}(\rm{Iso}_{HCAL})$: they are defined as the 
    scalar sum of the transverse energy of the calorimeter ECAL (HCAL) towers 
    in a $\Delta R$ cone of 0.3 centered on the muon with a veto cone of 
    $\Delta R = 0.01 $ around it.
\end{itemize}

The combined isolation variable is defined as 
$\rm{Iso}_{Total} = \rm{Iso}_{Track}+\rm{Iso}_{ECAL}+\rm{Iso}_{HCAL}$, 
and we require $\frac{\rm{Iso}_{Total}}{\pt}~<~0.15~(0.10)$ for muons 
with $\pt$ greater (smaller) than 20 $\GeVc$.

Due to the presence of pileup, which introduces additional energy into 
the isolation cone, the efficiency of the isolation cut is expected
to decrease with increasing amount of pileup. The average efficiency loss for signal 
muons with $p_{T} > 20$ GeV from a Higgs Monte Carlo simulation sample with mass of 
$130$ GeV is found to be approximately $1\%$. One specific procedure to correct for 
pileup in the isolation requirement, the fastjet correction method, has been attempted 
and found to over-correct the background, thus giving comparatively worse performance. 
Therefore, we currently do not perform any pileup corrections, and continue to 
investigate other options. A more detailed discussion can be found in 
Appendix \ref{app:PUIso}.

