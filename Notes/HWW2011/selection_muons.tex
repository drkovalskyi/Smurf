Muons in CMS are reconstructed as either $StandAloneMuons$ (track
in the muon detector with low momentum resolution), $GlobalMuons$
(outside-in approach seeded by a $StandAloneMuon$ with a global fit
using hits in the muon, silicon strip and pixel 
detectors) and $TrackerMuons$ (inside-out approach seeded by an offline 
silicon strip track, using the muon detector only for muon identification 
without refitting the track). Most good quality muons are reconstructed as 
all three types at the same time and the momentum resolution is dominated by the inner
tracker system up to about 200~$\GeVc$ in transverse momentum. Details about the
optimization at low $\pt$ are given in Appendix~\ref{app:mus}. The specific
requirements to select good prompt isolated muons are the following:
\begin{itemize}
\item the muon must be found by both the global and tracker muon algorithms;
\item the global muon must have at least one good muon hit;
\item the tracker muon must have at least two matches to muon segments in 
      different muon stations;
\item more than 10 hits in the inner tracker;
\item at least one pixel hit;
\item $\chi^2/{\mathrm{ndof}} < 10$ on a global fit;
\item impact parameter in the transverse plane $|d_{0}| < 0.02~(0.01)$~cm for
      muons with $\pt$ greater (smaller) than 20 $\GeVc$,
      calculated with respect to the primary vertex;
\item longitudinal impact parameter $|d_{z}| <0.1$~cm,
      calculated with respect to the primary vertex;
\item pseudorapidity $|\eta|$ must be smaller than 2.4;
\item relative \pt\ resolution is better than 10\%.
\end{itemize}

Furthermore, the particle flow candidate-based isolation variable is 
used to reduce the contamination from the non-isolated muons originating from
jets. 

\begin{itemize}
\item $\rm{Iso}_{PF}$: defined as the scalar sum of the \pt\ of the 
    particle flow candidates satisfying the following requirements:
    \begin{itemize}
    \item $\Delta R~<~0.3$ to the muon in the $\eta \times \phi$ plane,
    \item $|d_{z}(\mathrm{PF Candidate}) - d_{z}(\mathrm{muon})| < 0.1$~cm, if the PF candidate is charged,
    \item \pt $>1.0$ GeV, if the PF candidate is classified as a neutral hadron or a photon.
    \end{itemize}
\end{itemize}

We require $\frac{\rm{Iso}_{PF}}{\pt}~<~0.13~(0.06)$ for muons in the barrel 
with $\pt$ greater (smaller) than 20 $\GeVc$. For muons in the endcap, we
require $\frac{\rm{Iso}_{PF}}{\pt}~<~0.09~(0.05)$ for muons with $\pt$ 
greater (smaller) than 20 $\GeVc$. Further details of the choice of
the isolation requirement is documented in Appendix \ref{app:pfIsoStudy}.

