It is worth remarking that an alternative training, based on DY events under the Z peak, has been tested. 
Given the approximate description of the fake missing energy in the simulation, we wanted to test whether using real data 
for training could increase the rejection power against DY events.

MVA training on data is a non trivial problem because of background events contaminating the Z peak sample. 
These backgrounds (Top, \W\W\, Wjets, \W\Z, \Z\Z) - negligible in the bulk of the distributions - dominate the tail of \met-related variables 
(top plots in Figures~\ref{fig:pmet}-\ref{fig:mt}) and the MVA should not be trained against non-DY events. 
The solution we found to this problem is to give negative weight to the background events (red and blue area in the plots) and to use a MVA method that can 
properly deal with them in the training (BDTB)~\cite{TMVA2007}. 
The final result, however, is not better than what achieved with the MC training, with a sensitivity improvement in the same range and a slightly worse signal efficiency 
for same DY rejection.

This test has been presented in~\cite{ref:dymvadata}. 
