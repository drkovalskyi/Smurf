In 2011 \hww\ analysis~\cite{ref:hwwpaper,ref:hwwsmurfs}, in order to cope with Drell-Yan (DY) background, the same-flavor (SF=ee+$\mu\mu$) final state adds 
on top of the opposite-flavor (OF=e$\mu$+$\mu$e) selection a few tight cuts on \met\ and kinematic variables.
%\begin{itemize}
%\item minimum of projected-pfMet and projected-trackMet greater than 37+0.5$\cdot$number of vertices \GeV;
%\item dilepton mass (\mll) greater than 20 \GeVcc;
%\item trailing lepton transverse momentum ($p_T$(l2)) greater than 15 \GeVc;
%\item azimuthal angle difference between the leading jet and the dilepton system less than 165 degrees (only for leading jet $p_T$$>$15 \GeVc);
%\item dilepton mass not within a 15 \GeVcc\ window centered on \Z\ boson mass.
%\end{itemize}
The effect of these cuts is translated into different yields in the final cards: for \mHi=120 \GeVcc\ analysis (Table~\ref{tab:cards2011}) the ratio of yields
between OF and SF is $\sim$2.5 for ggH, $\sim$4 for Wjets, $\sim$2.5 for Top and $\sim$2 for qqWW.
In summary, because of DY most of the signal in the SF final state is rejected; 
the dominant backgrounds are suppressed as well, but the cuts might be not optimal. 

This note describes a new analysis selection for SF \hww\ analysis based on a multi-variate approach; 
the goal is to recover as much signal as possible while keeping DY at the same level as in 2011 analysis.

In order to figure out how much sensitivity can be gained by optimizing the SF analysis, we made a simple test. 
All yields and systematic uncertainties in the data cards are kept constant except the ggH yield in the SF final state.
The resulting expected limit improvement (combining 0- and 1-jet, OF and SF) for different increase in signal for \mHi=120 \GeVcc\ is reported in Table~\ref{tab:toytest}:
to get 10\% improvement we need 1.5-1.6x more signal with backgrounds at the same level.

%%%%%
\begin{table}[!ht]
\begin{center}
\begin{tabular} {|c|cccccccccc|}
\hline
\multicolumn{11}{|c|}{Cut Based} \\
\hline
analysis         & qqH & ggH & qqWW & ggWW & VV & Top & Zjets & Wjets & Wgamma & Ztt \\
\hline
SF, 0-jet & 0.039 &  4.939 & 35.884 &  1.485  & 0.625 &  2.086 & 10.397 &  3.263 &  0.254 &  0.000 \\
OF, 0-jet & 0.110 & 11.522 & 59.164 &  3.110  & 1.185 &  4.021 &  0.118 & 11.146 &  5.592 &  0.110 \\
\hline
SF, 1-jet & 0.178 &  1.366 &  7.254 &  0.386  & 0.487 &  5.081 &  9.792 &  0.606 &  0.000 &  0.000 \\
OF, 1-jet & 0.465 &  4.243 & 17.049 &  0.952  & 1.567 & 11.337 &  0.111 &  6.724 &  1.059 &  0.238 \\
\hline
\hline
\multicolumn{11}{|c|}{Shape Based} \\
\hline
analysis         & qqH & ggH & qqWW & ggWW & VV & Top & Zjets & Wjets & Wgamma & Ztt \\
\hline
SF, 0-jet & 0.048 &  6.347 & 99.429  & 4.275  & 1.738 &  6.151 & 11.656 &  7.658 &  0.945 &  0.000 \\
OF, 0-jet & 0.138 & 14.375 & 141.440 &  6.930 & 2.964 &  9.558 &  0.609 & 26.398 &  7.770 &  0.664 \\
\hline
SF, 1-jet & 0.233 &  1.890 & 22.012  & 1.299  & 1.299 & 14.437 & 21.012 &  4.918 &  1.118 &  0.000 \\
OF, 1-jet & 0.634 &  5.532 & 44.800  & 2.226  & 3.538 & 27.554 &  0.637 & 15.612 &  1.690 &  1.949 \\
\hline
\end{tabular}
\caption{Yields for published 2011 analysis: \mHi=120 \GeVcc\ cards.}
\label{tab:cards2011}
\end{center}
\end{table}
%%%%%


%%%%%
\begin{table}[!ht]
\begin{center}
\begin{tabular} {|c|cc|}
\hline
Signal Increase & \multicolumn{2}{|c|}{Expected Limit (\mHi=120 \GeVcc)} \\
in SF analysis  & Cut Based & Shape Based \\
\hline
1.0x & -0\%  & -0\%  \\
1.2x & -4\%  & -3\%  \\
1.4x & -7\%  & -8\%  \\
1.6x & -11\% & -12\% \\
1.8x & -15\% & -16\% \\
2.0x & -20\% & -20\% \\
\hline
\end{tabular}
\caption{Test of sensitivity improvement by scaling the signal yield is the same flavor cards. 
Expected limits are considered combining 0- and 1-jet bins, OF and SF final states.}
\label{tab:toytest}
\end{center}
\end{table}
%%%%%

\clearpage
