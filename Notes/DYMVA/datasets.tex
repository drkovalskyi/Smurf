The datasets used for this analysis are summarized in 
Tables.~\ref{tab:DatasetsData} and~\ref{tab:DatasetsMC} for data and Monte 
Carlo (MC), respectively. The total integrated luminosity\footnote{The 2011 integrated luminosity has been 
corrected to 4.9~\ifb, but, for consistency with the published analysis, the old values is used.} is \intlumi. 
We used the official good run list~\cite{json}. For Monte Carlo simulation 
we use madgraph when possible, but different generators such as Pythia and Powheg are also used. 
For $gg \to \WW$ a dedicated generator is used. For \zz\ process we use Pythia, 
since MadGraph sample has a generator level cut on the dilepton invariant mass of 
50 \GeV. 

\begin{table}[!ht]
\begin{center}
\begin{tabular}{|c|c|}
\hline
 Dataset Description                   &   Dataset Name   \\
\hline
\hline
\multicolumn{2}{|c|}{$H \to \WW$ Signal Selection Samples} \\
\hline
Run2011A MuEl PromptReco            &  /MuEG/Run2011A-PromptReco-v*/AOD   \\
Run2011A DiMuon PromptReco          &  /DoubleMu/Run2011A-PromptReco-v*/AOD   \\
Run2011A SingleMuon PromptReco      &  /SingleMu/Run2011A-PromptReco-v*/AOD   \\
Run2011A DiElectron PromptReco      &  /DoubleElectron/Run2011A-PromptReco-v*/AOD   \\
Run2011B DiElectron PromptReco      &  /DoubleElectron/Run2011B-PromptReco-v1/AOD   \\
Run2011B DiMuon PromptReco          &  /DoubleMu/Run2011B-PromptReco-v1/AOD   \\
Run2011B MuEl PromptReco            &  /MuEG/Run2011B-PromptReco-v1/AOD   \\
Run2011B SingleMuon PromptReco      &  /SingleMu/Run2011B-PromptReco-v1/AOD   \\
Run2011B SingleElectron PromptReco  &  /SingleElectron/Run2011B-PromptReco-v1/AOD   \\
\hline
\hline
\multicolumn{2}{|c|}{Fake Rate Measurement Samples} \\
\hline
Run2010A Jet  PromptReco            & /Jet/Run2011A-PromptReco-v*/AOD	\\
Run2010B Jet  PromptReco            & /Jet/Run2011B-PromptReco-v*/AOD	\\
Run2011A Photon PromptReco          & /Photon/Run2011A-PromptReco-v4/AOD \\
Run2011B Photon PromptReco          & /Photon/Run2011B-PromptReco-v1/AOD \\
\hline
\end{tabular}
\caption{Summary of data datasets used.\label{tab:DatasetsData}}
\end{center}
\end{table}

\begin{table}[!ht]
\begin{center}
{\footnotesize
\begin{tabular}{|c|c|c|}
\hline
\multicolumn{3}{|c|}{With Pileup: Processed dataset name is always} \\
\multicolumn{3}{|c|}{Fall11-PU\_S6\_START42\_V14B-v*/AODSIM} \\
\hline
 Dataset Description                     &   Primary Dataset Name   & cross-section (pb)\\
\hline
qq $\rightarrow WW$                  	 &   /WWJetsTo2L2Nu\_TuneZ2\_7TeV-madgraph-tauola                      &  4.783 \\
gg $\rightarrow WW \to 2l 2\nu$          &   /GluGluToWWTo4L\_TuneZ2\_7TeV-gg2ww-pythia6                       &   0.153\\
$\ttbar$                              	 &   /TTTo2L2Nu2B\_7TeV-powheg-pythia6/                                &  157.5 \\
$t$ ($s$-chan)                 	 	 &   /T\_TuneZ2\_s-channel\_7TeV-powheg-tauola                         &  3.19 \\
$\bar{t}$ ($s$-chan)                 	 &   /Tbar\_TuneZ2\_s-channel\_7TeV-powheg-tauola                      &  1.44 \\
$t$ ($t$-chan)             	 	 &   /T\_TuneZ2\_t-channel\_7TeV-powheg-tauola                         &  41.92 \\
$\bar{t}$ ($t$-chan)                 	 &   /Tbar\_TuneZ2\_t-channel\_7TeV-powheg-tauola                      &  22.65 \\
$tW$                                     &   /T\_TuneZ2\_tW-channel-DR\_7TeV-powheg-tauola                     &  7.87 \\
Z[20-inf] $\rightarrow ee$	  	 &   /DYToEE\_M-20\_CT10\_TuneZ2\_7TeV-powheg-pythia                   &  1666.0 \\
Z[20-inf] $\rightarrow \mu\mu$        	 &   /DYToMuMu\_M-20\_CT10\_TuneZ2\_7TeV-powheg-pythia                 &  1666.0 \\	       
Z[20-inf] $\rightarrow \tau\tau$  	 &   /DYToTauTau\_M-20\_CT10\_TuneZ2\_7TeV-powheg-pythia-tauola        &  1666.0 \\
Z[10-20]  $\rightarrow ee$	  	 &   /DYToEE\_M-10To20\_CT10\_TuneZ2\_7TeV-powheg-pythia               &  3892.9 \\
Z[10-20]  $\rightarrow \mu\mu$    	 &   /DYToMuMu\_M-10To20\_CT10\_TuneZ2\_7TeV-powheg-pythia             &  3892.9 \\
Z[10-20]  $\rightarrow \tau\tau$  	 &   /DYToTauTau\_M-10To20\_CT10\_TuneZ2\_7TeV-powheg-pythia-tauola    &  3892.9 \\
Z[10-50]  $\rightarrow \Lep\Lep$         &   /DYJetsToLL\_M-10To50\_TuneZ2\_7TeV-madgraph                      & 13629 \\
Z[50-inf] $\rightarrow \Lep\Lep$         &   /DYJetsToLL\_TuneZ2\_M-50\_7TeV-madgraph-tauola                   & 3048.0 \\
W $\rightarrow$ $\ell\nu$           	 &   /WJetsToLNu\_TuneZ2\_7TeV-madgraph-tauola                         &  31314.0 \\
WZ                               	 &   /WZJetsTo3LNu\_TuneZ2\_7TeV-madgraph-tauola                       &  0.857 \\
ZZ                               	 &   /ZZ\_TuneZ2\_7TeV\_pythia6\_tauola/                               &  7.406 \\
$W\gamma^{*}\rightarrow l\mu\mu$         &   /WGstarToLNu2Mu\_TuneZ2\_7TeV-madgraph-tauola                     &  1.60 \\ 
$W\gamma^{*}\rightarrow lee$             &   /WGstarToLNu2E\_TuneZ2\_7TeV-madgraph-tauola                      &  5.55 \\ 
$gg \to H \to WW \to 2\ell2\nu$          &   /GluGluToHToWWTo2L2Nu\_M-*\_7TeV-powheg-pythia6                   & vary \\
$gg \to H \to WW \to \ell\tau2\nu$       &   /GluGluToHToWWTo2L2Nu\_M-*\_7TeV-powheg-pythia6                   & vary \\
$gg \to H \to WW \to 2\tau2\nu$          &   /GluGluToHToWWTo2Tau2Nu\_M-*\_7TeV-powheg-pythia6                 & vary \\
$qqH,~H \to WW \to 2\ell2\nu$            &   /VBF\_HToWWTo2L2Nu\_M-*\_7TeV-powheg-pythia6                      & vary \\
$qqH,~ H \to WW \to \ell\tau2\nu$	 &   /VBF\_HToWWTo2Tau2Nu\_M-*\_7TeV-powheg-pythia6                    & vary \\
$qqH,~H \to WW \to 2\tau2\nu$	         &   /VBF\_HToWWToLNuTauNu\_M-*\_7TeV-powheg-pythia6                   & vary \\
$WH/ZH/\ttbar H,~H\to WW$                &   /WH\_ZH\_TTH\_HToWW\_M-*\_7TeV-pythia6                            & vary \\
\hline
\hline
\end{tabular}
}
\caption{Summary of Monte Carlo datasets used.\label{tab:DatasetsMC}.}
\end{center}
\end{table}

Analysis objects and selections are defined in~\cite{ref:hwwsmurfs}. 
A short summary of the selection is:
\begin{enumerate}
    \item We select events that pass pre-defined lepton triggers.
    \item We then select those events with two oppositely charged 
      high $\pt$ isolated leptons ($ee$, $\mu\mu$, $e\mu$).
    \item We apply a common $\WW$ preselection, which requires in brief: 
      \begin{itemize}
      \item categorize events by the number of reconstructed jets;
      \item exactly two high $\pt$ isolated leptons;
      \item large transverse missing energy due to the neutrinos;
      \item no b-tagged jets.
      \end{itemize}
    \item Finally, we perform two \emph{Higgs mass dependent} event selections: one for cut based and one for shape based analysis. 
\end{enumerate}
The definion of the analysis objects relevant for the present study are the following.
The \met\ variable used in the analysis is: min(proj-pfMet,proj-trackMet),
where ``pfMet'' is the \met~reconstructed with the particle flow algorithm, ``trackMet'' is the \met~ constructed from charged particles consistent 
with originating from the primary vertex.
The ``proj-met'' variable (where met is either pfMet or trackMet) is defined as:
\begin{equation}
\text{proj-met} = 
\begin{cases} \met & \text{if $\Delta\phi_{min}>\frac{\pi}{2}$,}
\\
\met\sin(\Delta\phi_{min}) & \text{if $\Delta\phi_{min}<\frac{\pi}{2}$}
\end{cases}
\end{equation}
\begin{equation}
\text{with } \Delta\phi_{min} =  min(\Delta\phi(\ell_1,\met),\Delta\phi(\ell_2,\met))\\
\end{equation}
where $\Delta\phi(\ell_i,\met)$ is the angle between \met\ and lepton $i$ in the transverse plane;
the main purpose for using ``proj-met'' is to reduce the impact of lepton mismeasurement and, secondarily, to further suppress the contribution from \dytt.\\
Jets are reconstructed using calorimeter and tracker information using a particle flow 
algorithm. The anti-${\rm k_T}$ clustering algorithm with ${\rm R=0.5}$ is used. 
We apply the standard jet energy corrections to the reconstructed jets, where the L1 Fast Jet
corrections are included. To exclude electrons and muons from the jet sample, these 
jets are required to be separated from the selected leptons in $\Delta R$ 
by at least $\Delta R^{\mathrm{jet-lepton}}>0.3$.

\clearpage
