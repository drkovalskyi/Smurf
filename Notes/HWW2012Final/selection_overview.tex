The fully leptonic final state consists of two isolated leptons
and large missing energy from the two undetectable neutrinos.
This is the same final state as the non-resonant $\WW$ background.
The Higgs cross-section is several orders of magnitude lower than
the major reducible background processes: \ttbar{}, \wjets{} and Drell-Yan. 
We thus perform several steps to select and extract the Higgs boson signal from data:

\begin{enumerate}
    \item We select events that pass pre-defined lepton triggers.
    \item We then select those events with two oppositely charged 
    high $\pt$ isolated leptons ($ee$, $\mu\mu$, $e\mu$) requiring:
        \begin{itemize}    
            \item $\pt>20~\GeV$ for the leading lepton;
            \item $\pt>10~\GeV$ for the trailing lepton;
            \item standard identification and isolation requirements 
	    on both leptons.
        \end{itemize}    
      \item We apply a common $\WW$ preselection, which requires in brief: 
         \begin{itemize}
             \item categorize events by the number of reconstructed jets;
             \item exactly two high $\pt$ isolated leptons;
             \item large transverse missing energy due to the neutrinos.
          \end{itemize}
    \item Finally, we perform two \emph{Higgs mass dependent} event selections, one cut-based and one using a multivariate technique 
described in detail in Section~\ref{sec:signal_selection}. 
\end{enumerate}

The $\WW$ preselection steps are described in detail in \cite{hcp2012Note}.
We did not change any events selections at WW-level except for $\pt^{ll}$ cut.     
The $\pt^{ll}$ requirement is relaxed from $>45~\GeV$ to $>30~\GeV$ to incorperate 
the analysis with Spin-Parity hypotheses separtation \cite{spinNote}. 
