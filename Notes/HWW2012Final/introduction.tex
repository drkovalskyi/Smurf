The standard model (SM) of particle physics successfully describes the
majority of high-energy experimental data~\cite{pdg}. One of the key
remaining questions is the origin of the masses of $\W$~and
$\Z$~bosons.  In the SM in its simplest implementation it is
attributed to the spontaneous breaking of electroweak symmetry caused
by a new scalar field~\cite{Higgs1, Higgs2, Higgs3}. The existence of
the associated field quantum, the Higgs boson, has yet to be
experimentally confirmed.

The $\Hi\to\WW$ channel is particularly sensitive for Higgs boson
searches in the intermediate mass range
($120-200~\GeV$)~\cite{dittmar}. This document describes the search
for the Higgs boson in the $\Hi\To\WW \To \Lep\Lprime\Nu\Nubar$
channel, for Higgs boson masses in the range of $110-600~\GeV$ using the 
full 2011 (7~\TeV) and 2012 (8~\TeV) datasets.
    
The study is based on the analysis prepared for the HCP2012
conference ~\cite{HWWHCP2012}. With the amount of data that CMS
collected the analysis has enough sensitivity for an observation of
the Standard Model Higgs in this channel alone. Given that for
126~\GeV\ Higgs the signal is extracted from a sample with significant
number of sizable backgrounds, the focus of this update is to make
sure that the associated systematic effects are well controlled.

The main analysis strategy is to select events with two opposite
charged leptons, large missing energy and little jet activity. The two
leptons are required to be isolated electrons or muons with $\pt$
greater than 10~\GeV{}. A cut-based analysis optimized for 126~\GeV\
Higgs is systematics limited with about 10~\ifb{}. Main analysis
sensitivity comes from a two-dimentional binned maximum-likelihood fit
that uses two mass-like variables (\mll\ and \mt{}) to discrimate
different backgrounds and extract signal. The performance of the fit
depends on how well we know shape of various backgrounds and how well
we account for uncertainties on the shapes and various systematic
effects.

Here is a list of important developments with respect to the 2012 HCP
analysis \cite{HWWHCP2012}:
\begin{itemize}
\item 
Update the analysis to full 2012 dataset (19.2~\ifb).
\item 
Use 2D shape analysis for different flavor final states to re-analyze
7~\TeV\ dataset.
\item
Spin analysis using 7 and 8~\TeV\ data.
\item 
Extensive study of fit stubility using toys Monte Carlo study biasing
the distributions (shape and normalization) in multiple ways
\item
Wide use of control samples to validate all key background shapes
\begin{itemize}
  \item {\bf Top-tagged events}. Essentially background free top only
  control sample in 1-jet case and fairly pure top sample in 0-jet
  case that allows to validate top background shapes.  

  \item {\bf Same-sign dilepton events}. Control sample to validate
  shapes and yields for W$\gamma$, W$\gamma*$, Wjets, WZ
  backgrounds. $\mu\mu$ with relaxed selection are helpful to look at
  muon fakes and W$\gamma*$.
  
  \item {\bf Tri-lepton events}. Assuming that one of the three
  leptons is neutrino we can get a clean control sample with a similar
  production mechanism.

  \item {\bf W + photon events}. High statistics sample of events
  usable for W$\gamma$ and W$\gamma*$ background studies.
\end{itemize}
\item 
New alternative WW background NLO Monte Carlo samples to study the fit
sensitivity to all standard ways to model WW that are currently available.
\end{itemize}

This analysis note covers only 0 and 1 jet final states. A search for
the Higgs boson in VBF channel will be documented separately.
