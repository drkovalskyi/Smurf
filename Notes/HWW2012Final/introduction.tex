The standard model (SM) of particle physics successfully describes the
majority of high-energy experimental data~\cite{pdg}. One of the key
remaining questions is the origin of the masses of $\W$~and
$\Z$~bosons.  In the SM in its simplest implementation it is
attributed to the spontaneous breaking of electroweak symmetry caused
by a new scalar field~\cite{Higgs1, Higgs2, Higgs3}. The existence of
the associated field quantum, the Higgs boson, has yet to be
experimentally confirmed. 

Direct searches at the CERN $\text{e}^+{\text e}^-$ LEP collider set a limit on 
the Higgs boson mass $m_{\text H} > 114.4$ GeV at 95\% confidence 
level(CL)~\cite{lephiggs}. At Tevatron collider, the 
CDF and D0 experiments state the evidence for a new particle that 
decays to $b\bar{b}$ quark paris in the mass range 125-135 GeV 
with a significance of up to 3.3 standard deviations~\cite{tevhiggs}. 
The LHC experiments, ATLAS and CMS, report the discovery of a new 
boson at approximately 125 GeV with 5 or more standard deviations 
each~\cite{atlasdiscovery, cmsdiscovery}. 

The $\Hi\to\WW$ channel is particularly sensitive for Higgs boson
searches in the intermediate mass range
($120-200~\GeV$)~\cite{dittmar}. It contributes significantly to 
the discovery of this new boson near 125 GeV. 
Based on the full 7~\TeV dataset (4.9 $\ifb$) and the 12.1 $\ifb$ 
at 8~\TeV, we observed in this channel an excess of events 
above background which is consistent with the expectations 
from a Standard Model Higgs boson with mass 125 GeV, with 
a statistical significance of 3.1 standard deviations~\cite{hwwhcp2012pas}. 

Afte this groundbreaking discovery, the experimental determination of the properties 
of the newly discovered boson is currently the most crucial task in high energy physics. 
To confirm the SM Higgs mechanism it is crucial to measure the couplings of the 
new particle with the known SM particles such as the $WW$ channel. 
The precise measurement of the signal strength in the $WW$ channel therefore is of 
great importance for the future updates. 
In addtion the $WW$ channel can also be used to extract information about the 
spin of the new particle. Even though this channel is not 
fully reconstructed, kinematic information such as the 
transverse higgs mass, dilepton invariant mass and the opening angle between the 
two leptons can be used to test the consistency of data with beyond the standard model 
senario, such as the graviton like spin 2 resonance with minimal couplings to the 
standard model particles~\cite{hwwspinpaper}. 


This document describes the search
for the Higgs boson in the $\Hi\To\WW \To \Lep\Lprime\Nu\Nubar$
channel, for Higgs boson masses in the range of $110-600~\GeV$ using the 
full 2011 (\intlumiSevenTeV at 7\TeV) and 2012 (\intlumiEightTeV, at 8~\TeV) datasets. 
This update is based on the analysis prepared for the HCP2012
conference~\cite{hwwhcp2012pas}. The expected significance for 
the SM Higgs boson with a mass near 125 GeV is about $5\sigma$ in 
this channel alone. 
Given that for 125~\GeV\ Higgs the signal is extracted from a sample 
with significant number of sizable backgrounds, the focus of this update is to 
validate the key systematic effects in the 2D shape analysis.  

The main analysis strategy is to select events with two opposite
charged leptons, large missing energy and little jet activity. The two
leptons are required to be isolated electrons or muons with $\pt$
greater than 10~\GeV{}. A cut-based analysis optimized for 125~\GeV\
Higgs is systematics limited with about 10~\ifb{}. Main analysis
sensitivity comes from a two-dimentional binned maximum-likelihood fit
that uses two mass-like variables (\mll\ and \mt{}) to discrimate
different backgrounds and extract signal. The performance of the fit
depends on how well we know shape of various backgrounds and how well
we account for uncertainties on the shapes and various systematic
effects.

Here is an overview of important developments with respect to the 2012 HCP
analysis~\cite{hwwhcp2012pas}:
\begin{itemize}
\item 
Update the analysis to full 2012 dataset (\intlumiEightTeV), 
\item 
Use 2D shape analysis for different flavor final states to re-analyze
7~\TeV\ dataset consistently, 
\item
Add an analysis to distinguish between the SM Higgs resonance and the graviton like 
spin 2 resonance with minimal couplings to the $WW$ pairs (referred to as $2_m^+$ model), 
\item 
Loosen the selections on the minimum of the dilepton $pT$ and transverse Higgs mass in the 
different flavor shape analysis to improve the expected separation between the SM Higgs and 
the $2_m^+$ graviton, 
\item 
Study the fit bias and stablitiy of the 2D shape based fits using pseudo-experiments, 
\item
Validate key background normalization and shapes by performing the analysis fits on signal 
free background enriched control samples
\begin{itemize}
  \item {\bf Top-tagged events}. Essentially background free top only
  control sample in 1-jet case and fairly pure top sample in 0-jet
  case that allows to validate top background shapes.  

  \item {\bf Same-sign dilepton events}. Control sample to validate
  shapes and yields for W$\gamma$, W$\gamma*$, Wjets, WZ
  backgrounds. $\mu\mu$ with relaxed selection are helpful to look at
  muon fakes and W$\gamma*$.
  
  \item {\bf Tri-lepton events}. Assuming that one of the three
  leptons is neutrino we can get a clean control sample with a similar
  production mechanism.

  \item {\bf W + photon events}. High statistics sample of events
  usable for W$\gamma$ and W$\gamma*$ background studies.
\end{itemize}
\item 
Study impacts of the WW shape with alternative generators such as WW@NLO and Powheg. 

\end{itemize}

This analysis note covers only 0 and 1 jet final states. A search for
the Higgs boson in VBF channel will be documented separately.
