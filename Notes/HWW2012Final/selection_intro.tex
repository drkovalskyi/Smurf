To enhance the sensitivity to the Higgs boson signal, two different
approaches are performed. The first one is a cut-based approach where
further requirements on a few observables are applied, while the
second one makes use of 2D fit using \mll and \mt. Both of them cover a
large Higgs boson mass ($m_{\rm{H}}$) range, and each is separately
optimized for different $m_{\rm{H}}$ hypotheses. The first method is
the simplest approach with smaller systematic uncertainties. The
second one is more sensitive, since it uses more information, taking
into account the correlation between the two variables and the shape on the 2D plane.

All analyses are further split in the corresponding 0-jet and 1-jet bins 
and two lepton flavor combinations($ee/\M\M$ and $e\M$).

No changes have been introduced to the cut-based approach since HCP 2012 \cite{hcp2012Note}.
The 2D analysis introduced two changes in selections and binning of templates.
As discussed in section \ref{sec:selection}, the cut on $\pt^{ll}$ is relaxed to 30\GeV. 
\mt cut is relaxed from $>80\GeV$ to $60\GeV$ to increase selection efficiency for 
the spin-two hypothesis \cite{spinNote}. New binning is chosen to get 
enough granuarity in the signal enriched regions to distinguish between different hypotheses. 
\begin{itemize}
    \item \mt (14 bins)  : [60,70,80,90,100,110,120,140,160,180,200,220,240,260,280]
    \item \mll (10 bins) : [12,30,45,60,75,100,125,150,175,200].
\end{itemize}
