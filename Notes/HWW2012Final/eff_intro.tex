 
We used the tag and probe method on \dyll~events to provide an unbiased, high-purity, 
lepton sample with which to measure both online and offline selection efficiencies.
The details of the method used are the same as the HCP analysis described
in Ref. \cite{hcp2012Note}, so we report only the results, and differences
from this reference.

To reduce the background contamination in the failing probe sample,
we employ an N-1 approach when measuring the simulation-to-data scale factor ($\rm{SF}$)
for the electron and muon selection. In this analysis,
we include an additional systematic uncertainty to account for
the potential bias from this selection efficiency factorisation scheme, compared
to measuring the efficiency in a single step. 

We define $\rm{SF}$ to be the product
of the scale factors for the ID and isolation parts of the selection,
$\rm{SF_{ID}}=\varepsilon_{ID | Iso}^{data}/\varepsilon_{ID | Iso}^{mc}$
and $\rm{SF_{Iso}} = \varepsilon_{Iso | ID}^{data}/\varepsilon_{Iso | ID}^{mc}$.
We then compare the efficiency of the full selection measured in a single step
in simulation with the equivalent efficiency from the N-1 factorisation described above:
\hbox{$\delta_{\varepsilon} = |1 - \varepsilon_{Iso+ID}^{mc} / (\varepsilon_{ID | Iso}^{mc} \times \varepsilon_{Iso | ID}^{mc})|$}.
The systematic uncertainty on $\rm{SF}$ is then taken to be $\delta_{\rm{SF}} = |1-\rm{SF}| \times \delta_{\varepsilon}$.

The systematic uncertainties on the offline selection due to the 
reconstruction efficiency are unchanged from the previous
analysis at 2\% and 1.5\% for electrons and muons respectively.
The trigger efficiencies
are measured with respect to the full offline selection, and do not
require any additional factorisation systematics.

