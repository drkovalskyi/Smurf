We use a combination of data-driven methods and detailed Monte Carlo
simulation studies to estimate background contributions.  From data we
can estimate the following backgrounds: $\Wjets$, $\dyll$, top
, and \WW{} backgrounds. The background from the remaining processes 
are taken from simulation.

Background composition and yields depend on the final state and on
the Higgs boson mass hypothesis under study. In the 0-Jet final state, 
the non-resonant \WW{} background dominates, while \wjets\ background contribution 
becomes sizable in the low Higgs mass hypotheses. 
In the 1-jet final state, the largest background contribution comes from 
top decays, while the non-resonant \ww\ background contribution is the second largest. 

For the backgrounds that can be estimated from data, 
we perform a data-driven background estimate in the signal region 
if the expected background contribution is sizable. 
If the expected contribution in the signal region is limited by statistics, 
we first estimate the background contribution with the $WW$ preselection from data 
and then extrapolate this estimation to the signal region using MC. The particular
choice of which backgrounds are estimated in the first or second way depends on the
integrated luminosity of the data sample that we analyze.

In the background estimation, the following improvements have been made since HCP 2012 \cite{hcp2012Note}.
\begin{itemize}
    \item{\Wjets}
    \begin{itemize}
        \item Subtraction of EWK(W/Z) contamination in fake rate estimation (see Section \ref{sec:bkg_fakes}).
        \item Split to the cases where e is a fake and \M~is a fake 
    \end{itemize}
    \item{\wgamma~and \Wgstar}
    \begin{itemize}
        \item split \wgamma~and \Wgstar 
        \item \wgamma~: To enhance statistics in the template, we select lepton + photon events and apply 
                        conversion($\gamma \to e$) probability as a function of photon $\eta$. The conversion
                        probability is measured in MC.
        \item \Wgstar~: The k-factor is measured using the same technique 
	                described in \cite{hcp2012Note}. The measured k-factor 
			is 1.5 with 30\% of systematic uncertainty. In addition,
			a specific study has been performed for
			$\Wgstar(\to e^+e^-)$ events.
    \end{itemize}
\end{itemize}
