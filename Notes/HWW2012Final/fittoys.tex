One of the main components of the analysis is the 2D fit to the data in the signal region, 
as it extracts the signal yield and significance.
The accuracy on the signal extraction depends on the ability of the fit in determining 
and constraining the backgrounds.
In fact, we first estimate the contribution from the various background processes 
either from data control regions or from MC (see Sec.~\ref{sec:backgrounds}), 
but then we rely on the fit to determine the final yield values. 

{\bf FIXME: using HCP analysis}

We validate the fit performance in presence of a 125 \GeV\ SM Higgs using toy experiments 
under different initial conditions.
Goal of the study is to determine the fit resolution and to check that it does not lead to 
biases in the signal strenght measurement. The procedure is defined as follows.
\begin{itemize}
\item 1k toy experiments are thrown with a simple poissonian sampling of S+B central shape (statistical only sampling)
or including systematic variations, both in normalization and shape (statistical+systematic sampling). 
\item Toy experiments can be thrown from nominal or biassed inputs; biases can due either to normalizationor to shape. 
In case of nominal input, toys are sampled from the sum of central shapes normalized to the process yield used in the cards;
in case of normalization bias, one of the yields is varied with respect to the value in the cards; 
in case of shape bias, one of the alternative shapes is used in place of the central when sampling the toys.
\item For each toy a new data card is created, where the data yield and shape are obtained from the sampling.
In case of nominal analysis, all other components in the cards are unchanged; 
in some cases, one systematic value can be inflated to test the ability of the fit in determining 
the nuisance parameter without constraints.
\item Data cards are processed as in the real data analysis; in particular we look at post-fit normalizations 
and at the signal strenght for the signal+background fit.
\end{itemize}

Results are summarized in Table~\ref{tab:toy_summary}.
We first test the fit performance without any input bias. 
The signal is estimated without large bias ($\leq 6$\%) and the resolution is 38\% (51\%) in the 0-jet (1-jet) bin 
when toy are sampled with statistical variations only. Including systematic effects in the toys does not introduce
additional bias but worses the signal resolution by $\sim$25\%. All background processes are estimated without bias.
Then we tested the fit performance in case of problems with the main backgrounds:
\begin{itemize}
\item {\bf Wjets} has a large normalization uncertainty (36\%), about the same size as the total signal yield (Table~\ref{tab:yield_summary}). 
Its shape is well constrained but not very different from the signal one. 
Therefore, it is crucial to verify that the fit is stable with respect to Wjets variations.
First, we tested the ability of the fit to distinguish the signal and Wjets shapes by relaxing the Wjets normalization without any input 
bias to the toys: the signal resolution is slightly degraded but no bias is introduced. 
Then, we applied an input bias ($\pm$1$\sigma$ normalization and shape) to the toys and analyzed them with the nominal analysis. 
The normalization bias is a large effect and the fit is only partially able to absorb it into the Wjets component; 
the resulting bias on the signal is $<$20\% (resolution is unaffected).
\item {\bf qqWW} is the largest background; it is well constrained both in terms of normalization and shape, but small variations of this 
process have a large size with respect to the signal. 
We tested variations on normalization ($\pm$9\%, roughly equal to 1$\sigma$ in the 0-jet bin), 
on shape (using the alernative shape from MCatNLO as input to the toys) and on both. 
In all cases, the effect on the signal is small, with biases $\leq$7\%. 
As expected, when relaxing the qqWW normalization uncertainty, it is easier for the fit to account the observed bias to the qqWW process,
and consequently reduce the already small signal bias (by a factor of 2).
\item {\bf Top} is the main background in the 1-jet bin. 
We tested variations on normalization ($\pm$5\%, roughly equal to 1$\sigma$ in the 1-jet bin) and on shape 
(using the alernative shape from MadGraph as input to the toys) separately. No bias is introduced for the signal. 
It is interesting to note that, in case of normalization bias in the 1-jet bin, the fit prefers to account the observed bias 
into the qqWW component since it has similar shape but larger normalization error.
\end{itemize}

In summary, we find the fit to be very stable with respect to biases at the level of 1$\sigma$ both in terms of shape and 
normalization for the main backgrounds. The largest observed bias is 18\% when injecting a 36\% more Wjets than expected.

%%%%%%%%%%%%%%%%%%%%%%%%%%%%%% 
\begin{table}
\begin{center}
\begin{tabular}{c | c  | c c | c c | c c | c c}
\hline
          &      & \multicolumn{2}{c|}{ggH} & \multicolumn{2}{c|}{qqWW} & \multicolumn{2}{c|}{Top} & \multicolumn{2}{c}{Wjets} \\
N$_{jets}$ & Test & bias & $\sigma$ & bias & $\sigma$ & bias & $\sigma$ & bias & $\sigma$ \\
\hline
0 & default                             & 5  & 38 & 1 & 3 & 1 & 11 & 2  & 19 \\
0 & sampling: stat.+syst.               & 6  & 47 & 1 & 8 & 4 & 17 &-3  & 27 \\
\hline
0 & analysis: inflated Wjets error      & 1  & 45 &-1 & 4 & 1 & 11 & 6  & 36 \\
0 & input: Wjets +36\% bias             & 18 & 38 & 1 & 3 & 0 & 10 & 17 & 21 \\
0 & input: Wjets -36\% bias             &-11 & 35 & 2 & 3 & 1 & 10 &-14 & 13 \\
0 & input: Wjets from altern. shape     & 9  & 38 &-1 & 3 & 1 & 12 &-2  & 15 \\
\hline
0 & input: qqWW +9\% bias               & 6  & 40 & 7 & 3 & 4 & 11 & 4  & 18 \\
0 & in: qqWW +9\%; ana: infl. qqWW err. & 3  & 40 & 8 & 4 & 2 & 11 & 0  & 16 \\
0 & input: qqWW from altern. shape      & 2  & 37 & 0 & 3 & 1 & 10 &-2  & 16 \\
0 & input: qqWW +9\% bias and altern. shape & 7  & 40 & 7 & 3 & 1 & 11 & 4  & 15 \\
0 & input: qqWW -9\% bias and altern. shape & 0  & 36 &-7 & 3 &-3 & 11 &-7  & 15 \\
\hline
0 & input: Top +5\% bias                & 4  & 39 & 0 & 3 & 3 & 11 & 1  & 18 \\
0 & input: Top from altern. shape       & 1  & 35 &-1 & 3 & 2 & 11 & 0  & 18 \\
\hline
\hline
1 & default                             &-1  & 51 & 0 & 6  & 0 & 4 & 1  & 17 \\
1 & sampling: stat.+syst.               &-6  & 66 & 6 & 13 & 0 & 8 &-5  & 24 \\
\hline
1 & analysis: inflated Wjets error      &-0  & 55 & 0 & 7  & 0 & 4 & 0  & 31 \\
1 & input: Wjets +36\% bias             & 12 & 48 & 1 & 6  & 0 & 4 & 21 & 22 \\
1 & input: Wjets -36\% bias             &-15 & 49 &-3 & 6  &-1 & 4 &-16 & 15 \\
1 & input: Wjets from altern. shape     & 8  & 51 & 0 & 5  &-1 & 4 &-3  & 17 \\
\hline
1 & input: qqWW +9\% bias               & 4  & 50 & 4 & 6  & 0 & 4 & 5  & 18 \\
1 & in: qqWW +9\%; ana: infl. qqWW err. & 2  & 51 & 7 & 11 & 0 & 5 & 1  & 20 \\
1 & input: qqWW from altern. shape      &-2  & 49 & 0 & 6  & 1 & 4 & 0  & 18 \\
1 & input: qqWW +9\% bias and altern. shape & 3  & 51 & 5 & 6 & 0 & 4 & 4  & 19 \\
1 & input: qqWW -9\% bias and altern. shape &-5  & 49 &-4 & 5 &-2 & 4 &-4  & 17 \\
\hline
1 & input: Top +5\% bias                & 3  & 52 & 3 & 6  & 1 & 4 & 2  & 20 \\
1 & input: Top from altern. shape       & 2  & 50 &-1 & 6  & 0 & 4 &-1  & 18 \\
\hline
\hline
\end{tabular}
\caption{Summary of results from toy experiments. Values of bias as $\sigma$ are \%.
In the Test column, ``default'' stands for: sampling is statistical only, input is nominal, analysis is nominal; 
in all other cases, the difference with respect to default is stated.}
\label{tab:toy_summary}
\end{center}
\end{table}


%%%%%%%%%%%%%%%%%%%%%%%%%%%%%% 
\begin{table}
\begin{center}
\begin{tabular}{c | c c c c | c c c }
\hline
N$_{jets}$ & ggH & qqWW & Top & Wjets & ggH/qqWW & ggH/Top & ggH/Wjets \\
\hline
0 & 82.2 & 1632.0 & 216.1 & 198.2 & 5\% & 38\% & 41\% \\
1 & 38.5 &  553.6 & 709.5 & 157.6 & 7\% & 5\%  & 24\% \\
\hline
\end{tabular}
\caption{Summary of yields and relative fraction of the signal with respect to the main backgrounds.}
\label{tab:yield_summary}
\end{center}
\end{table}

