In this section we describe an alternate approach,
using a binned fit to the 2D distribution of two physical variables.
The analysis method is described in brief here,
and in detail detail in Ref.~\cite{2DNote},~\cite{hcp2012Note}.

This method uses two physical variables : \mll~and \mt.
They are defined as follows:

\begin{itemize}
\item the dilepton mass $\mll$;
\item transverse Higgs mass,
$\mt^{\ell\ell\met} = \sqrt{2\pt^{ll}\met(1-cos(\Delta\phi_{\ell\ell-\met}))}$ where
$\Delta\phi_{\ell\ell-\met}$ is the angle between dilepton
direction and \met in the transverse plane.
\end{itemize}

The 2D templates for signal and background processes are constructed 
with the WW preselection with relax dilepton transverse momemtun requirement, 
$\pt^{ll}>30\GeV$ for low mass hypotheses($\mHi<300~\GeV$). For the high mass hypotheses ($\mHi\ge300~\GeV$),
we apply $\ptlmax > 50~\GeV$ to increase S/B. The bin size is chosen to avoid empty 
bins in the background templates, given the the available simulation and control sample statistics.  
Table~\ref{tab:binning_range} summarizes the bin size and range of templates. 


\fixme \textcolor{red}{how about draw empty templetes with grid?} 
\vspace{25pt}
\begin{table}[!htb]
\centering
\begin{tabular}{c | c | c }
\hline \hline
     Variable  & $\mHi< 300~\GeV$  & $\mHi\ge 300~\GeV$    \\
	\hline \hline
	\mt       & [60,280] 10 bins  & [80,380] 10 bins      \\
	\mll      & [0,200] 8 bins    & [0,450] 8 bins        \\
	\hline
	\end{tabular}
	\label{tab:binning_range}
	\caption{Summary of template parameters. For the high-\mHi~templates, 
			 overflow to \mll=600~\GeV~and~\mt=600~\GeV~is included
		  	 in the content of the last bin.}
\end{table}

We have applied the method to the $e\mu$ final state in the 0 and 1 jet bins, 
which are the most sensitive channels.  

