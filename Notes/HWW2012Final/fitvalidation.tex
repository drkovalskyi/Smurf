In this section we summarize the tests on fit behavior using data. 
There are following control regions where we can do the tests.

\begin{itemize}
    \item{\textbf{Same-sign \M\M~events} \\}  
        The sign requirement on the leptons is flipped. To get more statistics, 
        events in $|\mll - \mZ| < 15 \GeV$ are includeid. Instead of DYMVA, $minMET>35\GeV$
        is applied because requiring same sign reduces DY contribution (i.e. charge flip rate
        of muons is very small). This contral region is dominated by W+jets(\M), \Wgstar, and WZ/ZZ.
    \item{\textbf{Top-enriched sample} \\} 
        The top-veto requirement is flipped. This control region is dominated by top, WW, and Wjets.
%    \item{\textbf{WW-enriche} : }
%        $\mll > 70~\GeV$ is applied. This control region is dominated by WW, top, and Wjets.
%    \item Tri-lepton : ...  
\end{itemize}

The rest of selections are identical to the 2D analysis at $\mHi=125\GeV$.
In each test we perform a Maximum Likelihood fit with signal strength floated 
and check how normalizations of signal and background change.
Since signal contains only opposite sign di-lepton pairs, we use these events as signal 
in the test on same sign \M\M~events. The full 8 TeV data($\mathcal{L}=\intlumiEightTeV$) is used for the study.

Here is the summary of tests. Details(tables and plots) can be found in the appendix \ref{sec:appendix_fitvalidataion}. 
\begin{itemize}
    \item{\textbf{Same-sign \M\M~events} (Table \ref{tab:fitval_norm_ssmm_0j}-\ref{tab:fitval_norm_ssmm_1j}) \\ }  
        Signal contribution is close to 0 after fit. This means fit is able to 
        get the correct normalization when signal does not exist. Of the background 
        processes, WZ/ZZ and \Wgstar are not shifted, but Wjets($\mu$) is decreased by 30-40\% by fit.   
        This indicates a possible over-esitmation of Wjets($\mu$) background.  
    \item{\textbf{Top-enriched sample} (Table \ref{tab:fitval_norm_top_0j}-\ref{tab:fitval_norm_top_1j}) \\ } 
        In 0jet bin signal changed by -200\%, but the size of signal with respect to the total background 
        is only 1 \%. Therefore, change in the signal component is not significant. 
        Shift in backgrounds is not significant except for top in the 1jet bin. 
        Given that 94\% of the total background is from top, fit adjusts top to match data.  
%    \item{\textbf{WW-enriche}(Table \ref{tab:fitval_norm_ww_0j}-\ref{tab:fitval_norm_ww_1j}) : } 
%    \item Tri-lepton : ...  
\end{itemize} 

The conclusion is that fit does not show any anomalous behavior
when it is performed on the two control samples. Fit is able to assign 
proper normalizations of processes based on data.   

