Triggering on Higgs boson decays in the dilepton final state increases 
in difficulty with increasing instantaenous luminosity.
Single lepton triggers can only be sustained with very tight identification and
isolation requirements and large transverse momentum thresholds.
This means that double lepton triggers are the only viable option to maintain
sensitivity to a low mass Higgs boson, where the leptons transverse momentum
can be small.

We designed a suite of signal and control triggers appropriate for this analysis.
These dilepton triggers have a high efficiency to collect Higgs boson events
and are sufficiently loose to collect control events to estimate
fake lepton backgrounds and selection efficiencies with adequate precision.
We describe the features and motivations for the analysis triggers in Section~\ref{sec:mainTriggers},
and additional triggers to collect control events in 
Section~\ref{sec:utilityTriggers}.

\subsubsection{Analysis Triggers}
\label{sec:mainTriggers}

The main dielectron triggers, described in Table~\ref{tab:triggers_ee}, 
require two HLT electron candidates with loose shower shape and calorimeter isolation requirements 
and a match to a Level-1 seed for both legs. 
To accomodate the offline selection of $E_{T}>20,10$~GeV for the leading and trailing
electrons, $E_{T}>17,8$~GeV is required at the HLT level.
Controlling the total trigger rate is most challenging in
the dielectron channel, due to large fake electron background rates.
Additional requirements must be added to the track-to-cluster matching
and track isolation to control the total trigger rate at 
instantaneous luminosities above $1\times10^{33}~cm^{-1}sec^{-1}$.
In order to suppress contributions from pileup, the distance between the z vertex positions 
of the two electrons is required to be less than 0.2 cm.

The identification and isolation requirements are described in Table~\ref{tab:HLTElectronCuts}.
Because the electron HLT uses simplified algorithms compared to the offline selections,
the variables used online and offline do not always correspond exactly.
A detailed comparison between the online and offline selection requirements is given in
Appendix~\ref{app:online_vs_offline}.

\begin{table}[!ht]
  \caption{Analysis triggers for the $ee$ final state. 
The identification and isolation requirements are described in Table~\ref{tab:HLTElectronCuts}.}
    \vspace{5pt}
   \label{tab:triggers_ee}
  \begin{center}
 {\small
  \begin{tabular} {l|l|l|c}
\hline
  Dataset & Trigger name & L1 seed & Description\\
  \hline \hline
  \multirow{2}{*}{DoubleElectron} & HLT\_Ele17\_CaloIdT\_CaloIsoVL\_TrkIdVL\_TrkIsoVL\_ 	&  L1\_DoubleEG\_13\_7  & $p_T>17,8~\GeVc$ \\
                                  & Ele8\_CaloIdT\_CaloIsoVL\_TrkIdVL\_TrkIsoVL\_v[15-17] 	&                  & \\ %190456-190738 %190762-191419 %191512-194533
  \hline
  \end{tabular}
}
  \end{center}
\end{table}

\begin{table}[!ht]
 \caption{Summary of requirements applied to electrons in the triggers used for this analysis.
The selection requirements are given for electrons in the barrel (endcap).
L=Loose, VL=Very loose, T=Tight, VT=Very Tight.}
    \vspace{5pt}
 \label{tab:HLTElectronCuts}
 \centering
 \begin{tabular}{l|c}
   \hline
   name                       &  criterion \\
   \hline \hline
   \multirow{2}{*}{CaloId\_T} & $\mathrm{H/E < 0.15 (0.10) }$ \\
                               & $\sigma_{\eta\eta}\mathrm{< 0.011\;(0.031)}$ \\
    \hline
   \multirow{2}{*}{CaloId\_VT} & $\mathrm{H/E < 0.05 (0.05) }$ \\
                               & $\sigma_{\eta\eta}\mathrm{< 0.011\;(0.031)}$  \\
    \hline \hline
    \multirow{2}{*}{TrkId\_VL} & $|\Delta\eta|\mathrm{< 0.01\; (0.01)}$ \\
                               & $|\Delta\phi|\mathrm{< 0.15\;(0.10)}$  \\
    \hline
    \multirow{2}{*}{TrkId\_T} & $|\Delta\eta|\mathrm{< 0.008\; (0.008)}$ \\
                              & $|\Delta\phi|\mathrm{< 0.07\;(0.05)}$ \\
    \hline \hline
    \multirow{2}{*}{CaloIso\_VL} & $\mathrm{ECalIso/E_T <0.2\;(0.2)}$ \\
                                 & $\mathrm{HCalIso/E_T <0.2\;(0.2)}$ \\
    \hline
    \multirow{2}{*}{CaloIso\_T} & $\mathrm{ECalIso/E_T <0.15\;(0.075)}$ \\
                                 & $\mathrm{HCalIso/E_T <0.15\;(0.075)}$ \\
    \hline
    \multirow{2}{*}{CaloIso\_VT} & $\mathrm{ECalIso/E_T <0.05\;(0.05)}$ \\
                                 & $\mathrm{HCalIso/E_T <0.05\;(0.05)}$ \\
    \hline \hline
    TrkIso\_VL                   & $\mathrm{TrkIso/E_T <0.2\;(0.2)}$ \\
    \hline
    TrkIso\_T                   & $\mathrm{TrkIso/E_T <0.15\;(0.075)}$ \\
    \hline
    TrkIso\_VT                   & $\mathrm{TrkIso/E_T <0.05\;(0.05)}$ \\
    \hline \hline
    \multirow{8}{*}{WP80} 		& $\mathrm{H/E < 0.10 (0.05) }$ \\
                               	& $\sigma_{\eta\eta}\mathrm{< 0.01\;(0.03)}$ \\
    							& $|\Delta\eta|\mathrm{< 0.007\; (0.007)}$ \\
                               	& $|\Delta\phi|\mathrm{< 0.06\;(0.03)}$  \\
                               	& $|\frac{1}{E}-\frac{1}{p}|\mathrm{< 0.05\;(0.05)}$  \\
    							& $\mathrm{ECalIso/E_T <0.15\;(0.10)}$ \\
                                & $\mathrm{HCalIso/E_T <0.10\;(0.10)}$ \\
                       			& $\mathrm{TrkIso/E_T <0.05\;(0.05)}$\\
    \hline
 \end{tabular}
\end{table}

The main dimuon triggers
require two HLT muon candidates with transverse momentum greater than $17,8$~$\GeVc$ and
a match to a Level-1 seed is required for both legs. 
These are described in Table~\ref{tab:triggers_mm}.
In addition, the longitudinal distance between the z vertex positions of the two muons 
is required to be less than 0.2 cm to suppress pileup events.
\begin{table}[!ht]
  \caption{Analysis triggers for the $\mu\mu$ final state. }
    \vspace{5pt}
   \label{tab:triggers_mm}
  \begin{center}
 {\small
  \begin{tabular} {l|l|l|c}
\hline
  Dataset & Trigger name & L1 seed & Description\\
  \hline \hline
  \multirow{2}{*}{DoubleMu}	& HLT\_Mu17\_Mu8\_v[16-17] 	& L1\_DoubleMu\_10\_Open  & $p_T>17,8~\GeVc$ \\ %190456-193686 %193806-194533
   							& HLT\_Mu17\_TkMu8\_v[9-10] 	& L1\_DoubleMu\_10\_Open  & $p_T>17,8~\GeVc$ \\ %190456-193686 %193806-194533
  \hline
  \end{tabular}
}
  \end{center}
\end{table}

In the electron-muon channel we use two complementary triggers, which require
both muon and electron HLT candidates.
These are summarised in Table~\ref{tab:triggers_em}.
Finally, to recover any residual inefficiency,
we also allow events that passed only the single electron
or single isolated muon triggers summarised in Table~\ref{tab:triggers_single}.

\begin{table}[!ht]
  \caption{Analysis triggers for the $e\mu$ final state.
The identification and isolation requirements for electrons are described in Table~\ref{tab:HLTElectronCuts}.}
    \vspace{5pt}
   \label{tab:triggers_em}
  \begin{center}
 {\small
  \begin{tabular} {l|l|l|c}
\hline
  Dataset & Trigger name & L1 seed & Description\\
  \hline \hline
  \multirow{2}{*}{MuEG} & HLT\_Mu17\_Ele8\_CaloIdT\_CaloIsoVL\_TrkIdVL\_TrkIsoVL\_v[4-7] 	& L1\_Mu12\_EG7 	& $p_T>17,8~\GeVc$ \\
  															%190456-190738 %190762-191419 %191512-193686 %193806-194533
  						& HLT\_Mu8\_Ele17\_CaloIdT\_CaloIsoVL\_TrkIdVL\_TrkIsoVL\_v[4-7] 	& L1\_MuOpen\_EG12 	& $p_T>8,17~\GeVc$ \\ 
  															%190456-190738 %190762-191419 %191512-193686 %193806-194533
 \hline
  \end{tabular}
}
  \end{center}
\end{table}

\begin{table}[!ht]
  \caption{Single lepton triggers to recover lost efficiency. These trigges are also used for efficiency studies.
The identification and isolation requirements for electrons are described in Table~\ref{tab:HLTElectronCuts}.}
    \vspace{5pt}
   \label{tab:triggers_single}
  \begin{center}
 {\small
  \begin{tabular} {l|l|l|c}
\hline
  Dataset & Trigger name & L1 seed & Description\\
  \hline \hline
  SingleEle & HLT\_Ele27\_WP80\_v[8-10] & L1\_SingleEG20 OR L1\_SingleEG22  & $p_T>27~\GeVc$ \\ %190456-190738 %190762-191419 %191512-194533
  \hline \hline
  SingleMu & HLT\_IsoMu24\_eta2p1\_v[11-13]   & L1\_SingleMu16er  & $p_T>24~\GeVc$ \\  %190456-190738 %190762-193686 %193806-194533
  \hline 
  \end{tabular}
}
  \end{center}
\end{table}

\subsubsection{Utility Triggers}
\label{sec:utilityTriggers}

We now describe additional triggers that are introduced to collect control or
calibration events not covered by the main analysis triggers.

Because the main dielectron analysis triggers make requirements on
both legs, events collected with those triggers cannot be used to measure
efficiencies without introducing unacceptable bias.
Thus, to measure the electron selection and trigger efficiency
we introduce three specialised tag and probe triggers designed to maximize
the number of useful \dyll~events for both low and high $p_{T}$ electrons,
while keeping the total trigger rate at a reasonable level. 
The tag and probe method is described later in Section~\ref{sec:efficiency}.

The first and second triggers probe low $p_T$ electrons and apply very tight identification 
and isolation requirements on the tag leg to reduce the background rate.
The third trigger probes higher $p_{T}$ electrons.
These triggers are described in Table~\ref{tab:triggers_util} and labeled ``eff".

\begin{table}[!ht]
  \caption{Utility triggers.
The identification and isolation requirements for electrons are described in Table~\ref{tab:HLTElectronCuts}.
%The identification and isolation requirements for photons are described in Table~\ref{tab:PhotonPlusLeptonTriggerCuts}.
Triggers labeled ``eff" are used for efficiency studies, ``FR" are used for fake rate studies,
and ``$\zeta$ method " are used for Drell-Yan background estimation.}
    \vspace{5pt}
   \label{tab:triggers_util}
  \begin{center}
 {\small
  \begin{tabular} {l|l|l|c}
\hline
  Dataset & Trigger name & L1 seed & Description\\
 \hline \hline 
  \multirow{6}{*}{DoubleElectron} 	
  	& HLT\_Ele17\_CaloIdVT\_CaloIsoVT\_ 		& L1\_DoubleEG\_13\_7  	& $p_T>17,8~\GeVc$, eff\\
    & TrkIdT\_TrkIsoVT\_Ele8\_Mass50\_v[3-5] 	&                 		& \\ %190456-190738 %190762-191419 %191512-194533
  	& HLT\_Ele20\_CaloIdVT\_CaloIsoVT\_			& L1\_SingleIsoEG18er	& $p_T>20,4~\GeVc$, eff\\
	& TrkIdT\_TrkIsoVT\_SC4\_Mass50\_v[3-5] 	& 						& \\ %190456-190738  %190762-191419 %191512-194533
   	& HLT\_Ele32\_CaloIdT\_CaloIsoT\_			& L1\_SingleEG22   		& $p_T>32,17~\GeVc$, eff\\
	& TrkIdT\_TrkIsoT\_SC17\_Mass50\_v[3-5]		& 						& \\ %190456-190738  %190762-191419  %191512-194533 
	\hline \hline
  \multirow{9}{*}{DoubleElectron} 	
	& HLT\_Ele8\_CaloIdT\_TrkIdVL\_v[2-4]		& L1\_SingleEG5 		& $p_T>8~\GeVc$, FR \\ %190456-190738 %190762-191419 %191512-194533
 	& HLT\_Ele8\_CaloIdT\_CaloIsoVL\_			& L1\_SingleEG7 		& $p_T>8~\GeVc$, FR \\ 
 	& TrkIdVL\_TrkIsoVL\_v[12-14]				&  						& \\ %190456-190738 %190762-191419 %191512-194533
	& HLT\_Ele17\_CaloIdT\_CaloIsoVL\_  		& L1\_SingleEG12		& $p_T>17~\GeVc$, FR \\ 
	& TrkIdVL\_TrkIsoVL\_v[3-5]  				& 						& \\ %190456-190738 %190762-191419 %191512-194533
   	& HLT\_Ele8\_CaloIdT\_CaloIsoVL\_      		& L1\_SingleEG7         & $p_T>8~\GeVc$, FR \\
   	& TrkIdVL\_TrkIsoVL\_Jet30\_v[3-5]      	& 						& \\ %190456-190738 %190762-191419 %191512-194533
 	& HLT\_Ele17\_CaloIdT\_CaloIsoVL\_			& L1\_SingleEG12		& $p_T>17~\GeVc$, FR \\ 
 	& TrkIdVL\_TrkIsoVL\_Jet30\_v[3-5]			& 						& \\ %190456-190738 %190762-191419 %191512-194533
	\hline \hline
  \multirow{2}{*}{DoubleMu}     
    & HLT\_Mu8\_v16 	&  L1\_SingleMu3  		& $p_T>8~\GeVc$, FR\\ %190456 - 194533
    & HLT\_Mu17\_v3    	&  L1\_SingleMu12      	& $p_T>17~\GeVc$, FR\\ %190456 - 194533   
	\hline \hline
  \multirow{5}{*}{DoubleElectron} 	
	& HLT\_Photon22\_R9Id90\_HE10\_Iso40\_EBOnly\_v[2-4]					& L1\_SingleEG22		& $\zeta$ method \\ %190456-190738 190762-191419 %191512-194731
	& HLT\_Photon36\_R9Id90\_HE10\_Iso40\_EBOnly\_v[2-4]					& L1\_SingleEG22		& $\zeta$ method \\ %190456-190738 190762-191419 %191512-194731
	& HLT\_Photon50\_R9Id90\_HE10\_Iso40\_EBOnly\_v[2-4]					& L1\_SingleEG22		& $\zeta$ method \\ %190456-190738 190762-191419 %191512-194731
	& HLT\_Photon75\_R9Id90\_HE10\_Iso40\_EBOnly\_v[2-4]					& L1\_SingleEG22		& $\zeta$ method \\ %190456-190738 190762-191419 %191512-194731
	& HLT\_Photon90\_R9Id90\_HE10\_Iso40\_EBOnly\_v[2-4]					& L1\_SingleEG22		& $\zeta$ method \\ %190456-190738 190762-191419 %191512-194731
    \hline 
  \end{tabular}
}
  \end{center}
\end{table}

Another set of specialised triggers are used to record events
enriched in fake electrons and muons for the measurement of jet induced backgrounds.
This is done using the fake rate method, which is described in detail in
Section~\ref{sec:bkg_fakes}.

We introduce five triggers for the electron and two for muon fake rate measurements,
described in Table~\ref{tab:triggers_util} and labeled ``FR".
For electrons, because these triggers are prescaled, the first three impose different $p_T$ thresholds 
to collect a sufficient sample over a large $p_T$ range.
The fourth and fifth triggers require an additional jet with corrected $E_{T}>30$~GeV
to perform systematic studies on the fake rate measurement.
For muon, we use two different $\pt$ thresholds to collect sufficient sample
over a large $\pt$ range because these triggers are prescaled. 
The five photon triggers are used to collect events populated by photons, 
which are used to estimate Drell-Yan background. 
%The photon selection requiriements are summarised in Table~\ref{tab:PhotonPlusLeptonTriggerCuts}.

%\begin{table}[htb]
% \caption{Summary of requirements applied in the photon triggers used for this analysis.}
%    \vspace{5pt}
%  \label{tab:PhotonPlusLeptonTriggerCuts}
%  \centering
%  \begin{tabular}{l|c}
%    \hline
%    name                        &  criterion \\
%    \hline \hline 
%    \multirow{1}{*}{R9Id90} 	& $\mathrm{R9 > 0.9 }$ \\
%    \hline 
%    \multirow{1}{*}{HE10} 		& $\mathrm{H/E < 0.1 }$ \\
%    \hline 
%    \multirow{3}{*}{Iso40}     	& $\mathrm{ECalIso} < 4.0 $ \\
%                                & $\mathrm{HCalIso} < 4.0 $ \\
%                                & $\mathrm{TrkIso}  < 4.0 $ \\
%    \hline 
%  \end{tabular}
%\end{table}

