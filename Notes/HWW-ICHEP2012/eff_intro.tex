 
We used the tag and probe method on \dyll~events to provide an unbiased, high-purity, 
lepton sample with which to measure both online and offline selection efficiencies.
This method, which is now described, 
has been used successfully in previous CMS analyses \cite{ref:tagprobe_mit_w}\cite{ref:tagprobe_snt_top}.

\subsubsection{Method}
For both electrons and muons we used the lowest threshold unprescaled single trigger sample
available in the Prompt Reco.  This corresponds to:

\begin{itemize}
    \item Muons: HLT\_IsoMu24\_eta2p1\_v*
    \item Electrons: HLT\_Ele27\_WP80\_v*
\end{itemize}

At least one of the leptons, the {\it tag}, was required to pass the full selection criteria
while the other lepton, the {\it probe}, was required to pass a set of identification criteria leaving 
it unbiased with respect to the criterion under study. By requiring that the tag was able to have passed 
the single lepton trigger on which the events were acquired, we reduced the bias due to the trigger on 
the probe. Also, the tight criteria imposed on the tag coupled with the invariant mass requirement 
improves the purity of the sample. 

To reduce the background to the offline selection measurements,
the selections were split into the ID and isolation parts.  
The efficiency of the ID part was measured with respect to the isolation
requirements, and vice versa, in both data and simulation.
This is referred to as the N-1 method.
The bias on the efficiency from changing the denominator is
expected to be negligible is the scale factors for each step used
are close to one.
To estimate and subtract any residual background contribution in the data measurements,
a simultaneous fit was performed to the mass distributions
of passing and failing probes in the range $60<M_{ll}<120$ GeV.
The signal model for electrons was taken from simulation, 
with a gaussian smearing component to take into account the resolution.
The background model is an exponential times an error function.
In the simulation measurements, simple counting was used.
This method and its associated systematics are discussed in detail in Reference \cite{ref:tagprobe_mit_w}
The trigger efficiency is measured with respect to the full offline selection,
and thus the probe sample is very pure.  In this case the efficiencies were
extracted by simple counting in the mass range $81<M_{ll}<101$ GeV.

To produce overall data-MC scale factors to apply in the analysis, we factorise the efficiency measurements
into two steps such that

\begin{equation}
\varepsilon_{total} = \varepsilon_{offline} \times \varepsilon_{trigger}.
\end{equation}

The offline efficiency $\varepsilon_{offline} = \varepsilon_{offline}^{l1} \times \varepsilon_{offline}^{l2}$
is the product of the efficiencies of the two leptons and is discussed in more detail in Sections \ref{sec:eff_electron}
and \ref{sec:eff_muon} for electrons and muons respectively.
The trigger efficiency is measured with respect to the offline selection and
is discussed in more detail in Section \ref{sec:eff_trigger}.

