Jet induced fake leptons are an important source of background for many 
physics channels. 
In this analysis the main sources of fake leptons are
$\Wjets$ and QCD events, where at least one of the jets or a
constituent is misidentified as an isolated lepton. 
The dominant background is $\Wjets$ because there is already one prompt, 
well isolated, lepton from the $W$ boson decay.
Fake non-prompt leptons arise from the leptonic decay
of heavy quarks, misidentified hadrons or electrons from 
photon conversion.

A data-driven approach, described in detail in~\cite{fakeLeptonNote1} 
and~\cite{fakeLeptonNote2}, is pursued to estimate this background. Only 
a summary of the method is described here, more details can be found 
in~\cite{HWW2011}. A set of loosely selected lepton-like objects, referred to as the 
``fakeable object'' or ``denominator'' from here on, is defined in a 
sample of events dominated by dijet production. 
The efficiency for these denominator objects to pass 
the full lepton selection critera is measured. 
This background efficiency, typically referred to as the ``fake rate'', 
is parameterized as a function of the $\pt$ and $\eta$ of the denominator 
object in order to capture any dependence on kinematic and geometric quantities. 
We will denote the fake rate symbollically by $\epsilon_{\mathrm{fake}}$.
These fake rates are, then, used as weights to extrapolate
the background yield from a sample of loose denominator objects to the sample
of fully selected leptons.

\subsubsection{Denominator Object Definitions}
\label{sec:fakerateDenominatorObjectDef}
The denominator object definition has significant impact on the
systematic uncertainty of the method, due to the fact that 
the sample dependence uncertainties for extrapolating in different 
isolation and lepton quality criteria are typically different. We consider 
the following definition:

\begin{itemize}
  \item $\sigma_{i\eta i\eta} < 0.01/0.03$ (barrel/endcap)
  \item $|\Delta\phi_{in}| < 0.15/0.10$
  \item $|\Delta\eta_{in}| < 0.007/0.009$
  \item $H/E< 0.12/0.10$
  \item full conversion rejection
  \item $|d_{0}| < 0.02$~cm
  \item $|d_z| < 0.1$~cm
  \item $\frac{\sum_{\rm trk}\Et}{\pt^{\rm ele}}<0.2$
  \item $\frac{\left[\sum_{\rm ECAL}\Et\right]-1}{\pt^{\rm ele}}<0.2$
  \item $\frac{\sum_{\rm HCAL}\Et}{\pt^{\rm ele}}<0.2$
\end{itemize}

The situation for muons is simpler. The loose muon selection requirements can differ from
the tight selection of Section~\ref{sec:sel_muons}, only in less stringent cuts on $d_0$
and isolation. We consider the following definition:
\begin{itemize}
 \item $|d_{0}| < 0.2$~cm
 \item $\textrm{Isolation MVA output} > -0.6$
\end{itemize}
The detailed of muon isolation MVA can be found in Sec.\ref{sec:sel_muons}.

\subsubsection{Fake Rate Measurement}
\label{sec:fakerateMeasurement}
The fake rates are measured in calibration data samples dominated by fake leptons 
resulting from jets in QCD dijet events. 
The QCD dijet event sample is collected using a combination of different 
electron and muon triggers.

In order to suppress 
contamination due to signal leptons from the decay of W and Z bosons we require
that the missing transverse energy is less than $20$ GeV, and that 
the event contains only a single reconstructed lepton. In order to control the 
average $p_{T}$ of the jet that fakes the lepton, we impose a $p_{T}$ requirement 
on the leading jet in the event and require that the lepton denominator object is 
separated from the leading jet by $\Delta$R $ > 1.0$. The nominal fake rates for
electrons are measured requiring that the leading jet $p_{T}$ is greater than 
$35$ GeV, and the nominal fake rates for muons are measured with the requirement 
that the leading jet $p_{T}$ is greater than $15$ GeV. 
From these selected event samples, we measure the fake rate 
($\epsilon_{\mathrm{fake}}$) by counting the number of denominator 
objects which pass the full lepton selection, in bins of $p_{T}$
and $\eta$.

\subsubsection{Application of Fake Rates}
\label{sec:fakerateApplication}

Having measured the fake rates, parameterized in the kinematic quantities of interest,
we then use them as weights in order to extrapolate the yield of the sample of loose
leptons to the sample of fully selected leptons. This is done by selecting events
passing the full event selection described in Sec.\ref{sec:selection}, 
with the exception that one of the two lepton
candidates is required to pass the denominator selection cuts but fail the full 
lepton selection cuts. This lepton is from here on denoted the ``failing leg''. 
The other lepton is required to pass the full selection.
The data sample selected in this way is denoted the ``tight + fail'' sample.
Each of the events passing this selection is given a weight computed from
the fake rate in the particular $p_{T}$ and $\eta$ bin of the 
failing leg, as follows:

\begin{eqnarray}
  w_{i} = \frac{\epsilon_{\mathrm{fake}}(p_{\mathrm{T i}},\eta_{i})}{1 - \epsilon_{\mathrm{fake}}(p_{\mathrm{T i}},\eta_{i})}
\end{eqnarray}

where $i$ is an index denoting the failing leg, and $p_{\mathrm{T i}}$ and $\eta_{i}$
are the transverse momentum and pseudorapidity of the failing leg. 
Summing the weights $w_{i}$ over all such events in the tight + fail sample yields
the total jet induced background prediction.

This tight + fail extrapolation prediction will in fact 
double count the QCD component of the background, where both leptons are jet induced
fakes. This is essentially a combinatorial artifact, due to the fact that in the tight
plus fail selection, one is unable to uniquely distinguish which lepton is required to
be the tight one and which lepton is required to be the failing one, and therefore
one customarily selects both combinations. This double fake background is 
typically very small and accounts for roughly a few percent of the total jet
induced background. In order to estimate the amount of double counting,
we perform the fake rate extrapolation on both lepton legs, selecting events
which pass all event selection criteria, except that both leptons are required
to pass the denominator selection, but fail the full lepton selection. This
event sample is denoted as the ``fail + fail'' sample. Events in the fail + fail
sample are then given weights as follows:

\begin{eqnarray}
  w_{i,j} = \frac{\epsilon_{\mathrm{fake}}(p_{\mathrm{T i}},\eta_{\mathrm{i}})}{1 - \epsilon_{\mathrm{fake}}(p_{\mathrm{T i}},\eta_{\mathrm{i}})} \times \frac{\epsilon_{\mathrm{fake}}(p_{\mathrm{T j}},\eta_{\mathrm{j}})}{1 - \epsilon_{\mathrm{fake}}(p_{\mathrm{T j}},\eta_{\mathrm{j}})}
\end{eqnarray}

where $i$ and $j$ denote the two failing leg, and $p_{\mathrm{T i/j}}$ and $\eta_{\mathrm{i/j}}$
are the transverse momentum and pseudorapidity of the first and second leg.
Summing the weights $w_{i,j}$ over all such events in the fail + fail sample yields
the total QCD double fake background. This prediction is then subtracted from the
tight + loose prediction in order to account for the double counting. 

In this procedure, an over-estimation of the fake lepton contribution due to 
contamination from real dilepton events, and from $W+\gamma$ events may occur. These contributions 
are subtracted using the Monte Carlo simulation prediction with the procedure described 
in~\cite{fakeLeptonNote1} and~\cite{fakeLeptonBkgSpillage1}.
