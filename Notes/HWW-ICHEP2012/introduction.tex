The standard model (SM) of particle physics successfully describes the
majority of high-energy experimental data~\cite{pdg}. One of the key
remaining questions is the origin of the masses of $\W$~and
$\Z$~bosons.  In the SM in its simplest implementation it is
attributed to the spontaneous breaking of electroweak symmetry caused
by a new scalar field~\cite{Higgs1, Higgs2, Higgs3}. The existence of
the associated field quantum, the Higgs boson, has yet to be
experimentally confirmed.

The $\Hi\to\WW$ channel is particularly sensitive for Higgs boson
searches in the intermediate mass range
($120-200~\GeVcc$)~\cite{dittmar}. This document describes the search
for the Higgs boson in the $\Hi\To\WW \To \Lep\Lprime\Nu\Nubar$
channel, for Higgs boson masses in the range of $115-600~\GeVcc$ using
2011 and 2012 datasets available by ICHEP 2012 conference.
    
The study is based on the analysis of full 2011
dataset~\cite{HWW2011}. Several modifications and improvements have
been added, both to cope with the more difficult conditions due to the
higher pileup in 2012, and also to extend the sensitivity.

The main analysis strategy is to select events with two opposite
charged leptons, large missing energy and little jet activity. The two
leptons are required to be isolated electrons or muons with \pt
greater than 10~\GeV{}.

Here is a list of important changes with respect to the 2011 analysis:
\begin{itemize}
\item 
Revised lepton selection using multivariate techniques that allows to
increase signal efficiency by about 5\% at the same level of
background.
\item 
New MVA based Jet ID that helps to reduce top mis-tag rate in 0-jet
case and reduces pileup dependence of the jet veto.
\item
Alternative b-tagging algorithm in 1 and 2 jet cases (JetProbabilityB)
\item
New multivariate technique to reduce \dyll\ background in \ee\
and \mm\ final states
\item 
Significantly improved 2-jet VBF analysis
\item 
New method for $\dyll$ background estimation that works as a
cross-check to the old one and allows to extract MVA shapes from data
\end{itemize}

%% Since the last update of the analysis, total amount of usable data
%% increased by a factor of 3 (4.7~\ifb{}). With 4~\ifb\ of data
%% available we performed a study of the detector performance under the
%% new machine conditions. In this study we compared results for two
%% independent datasets corresponding to Run2011A and Run2011B data
%% taking periods. This also allowed us to review and refine our
%% background estimation techniques. The details of the study can be
%% found in Appendix~\vref{app:a_vs_b}.

%% After a detail review of the pile conditions
%% (Sec.~\ref{sec:pileupReweighting}), efficiency measurements
%% (Sec.~\ref{sec:alleff}) and general description of the systematic
%% uncertainties (Sec.~\ref{sec:systematics}) we show full 2011 data
%% results (Sec.~\vref{sec:dataresults}). In that section we summarize
%% all the background estimations and the expected background yields for
%% all Higgs mass hypothesis. At the end of the section the final results
%% for the Higgs search are shown
%% (Sec.~\vref{sec:search_results}). Summary of the analysis can be found
%% in Section.~\vref{sec:summary}.
