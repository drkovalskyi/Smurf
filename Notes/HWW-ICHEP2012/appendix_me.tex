\subsection{Event Probability Calculation}

In the Matrix Element method we calculate event-by-event based probability assuming a
certain hypothesis. In this analysis we calculate event probabilities 
for gluon fusion Higgs boson production ($ggH$), electroweak $q\bar{q}\rightarrow WW$ pair 
production ($WW$) and the $\Wjets$ process where a $W$ boson is produced in association with one hadronic jet. 
The Matrix Element functions used in the determination of the differential cross sections
for this analysis are obtained from MCFM v6.2 at the leading order. 

The probability is denoted by $P(x_{obs};\alpha)$,
where $\alpha$ is a set of physics 
parameters of the specific model and $x_{obs}$ are the measured kinematic quantities.
In the case of Standard Model Higgs Boson production,
 $\alpha$ is $(m_H, \Gamma_H)$, where  $m_H$ is the Higgs mass 
and $\Gamma_H$ is the Higgs width. The event probability density is given by
\begin{equation}
P(x_{obs};\alpha) =
 \frac{1}{ < \sigma(\alpha) > }
 \int \frac {d \sigma_{0} (y;\alpha) }{ dy }
 \epsilon (y) G(x_{obs},y) dy,  
\label{eqn:EvtProb}  
\end{equation}
where $y$ denotes the true values of the observables,
$\frac{d \sigma_{LO}}{dy}$ is the  parton-level differential cross-section differential
in those observables, $\epsilon(y)$ is the detector acceptance and efficiency function
and $G(x_{obs},y)$ is the transfer function between the true and measured values of the
observables, representing the detector resolution.
Equation (\ref{eqn:EvtProb}) integrates over all possible true values of the
observables, $y$, consistent with the measured quantities $x_{obs}$.
The constant $<\sigma(\alpha)>$ normalizes the total event probability to unity
There are eight observables, $x_{obs}$, representing all the 
lepton kinematic information: lepton momenta $\vec{l}^+$, $\vec{l}^-$ and missing 
transverse momentum, \met$_x$ and \met$_y$.

The efficiency $\epsilon(y)$ is defined as the probability for a parton-level object with momentum 
$p$ to be reconstructed as a lepton with momentum $q$. It is parameterized 
as a function of transverse momentum and pseudo-rapidity and extracted from $WW$ Monte Carlo. 
Note that the efficiency function in Equation~\ref{eqn:EvtProb} can be factorized out of
the integral when calculating event probabilities.
The lepton momenta are smeared according to the measured lepton mometum resolutions. 

The $WW$ leptonic final state is not fully reconstructed with two neutrinos in the final state. 
The $z$-components of the neutrino momenta as well as the individual 
neutrino transverse components are missing. We therefore integrate over 
these unknown quantities, performed using the importance sampling integration method.

For the $\Wjets$ events reconstructed in the dilepton final state, one of the reconstructed 
leptons has been faked by a parton fragmenting and hadronizing into a QCD jet 
which then fakes the signature of a lepton in the detector. 
To account for this effect properly, we multiply the differential cross-section for the $\Wjets$ 
process by the probability for a parton to be reconstructed as a lepton with the measured kinematics. 
This probability can be factorized into two terms:
\begin{eqnarray}
\begin{array}{lcl}
P(parton\rightarrow lepton)=P(parton\rightarrow FO)\times P(FO\rightarrow lepton)
\end{array} 
\end{eqnarray} 
where FO refers to a so-called "fakeable object" (see Sec.~\ref{sec:bkg_fakes}). 
The first term in the product is measured using Monte Carlo and parametrized
in $p_{T}$ and $\eta$.  The second term is the fake rate measured in the data 
(see Sec.~\ref{sec:bkg_fakes}).
We verify the $P(parton\rightarrow lepton)$ values we obtain from this method by 
comparing them to ones measured in a $\gamma+jet$ Monte Carlo sample and $\gamma+jet$ data.
The probabilities agree within the uncertainty of $30\%$.  

As we are using the LO matrix element based calculations we account for the 
transverse recoil %and thus improve the performance of our discriminant on data
by integrating over the possible values of the system boost $k_{T}(k_{x},k_{y})$. 
The $k_T$ model is extracted for each jet bin from Monte Carlo for each process separately. 

\subsection{Likelihood Ratio Discriminator}
The probabilities for all processes under consideration are combined 
to construct a single discriminant, called the Likelihood Ratio ($LR$), defined as
\begin{equation}
\label{eqn:LR}
LR = \frac { P_s} { P_s + \sum_i k_{bi} P_{bi}},
\end{equation}
where $P_s$  is the probability for the signal, $P_{bi}$ is the probability for background
process $i$, and
$k_{bi}$ is the expected fractional contribution of background $i$,
satisfying the sum $\sum k_{bi} =1$.
Because signal events are expected to have $P_s>P_b$ and vice-versa for background events, 
the value of $LR$ is close to one for signal and zero for background processes.
The calculation of $P_s$ is a function of Higgs mass, so the likelihood ratio
shape depends on $m_H$. This is true for both signal and background templates of $LR$. 
This variable is then used in a shape based analysis to extract the final results. 

\subsection{\texorpdfstring{Results with $\intlumiEightTeV$}{Results on data}}

Figures~~\ref{fig:me_lr_115_120},~\ref{fig:me_lr_130_140},
~\ref{fig:me_lr_130_140} and~~\ref{fig:me_lr_160_200} show the matrix element output 
distributions, comparing data and MC for several higgs mass hypotheses. We then perform the 
shape analysis based on these output in the same way as in the BDT based shape analysis. 
The same treatment of the shape variations are applied as well. 
The expected and observed upper limits at 95\%C.L. for each mass point are shown in Table~\ref{tab:limits_me_5fb_0j} 
and Figure~\ref{fig:limits_me_5fb} for the dataset corresponding to $\intlumiEightTeV$. 
The sensitivity performance of the matrix element method are
consistent with the BDT-based approach shown in Section~\ref{app:appendix_limits_bychannel}.


%%%%%%%%%%%%%%%%%%%%%%        
\begin{figure}[!hbtp]
\centering
\subfigure[$mH=115 GeV$]{
\centering                    
\label{subfig:lr115}      
\includegraphics[width=.45\textwidth]{figures/hww_analysis19_115_lr_incl_0j.pdf}}
\subfigure[$mH=120 GeV$]{               
\centering
\label{subfig:lr120}      
\includegraphics[width=.45\textwidth]{figures/hww_analysis19_120_lr_incl_0j.pdf}}
\caption{Likelihood ratio for Higgs mass hypotheses 115 and 120 \GeV.}
\label{fig:me_lr_115_120}
\end{figure}


\begin{figure}[!hbtp]
\centering
\subfigure[$mH=125 GeV$]{
\label{subfig:lr125}
\includegraphics[width=.45\textwidth]{figures/hww_analysis19_125_lr_incl_0j.pdf}}
\subfigure[$mH=130 GeV$]{
\centering
\label{subfig:lr130}
\includegraphics[width=.45\textwidth]{figures/hww_analysis19_130_lr_incl_0j.pdf}}
\caption{Likelihood ratio for Higgs mass hypotheses 125 and 130 \GeV.}
\label{fig:me_lr_130_140}
\end{figure}


\begin{figure}[!hbtp]
\subfigure[$mH=140 GeV$]{
\centering
\label{subfig:lr140}
\includegraphics[width=.45\textwidth]{figures/hww_analysis19_140_lr_incl_0j.pdf}}
\subfigure[$mH=150 GeV$]{
\centering
\label{subfig:lr150}
\includegraphics[width=.45\textwidth]{figures/hww_analysis19_150_lr_incl_0j.pdf}}
\caption{Likelihood ratio for Higgs mass hypotheses 140 and 150 \GeV.}
\label{fig:me_lr_140_150}
\end{figure}

\begin{figure}[!hbtp]
\subfigure[$mH=160 GeV$]{
\centering
\label{subfig:lr160}
\includegraphics[width=.45\textwidth]{figures/hww_analysis19_160_lr_incl_0j.pdf}}
\subfigure[$mH=200 GeV$]{
\centering
\label{subfig:lr200}
\includegraphics[width=.45\textwidth]{figures/hww_analysis19_200_lr_incl_0j.pdf}}
\caption{Likelihood ratio for Higgs mass hypotheses 160 and 200 \GeV.}
\label{fig:me_lr_160_200}
\end{figure}

%%%%%%%%%%%%%%%%%%%%%%


%%%%%%%%%%%%%%%%%%%%%%%%%%%%%%
\begin{figure}[!hbtp]
\centering
\subfigure[ME 0-Jet OF]{
\label{subfig:me_0jof}
\includegraphics[width=.45\textwidth]{figures/limits_0jof_shape_me-CLs-asymptotic_zoom_log.pdf}
}
\centering
\subfigure[ME 0-Jet SF]{
\label{subfig:me_0jsf}
\includegraphics[width=.45\textwidth]{figures/limits_0jsf_shape_me-CLs-asymptotic_zoom_log.pdf}
}\\
\centering
\subfigure[ME 0-Jet combined]{
\label{subfig:me_0j}
\includegraphics[width=.45\textwidth]{figures/limits_0j_shape_me-CLs-asymptotic_zoom_log.pdf}
}
\caption{ Shape analysis upper limits based on the matrix element outputs at 95\% C.L. for $\intlumiEightTeV$ data. }
\label{fig:limits_me_5fb}
\end{figure}
%%%%%%%%%%%%%%%%%%%%%%%%%%%%%%

%%%%%%%%%%%%%%%%%%%%%%%%%%%%%%
\begin{table}
\begin{center}
\begin{tabular}{c c c c c c}
\hline\hline
 Higgs Mass   & Observed & Median expected & Expected range for 68\% & Expected range for 95\%   \\
\hline
\multicolumn{5}{c} {0-Jet} \\
\hline
110 & 7.0 & 6.7 & [4.8, 9.3] & [3.6, 12.5] \\
115 & 3.9 & 3.5 & [2.5, 4.8] & [1.9, 6.5] \\
120 & 2.3 & 2.1 & [1.5, 2.9] & [1.1, 3.9] \\
125 & 2.2 & 1.3 & [1.0, 1.9] & [0.7, 2.5] \\
130 & 1.5 & 1.0 & [0.7, 1.3] & [0.5, 1.8] \\
135 & 0.9 & 0.7 & [0.5, 1.0] & [0.4, 1.4] \\
140 & 0.8 & 0.6 & [0.5, 0.9] & [0.3, 1.2] \\
145 & 0.8 & 0.5 & [0.4, 0.7] & [0.3, 0.9] \\
150 & 0.8 & 0.4 & [0.3, 0.5] & [0.2, 0.7] \\
160 & 0.3 & 0.2 & [0.1, 0.3] & [0.1, 0.4] \\
170 & 0.4 & 0.2 & [0.2, 0.3] & [0.1, 0.4] \\
180 & 0.4 & 0.3 & [0.2, 0.4] & [0.2, 0.5] \\
190 & 0.6 & 0.4 & [0.3, 0.6] & [0.2, 0.8] \\
200 & 0.8 & 0.6 & [0.4, 0.8] & [0.3, 1.0] \\
\hline
\multicolumn{5}{c} {0-Jet bin OF} \\
\hline
110 & 9.6 & 8.2 & [5.9, 11.3] & [4.4, 15.2] \\
115 & 6.0 & 4.2 & [3.1, 5.9] & [2.3, 7.9] \\
120 & 3.2 & 2.6 & [1.9, 3.7] & [1.4, 4.9] \\
125 & 2.3 & 1.6 & [1.1, 2.2] & [0.8, 3.0] \\
130 & 1.6 & 1.2 & [0.8, 1.6] & [0.6, 2.2] \\
135 & 1.1 & 0.9 & [0.6, 1.2] & [0.5, 1.6] \\
140 & 0.9 & 0.7 & [0.5, 1.0] & [0.4, 1.3] \\
145 & 0.8 & 0.6 & [0.4, 0.8] & [0.3, 1.1] \\
150 & 0.6 & 0.4 & [0.3, 0.6] & [0.2, 0.8] \\
160 & 0.3 & 0.2 & [0.2, 0.3] & [0.1, 0.5] \\
170 & 0.3 & 0.3 & [0.2, 0.4] & [0.1, 0.5] \\
180 & 0.4 & 0.4 & [0.3, 0.5] & [0.2, 0.7] \\
190 & 0.5 & 0.6 & [0.4, 0.8] & [0.3, 1.0] \\
200 & 0.8 & 0.7 & [0.5, 1.0] & [0.4, 1.3] \\
\hline
\multicolumn{5}{c} {0-Jet bin SF} \\
\hline
110 & 10.2 & 12.7 & [9.1, 17.7] & [6.8, 23.7] \\
115 & 4.8 & 6.4 & [4.6, 8.9] & [3.4, 11.9] \\
120 & 3.8 & 3.6 & [2.6, 5.0] & [1.9, 6.8] \\
125 & 3.3 & 2.4 & [1.7, 3.3] & [1.3, 4.4] \\
130 & 2.1 & 1.5 & [1.1, 2.1] & [0.8, 2.8] \\
135 & 2.0 & 1.3 & [0.9, 1.8] & [0.7, 2.4] \\
140 & 1.0 & 1.0 & [0.7, 1.4] & [0.6, 1.9] \\
145 & 1.4 & 0.8 & [0.6, 1.1] & [0.4, 1.5] \\
150 & 1.1 & 0.5 & [0.4, 0.7] & [0.3, 1.0] \\
160 & 0.6 & 0.3 & [0.2, 0.4] & [0.2, 0.5] \\
170 & 0.5 & 0.3 & [0.2, 0.4] & [0.2, 0.6] \\
180 & 0.7 & 0.4 & [0.3, 0.6] & [0.2, 0.8] \\
190 & 1.1 & 0.6 & [0.4, 0.9] & [0.3, 1.2] \\
200 & 1.2 & 0.8 & [0.6, 1.1] & [0.4, 1.5] \\
\hline\hline
\end{tabular}
\end{center}
\caption{Expected and observed upper limits for SM Higgs using {\bf shape analysis based 
    on matrix element output} for $\intlumiEightTeV$ data the matrix element outputs 
  for the {\bf 0-jet} bin two sub-channels. }
\label{tab:limits_me_5fb_0j}
\end{table}
%%%%%%%%%%%%%%%%%%%%%%%%%%%%%%

\clearpage


\subsection{Correlation between likelihood ratio and BDT output}

In this section we examine the correlation
between the matrix element LR discriminator 
and the BDT output.  
To do this, we select events in the samples of simulated
Higgs boson decays with $m_{H}=125 \GeV$ and
simulated $qq\rightarrow WW$ decays, using the standard
BDT analysis preselection in the zero jet bin.
The data are selected in the same way.
We then calculate the correlation co-efficients 
between the two discriminators, and between each
discriminator and relevant kinematic variables.

We observe the correlation co-efficient between the
matrix element LR discriminator and the BDT output to be $0.66$ in the 
signal sample and $0.78$ in the background sample.
While the matrix element and the BDT have similar
expected sensitivity to the Higgs boson signal, they are not
$100\%$ correlated.  This means that the matrix element
discrimnator can be used to provide a partially independent
cross check of the BDT based analysis.

The correlation co-efficients between the
two discriminating variables, and the event kinematic variables
are given in Table \ref{tab:bdt_me_correlations} using the simulated samples.
We observe that the value of both discriminators is highly
correlated with $M_{ll}$, however less so for the 
matrix element than the BDT.

The correlation between the output of the two discriminators
is shown for the data and the expected signal and background
in Figures \ref{fig:me_correlations_all}, 
\ref{fig:me_correlations_sf} and \ref{fig:me_correlations_of}.
The output assigned to data events in both discriminators
is qualitatively consistent with expectations.
Signal like events have a high score in both discriminators,
and background like events have a low score in both discriminators.

\vspace{30pt}
\begin{table}[ht!]
\begin{center}
\begin{tabular}{l|c|c}
\hline
Variable        &   Signal Events   & Background Events \\ \hline
\hline
\multicolumn{3}{c}{Matrix Element LR} \\
\hline

$p_{T} (trail)$ &   -0.508485       &  -0.553587 \\
$M_{ll}$        &   -0.489669       &  -0.709396 \\
$p_{T} (lead)$  &   -0.327346       &  -0.306577 \\
$M_{T}$         &   -0.270782       &  -0.212003 \\
$MET$           &  -0.247144        &  -0.201313 \\
$\Delta\phi$    &   -0.170118       &  -0.310511  \\
\hline
\multicolumn{3}{c}{BDT} \\
\hline
$M_{ll}$        &  -0.738705       &  -0.873989 \\
$p_{T} (trail)$ &  -0.55488        &  -0.542548 \\
$\Delta\phi$    &  -0.403576       &  -0.510162  \\
$M_{T}$         &  -0.2568         &  -0.266183  \\
$p_{T} (lead)$  &  -0.217102       &  -0.342558 \\
$MET$           &  -0.138572       &  -0.156602 \\
\hline
\end{tabular}
\end{center}
\caption{ The correlations of the matrix element LR and BDT discriminators, with
event kinematic variables in simulated events.  Rows are sorted by highest correlation in signal events. }
\label{tab:bdt_me_correlations}
\end{table}

\begin{figure}[!hbtp]
\centering
\subfigure[$m_{H}=125$~GeV]{\label{subfig:fig_me_correlations_all_s}
\includegraphics[width=.45\textwidth]{figures/bdt_me_hww125_all.pdf}}
\subfigure[$qq\rightarrow WW$]{\label{subfig:fig_me_correlations_all_b}
\includegraphics[width=.45\textwidth]{figures/bdt_me_qqww_all.pdf}}
\caption{Correlation between BDT and ME LR in data
and simulated $qq\rightarrow WW$ and $m_{H}=125$~GeV decays.
Events are shown in all lepton channels.  Correlation is $0.66$ for signal and $0.78$ for background.}.
\label{fig:me_correlations_all}
\end{figure}

\begin{figure}[!hbtp]
\centering
\subfigure[$m_{H}=125$~GeV]{\label{subfig:fig_me_correlations_sf_s}
\includegraphics[width=.45\textwidth]{figures/bdt_me_hww125_sf.pdf}}
\subfigure[$qq\rightarrow WW$]{\label{subfig:fig_me_correlations_sf_b}
\includegraphics[width=.45\textwidth]{figures/bdt_me_qqww_sf.pdf}}
\caption{Correlation between BDT and ME LR in data
and simulated $qq\rightarrow WW$ and $m_{H}=125$~GeV decays.
Events are shown in the $\mu\mu$ and $ee$ channels. Correlation is $0.66$ for signal and $0.79$ for background.}
\label{fig:me_correlations_sf}
\end{figure}

\begin{figure}[!hbtp]
\centering
\subfigure[$m_{H}=125$~GeV]{\label{subfig:fig_me_correlations_of_s}
\includegraphics[width=.45\textwidth]{figures/bdt_me_hww125_of.pdf}}
\subfigure[$qq\rightarrow WW$]{\label{subfig:fig_me_correlations_of_b}
\includegraphics[width=.45\textwidth]{figures/bdt_me_qqww_of.pdf}}
\caption{Correlation between BDT and ME LR in data
and simulated $qq\rightarrow WW$ and $m_{H}=125$~GeV decays.
Events are shown in the $e\mu$ and $\mu e$ channels. Correlation is $0.69$ for signal and $0.80$ for background.}
\label{fig:me_correlations_of}
\end{figure}


\clearpage

