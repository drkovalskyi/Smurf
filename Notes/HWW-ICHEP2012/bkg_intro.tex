We use a combination of data-driven methods and detailed Monte Carlo
simulation studies to estimate background contributions.  From data we
can estimate the following backgrounds: $\Wjets$, $\dyll$, $\WZ$ and
$\ZZ$ (for events where both leptons come from a $\dyll$), top
background, $\dytt$, and \WW{}. The background from the remaining processes 
are taken from simulation.

Background composition and yields depend on the final state and on
the Higgs boson mass hypothesis under study. In the 0-Jet final state, 
the non-resonant \WW{} background dominates, while \wjets\ background contribution 
becomes sizable in the low Higgs mass hypotheses. 
In the 1-jet and 2-jet final states, the largest background contribution comes from 
top decays, while the non-resonant \ww\ background contribution is the second largest. 

For the backgrounds that can be estimated from data, 
we perform a data-driven background estimate in the signal region 
if the expected background contribution is sizable. 
If the expected contribution in the signal region is limited by statistics, 
we first estimate the background contribution with the $WW$ preselection from data 
and then extrapolate this estimation to the signal region using MC. The particular
choice of which backgrounds are estimated in the first or second way depends on the
integrated luminosity of the data sample that we analyze.
