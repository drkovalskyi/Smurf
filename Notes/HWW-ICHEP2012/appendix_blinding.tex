The analysis consists of two types of analyses: a cut based counting
experiment and a shape based maximum-likelihood fit. In both cases
data-driven background estimations play a critical role in measuring
the expected background event yields.

The main control sample that is effectively signal free is based on a
selection for Standard Model WW process. Higgs signal contribution is
expected to be small at this selection level and tests for background
estimations as well as WW cross-section measurements should not lead
to a bias for the Higgs search.

For the cut-based analysis the main thing that should be avoided is
looking at final event yields at Higgs selection level with full
background estimation. Normally these numbers are available only in a
form of final data cards, which can easily be blinded by replacing
observed event yields with some random numbers derived from the
expected event yield.

For the shape-analysis it is harder to introduce a bias, because all
the inputs are provided at WW selection level with a few additional
cuts that keep too much background to judge if signal is present or
not. It is critical to avoid looking at the observed distributions
overlaid with stacked background distributions. It is Ok to use CLs
procedure to extracted expected upper limits even though it uses
actual data to fit for central values of nuisance parameters. The main
thing to be avoided is extraction of the observed upper limits, which
can be simply turned off in the statistical tools.

The criteria for analysis readiness for unblinding should be based on
an assessment of how well backgrounds are understood, how reliable their
estimations are and how reasonable the associated uncertainties look.
