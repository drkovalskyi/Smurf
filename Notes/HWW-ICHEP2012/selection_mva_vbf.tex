For the VBF channel we performed a similar multivariate analysis as in the 0 and 1 jet bins. 
In this channel the main background is the top events. Therefore the discriminant is 
constructed to distinguish signal against the top background. 
In additional to the $\WW$ preselection we apply a loose cut on the 
$\mll$ and $\mt$ to enhance the signal to background ration. 
The selection on the $\mll$ is the same as in the 0 and 1 jet bins. 
For $\mt$ a loose window selection between 30 GeV and the higgs mass is used. 

The input variables used in the MVA analysis include the same ones 
in the 0-jet bin, adding the following three VBF jet related ones:
%%%%%%%%%%%%%%%%%%
\begin{itemize}
\item $\Delta\eta (j_1-j_2)$
\item $m_{j_1j_2}$ 
\item the leading jet $\eta$.
\end{itemize}
%%%%%%%%%%%%%%%%%%
The full shape of the classifier output is used for the final analysis. Compared to the 
cut-based analysis this MVA based approach improves the search sensitivity by 
about 20-30\%. 

The systematic uncertainties related to the dilepton variables can be 
addressed in the same way as in the 0 and 1 jet bins. However we need to 
do a careful study of the shape systematics due to the jet related variables. 
This includes the potential data/MC differences in these variables and also 
the the theoretical uncertainties addressing how well we can rely on a 
particular MC to produce the true distributions. 
{\bf This part has not yet been done systematically. The only systematics related to 
jet considered is the jet energy scale uncertainties where we scale the 
jets up and down by 5\%}
For the signal we do not yet have any alternative MC generator to be compared 
to. We can compare the generator level jet kinematics with the MCFM predicition.
For the main top background this can be validated by comparing the 
top enriched sample from data and MC. 
However for other backgrounds such as the $\WW$ we need to do a more careful 
study to convince ourselves that the shapes from the MC is reasonable. 
