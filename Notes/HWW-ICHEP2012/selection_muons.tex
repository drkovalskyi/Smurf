%The muon selection is unchanged with respect to~\cite{HWW2011}. 
Muons in CMS are reconstructed as either $StandAloneMuons$ (track
in the muon detector with low momentum resolution), $GlobalMuons$
(outside-in approach seeded by a $StandAloneMuon$ with a global fit
using hits in the muon, silicon strip and pixel 
detectors) and $TrackerMuons$ (inside-out approach seeded by an offline 
silicon strip track, using the muon detector only for muon identification 
without refitting the track). Most good quality muons are reconstructed as 
all three types at the same time and the momentum resolution is dominated by the inner
tracker system up to about 200~$\GeVc$ in transverse momentum.  
We require the muon to be reconstructed as $GlobalMuon$, with $\chi^2/{\mathrm{ndof}} < 10$ 
on the global fit, must have at least one good muon hit, and at least two 
matches to muon segments in different muon stations; 
or $TrackerMuon$, provided it satisfies the ``Tracker Muon Last Station 
Tight" selection requiring at least two muon segments matched at 
3$\sigma$ in local X and Y coordinates, with one being in the outermost muon station.

In addition, the following specific requirements to select good prompt isolated 
muons are the following:
\begin{itemize}
\item more than 5 tracker layers; 
\item at least one pixel hit;
\item impact parameter in the transverse plane $|d_{0}| < 0.02~(0.01)$~cm for
      muons with $\pt$ greater (smaller) than 20 $\GeVc$,
      calculated with respect to the primary vertex;
\item longitudinal impact parameter $|d_{z}| <0.1$~cm,
      calculated with respect to the primary vertex;
\item pseudorapidity $|\eta|$ must be smaller than 2.4;
\item relative \pt\ resolution is better than 10\%;
\item decay in flight with the kink finding algorithm: $\chi^2/{\mathrm{ndof}} < 20$
\item identified as a PF Muon. 
\end{itemize}

Furthermore, the MVA-based isolation variable is used to reduce contamination from non-isolated muons 
originating from jets. We used energy deposits of PF candidates in three different categories, charged hadron, gamma, 
and neutral hadrons in concentric isolation cones. 
Neutral components are corrected by subtracting pileup contribution which is calculated by $\rho \times A_{eff}$
where $\rho$ is the event-by-event energy density and $A_{eff}$ is the effective area.  
Effective areas are from Fall 11 simulation and values are shown in Table~\ref{tab:muAeff}.  
Exact definition of input variables to the MVA is the following. 
\begin{itemize}
\item PF charged hadron
	\begin{itemize}
    \item minimum of $\textrm{Iso}_\textrm{PF charged, 01}/\pt$ and 2.5	
    \item minimum of $\textrm{Iso}_\textrm{PF charged, 12}/\pt$ and 2.5	
    \item minimum of $\textrm{Iso}_\textrm{PF charged, 23}/\pt$ and 2.5	
    \item minimum of $\textrm{Iso}_\textrm{PF charged, 34}/\pt$ and 2.5	
    \item minimum of $\textrm{Iso}_\textrm{PF charged, 45}/\pt$ and 2.5	
	\end{itemize}
\item PF gamma : If negative, 0.0 is assigned
	\begin{itemize}
    \item minimum of $\left[\textrm{Iso}_\textrm{PF gamma, 01} - \rho \times A_{eff}\right]/\pt$ and 2.5 
    \item minimum of $\left[\textrm{Iso}_\textrm{PF gamma, 12} - \rho \times A_{eff}\right]/\pt$ and 2.5 
    \item minimum of $\left[\textrm{Iso}_\textrm{PF gamma, 23} - \rho \times A_{eff}\right]/\pt$ and 2.5 
    \item minimum of $\left[\textrm{Iso}_\textrm{PF gamma, 34} - \rho \times A_{eff}\right]/\pt$ and 2.5 
    \item minimum of $\left[\textrm{Iso}_\textrm{PF gamma, 45} - \rho \times A_{eff}\right]/\pt$ and 2.5 
	\end{itemize}
\item PF neutral hadron : If negative, 0.0 is assigned
	\begin{itemize}
    \item minimum of $\left[\textrm{Iso}_\textrm{PF neutral, 01} - \rho \times A_{eff}\right]/\pt$ and 2.5 
    \item minimum of $\left[\textrm{Iso}_\textrm{PF neutral, 12} - \rho \times A_{eff}\right]/\pt$ and 2.5 
    \item minimum of $\left[\textrm{Iso}_\textrm{PF neutral, 23} - \rho \times A_{eff}\right]/\pt$ and 2.5 
    \item minimum of $\left[\textrm{Iso}_\textrm{PF neutral, 34} - \rho \times A_{eff}\right]/\pt$ and 2.5 
    \item minimum of $\left[\textrm{Iso}_\textrm{PF neutral, 45} - \rho \times A_{eff}\right]/\pt$ and 2.5 
	\end{itemize}
\end{itemize}
where we defined 
\begin{eqnarray} 
\textrm{Iso}_\textrm{PF,XY} = \textrm{the scalar sum of \pt\ of PF candidates in the cone of 
									  0.X} < \Delta R^{\mu-\textrm{PF candidate}} < \textrm{0.Y}.
\end{eqnarray} 
We require the MVA output to be greater than 0.82 (0.86) in $10~<~p_T~<~20$ GeV 
and 0.86 (0.82) in $p_T~>~20$ GeV. Cut values correspond to the ones in barrel (endcap).

\begin{table}[htp]
	\centering
		\begin{tabular}{c|c|c|c|c|c|c}
			\hline 
				\multicolumn{7}{c}{PF gamma} \\
	  	    \hline
			 	$|\eta|$     & $0.0 - 1.0$ & $1.0 - 1.479$ & $1.479 - 2.0$ & $2.0 - 2.2$ & $2.2 - 2.3$ & $2.3-$ \\       		
	  	    \hline \hline
				$0.0<dR<0.1$ & 0.004& 0.002& 0.003& 0.009& 0.003& 0.011 \\
				$0.1<dR<0.2$ & 0.012& 0.008& 0.006& 0.012& 0.019& 0.024 \\
				$0.2<dR<0.3$ & 0.026& 0.020& 0.012& 0.022& 0.027& 0.034 \\
				$0.3<dR<0.4$ & 0.042& 0.033& 0.022& 0.036& 0.059& 0.068 \\
				$0.4<dR<0.5$ & 0.060& 0.043& 0.036& 0.055& 0.092& 0.115 \\
	  	    \hline \hline 
				\multicolumn{7}{c}{PF neutral hadron} \\
	  	    \hline 
			 	$|\eta|$     & $0.0 - 1.0$ & $1.0 - 1.479$ & $1.479 - 2.0$ & $2.0 - 2.2$ & $2.2 - 2.3$ & $2.3-$ \\       		
	  	    \hline \hline
				$0.0<dR<0.1$ & 0.002& 0.004& 0.004& 0.004& 0.010& 0.014 \\
			    $0.1<dR<0.2$ & 0.005& 0.007& 0.009& 0.009& 0.015& 0.017 \\
			    $0.2<dR<0.3$ & 0.009& 0.015& 0.016& 0.018& 0.022& 0.026 \\ 
				$0.3<dR<0.4$ & 0.013& 0.021& 0.026& 0.032& 0.037& 0.042 \\ 
				$0.4<dR<0.5$ & 0.017& 0.026& 0.035& 0.046& 0.063& 0.135 \\ 
			\hline
		\end{tabular}
		\caption{ Effective areas used for muon isolation. They were calculated with Fall11 MC sample.}
	\label{tab:muAeff}
\end{table}

%\begin{itemize}
%\item $\rm{Iso}_{PF}$: defined as the scalar sum of the \pt\ of the 
%    particle flow candidates satisfying the following requirements:
%    \begin{itemize}
%    \item $\Delta R~<~0.3$ to the muon in the $\eta \times \phi$ plane,
%    \item $|d_{z}(\mathrm{PF Candidate}) - d_{z}(\mathrm{muon})| < 0.1$~cm, if the PF candidate is charged,
%    \item \pt $>1.0$ GeV, if the PF candidate is classified as a neutral hadron or a photon.
%    \end{itemize}
%\end{itemize}
%
%We require $\frac{\rm{Iso}_{PF}}{\pt}~<~0.13~(0.06)$ for muons in the barrel 
%with $\pt$ greater (smaller) than 20 $\GeVc$. For muons in the endcap, we
%require $\frac{\rm{Iso}_{PF}}{\pt}~<~0.09~(0.05)$ for muons with $\pt$ 
%greater (smaller) than 20 $\GeVc$.
