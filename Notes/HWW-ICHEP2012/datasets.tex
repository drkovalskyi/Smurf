%UPDATEME%
The datasets used for this analysis are summarized in 
Tables~\ref{tab:DatasetsData} and~\ref{tab:DatasetsMC} for data and Monte 
Carlo, respectively. The total integrated luminosity is \intlumiEightTeV. 
Since some portion of data datasets was not processed, 
we use a json excluding the missing events~\cite{json}. For Monte Carlo simulation 
we use madgraph when possible, but different generators such as Pythia and Powheg~\cite{powheg} 
are also used.  For $gg \to \WW$ a dedicated generator is used. 
%For \wz\ and \zz\ processes we use Pythia, since MadGraph samples are mixed with $\WW$ in
%a single $VV$ sample, which is difficult to use properly.

\begin{table}[!ht]
\begin{center}
\begin{tabular}{|c|c|}
\hline
 Dataset Description                   &   Dataset Name   \\
\hline \hline
\multirow{2}{*}{MuEl PromptReco}   		&  /MuEG/Run2012A-PromptReco-v1/AOD   \\
            							&  /MuEG/Run2012B-PromptReco-v1/AOD   \\
\multirow{2}{*}{DiMuon PromptReco}     	&  /DoubleMu/Run2012A-PromptReco-v1/AOD   \\
          								&  /DoubleMu/Run2012B-PromptReco-v1/AOD   \\
\multirow{2}{*}{DiElectron PromptReco} 	&  /DoubleElectron/Run2012A-PromptReco-v1/AOD   \\
      									&  /DoubleElectron/Run2012B-PromptReco-v1/AOD   \\
\multirow{2}{*}{SingleMuon PromptReco}  &  /SingleMu/Run2012A-PromptReco-v1/AOD   \\
      									&  /SingleMu/Run2012B-PromptReco-v1/AOD   \\
\multirow{2}{*}{SingleElectron PromptReco} 	&  /SingleElectron/Run2012A-PromptReco-v1/AOD   \\
      										&  /SingleElectron/Run2012B-PromptReco-v1/AOD   \\
\hline
\end{tabular}
\caption{Summary of data datasets used.\label{tab:DatasetsData}}
\end{center}
\end{table}

\begin{table}[!ht]
\begin{center}
{\footnotesize
\begin{tabular}{|c|c|c|}
\hline
\multicolumn{3}{|c|}{With Pileup: Processed dataset name is always} \\
\multicolumn{3}{|c|}{/Summer12-PU\_S7\_START52\_V9-v*/AODSIM} \\
\hline
 Dataset Description              		&   Primary Dataset Name   & cross-section (pb)\\
\hline
$\ttbar$                              	&   /TTJets\_TuneZ2star\_8TeV-madgraph-tauola                          	& 	225.2 	\\
tW                  	 	 			&   /T\_tW-channel-DR\_TuneZ2star\_8TeV-powheg-tauola                  	&  	11.18 	\\
$\bar{\textrm{t}}$W                   	&   /Tbar\_tW-channel-DR\_TuneZ2star\_8TeV-powheg-tauola               	&  	11.18 	\\
gg $\rightarrow WW \to 2l 2\nu$         &   /GluGluToWWTo4L\_TuneZ2star\_8TeV-gg2ww-pythia6                     &   1.74	\\
qq $\rightarrow WW$                  	&   /WWJetsTo2L2Nu\_TuneZ2star\_8TeV-madgraph-tauola                    &  	5.81  	\\
WZ                               	 	&   /WZ\_TuneZ2star\_8TeV\_pythia6\_tauola                        		&  	22.45 	\\
Z[10-50] 	  	 						&   /DYJetsToLL\_M-10To50filter\_8TeV-madgraph                   		&  	860.5 	\\
Z[50-inf] 	  	 						&   /DYJetsToLL\_M-50\_TuneZ2Star\_8TeV-madgraph-tarball           		&  	3532.8 	\\
ZZ $\rightarrow 2l 2\nu$    	 		& 	/ZZJetsTo2L2Nu\_TuneZ2star\_8TeV-madgraph-tauola                    &   0.365	\\
ZZ $\rightarrow 2l 2q$    	 			&   /ZZJetsTo2L2Q\_TuneZ2star\_8TeV-madgraph-tauola                     &   1.28	\\
ZZ $\rightarrow 4l$    	 				&   /ZZJetsTo4L\_TuneZ2star\_8TeV-madgraph-tauola                       &   0.0921	\\
$gg \to H \to WW \to 2l 2\nu$         	&   /GluGluToHToWWTo2LAndTau2Nu\_M-*\_8TeV-powheg-pythia6             	& 	vary 	\\
$qqH,~H \to WW \to 2l 2\nu$           	&   /VBF\_HToWWTo2LAndTau2Nu\_M-*\_8TeV-powheg-pythia6                 	& 	vary 	\\
$WH/ZH/\ttbar H,~H\to WW$              	&   /WH\_ZH\_TTH\_HToWW\_M-*\_8TeV-pythia6                            	& 	vary 	\\
\hline
\hline
\end{tabular}
}
\caption{Summary of Monte Carlo datasets used.\label{tab:DatasetsMC}. The cross sections for a SM Higgs boson
is taken from the LHC Higgs cross-section working group~\cite{LHCHiggsCrossSectionWorkingGroup:2011ti}}
\end{center}
\end{table}

Last year we adjusted the Higgs $\pt$ spectrum because 
the spectrum in the simulation (POWHEG 2.0) was harder than the one 
in the most precise calculation to NNLO with resummation to NNLL order.
In the new version of Powheg (POWHEG 2.1), this problem has been mostly 
resolved, thus we do not apply any corrections for the Higss $\pt$.
