Jets are reconstructed using calorimeter and tracker information using a particle flow 
algorithm~\cite{jetpas}. The anti-${\rm k_T}$ clustering algorithm~\cite{antikt} 
with ${\rm R=0.5}$ is used. We apply the standard jet energy 
corrections~\cite{jes} to the reconstructed jets, where the L1 Fast Jets 
corrections are included. The latter corrections are rather important since 
they help in flatening the reconstruction efficiency as a function of the 
number of overlapping events.
To exclude electrons and muons from the jet sample, these 
jets are required to be separated from the selected leptons in $\Delta R$ 
by at least $\Delta R^{\mathrm{jet-lepton}}>0.3$.
In order to reject jets from pileup, MVA-based jet identification~\cite{mvajetid} is applied. 
This idenfication takes advantage of difference in shape of energy deposit in a jet cone.
Jets from pileup are soft, so they need to be overlaid to pass the jet $\pt$ threshold. 
Therefore, energy deposit of pileup jets are more spread than the one from hard interactions.
We select jets of which MVA outputs are greater than the values 
which correspond to the loose working point in Table~\ref{tab:jetidcut}.

\begin{table}[htp]
	\centering
		\begin{tabular}{c|c|c|c|c}
			\hline
									&  $0<|\eta|<2.5$ 	& $2.5<|\eta|<2.75$		& $2.75<|\eta|<3.0$ 	& $3.0<|\eta|<4.7$ 		\\ 
			\hline \hline
				$\pt<10$ \GeV		& 0.0 				& 0.0					& 0.0	 				& 0.2					\\ 
				$10<\pt<20$	\GeV 	& -0.4 				& -0.4					& -0.4	 				& 0.4					\\
				$20<\pt<30$	\GeV	& 0.0 				& 0.0					& 0.2	 				& 0.6					\\ 
				$\pt>30$ \GeV 		& 0.0 				& 0.0					& 0.6	 				& 0.2					\\
			\hline 
		\end{tabular}
		\caption{Cut values on jet identification MVA outputs. MVA output is required to be greater than 
				these values to be counted as a jet.}
	\label{tab:jetidcut}
\end{table}

In this analysis we use high $\pt$ jets to define the analysis jet bin
and low $\pt$ jets to do the top events veto.
We define:
\begin{itemize}
\item {\it counted jet}: a reconstructed jets with $\pt > ~30 ~\GeV$ within $|\eta|<4.7$;
\item {\it low $\pt$ jet}: a reconstructed jets with $10~ < \pt < ~30~\GeV$ within $|\eta|<4.7$
\end{itemize}

We analyze the events separately based on the number of counted jets in the event.
In the events containing two or more jets, there are additional requirements. 
Our focus in these events is on VBF production where there are two energetic jets  
well separated in forward and backward regions without significant hadronic 
activities in the central region.  
Therefore, we reject events where there is a third jet with $\pt > 30 \GeV$ 
between the first and second energetic jets. In addition, events with more than 
3 jets are rejected because there can be more than 3 jets in background processes such as \ttbar 
while there are less jets in the signal process. 
