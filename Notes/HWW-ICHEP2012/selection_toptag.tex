Because the production cross-section is substantially higher than the
$\WW$ cross-section, top backgrounds pose a significant challenge.
To reduce the top background, we introduce two top tagging methods.
Both methods rely on the fact that top quarks decay to $Wb$ with
almost 100\% probability.

The first method vetoes events
containing soft muons from the $b$-quark decays.
The requirements used to select soft muons are:

\begin{itemize}
    \item $\pt > 3$ GeV;
    \item reconstructed as a TrackerMuon;
    \item meets $\mathrm{TMLastStationAngTight}$ muon id requirements;
    \item more than 5 tracker layers;
    \item the transverse impact parameter with respect to the Primary Vertex, $|d_{0}| < 0.2$~cm;
    \item the longitudinal impact parameter with respect to the Primary Vertex $|d_{z}| <0.2$~cm;
    \item non-isolated $({\rm{Iso}_{Total}}/{\pt}~>~0.1)$ if $\pt>20~\GeV$.
\end{itemize}

The second method uses standard $b$-jet tagging~\cite{HWW2011}.
In this method, we use different taggers for jets with $\pt>~30$ GeV and $\pt<~30$ GeV. 
For jets with $\pt>~30$ GeV, $\mathrm{jetBProbability}$~\cite{btag} algorithm is used with 
a discriminator cut value at 1.05. For jets with $\pt<~30$ GeV, 
$\mathrm{TrkCountingHighEff}$~\cite{btag} algorithm is used with a discriminator cut value 
at 2.1. Events containing jets tagged with these btaggers are rejected 
if discriminators are less than the cut values. These algorithms are applied to jets 
with the same definition as Section \ref{sec:sel_jets}, with the exception that 
we consider jets with $\Et ~>~10~\GeV$. The MVA-based jet identification which is applied 
for selecting jets reduces mistag rate and dependency on pile-up.
%This requirement is different with respect to~\cite{HWW2011}, 
By using the expected tagging efficiency for the two methods,
it is possible to estimate the residual top background after the vetoes
have been applied. This is described in detail in Section~\ref{sec:bkg_top}.
