 
To determine the efficiency of the dilepton triggers, 
we derive the efficiency of the requirements imposed on each leg separately.
This requires a modification to the tag and probe method described above in 
the case of the double electron and double muon triggers used in 2012.
In these cases the trigger objects are saved after the requirement that there are two valid objects, 
thus there is a 100\% correlation between the decision we can probe on each lepton.
This means that we must pick exactly one tag candidate for each event a priori, which we do 
randomly. 
If the randomly selected tag candidate meets the tight requirements then we are free to 
probe the other lepton.

In the 2012 dataset, the double electron and double muon triggers contain 
a final selection requirement on the $dZ$ of the vertices of both leptons, 
in addition to the per lepton requirements.  Thus we split the trigger efficiency 
measurement into two parts: the efficiency for the leading and trailing leg 
requirements with respect to the offline selection, and the efficiency of the 
last step given both legs have passed the previous steps.
While the $dZ$ cut was desgined to be highly efficient,
for technical reasons there was a roughly 15\% inefficiency in
the early part of the 2012A dataset for the double muon triggers.
In the full dataset, the inefficiency roughly 5\%.
In this analysis, the inefficiency is absorbed by the single triggers, 
leading to a negligible overall efficiency loss.

Because there are different requirements imposed on the two legs of the 
double electron trigger at level-1, we measure the efficiency of 
both legs separately.
The efficiency of the leading $p_T$ threshold leg with respect to an electron passing
offline selection is tabulated in 
Table \ref{tab:eff_ele_lead_dbl} of Appendix 
\ref{app:appendix_efficiency_trigger}. The efficiency for the trailing 
leg is given in Table \ref{tab:eff_ele_trail_dbl}. 
The efficiency of the single electron trigger with respect to
an electron passing offline selection is given in Table \ref{tab:eff_ele_sgl}.

The efficiency of the leading and trailing legs of the double muon trigger
is summarized in Tables \ref{tab:eff_muon_lead_dbl} and
\ref{tab:eff_muon_trail_dbl}. The efficiency of the single
muon trigger is given in Table \ref{tab:eff_muon_sgl}.

In the case of the $e\mu$ triggers, we assume that the leading
and trailing electron and muon legs can be modelled by the measurements
of those legs in the double electron and muon triggers. This assumption was cross 
checked in the 2011 analysis by comparing the model with
a direct measurement of the $e\mu$ trigger efficiency in
dilepton $t\bar{t}$ events requiring that the event has missing transverse
energy greater than $20$ \GeV.
The efficiency of 
the muon leg was measured using events passing the single electron trigger,
while the efficiency of the electron leg was measured using events passing the
single muon trigger. The results were found to be consistent
with the model based on the double trigger measurements within the 
statistical uncertainties. 

Having measured the per lepton trigger efficiencies 
and for the double and single trigger,
we compute the efficiency for dilepton events to be selected.
We do this by taking into account the two ways an event can be selected: 
the double trigger can pass or the double trigger can fail because one leg is bad
but the good leg can pass the single trigger.  If the double trigger fails
because of the $dZ$ cut, which is assumed to be uncorrelated with the per leg requirements,
then both legs may be eligible to pass the single trigger. 
If both legs are bad in the double trigger they will also both be bad in the single trigger
because the requirements of the single trigger are tighter than any single leg of the double trigger.
Thus taking into account combinatorics, the event efficiency $\varepsilon_{\ell\ell'}(p_T,\:\eta,\:p'_T,\:\eta')$
is given in Equation \ref{eqn:evteff}, where $\varepsilon_{S}(p_T,\:\eta)$ is the single 
lepton trigger efficiency,
$\varepsilon_{\mathrm{D,leading}}(p_T,\:\eta)$ is the efficiency of the leading leg of the 
appropriate double trigger, and $\varepsilon_{\mathrm{D,trailing}}(p_T,\:\eta)$ is the 
efficiency of the trailing leg of the appropriate double trigger, and $\varepsilon_{dZ}$ is the efficiency
of the $dZ$ cut in the double trigger.

\begin{eqnarray}
\label{eqn:evteff}
\varepsilon_{\ell\ell'}(p_T,\:\eta,\:p'_T,\:\eta') & = & 1 - [(1-\varepsilon_{dZ})\varepsilon_{\mathrm{D,leading}}(p_T,\:\eta)\varepsilon_{\mathrm{D,leading}}(p_T',\:\eta') \\
               &   & (1-\varepsilon_{\mathrm{D,leading}}(p_T,\:\eta))(1-\varepsilon_{\mathrm{D,leading}}(p_T',\:\eta')) \\
               &   & +~\varepsilon_{\mathrm{D,leading}}(p_T,\:\eta)(1-\varepsilon_{\mathrm{D,trailing}}(p_T',\:\eta')) \\
               &   & +~\varepsilon_{\mathrm{D,leading}}(p_T',\:\eta')(1-\varepsilon_{\mathrm{D,trailing}}(p_T,\:\eta))] \\
               &   & +~\varepsilon_{S}(p_T',\:\eta')(1-\varepsilon_{\mathrm{D,trailing}}(p_T,\:\eta)) \nonumber\\
               &   & +~\varepsilon_{S}(p_T,\:\eta)(1-\varepsilon_{\mathrm{D,trailing}}(p_T',\:\eta')) \\
               &   & +~(1-(1-\varepsilon_{S}(p_T,\:\eta))(1-\varepsilon_{S}(p_T',\:\eta'))(1-\varepsilon_{dZ})
\end{eqnarray}

The procedure of Equation \ref{eqn:evteff} is applied to simulated Higgs boson decays to obtain an event-by-event weight factor. We find a 
trigger efficiency with respect to the offline selection of around $99\%$ for WW events.
In the current analysis, we set the $\varepsilon_{dZ}$ term to one.  

