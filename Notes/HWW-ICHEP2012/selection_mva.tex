To make maximal use of the event information we have performed a multivariate analysis 
using a multivariate classifier based on the Boosted Decision Tree (BDT) technique. 
The BDT is implemented using the TMVA~\cite{tmva} toolkit and has been 
successfully applied in high energy physics to increase the 
statistical significance of a signal extraction
It requires less training than other multivariate classifiers and 
it is insensitive to the inclusion of poorly discriminating input variables.

To improve the search sensitivity, we apply a loose cut in addition to the 
$\WW$ preselection, based on the
maximum $\mll$ Table~\ref{tab:presel_tmva_analysis} and a cut on $\mt$: (80-$\mHi$) $\GeV$ to 
supress the $\dytt$ and $\wgamma$ backgrounds. 
In addition to the selection variables for the cut-based analysis, the multivariate signal extraction 
procedure uses the following ones: 
\begin{itemize}
\item $\Delta R_{\Lep\Lep}\equiv\sqrt{\deletall^2 + \delphill^2}$ between the leptons, 
with $\deletall$ the $\eta$ difference between the leptons, 
which has similar properties as $\delphill$
\item lepton flavors ($\mu\mu$, $ee$, $e\mu$ or $\mu e$ );
\item finally, for the 1-jet bin, the azimutal angles between the dilepton 
system and $\met$, and between the dilepton system and the 
highest $\pt$ jet, are included.
\end{itemize}

The training has been carried out separately in the 0-jet and 1-jet bins 
for different Higgs masses using the corresponding signal samples. We use a new 
training with respect to last year~\cite{HWW2011}, and the full shape of the 
classifier output is used as final discriminant variable. As a cross-check, we 
report the results using last year's training in 
App.~\ref{app:appendix_mll_bdt2011}. Two other shape-based approaches, using the $\mll$ mass 
distribution and a Matrix Element technique, are also reported in 
Apps.~\ref{app:appendix_mll_bdt2011} and~\ref{app:appendix_me}.

\begin{table}
\begin{center}
\begin{tabular}{|r|c|c|c|c|c|c|c|c|c|c|c|}
\hline
$\mHi~~~~~[\GeV]$   & [110-125) & [125-130] & (130,140] & 150 & 160 & 170 & 180 & 190 & 200 \\
\hline
$\mll<~~~[\GeV]$    &  70 &  80 &  90 & 100 & 100 & 100 & 110 & 120 & 130\\
\hline
\end{tabular}
%\vspace{0.5cm}
\begin{tabular}{|r|c|c|c|c|c|c|c|c|c|}
\hline
$\mHi~~~~~[\GeV]$    &  250 & 300 & 350 & 400 & 450 & 500 & 550 & 600 \\
\hline
$\mll<~~~[\GeV]$     &  250 & 300 & 350 & 400 & 450 & 500 & 550 & 600 \\
\hline
\end{tabular}
\caption{$\mll$ upper limit requirement as a function of the Higgs mass used to 
enrich the background datasets of signal-like events. These samples are employed 
in the training of the multivariate classifier used for the signal 
extraction.\label{tab:presel_tmva_analysis}}
\end{center}
\end{table}


Aside from the BDT based multivariate technique, we also employ a Matrix Element technique~\cite{MENote}. 
The Matrix Element method works by calculating the probability for each recorded
event to originate from a specific physics process.
This is done by comparing the differential cross sections predicted by Matrix Element 
calculations for the signal and background processes given the kinematic observables
on an event-by-event basis.
The discriminating power arises because the differential cross sections for 
signal and background events are largest in different regions of the available
kinematic phase space. More details are given in Appendix~\ref{app:appendix_me}. 

As a cross-check we also perform a shape-based analysis using a single kinematic observable 
chosen as the dilepton mass. 
