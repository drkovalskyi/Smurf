To make maximal use of the event information we have performed a multivariate analysis 
using a multivariate classifier based on the Boosted Decision Tree (BDT) technique. 
The BDT is implemented using the TMVA~\cite{tmva} toolkit and has been 
successfully applied in high energy physics to increase the 
statistical significance of a signal extraction
It requires less training than other multivariate classifiers and 
it is insensitive to the inclusion of poorly discriminating input variables.

In addition to the $\WW$ preselection, we apply a loose cut on the
maximum $\mll$ to enhance the signal-to-background ratio shown in 
Table~\ref{tab:presel_tmva_analysis} and a cut on $\mt$: (80-$\mHi$) $\GeV$ to 
supress the $\dytt$ and $\wgamma$ backgrounds. 
In addition to the selection variables for the cut-based analysis, the multivariate signal extraction 
procedure uses the following ones: 
\begin{itemize}
\item $\Delta R_{\Lep\Lep}\equiv\sqrt{\deletall^2 + \delphill^2}$ between the leptons, 
with $\deletall$ the $\eta$ difference between the leptons, 
which has similar properties as $\delphill$
\item lepton flavors ($\mu\mu$, $ee$, $e\mu$ or $\mu e$ );
\item finally, for the 1-jet bin, the azimutal angles between the dilepton 
system and $\met$, and between the dilepton system and the 
highest $\pt$ jet, are included.
\end{itemize}

The training has been carried out separately in the 0-jet and 1-jet bins 
for different Higgs masses using the corresponding signal samples. We use a new 
training with respect to last year~\cite{HWW2011}, and the full shape of the 
classifier output is used as final discriminant variable. As a cross-check, we 
report the results using last year's training in 
App.~\ref{app:appendix_mll_bdt2011}. Two other shape-based approaches, using the $\mll$ mass 
distribution and a Matrix Element technique, are also reported in 
Apps.~\ref{app:appendix_mll_bdt2011} and~\ref{app:appendix_me}.

\begin{table}
\begin{center}
\begin{tabular}{|r|c|c|c|c|c|c|c|c|c|c|c|}
\hline
$\mHi~~~~~[\GeV]$   & [110-125) & [125-130] & (130,140] & 150 & 160 & 170 & 180 & 190 & 200 \\
\hline
$\mll<~~~[\GeV]$    &  70 &  80 &  90 & 100 & 100 & 100 & 110 & 120 & 130\\
\hline
\end{tabular}
%\vspace{0.5cm}
\begin{tabular}{|r|c|c|c|c|c|c|c|c|c|}
\hline
$\mHi~~~~~[\GeV]$    &  250 & 300 & 350 & 400 & 450 & 500 & 550 & 600 \\
\hline
$\mll<~~~[\GeV]$     &  250 & 300 & 350 & 400 & 450 & 500 & 550 & 600 \\
\hline
\end{tabular}
\caption{$\mll$ upper limit requirement as a function of the Higgs mass used to 
enrich the background datasets of signal-like events. These samples are employed 
in the training of the multivariate classifier used for the signal 
extraction.\label{tab:presel_tmva_analysis}}
\end{center}
\end{table}
