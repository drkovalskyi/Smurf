\subsection{Data and MC Samples}

We use the datasets listed in Table~\ref{tab:DatasetsMC} for Monte Carlo (MC) and in Table~\ref{tab:DatasetsData} for data.
The pileup scenario used to produce the MC is PU\_S4 or PU\_S6, and is re-weighted
to match the observed pileup in the data analysed.
Because the 2011A and 2011B data taking periods have different pileup distributions,
they are considered separately.

The \gjets~data are recorded by using the prescaled photon triggers described in Table~\ref{tab:triggers_photon}.
Because the prescale depends on the $p_T$ threshold of the triggers, we re-weight the \gjets~data to take into
account the lowest available trigger prescale for each event to produce a consistent physical data sample.
At large MET there can be a significant contribution to the \gjets~data sample from processes with real MET.
These contributions are subtracted using MC according to the cross sections listed in Table~\ref{tab:DatasetsMC}.

The \dyll~data, i.e. same flavor dilepton events.  
We subtract the contribution from backgrounds that decay equally to $\mu e+e\mu$ (OF) as they do to $ee+\mu\mu$ (SF),
e.g. $\mathrm{WW}$ and $t\bar{t}$, by subtracting the $\mu e$ and $e\mu$ events from the same flavor sample.
The backgrounds from \W\Z~ and \Z\Z~ (\V\Z) events are subtracted using MC.

\begin{table}[!ht]
\begin{center}
{\footnotesize
\begin{tabular}{|c|c|c|}
\hline
 Dataset Description                     &   Primary Dataset Name   & cross-section (pb)\\
\hline
\hline
\multicolumn{3}{|c|}{Drell-Yan samples} \\
\hline
Z[20-inf] $\rightarrow ee$	  	 &   /DYToEE\_M-20\_CT10\_TuneZ2\_7TeV-powheg-pythia                   &  1666.0 \\
Z[20-inf] $\rightarrow \mu\mu$        	 &   /DYToMuMu\_M-20\_CT10\_TuneZ2\_7TeV-powheg-pythia                 &  1666.0 \\
\hline
\multicolumn{3}{|c|}{Dilepton background samples} \\
\hline
WZ                               	 &   /WZJetsTo3LNu\_TuneZ2\_7TeV-madgraph-tauola                       &  0.857 \\
ZZ                               	 &   /ZZ\_TuneZ2\_7TeV\_pythia6\_tauola/                               &  7.406 \\
\hline
\multicolumn{3}{|c|}{Photon samples} \\
\hline
$\gamma$ with 0$<$$p\_T$$<$15 \GeVc       &   /G\_Pt-0to15\_TuneZ2\_7TeV\_pythia6                                    & 8.42E+07 \\
$\gamma$ with 15$<$$p\_T$$<$30 \GeVc      &   /G\_Pt-15to30\_TuneZ2\_7TeV\_pythia6                                   & 1.72E+05 \\
$\gamma$ with 30$<$$p\_T$$<$50 \GeVc      &   /G\_Pt-30to50\_TuneZ2\_7TeV\_pythia6                                   & 1.67E+04 \\
$\gamma$ with 50$<$$p\_T$$<$80 \GeVc      &   /G\_Pt-50to80\_TuneZ2\_7TeV\_pythia6                                   & 2.72E+03 \\
$\gamma$ with 80$<$$p\_T$$<$120 \GeVc     &   /G\_Pt-80to120\_TuneZ2\_7TeV\_pythia6                                  & 4.47E+02 \\
$\gamma$ with 120$<$$p\_T$$<$170 \GeVc    &   /G\_Pt-120to170\_TuneZ2\_7TeV\_pythia6                                 & 8.42E+01 \\
$\gamma$ with 170$<$$p\_T$$<$300 \GeVc    &   /G\_Pt-170to300\_TuneZ2\_7TeV\_pythia6                                 & 2.26E+01 \\
$\gamma$ with 300$<$$p\_T$$<$470 \GeVc    &   /G\_Pt-300to470\_TuneZ2\_7TeV\_pythia6                                 & 1.49E+00 \\
$\gamma$ with 470$<$$p\_T$$<$800 \GeVc    &   /G\_Pt-470to800\_TuneZ2\_7TeV\_pythia6                                 & 1.32E-01 \\
$\gamma$ with 800$<$$p\_T$$<$1400 \GeVc   &   /G\_Pt-800to1400\_TuneZ2\_7TeV\_pythia6                                & 3.48E-03 \\
$\gamma$ with 1400$<$$p\_T$$<$1800 \GeVc  &   /G\_Pt-1400to1800\_TuneZ2\_7TeV\_pythia6                               & 1.27E-05 \\
$\gamma$ with $p\_T$$>$1800 \GeVc         &   /G\_Pt-1800\_TuneZ2\_7TeV\_pythia6                                     & 2.94E-07 \\
\hline
\multicolumn{3}{|c|}{Photon background samples} \\
\hline
W $\rightarrow$ $\ell\nu$           	 &   /WJetsToLNu\_TuneZ2\_7TeV-madgraph-tauola                         &  31314.0 \\
W+$\gamma$ $\rightarrow$ $\mu\nu\gamma$  &   /WGToMuNuG\_TuneZ2\_7TeV-madgraph                                   &  137.3 \\
W+$\gamma$ $\rightarrow$ $e\nu\gamma$    &   /WGToENuG\_TuneZ2\_7TeV-madgraph                                    &  137.3 \\
W+$\gamma$ $\rightarrow$ $\tau\nu\gamma$ &   /WGToTauNuG\_TuneZ2\_7TeV-madgraph-tauola                           &  137.3 \\
Z+$\gamma$ $\rightarrow$ $\nu\nu\gamma$  &   /ZGToNuNuG\_TuneZ2\_7TeV-madgraph                                   &  3.426 \\
Z+$\gamma$ $\rightarrow$ $\mu\mu\gamma$  &   /ZGToMuMuG\_TuneZ2\_7TeV-madgraph                                   &  41.975 \\
Z+$\gamma$ $\rightarrow$ $ee\gamma$      &   /ZGToEEG\_TuneZ2\_7TeV-madgraph                                     &  42.75 \\
Z+$\gamma$ $\rightarrow$ $\tau\tau\gamma$  &   /ZGToTauTauG\_TuneZ2\_7TeV-madgraph-tauola                        &  38.4375 \\
\hline
\end{tabular}
}
\caption{Summary of MC datasets used. Processed dataset name is either Summer11-PU\_S4\_START42\_V11-v*/AODSIM or Fall11-PU\_S6\_START42\_V14B-v*/AODSIM}
\label{tab:DatasetsMC}
\end{center}
\end{table}


\begin{table}[!ht]
\begin{center}
\begin{tabular}{|l|l|}
\hline
 Dataset Description                   &   Dataset Name   \\
\hline
\hline
\multicolumn{2}{|c|}{Dilepton Samples} \\
\hline
Run2011A DiElectron May10ReReco      &  /DoubleElectron/Run2011A-May10ReReco-v1/AOD \\
Run2011A DiMuon May10ReReco          &  /DoubleMu/Run2011A-May10ReReco-v1/AOD \\
Run2011A MuEl May10ReReco            &  /MuEG/Run2011A-May10ReReco-v1/AOD \\
Run2011A SingleMuon May10ReReco      &  /SingleMu/Run2011A-May10ReReco-v1/AOD \\
Run2011A SingleElectron May10ReReco  &  /SingleElectron/Run2011A-May10ReReco-v1/AOD   \\
\hline
Run2011A DiElectron PromptReco      &  /DoubleElectron/Run2011A-PromptReco-v4/AOD   \\
Run2011A DiMuon PromptReco          &  /DoubleMu/Run2011A-PromptReco-v4/AOD   \\
Run2011A MuEl PromptReco            &  /MuEG/Run2011A-PromptReco-v4/AOD   \\
Run2011A SingleMuon PromptReco      &  /SingleMu/Run2011A-PromptReco-v4/AOD   \\
Run2011A SingleElectron PromptReco  &  /SingleElectron/Run2011A-PromptReco-v4/AOD   \\
\hline
Run2011A DiElectron Aug05ReReco      &  /DoubleElectron/Run2011A-05Aug2011-v1/AOD \\
Run2011A DiMuon Aug05ReReco          &  /DoubleMu/Run2011A-05Aug2011-v1/AOD \\
Run2011A MuEl Aug05ReReco            &  /MuEG/Run2011A-05Aug2011-v1/AOD \\
Run2011A SingleMuon Aug05ReReco      &  /SingleMu/Run2011A-05Aug2011-v1/AOD \\
Run2011A SingleElectron Aug05ReReco  &  /SingleElectron/Run2011A-05Aug2011-v1/AOD   \\
\hline
Run2011A DiElectron PromptReco      &  /DoubleElectron/Run2011A-03Oct2011-v1/AOD   \\
Run2011A DiMuon PromptReco          &  /DoubleMu/Run2011A-03Oct2011-v1/AOD   \\
Run2011A MuEl PromptReco            &  /MuEG/Run2011A-03Oct2011-v1/AOD   \\
Run2011A SingleMuon PromptReco      &  /SingleMu/Run2011A-03Oct2011-v1/AOD   \\
Run2011A SingleElectron PromptReco  &  /SingleElectron/Run2011A-03Oct2011-v1/AOD   \\
\hline
Run2011B DiElectron PromptReco      &  /DoubleElectron/Run2011B-PromptReco-v1/AOD   \\
Run2011B DiMuon PromptReco          &  /DoubleMu/Run2011B-PromptReco-v1/AOD   \\
Run2011B MuEl PromptReco            &  /MuEG/Run2011B-PromptReco-v1/AOD   \\
Run2011B SingleMuon PromptReco      &  /SingleMu/Run2011B-PromptReco-v1/AOD   \\
Run2011B SingleElectron PromptReco  &  /SingleElectron/Run2011B-PromptReco-v1/AOD   \\
\hline
\multicolumn{2}{|c|}{Photon+Jets Samples} \\
\hline
Run2011A Photon May10ReReco         & /Photon/Run2011A-May10ReReco-v1/AOD \\
Run2011A Photon PromptReco          & /Photon/Run2011A-PromptReco-v4/AOD \\
Run2011A Photon May10ReReco         & /Photon/Run2011A-05Aug2011-v1/AOD \\
Run2011A Photon PromptReco          & /Photon/Run2011A-03Oct2011-v1/AOD \\
Run2011B Photon PromptReco          & /Photon/Run2011B-PromptReco-v1/AOD \\
\hline
\end{tabular}
\caption{Summary of data datasets used.}
\label{tab:DatasetsData}
\end{center}
\end{table}



%%%%%
\begin{table}[!ht]
\begin{center}
\begin{tabular} {l}
\hline
Trigger name \\
\hline 
\hline
HLT\_Photon50\_CaloIdVL\_IsoL \\
HLT\_Photon50\_CaloIdVL \\
HLT\_Photon75\_CaloIdVL\_IsoL \\
HLT\_Photon75\_CaloIdVL \\
HLT\_Photon90\_CaloIdVL\_IsoL \\
HLT\_Photon90\_CaloIdVL \\
\hline 
\end{tabular}
\caption{Photon triggers used for the $\met$ modelling of the $\dyll$ backgrounds. }
\label{tab:triggers_photon}
\end{center}
\end{table}
%%%%%

\clearpage

\subsection{Basic Event Selection}

\subsubsection{Photon Events}

Photons are selected in events that pass one of the triggers described in Table \ref{tab:triggers_photon}.
The photon selection criteria are based on the $\mathrm{V}\gamma$ selection \cite{ref:vgamma}:

\begin{itemize}
\item Photon in ECAL Barrel: $|\eta_\gamma|$$<$1.479
\item Photon R9 $<$ 0.9
\item Pixel Seed Veto
\item H/E $<$ 0.05
\item Shower shape: $\sigma_{i\eta i\eta}$ $<$ 0.011
\item Hollow cone track isolation: trkSumPtHollowConeDR04 $<$ 2.0 + 0.001$\cdot$$E_T$ + 0.0167$\cdot$$\rho_{25}$
\item Jurrasic ECAL Isolation: ecalRecHitSumEtConeDR04 $<$ 4.2 + 0.006$\cdot$$E_T$ + 0.183$\cdot$$\rho_{25}$
\item Tower-based HCAL Isolation: hcalTowerSumEtConeDR04 $<$ 2.2 + 0.0025$\cdot$$E_T$ + 0.062$\cdot$$\rho_{25}$
\item Additionally apply spike cleaning: $\sigma_{i\eta i\eta}$ $>$ 0.001 and $\sigma_{i\phi i\phi}$ $>$ 0.001
\end{itemize}

where $E_T$ is the photon transverse energy and $\rho_{25}$ is the energy density from kt6PFJets in the tracker region ($|\eta|$$<$2.5).

\subsubsection{Dilepton Events}

We use the same event selection as the 2011 Higgs boson search in the  $\mathrm{WW}\rightarrow\ell\nu\ell\nu$ channel.
This selection is described in more detail in Ref.~\ref{ref:hwwsmurfs}, and summarized here in brief:

\begin{enumerate}
    \item We select events that pass pre-defined lepton triggers.
    \item We then select those events with two oppositely charged 
    high $\pt$ isolated leptons ($ee$, $\mu\mu$, $e\mu$) requiring:
        \begin{itemize}    
            \item $\pt>20~\GeVc$ for the leading lepton;
            \item $\pt>15/10~\GeVc$ for the trailing lepton (same-flavor/opposite flavor);
            \item standard identification and isolation requirements on both leptons.
        \end{itemize}    
      \item We apply a common $\WW$ preselection, which requires in brief: 
         \begin{itemize}
             \item categorize events by the number of reconstructed jets;
             \item exactly two high $\pt$ isolated leptons;
             \item large transverse missing energy due to the neutrinos;
             \item no b-tagged jets.
          \end{itemize}
    \item Finally, we perform two \emph{Higgs mass dependent} event selections: one for cut based and one for BDT shape based analysis. 
\end{enumerate}
