Drell-Yan (\dyll) events represent the major background to a \hww~ signal in the same flavor final state.
After applying tight cuts on \met, they are highly suppressed but the expected signal yield is significantly 
reduced and the remaining \dyll~ background is difficult to estimate.
The net effect is that the sensitivity of the \hww~ analysis is dominated by the opposite flavor final state.

In the published 2011 analysis\cite{ref:hwwpaper}, the \dyll~ background is estimated using a method called \routin\cite{ref:hwwsmurfs}, 
which extrapolates to the signal region the \dyll~ yield in the $Z$ peak region. 
Results from the \routin method are generally stable but suffer from large statistical and systematic uncertainties.

In addition, the analysis relied on MC for deriving the shapes used in the BDT analysis; 
given that the Drell-Yan Monte Carlo sample with generator cut \mll$<$20 \GeVcc contained too few events and the statistical uncertainty 
in that region was too large, in the same flavor final state a \mll$>$ 20\GeVcc cut was applied and a significant fraction 
of a low-mass Higgs signal was lost.

The present note describes a new method for \dyll~estimation constituting a valid alternative 
to \routin, that can cross-check its results, provide shapes from data and possibly reduce the uncertainties.
The main idea is to use a ``fake-rate'' mathod, where the rate for \dyll\ events passing the final \met\ selection
is evaluated on a \gjets\ sample and applied to same flavor dilepton events in a loose \met\ region. 
