
The dilepton + MET final state can arise from decay of two $W$ bosons to leptons, with the neutrinos providing MET. In addition to Standard Model processes, this final state arises in many models of new physics.

If both leptons are required to have the same flavor ($e^{+}e^{-}$ or $\mu^{+}\mu^{-}$) then a large contribution may be introduced from the Drell-Yan process, $q\bar{q}\rightarrow l^{+}l^{-}$. In such events, fake MET can be generated by the mis-meausrement of one or both of the leptons or recoiling hadronic jets if present.
To reduce the contribution of the Drell-Yan (\dyll) background, events can be vetoed when the reconstructed dilepton invariant mass is within a window around the $Z$ pole, where the Drell-Yan cross section is enhanced. 

In the published 2011 Higgs boson search in the $\mathrm{WW}\rightarrow\ell\nu\ell\nu$ channel  \cite{ref:hwwsmurfs}, the \dyll background was estimated using the control region formed by the vetoed same flavor events close to the Z mass to set the normalisation for the \dyll simulation.  This method, referred to commonly as ``$R_{out/in}$'' was originally described in Ref \cite{ref:routin} and details specific to the HWW analysis can be found in Ref \cite{ref:hwwsmurfs}.  Because the control region used in the $R_{out/in}$ method is generally statistically limited as the signal MET cut is approached, it can suffer from large uncertainties in both data and simulation.

To maintain the stability and systematic uncertainties in the $R_{out/in}$ estimate, it was necessary to remove events with $m_{ll}<20$GeV in the 2011 analysis due to poor simulation coverage of this region.  This requirement implies an efficiency loss for low mass Higgs Boson decays.

This note describes an alternate method for \dyll estimation, which we refer to as the ``$\zeta$-method''.  This method is inspired by ``fake-rate'' methods, in which the efficiency to pass a tight numerator cut with respect to a looser denominator cut is measured in a control sample enriched in background.  This fake-rate is then applied to the background dominated denominator and not numerator selection in the signal sample.  This method is commonly used to estimate fake lepton backgrounds by extrapolating in lepton ID or isolation.  In the case considered in this note the variable of interest is the MET.  We first evaluate the efficiency to pass the signal MET cut given a loose denominator cut in a photon+jets control sample, and then apply this efficiency to the \dyll dominated dilepton events that pass the denominator cut and fail the numerator cut.
