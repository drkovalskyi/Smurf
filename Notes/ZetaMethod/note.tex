\documentclass{cmspaper}
\usepackage{graphicx}
\usepackage{subfigure}
\usepackage{amsmath}
\usepackage{amssymb}
\usepackage[pdfborder=0 0 0,
            colorlinks,
            urlcolor = blue,
            linkcolor = black,
            citecolor = black,
            menucolor = black,]
           {hyperref}
%% \usepackage[colorlinks]{hyperref}
%% \usepackage{url}
\usepackage[toc,page]{appendix}
\renewcommand{\appendixname}{Appendix}
%% \renewcommand{\appendixtocname}{List of appendices}

\input{commands}
% useful definitions

% processes
\def\dyee {\ensuremath{Z/\gamma^*\to ee}}
\def\dymm {\ensuremath{Z/\gamma^*\to\mu\mu}}
\def\dytt {\ensuremath{Z/\gamma^*\to\tau\tau}}
\def\zee {\ensuremath{Z\to ee}}
\def\zmm {\ensuremath{Z\to\mu\mu}}
\def\ztt {\ensuremath{Z\to\tau\tau}}
\def\ttbar {\ensuremath{t\bar{t}}}
\def\wwll {\ensuremath{WW\to l^+l^-}}
\def\wwlulu{\ensuremath{WW\to l^+\nu l^-\bar{\nu}}}
\def\ww {\ensuremath{WW}}
\def\wz{\ensuremath{WZ}}
\def\zz{\ensuremath{ZZ}}
\def\wgamma{\ensuremath{W\gamma}}
\def\wjets{\ensuremath{W+}jets} 
\def\tw{\ensuremath{tW}} 
\def\singletopt{\ensuremath{t} ($t$-chan)} 
\def\singletops{\ensuremath{t} ($s$-chan)} 
\def\all{all}
\def\ee{\ensuremath{ee}}
\def\emu{\ensuremath{e\mu}}
\def\mm{\ensuremath{\mu\mu}}

%units

%others
\def\pt{\ensuremath{p_T}}
\def\ipb{pb\ensuremath{^{-1}}}
\def\ifb{fb\ensuremath{^{-1}}}
\def\et{\ensuremath{E_T}}
\def\met{\ensuremath{E\!\!\!\!/_T}}
\def\fBrem{\ensuremath{f_{\rm brem}}}
\def\pin{\ensuremath{p_{\rm in}}}
\def\pout{\ensuremath{p_{\rm out}}}


\setcounter{topnumber}{1}
\setcounter{bottomnumber}{1}

\begin{document}
\begin{titlepage}

  \analysisnote{2012/XXX}

  \date{\today}

  \title{Estimating the Drell-Yan Contribution to a Dilepton + MET Selection using Photon+Jets Events
  in HWW Analysis}
  
  \input{authors}

  \begin{abstract}
The selection of events containing dileptons + MET is important for both Standard Model measurements and searches for new physics. If both leptons have the same flavour, then a background may be introduced from the Drell-Yan process where the MET arises due to instrumental mis-measurement.  A data driven method to estimate this background by using a photon+jets control region is described in this note in the context of the Higgs boson search in the {$\mathrm{WW}\rightarrow\ell\nu\ell\nu$} channel.
   \end{abstract} 

\end{titlepage}
\tableofcontents
\listoftables
\listoffigures
\newpage 

\section{Introduction}
   \label{sec:introduction}
   Drell-Yan (\dyll) events represent the major background to a \hww~ signal in the same flavor final state.
After applying tight cuts on \met, they are highly suppressed but the expected signal yield is significantly 
reduced and the remaining \dyll~ background is difficult to estimate.
The net effect is that the sensitivity of the \hww~ analysis is dominated by the opposite flavor final state.

In the published 2011 analysis\cite{ref:hwwpaper}, the \dyll~ background is estimated using a method called \routin\cite{ref:hwwsmurfs}, 
which extrapolates to the signal region the \dyll~ yield in the $Z$ peak region. 
Results from the \routin method are generally stable but suffer from large statistical and systematic uncertainties.

In addition, the analysis relied on MC for deriving the shapes used in the BDT analysis; 
given that the Drell-Yan Monte Carlo sample with generator cut \mll$<$20 \GeVcc contained too few events and the statistical uncertainty 
in that region was too large, in the same flavor final state a \mll$>$ 20\GeVcc cut was applied and a significant fraction 
of a low-mass Higgs signal was lost.

The present note describes a new method for \dyll~estimation constituting a valid alternative 
to \routin, that can cross-check its results, provide shapes from data and possibly reduce the uncertainties.
The main idea is to use a ``fake-rate'' mathod, where the rate for \dyll\ events passing the final \met\ selection
is evaluated on a \gjets\ sample and applied to same flavor dilepton events in a loose \met\ region. 

\section{Samples and Selections}
  \label{sec:datasets}
  %UPDATEME%
The datasets used for this analysis are summarized in 
Tables~\ref{tab:DatasetsData} and~\ref{tab:DatasetsMC} for data and Monte 
Carlo, respectively. The total integrated luminosity is \intlumiEightTeV. 
We used the official good run list~\cite{json}. For Monte Carlo simulation 
we use madgraph when possible, but different generators such as Pythia and Powheg~\cite{powheg} 
are also used.  For $gg \to \WW$ a dedicated generator is used. For \wz\ and \zz\
processes we use Pythia, since MadGraph samples are mixed with $\WW$ in
a single $VV$ sample, which is difficult to use properly.

\begin{table}[!ht]
\begin{center}
\begin{tabular}{|c|c|}
\hline
 Dataset Description                   &   Dataset Name   \\
\hline \hline
\multirow{2}{*}{MuEl PromptReco}   		&  /MuEG/Run2012A-PromptReco-v1/AOD   \\
            							&  /MuEG/Run2012B-PromptReco-v1/AOD   \\
\multirow{2}{*}{DiMuon PromptReco}     	&  /DoubleMu/Run2012A-PromptReco-v1/AOD   \\
          								&  /DoubleMu/Run2012B-PromptReco-v1/AOD   \\
\multirow{2}{*}{DiElectron PromptReco} 	&  /DoubleElectron/Run2012A-PromptReco-v1/AOD   \\
      									&  /DoubleElectron/Run2012B-PromptReco-v1/AOD   \\
\multirow{2}{*}{SingleMuon PromptReco}  &  /SingleMu/Run2012A-PromptReco-v1/AOD   \\
      									&  /SingleMu/Run2012B-PromptReco-v1/AOD   \\
\multirow{2}{*}{SingleElectron PromptReco} 	&  /SingleElectron/Run2012A-PromptReco-v1/AOD   \\
      										&  /SingleElectron/Run2012B-PromptReco-v1/AOD   \\
\hline
\end{tabular}
\caption{Summary of data datasets used.\label{tab:DatasetsData}}
\end{center}
\end{table}

\begin{table}[!ht]
\begin{center}
{\footnotesize
\begin{tabular}{|c|c|c|}
\hline
\multicolumn{3}{|c|}{With Pileup: Processed dataset name is always} \\
\multicolumn{3}{|c|}{/Summer12-PU\_S7\_START52\_V9-v*/AODSIM} \\
\hline
 Dataset Description              		&   Primary Dataset Name   & cross-section (pb)\\
\hline
$\ttbar$                              	&   /TTJets\_TuneZ2star\_8TeV-madgraph-tauola                          	& 	225.2 	\\
tW                  	 	 			&   /T\_tW-channel-DR\_TuneZ2star\_8TeV-powheg-tauola                  	&  	11.18 	\\
$\bar{\textrm{t}}$W                   	&   /Tbar\_tW-channel-DR\_TuneZ2star\_8TeV-powheg-tauola               	&  	11.18 	\\
gg $\rightarrow WW \to 2l 2\nu$         &   /GluGluToWWTo4L\_TuneZ2star\_8TeV-gg2ww-pythia6                     &   1.74	\\
qq $\rightarrow WW$                  	&   /WWJetsTo2L2Nu\_TuneZ2star\_8TeV-madgraph-tauola                    &  	5.81  	\\
WZ                               	 	&   /WZ\_TuneZ2star\_8TeV\_pythia6\_tauola                        		&  	22.45 	\\
Z[10-50] 	  	 						&   /DYJetsToLL\_M-10To50filter\_8TeV-madgraph                   		&  	860.5 	\\
Z[50-inf] 	  	 						&   /DYJetsToLL\_M-50\_TuneZ2Star\_8TeV-madgraph-tarball           		&  	3532.8 	\\
ZZ $\rightarrow 2l 2\nu$    	 		& 	/ZZJetsTo2L2Nu\_TuneZ2star\_8TeV-madgraph-tauola                    &   0.365	\\
ZZ $\rightarrow 2l 2q$    	 			&   /ZZJetsTo2L2Q\_TuneZ2star\_8TeV-madgraph-tauola                     &   1.28	\\
ZZ $\rightarrow 4l$    	 				&   /ZZJetsTo4L\_TuneZ2star\_8TeV-madgraph-tauola                       &   0.0921	\\
$gg \to H \to WW \to 2l 2\nu$         	&   /GluGluToHToWWTo2LAndTau2Nu\_M-*\_8TeV-powheg-pythia6             	& 	vary 	\\
$qqH,~H \to WW \to 2l 2\nu$           	&   /VBF\_HToWWTo2LAndTau2Nu\_M-*\_8TeV-powheg-pythia6                 	& 	vary 	\\
$WH/ZH/\ttbar H,~H\to WW$              	&   /WH\_ZH\_TTH\_HToWW\_M-*\_8TeV-pythia6                            	& 	vary 	\\
\hline
\hline
\end{tabular}
}
\caption{Summary of Monte Carlo datasets used.\label{tab:DatasetsMC}. The cross sections for a SM Higgs boson
is taken from the LHC Higgs cross-section working group~\cite{LHCHiggsCrossSectionWorkingGroup:2011ti}}
\end{center}
\end{table}

Since some portion of data datasets were not processed, 
we use a json excluding missing events~\cite{json}. 
Last year we adjusted Higgs $\pt$ spectrum because 
the spectrum in the simulation (POWHEG 2.0) was harder than the one 
in the most precise calculation to NNLO with resummation to NNLL order.
In the new version of Powheg (POWHEG 2.1), this problem has been mostly resolved,
thus we do not apply any corrections for Higss $\pt$.

\section{The \zm~ Method}
   \label{sec:method}
   \subsection{Method definition}

The basic idea of the \zm~method is to use a ``fake-rate'' method~\cite{fakeLeptonNote1,fakeLeptonNote2} based on \met.

Given a background process $B$ and a particular cut $c$ in the analysis suppressing $B$, a fake-rate method
estimates the yield of $B$ passing $c$ as the number of $B$ events failing $c$  
times a pass-to-fail ratio $R_{p-f}$
\begin{equation}
N_B\left(\mathrm{pass}\right)=N_B\left(\mathrm{fail}\right) \cdot R_{p-f}~,
\end{equation}
where $R_{pf}$ has to measured in an independent control sample 
reproducing with good accuracy the behavior of $B$ with respect to $c$.

Assuming that both in photon and \dyll~ events the sources of \met are the hadronic recoil against and pile-up effects, 
the \zm~method uses the \gjets~ sample to model the \met distribution in \dyll~events.
It defines a pass-to-fail ratio (\zm$\equiv$$R_{p-f}$) in bins of $N_{jets}$ and $p_{T}\left(\gamma\right)$:
\begin{equation}
\zeta\left(N_{jets},p_{T}\left(\gamma\right)\right)=\frac{N\left(N_{jets},p_{T}\left(\gamma\right), \met>\mwp\right)}{N\left(N_{jets},p_{T}\left(\gamma\right),\met<\mwp\right)}
\end{equation}
where $\mwp$ represents the working point chosen for the final \met~cut in the analyis.

In the same flavor dilepton sample, events passing the final analysis
selection but with $\met<\mwp$ are selected and \zm~is used to extrapolate to the signal region:
\begin{equation}
N_{\dyll}\left(\met>\mwp,N_{jets},p_{T}\left(\dil\right)\right)= \zeta\left(N_{jets},p_{T}\left(\dil\right)\right) \cdot N_{\dyll}\left(\met<\mwp,N_{jets},p_{T}\left(\dil\right)\right)
\end{equation}
where, in data, $N_{\dyll}\left(\met<\mwp\right)$ is computed subtracting opposite flavor and \V\Z~ events.

A corollary of the method is that shapes can be obtained from data by filling the histogram with \dyll~ events in the \met$<$\mwp~region 
with weight \zm~ (the same procedure is already used in 2011 analysis for the \Wjets~ background\cite{ref:shapenote}).

\subsection{The \hww~case}

In the \hww~analysis we consider events with dilepton $p_T$$>$45 \GeVc~and $\mathrm{min}$-$\mathrm{pmet}$$>$20 \GeV~ as a baseline common to same and opposite flavor final state~\cite{ref:hwwsmurfs};
$\mathrm{min}$-$\mathrm{pmet}$ is defined as:
\begin{equation}
\text{min-pmet} = \text{min(proj-pfMet,proj-trackMet)} ,
\end{equation}
where ``pfMet'' is the \met~reconstructed with the particle flow algorithm, ``trackMet'' is the \met~ constructed from charged particles consistent 
with originating from the primary vertex.
The ``proj-met'' variable (where met is either pfMet or trackMet) is defined as:
\begin{equation}
\text{proj-met} = 
\begin{cases} \met & \text{if $\Delta\phi_{min}>\frac{\pi}{2}$,}
\\
\met\sin(\Delta\phi_{min}) & \text{if $\Delta\phi_{min}<\frac{\pi}{2}$}
\end{cases}
\end{equation}
\begin{equation}
\text{with } \Delta\phi_{min} =  min(\Delta\phi(\ell_1,\met),\Delta\phi(\ell_2,\met))\\
\end{equation}
where $\Delta\phi(\ell_i,\met)$ is the angle between \met\ and lepton $i$ in the transverse plane;
the main purpose for using ``proj-met'' is to reduce the impact of lepton mismeasurement and, secondarily, to further suppress the contribution from \dytt.
The final \met~ selection in the same flavor dilepton final state requires min-pmet$>$(37+nvtx/2) \GeV. 

In the photon sample this selection is reproduced by cutting on min-met=$\mathrm{min(pfMet,trackMet)}$ and photon $p_T$.
In photon events, ``proj-met'' variables cannot be defined because is not possible to project on lepton directions, so the \zm~ method will not 
predict the contribution from fake \met~due to lepton mismesurement. 
However, as it shown in Figures \ref{fig:met_0j}-\ref{fig:met_1j}, despite the slightly different definiton, the photon sample reproduces with 
good precision the \met in the \dyll~sample.

Is summary, we use the following definitions:
\begin{itemize}
\item dilepton sample pass region: min-pmet$>$(37+nvtx/2) \GeV, dilepton $p_T$$>$45 \GeVc
\item dilepton sample fail region: 20$<$min-pmet$<$(37+nvtx/2) \GeV, dilepton $p_T$$>$45 \GeVc
\item photon sample pass region: min-met$>$(37+nvtx/2) \GeV, photon $p_T$$>$45 \GeVc
\item photon sample fail region: 20$<$min-met$<$(37+nvtx/2) \GeV, photon $p_T$$>$45 \GeVc
\end{itemize}

%%%%%%%%
\begin{figure}[!hbtp]
\begin{center}
\subfigure[proj-pfMet in \dyll~ sample]{\label{subfig:pmet_0j}
\includegraphics[width=.4\textwidth]{figures/pmet_0j.png}}
\subfigure[pfMet in \gjets~ sample]{\label{subfig:met_0j}
\includegraphics[width=.4\textwidth]{figures/met_0j.png}}\\
\subfigure[proj-trackMet in \dyll~ sample]{\label{subfig:pTrackMet_0j}
\includegraphics[width=.4\textwidth]{figures/pTrackMet_0j.png}}
\subfigure[trackMet in \gjets~ sample]{\label{subfig:trackMet_0j}
\includegraphics[width=.4\textwidth]{figures/trackMet_0j.png}}
\caption{\met in the 0-jet bin. 
For visualization purposes, the \gjets~ and \dyll~contributions from MC are scaled by a had-hoc scale factor to macth data in the bulk of the distribution.
In the \gjets~ sample, data is shown after re-weighting to the \dyll~$p_T$ distribution in MC events.}
\label{fig:met_0j}
\end{center}
\end{figure}
%%%%%%%%

%%%%%%%%
\begin{figure}[!hbtp]
\begin{center}
\subfigure[proj-pfMet in \dyll~ sample]{\label{subfig:pmet_1j}
\includegraphics[width=.4\textwidth]{figures/pmet_1j.png}}
\subfigure[pfMet in \gjets~ sample]{\label{subfig:met_1j}
\includegraphics[width=.4\textwidth]{figures/met_1j.png}}\\
\subfigure[proj-trackMet in \dyll~ sample]{\label{subfig:pTrackMet_1j}
\includegraphics[width=.4\textwidth]{figures/pTrackMet_1j.png}}
\subfigure[trackMet in \gjets~ sample]{\label{subfig:trackMet_1j}
\includegraphics[width=.4\textwidth]{figures/trackMet_1j.png}}
\caption{\met in the 1-jet bin. 
For visualization purposes, the \gjets~ and \dyll~contributions from MC are scaled by a had-hoc scale factor to macth data in the bulk of the distribution.
In the \gjets~ sample, data is shown after re-weighting to the \dyll~$p_T$ distribution in MC events.}
\label{fig:met_1j}
\end{center}
\end{figure}
%%%%%%%%

%%%%%%%%
\begin{figure}[!hbtp]
\begin{center}
\subfigure[proj-pfMet in \dyll~ sample]{\label{subfig:pmet_2j}
\includegraphics[width=.4\textwidth]{figures/pmet_2j.png}}
\subfigure[pfMet in \gjets~ sample]{\label{subfig:met_2j}
\includegraphics[width=.4\textwidth]{figures/met_2j.png}}\\
\subfigure[proj-trackMet in \dyll~ sample]{\label{subfig:pTrackMet_2j}
\includegraphics[width=.4\textwidth]{figures/pTrackMet_2j.png}}
\subfigure[trackMet in \gjets~ sample]{\label{subfig:trackMet_2j}
\includegraphics[width=.4\textwidth]{figures/trackMet_2j.png}}
\caption{\met in the 2-jet bin. 
For visualization purposes, the \gjets~ and \dyll~contributions from MC are scaled by a had-hoc scale factor to macth data in the bulk of the distribution.
In the \gjets~ sample, data is shown after re-weighting to the \dyll~$p_T$ distribution in MC events.}
\label{fig:met_2j}
\end{center}
\end{figure}
%%%%%%%%

\clearpage

\subsection{Evaluation of \zm}

For each jet bin, \zm~ is evaluated both on MC and data samples in bins of photon $p_T$.
In data, the expected backgrounds with non-fake \met~ are subtracted from the photon sample.
Results for WW level in  are shown in Figures~\ref{fig:zeta_MC} and~\ref{fig:zeta_mh0}.
These are the values used for the \hww~ BDT shape analysis.

In the cut based analysis, an additional narrow cut on the transverse Higgs mass is applied~\cite{ref:hwwsmurfs}, where the following definition of $m_T$ is used:
\begin{equation}
m_{T} = \sqrt{2\pt^{ll}\met(1-cos(\Delta\phi_{\ell\ell-\met}))}
\end{equation}
where $\Delta\phi_{\ell\ell-\met}$ is the angle between dilepton direction and \met\ in the transverse plane.
For example, in the \mHi=120 \GeVcc~ analysis the signal region requires 70$<$$m_T$$<$120 \GeVcc.
Given that \met\ and $m_T$ are correlated variables, the \zm\ value depends on the $m_T$ cut value; 
therefore, \zm\ is derived separately for each Higgs mass analysis applying the corresponding $m_T$ cut on the photon sample.

A few representative results for \zm~derived from data for Higgs level cut based analysis are shown in Figures~\ref{fig:zeta_mh120}-\ref{fig:zeta_mh160}.

%%%%%%%%
\begin{figure}[!hbtp]
\begin{center}
\subfigure[0-jet]{\label{subfig:zeta_MC_0j_minmet}
\includegraphics[width=.4\textwidth]{figures/zeta_MC_0j_minmet.png}}
\subfigure[1-jet]{\label{subfig:zeta_MC_1j_minmet}
\includegraphics[width=.4\textwidth]{figures/zeta_MC_1j_minmet.png}}\\
\subfigure[2-jet]{\label{subfig:zeta_MC_2j_minmet}
\includegraphics[width=.4\textwidth]{figures/zeta_MC_2j_minmet.png}}
\caption{\zm~in MC at \WW~level.}
\label{fig:zeta_MC}
\end{center}
\end{figure}
%%%%%%%%

%%%%%%%%
\begin{figure}[!hbtp]
\begin{center}
\subfigure[0-jet]{\label{subfig:zeta_mass0_0j_minmet}
\includegraphics[width=.4\textwidth]{figures/zeta_mass0_0j_minmet.png}}
\subfigure[1-jet]{\label{subfig:zeta_mass0_1j_minmet}
\includegraphics[width=.4\textwidth]{figures/zeta_mass0_1j_minmet.png}}\\
\subfigure[2-jet]{\label{subfig:zeta_mass0_2j_minmet}
\includegraphics[width=.4\textwidth]{figures/zeta_mass0_2j_minmet.png}}
\caption{\zm~in data at \WW~level.}
\label{fig:zeta_mh0}
\end{center}
\end{figure}
%%%%%%%%

%%%%%%%%
\begin{figure}[!hbtp]
\begin{center}
\subfigure[0-jet]{\label{subfig:zeta_mass120_0j_minmet}
\includegraphics[width=.4\textwidth]{figures/zeta_mass120_0j_minmet.png}}
\subfigure[1-jet]{\label{subfig:zeta_mass120_1j_minmet}
\includegraphics[width=.4\textwidth]{figures/zeta_mass120_1j_minmet.png}}\\
\subfigure[2-jet]{\label{subfig:zeta_mass120_2j_minmet}
\includegraphics[width=.4\textwidth]{figures/zeta_mass120_2j_minmet.png}}
\caption{\zm~in data for \mHi=120 \GeVcc.}
\label{fig:zeta_mh120}
\end{center}
\end{figure}
%%%%%%%%

%%%%%%%%
\begin{figure}[!hbtp]
\begin{center}
\subfigure[0-jet]{\label{subfig:zeta_mass140_0j_minmet}
\includegraphics[width=.4\textwidth]{figures/zeta_mass140_0j_minmet.png}}
\subfigure[1-jet]{\label{subfig:zeta_mass140_1j_minmet}
\includegraphics[width=.4\textwidth]{figures/zeta_mass140_1j_minmet.png}}\\
\subfigure[2-jet]{\label{subfig:zeta_mass140_2j_minmet}
\includegraphics[width=.4\textwidth]{figures/zeta_mass140_2j_minmet.png}}
\caption{\zm~in data for \mHi=140 \GeVcc.}
\label{fig:zeta_mh140}
\end{center}
\end{figure}
%%%%%%%%

%%%%%%%%
\begin{figure}[!hbtp]
\begin{center}
\subfigure[0-jet]{\label{subfig:zeta_mass160_0j_minmet}
\includegraphics[width=.4\textwidth]{figures/zeta_mass160_0j_minmet.png}}
\subfigure[1-jet]{\label{subfig:zeta_mass160_1j_minmet}
\includegraphics[width=.4\textwidth]{figures/zeta_mass160_1j_minmet.png}}\\
\subfigure[2-jet]{\label{subfig:zeta_mass160_2j_minmet}
\includegraphics[width=.4\textwidth]{figures/zeta_mass160_2j_minmet.png}}
\caption{\zm~in data for \mHi=160 \GeVcc.}
\label{fig:zeta_mh160}
\end{center}
\end{figure}
%%%%%%%%

\clearpage

\section{Closure tests}
   \label{sec:closure}
   The performance of the \zm\ method are evaluated in the following cases:
\begin{itemize}
\item on MC and on data;
\item inside and outside the Z peak;
\item at \WW\ level and after \mHi=120 \GeVcc\ selection.
\end{itemize}

The selection applied includes all \hww\ analysis cuts, except that, in order to increase the \dyll\ sample, 
the $\Delta\phi$(ll,jet1)$<$165$^\circ$ cut and the VBF selection in the 2-jet bin are not applied.
When considering  \mHi=120 \GeVcc\ selection, additional cuts are: dilepton mass $<$40 \GeVcc (not applied for tests under Z peak), 
$\Delta\phi$($l_1$,$l_2$)$<$115$^\circ$ and 70$<$$m_T$$<$120 \GeVcc.
In data, \dyll\ is always considered after \V\Z\ and opposire flavor subtraction.


Tables~\ref{tab:mc_closure_ww_zv}-\ref{tab:data_closure_120_zv} summarize the result of the performed closure tests.
We define ``Observed'' as the \dyll\ yield in the pass region; ``Predicted'' as the \dyll\ yield extrapolating from the fail region using the \zm\ method;
``Bias'' as (Predicted-Observed)/Observed.
These results suggest a conservative systematic error of 40\% per channel (ee and $\mu\mu$).

%%%%%
\begin{table}[!ht]
\begin{center}
\begin{tabular} {|c|ccc|}
\hline
$N_{jets}$  & Observed & Predicted & Bias (\%) \\
\hline 
\hline
\multicolumn{4}{|c|}{di-muon final state} \\
\hline
0 & 10.0$\pm$2.9 &  7.8$\pm$2.0 & -22$\pm$30 \\
1 & 42.3$\pm$3.8 & 41.6$\pm$1.4 &  -2$\pm$11 \\
2 & 20.7$\pm$3.2 & 19.1$\pm$0.9 &  -8$\pm$15 \\
\hline 
\hline
\multicolumn{4}{|c|}{di-electron final state} \\
\hline
0 &  6.7$\pm$2.4 &  4.3$\pm$1.2 & -36$\pm$29 \\
1 & 28.9$\pm$3.8 & 23.9$\pm$0.9 & -17$\pm$11 \\
2 &  9.0$\pm$1.2 & 10.8$\pm$0.6 &  21$\pm$18 \\
\hline 
\end{tabular}
\caption{Closure test on MC at WW level with Z veto.}
\label{tab:mc_closure_ww_zv}
\end{center}
\end{table}
%%%%%



%%%%%
\begin{table}[!ht]
\begin{center}
\begin{tabular} {|c|ccc|}
\hline
$N_{jets}$  & Observed & Predicted & Bias (\%) \\
\hline 
\hline
\multicolumn{4}{|c|}{di-muon final state} \\
\hline
0 &  6.0$\pm$2.4 &  4.5$\pm$1.3 & -24$\pm$38 \\
1 & 14.5$\pm$3.5 & 12.7$\pm$1.0 & -13$\pm$22 \\
2 &  3.9$\pm$2.3 &  4.0$\pm$0.4 &   1$\pm$60 \\
\hline 
\hline
\multicolumn{4}{|c|}{di-electron final state} \\
\hline
0 & 4.3$\pm$2.3 & 2.3$\pm$0.9 & -47$\pm$35 \\
1 & 5.9$\pm$2.4 & 6.6$\pm$0.6 &  12$\pm$47 \\
2 & 1.0$\pm$0.5 & 1.7$\pm$0.2 &  67$\pm$86 \\
\hline 
\end{tabular}
\caption{Closure test on MC after \mHi=120 \GeVcc\ selection with Z veto.}
\label{tab:mc_closure_120_zv}
\end{center}
\end{table}
%%%%%

%%%%%
\begin{table}[!ht]
\begin{center}
\begin{tabular} {|c|ccc|}
\hline
$N_{jets}$  & Observed & Predicted & Bias (\%) \\
\hline 
\hline
\multicolumn{4}{|c|}{di-muon final state} \\
\hline
0 & 117$\pm$16 & 183$\pm$36 & 56$\pm$38 \\
1 & 977$\pm$33 & 963$\pm$53 & -1$\pm$6  \\
2 & 537$\pm$24 & 582$\pm$28 &  8$\pm$7 \\
\hline 
\hline
\multicolumn{4}{|c|}{di-electron final state} \\
\hline
0 &  95$\pm$13 & 106$\pm$22 &  12$\pm$28 \\
1 & 717$\pm$28 & 623$\pm$33 & -13$\pm$6  \\
2 & 468$\pm$22 & 384$\pm$18 & -18$\pm$5  \\
\hline 
\end{tabular}
\caption{Closure test on data at WW level in Z peak.}
\label{tab:data_closure_ww_zp}
\end{center}
\end{table}
%%%%%

%%%%%
\begin{table}[!ht]
\begin{center}
\begin{tabular} {|c|ccc|}
\hline
$N_{jets}$  & Observed & Predicted & Bias (\%) \\
\hline 
\hline
\multicolumn{4}{|c|}{di-muon final state} \\
\hline
0 &  44$\pm$25 &  47$\pm$10 &  7$\pm$7 \\
1 & 200$\pm$23 & 211$\pm$17 &  5$\pm$2 \\
2 & 128$\pm$18 & 123$\pm$10 & -4$\pm$2 \\
\hline 
\hline
\multicolumn{4}{|c|}{di-electron final state} \\
\hline
0 &  40$\pm$18 &  30$\pm$7 & -26$\pm$38 \\
1 & 168$\pm$18 & 118$\pm$8 & -29$\pm$9  \\
2 &  88$\pm$14 &  76$\pm$6 & -14$\pm$15 \\
\hline 
\end{tabular}
\caption{Closure test on data at WW level with Z veto.}
\label{tab:data_closure_ww_zv}
\end{center}
\end{table}
%%%%%

%%%%%
\begin{table}[!ht]
\begin{center}
\begin{tabular} {|c|ccc|}
\hline
$N_{jets}$  & Observed & Predicted & Bias (\%) \\
\hline 
\hline
\multicolumn{4}{|c|}{di-muon final state} \\
\hline
0 &  68$\pm$9  &  61$\pm$14 & -10$\pm$24 \\
1 & 250$\pm$16 & 298$\pm$32 &  19$\pm$15 \\
2 &  97$\pm$10 & 143$\pm$19 &  47$\pm$24 \\
\hline 
\hline
\multicolumn{4}{|c|}{di-electron final state} \\
\hline
0 &  42$\pm$7  &  35$\pm$8  & -17$\pm$24 \\
1 & 168$\pm$13 & 190$\pm$20 &  13$\pm$15 \\
2 &  95$\pm$10 &  91$\pm$10 &  -4$\pm$15 \\
\hline 
\end{tabular}
\caption{Closure test on data after \mHi=120 \GeVcc\ selection in Z peak.}
\label{tab:data_closure_120_zp}
\end{center}
\end{table}
%%%%%

%%%%%
\begin{table}[!ht]
\begin{center}
\begin{tabular} {|c|ccc|}
\hline
$N_{jets}$  & Observed & Predicted & Bias (\%) \\
\hline 
\hline
\multicolumn{4}{|c|}{di-muon final state} \\
\hline
0 & 25$\pm$10 & 22$\pm$5  & -13$\pm$42 \\
1 & 62$\pm$9  & 72$\pm$13 &  17$\pm$24 \\
2 & 42$\pm$8  & 35$\pm$8  & -18$\pm$24 \\
\hline 
\hline
\multicolumn{4}{|c|}{di-electron final state} \\
\hline
0 &  0$\pm$6 &  7$\pm$2 &    nan    \\
1 & 30$\pm$6 & 32$\pm$5 &  7$\pm$29 \\
2 & 14$\pm$5 & 17$\pm$4 & 26$\pm$51 \\
\hline 
\end{tabular}
\caption{Closure test on data after \mHi=120 \GeVcc\ selection with Z veto.}
\label{tab:data_closure_120_zv}
\end{center}
\end{table}
%%%%%

\clearpage

\section{Comparison with \routin~ method}
   \label{sec:comparison}
   We perform the full \hww\ analysis using the \zm\ method for Drell-Yan estimation and we compare it with the default analysis (\routin\ method).
First, we compare the analysis components that are affected by the Drell-Yan estimation method 
(data/MC scale factors and shapes for the BDT analysis) and then we compare the expected limits resulting from the two analyses.

The scale factors from the two estimation methods, reported in Table~\ref{tab:scalefactors}, are compatible within the uncertainties.
However, especially in the 0-jet bin, the error can be larger than 50\% and a proper comparison is difficult.

The \routin-based analysis uses as central shape MC events in the region with 30$<$\met$<$37 \GeV and as alternative shapes the MC in the signal region (``up'')
 and its mirror shape with respect to central (``down'').
With the \zm\ method, the central shape can be naturally taken from data weighting the events in the fail region with the corresponding \zm\ value;
the shape variations are still provided by the signal region in MC.
A comparison of the shapes used in the \routin- and \zm-based analyses is shown in Figures~\ref{fig:zjets_shape_mh115}-\ref{fig:zjets_shape_mh140}.
Shapes from the \zm\ method are generally suffering more from statistical fluctuations, but, when the unceratinty is not too large, 
similar features to the \routin\ case can be observed. As an example, both in Fig.~\ref{subfig:routin_zjets_shape_mh130_1jet} and 
Fig.~\ref{subfig:zeta_zjets_shape_mh130_1jet} there is a sharp peak at 0.7 and a broad bump below 0.2.

Considering the same flavor final states only, the resulting expected limits are compatible  within 10\% with the deafult analysis and 
show almost no difference when combined with the opposite flavor (Figures~\ref{tab:sf_limits} and \ref{tab:sfof_limits}).

%%%%%
\begin{table}[!ht]
\begin{center}
\begin{tabular} {|c|cc|}
\hline
Analysis  & \routin\ method & \zm\ method \\
\hline 
\hline
\multicolumn{3}{|c|}{0-jet} \\
\hline 
 WW level & 3.05$\pm$1.75 & 4.34$\pm$1.82 \\
 \mHi=115 & 5.28$\pm$3.51 & 2.12$\pm$1.47 \\
 \mHi=120 & 5.66$\pm$3.75 & 2.43$\pm$1.46 \\
 \mHi=130 & 4.07$\pm$2.80 & 2.69$\pm$1.33 \\
 \mHi=140 & 3.57$\pm$2.46 & 2.57$\pm$1.17 \\
 \mHi=150 & 2.76$\pm$2.86 & 3.03$\pm$1.41 \\
 \mHi=160 & 1.53$\pm$2.73 & 2.22$\pm$2.01 \\
 \mHi=170 & 0.75$\pm$1.65 & 1.00$\pm$1.00 \\
 \mHi=180 & 0.33$\pm$0.78 & 1.00$\pm$1.00 \\
 \mHi=190 & 1.08$\pm$1.50 & 1.00$\pm$1.00 \\
 \mHi=200 & 1.25$\pm$1.02 & 1.00$\pm$1.00 \\
\hline 
\hline
\multicolumn{3}{|c|}{1-jet} \\
\hline 
 WW level & 2.39$\pm$0.45 & 2.86$\pm$0.88 \\
 \mHi=115 & 5.05$\pm$2.22 & 7.74$\pm$2.81 \\
 \mHi=120 & 5.46$\pm$2.35 & 6.87$\pm$2.33 \\
 \mHi=130 & 3.39$\pm$1.75 & 4.99$\pm$1.64 \\
 \mHi=140 & 4.50$\pm$2.29 & 5.46$\pm$1.77 \\
 \mHi=150 & 5.25$\pm$2.00 & 4.23$\pm$1.33 \\
 \mHi=160 & 7.33$\pm$3.03 & 4.51$\pm$1.54 \\
 \mHi=170 & 9.58$\pm$4.18 & 1.33$\pm$1.94 \\
 \mHi=180 & 7.76$\pm$3.35 & 4.45$\pm$2.72 \\
 \mHi=190 & 4.53$\pm$1.77 & 3.84$\pm$1.69 \\
 \mHi=200 & 4.46$\pm$1.48 & 3.91$\pm$1.71 \\
\hline 
\hline
\multicolumn{3}{|c|}{2-jet} \\
\hline 
 WW level & 3.11$\pm$0.62 & 2.00$\pm$0.62 \\
\hline 
\end{tabular}
\caption{Drell-Yan data/MC scale factors for \hww\ analysis using 2011 data. Unceratinties include both statistic and systematic errors.}
\label{tab:scalefactors}
\end{center}
\end{table}
%%%%%


%%%%%%%%
\begin{figure}[!hbtp]
\begin{center}
\subfigure[0-jet, \routin~method]{\label{subfig:routin_zjets_shape_mh115_0jet}
\includegraphics[width=.4\textwidth]{figures/dyshape_routin/zjets_shape_mh115_0jet.png}}
\subfigure[0-jet, \zm~method]{\label{subfig:zeta_zjets_shape_mh115_0jet}
\includegraphics[width=.4\textwidth]{figures/dyshape_zeta/zjets_shape_mh115_0jet.png}}\\
\subfigure[1-jet, \routin~method]{\label{subfig:routin_zjets_shape_mh115_1jet}
\includegraphics[width=.4\textwidth]{figures/dyshape_routin/zjets_shape_mh115_1jet.png}}
\subfigure[1-jet, \zm~method]{\label{subfig:zeta_zjets_shape_mh115_1jet}
\includegraphics[width=.4\textwidth]{figures/dyshape_zeta/zjets_shape_mh115_1jet.png}}
\caption{\dyll~shape in \mHi=115 \GeVcc analysis.}
\label{fig:zjets_shape_mh115}
\end{center}
\end{figure}
%%%%%%%%

%%%%%%%%
\begin{figure}[!hbtp]
\begin{center}
\subfigure[0-jet, \routin~method]{\label{subfig:routin_zjets_shape_mh120_0jet}
\includegraphics[width=.4\textwidth]{figures/dyshape_routin/zjets_shape_mh120_0jet.png}}
\subfigure[0-jet, \zm~method]{\label{subfig:zeta_zjets_shape_mh120_0jet}
\includegraphics[width=.4\textwidth]{figures/dyshape_zeta/zjets_shape_mh120_0jet.png}}\\
\subfigure[1-jet, \routin~method]{\label{subfig:routin_zjets_shape_mh120_1jet}
\includegraphics[width=.4\textwidth]{figures/dyshape_routin/zjets_shape_mh120_1jet.png}}
\subfigure[1-jet, \zm~method]{\label{subfig:zeta_zjets_shape_mh120_1jet}
\includegraphics[width=.4\textwidth]{figures/dyshape_zeta/zjets_shape_mh120_1jet.png}}
\caption{\dyll~shape in \mHi=120 \GeVcc analysis.}
\label{fig:zjets_shape_mh120}
\end{center}
\end{figure}
%%%%%%%%

%%%%%%%%
\begin{figure}[!hbtp]
\begin{center}
\subfigure[0-jet, \routin~method]{\label{subfig:routin_zjets_shape_mh130_0jet}
\includegraphics[width=.4\textwidth]{figures/dyshape_routin/zjets_shape_mh130_0jet.png}}
\subfigure[0-jet, \zm~method]{\label{subfig:zeta_zjets_shape_mh130_0jet}
\includegraphics[width=.4\textwidth]{figures/dyshape_zeta/zjets_shape_mh130_0jet.png}}\\
\subfigure[1-jet, \routin~method]{\label{subfig:routin_zjets_shape_mh130_1jet}
\includegraphics[width=.4\textwidth]{figures/dyshape_routin/zjets_shape_mh130_1jet.png}}
\subfigure[1-jet, \zm~method]{\label{subfig:zeta_zjets_shape_mh130_1jet}
\includegraphics[width=.4\textwidth]{figures/dyshape_zeta/zjets_shape_mh130_1jet.png}}
\caption{\dyll~shape in \mHi=130 \GeVcc analysis.}
\label{fig:zjets_shape_mh130}
\end{center}
\end{figure}
%%%%%%%%

%%%%%%%%
\begin{figure}[!hbtp]
\begin{center}
\subfigure[0-jet, \routin~method]{\label{subfig:routin_zjets_shape_mh140_0jet}
\includegraphics[width=.4\textwidth]{figures/dyshape_routin/zjets_shape_mh140_0jet.png}}
\subfigure[0-jet, \zm~method]{\label{subfig:zeta_zjets_shape_mh140_0jet}
\includegraphics[width=.4\textwidth]{figures/dyshape_zeta/zjets_shape_mh140_0jet.png}}\\
\subfigure[1-jet, \routin~method]{\label{subfig:routin_zjets_shape_mh140_1jet}
\includegraphics[width=.4\textwidth]{figures/dyshape_routin/zjets_shape_mh140_1jet.png}}
\subfigure[1-jet, \zm~method]{\label{subfig:zeta_zjets_shape_mh140_1jet}
\includegraphics[width=.4\textwidth]{figures/dyshape_zeta/zjets_shape_mh140_1jet.png}}
\caption{\dyll~shape in \mHi=140 \GeVcc analysis.}
\label{fig:zjets_shape_mh140}
\end{center}
\end{figure}
%%%%%%%%

%%%%%
\begin{table}[!ht]
\begin{center}
\begin{tabular} {|c|ccc||ccc|}
\hline
\mHi & Expected & 68\% C.L. & 95\% C.L.  & Expected & 68\% C.L. & 95\% C.L. \\
\hline
\hline
 & \multicolumn{3}{|c||}{\routin, cut based}  & \multicolumn{3}{|c|}{\zm, cut based} \\
\hline
115 & 7.54 & [5.43, 10.49] & [4.05, 14.06]& 6.96 & [5.02, 9.69] & [3.74, 12.99]\\
120 & 4.28 & [3.08, 5.95] & [2.30, 7.98]  & 3.85 & [2.78, 5.36] & [2.07, 7.19] \\
130 & 1.74 & [1.26, 2.43] & [0.94, 3.25]  & 1.59 & [1.14, 2.21] & [0.85, 2.96] \\
140 & 1.04 & [0.75, 1.45] & [0.56, 1.94]  & 0.96 & [0.69, 1.33] & [0.51, 1.79] \\
150 & 0.69 & [0.50, 0.96] & [0.37, 1.29]  & 0.63 & [0.46, 0.88] & [0.34, 1.18] \\
160 & 0.38 & [0.27, 0.52] & [0.20, 0.70]  & 0.36 & [0.26, 0.50] & [0.19, 0.67] \\
170 & 0.38 & [0.27, 0.53] & [0.20, 0.71]  & 0.37 & [0.27, 0.52] & [0.20, 0.70] \\
180 & 0.48 & [0.35, 0.67] & [0.26, 0.90]  & 0.48 & [0.35, 0.67] & [0.26, 0.90] \\
190 & 0.72 & [0.52, 1.01] & [0.39, 1.35]  & 0.72 & [0.52, 1.00] & [0.39, 1.34] \\
200 & 0.91 & [0.66, 1.27] & [0.49, 1.70]  & 0.91 & [0.66, 1.27] & [0.49, 1.70] \\
250 & 2.08 & [1.50, 2.89] & [1.12, 3.88]  & 2.23 & [1.61, 3.11] & [1.20, 4.16] \\
300 & 2.01 & [1.45, 2.80] & [1.08, 3.75]  & 2.38 & [1.71, 3.31] & [1.27, 4.43] \\
\hline
\hline
 & \multicolumn{3}{|c||}{\routin, shape based}  & \multicolumn{3}{|c|}{\zm, shape based} \\
\hline
115 & 6.02 & [4.33, 8.37] & [3.23, 11.22] & 5.66 & [4.07, 7.87] & [3.03, 10.55] \\
120 & 3.12 & [2.25, 4.34] & [1.68, 5.82]  & 3.10 & [2.23, 4.32] & [1.66, 5.79] \\ 
130 & 1.31 & [0.95, 1.83] & [0.71, 2.45]  & 1.52 & [1.10, 2.12] & [0.82, 2.84] \\ 
140 & 0.79 & [0.57, 1.11] & [0.43, 1.48]  & 0.82 & [0.59, 1.15] & [0.44, 1.54] \\ 
150 & 0.57 & [0.41, 0.79] & [0.31, 1.06]  & 0.58 & [0.42, 0.80] & [0.31, 1.08] \\ 
160 & 0.34 & [0.24, 0.47] & [0.18, 0.63]  & 0.30 & [0.22, 0.42] & [0.16, 0.56] \\ 
170 & 0.35 & [0.25, 0.49] & [0.19, 0.65]  & 0.31 & [0.22, 0.43] & [0.17, 0.58] \\ 
180 & 0.44 & [0.31, 0.61] & [0.23, 0.81]  & 0.46 & [0.33, 0.64] & [0.25, 0.85] \\ 
190 & 0.66 & [0.48, 0.92] & [0.36, 1.24]  & 0.69 & [0.49, 0.95] & [0.37, 1.28] \\ 
200 & 0.78 & [0.56, 1.08] & [0.42, 1.45]  & 0.93 & [0.67, 1.30] & [0.50, 1.74] \\ 
250 & 1.75 & [1.26, 2.44] & [0.94, 3.27]  & 2.06 & [1.48, 2.86] & [1.10, 3.84] \\ 
300 & 1.75 & [1.26, 2.43] & [0.94, 3.26]  & 1.85 & [1.33, 2.57] & [0.99, 3.44] \\ 
\hline
\end{tabular}
\caption{Expected limits combining the same flavor final states.}
\label{tab:sf_limits}
\end{center}
\end{table}
%%%%%


%%%%%
\begin{table}[!ht]
\begin{center}
\begin{tabular} {|c|ccc||ccc|}
\hline
\mHi & Expected & 68\% C.L. & 95\% C.L.  & Expected & 68\% C.L. & 95\% C.L. \\
\hline
\hline
 & \multicolumn{3}{|c||}{\routin, cut based}  & \multicolumn{3}{|c|}{\zm, cut based} \\
\hline
115 & 3.26 & [2.35, 4.54] & [1.75, 6.08]  & 3.25 & [2.34, 4.52] & [1.74, 6.06] \\
120 & 2.01 & [1.45, 2.80] & [1.08, 3.75]  & 2.00 & [1.44, 2.79] & [1.07, 3.74] \\
130 & 0.96 & [0.69, 1.33] & [0.51, 1.79]  & 0.95 & [0.69, 1.33] & [0.51, 1.78] \\
140 & 0.62 & [0.45, 0.87] & [0.33, 1.16]  & 0.62 & [0.45, 0.86] & [0.33, 1.15] \\
150 & 0.46 & [0.33, 0.64] & [0.25, 0.86]  & 0.45 & [0.33, 0.63] & [0.24, 0.85] \\
160 & 0.26 & [0.19, 0.36] & [0.14, 0.48]  & 0.26 & [0.18, 0.36] & [0.14, 0.48] \\
170 & 0.27 & [0.19, 0.37] & [0.14, 0.50]  & 0.27 & [0.19, 0.37] & [0.14, 0.50] \\
180 & 0.38 & [0.28, 0.53] & [0.21, 0.71]  & 0.38 & [0.28, 0.53] & [0.21, 0.71] \\
190 & 0.56 & [0.41, 0.78] & [0.30, 1.05]  & 0.56 & [0.40, 0.78] & [0.30, 1.05] \\
200 & 0.70 & [0.50, 0.97] & [0.37, 1.30]  & 0.70 & [0.50, 0.97] & [0.37, 1.30] \\
250 & 1.40 & [1.01, 1.95] & [0.75, 2.61]  & 1.41 & [1.02, 1.96] & [0.76, 2.63] \\
300 & 1.52 & [1.10, 2.12] & [0.82, 2.84]  & 1.57 & [1.13, 2.19] & [0.84, 2.93] \\
\hline
\hline
  & \multicolumn{3}{|c||}{\routin, shape based}  & \multicolumn{3}{|c|}{\zm, shape based} \\
\hline
115 & 2.57 & [1.85, 3.57] & [1.38, 4.79]  & 2.55 & [1.84, 3.55] & [1.37, 4.76] \\
120 & 1.52 & [1.10, 2.12] & [0.82, 2.84]  & 1.52 & [1.10, 2.12] & [0.82, 2.84] \\
130 & 0.70 & [0.50, 0.97] & [0.38, 1.31]  & 0.73 & [0.52, 1.01] & [0.39, 1.35] \\
140 & 0.47 & [0.34, 0.66] & [0.25, 0.88]  & 0.48 & [0.34, 0.66] & [0.26, 0.89] \\
150 & 0.35 & [0.25, 0.49] & [0.19, 0.65]  & 0.35 & [0.25, 0.48] & [0.19, 0.65] \\
160 & 0.22 & [0.16, 0.31] & [0.12, 0.42]  & 0.21 & [0.15, 0.30] & [0.12, 0.40] \\
170 & 0.24 & [0.17, 0.33] & [0.13, 0.45]  & 0.23 & [0.16, 0.32] & [0.12, 0.43] \\
180 & 0.31 & [0.22, 0.43] & [0.17, 0.58]  & 0.31 & [0.23, 0.44] & [0.17, 0.58] \\
190 & 0.45 & [0.32, 0.62] & [0.24, 0.84]  & 0.44 & [0.32, 0.62] & [0.24, 0.83] \\
200 & 0.53 & [0.38, 0.74] & [0.28, 0.99]  & 0.55 & [0.40, 0.76] & [0.29, 1.02] \\
250 & 1.17 & [0.84, 1.63] & [0.63, 2.19]  & 1.21 & [0.87, 1.69] & [0.65, 2.26] \\
300 & 1.27 & [0.91, 1.76] & [0.68, 2.36]  & 1.21 & [0.87, 1.69] & [0.65, 2.26] \\
\hline
\end{tabular}
\caption{Expected limits combining the same flavor and opposite flavor final states.}
\label{tab:sfof_limits}
\end{center}
\end{table}
%%%%%

\section{Conclusion}
   The \zm\ and \routin\ methods for DY estimation provide compatible results both in terms of DY estimation and in terms of final limits.
In 2012 analysis, the two methods will cross-check each other and the one with lowest uncertainty will be used as default method.

\clearpage
\clearpage

\vspace*{-0.2cm}
\thebibliography{12}

\bibitem{pdg}
 K. Nakamura et al. (Particle Data Group), "Review of particle physics", J. Phys.G37 , 2010.

\bibitem{Higgs1}
F. Englert and R. Brout, "Broken symmetries and the masses of gauge bosons", Phys. Rev. Lett. 13,  1964.

\bibitem{Higgs2}
P. W. Higgs, "Broken symmetry and the mass of gauge vector mesons", Phys. Rev. Lett. 13, 1964.

\bibitem{Higgs3}
Guralnik, G.S. and Hagen, C.R. and Kibble, T.W.B., "Global Conservation Laws and Massless Particles", 
Phys.Rev.Lett. 13, 1964.

\bibitem{HWW2010}
CMS Collaboration, "Title: Measurement of WW Production and Search for the Higgs Boson in 
pp Collisions at $\sqrt{s}$ = 7 TeV", arXiv:1102.5429

\bibitem{VBTFCrossSectionNote}
J. Alcaraz Maestre, \textit{et al.}, "Updated Measurements of Inclusive W and Z Cross Sections 
at $\sqrt{s}=7$ TeV", CMS AN-2010/264.

\bibitem{ggWWError}
F.~ Stoeckli, "http://indico.cern.ch/getFile.py/access?contribId=0\&resId=1\&materialId=slides\&confId=49009", 
EWK Diboson meeting of March 12 2009.

\bibitem{json}
{\small
/afs/cern.ch/cms/CAF/CMSCOMM/COMM\_DQM/certification/Collisions11/7TeV/Prompt/Cert\_160404-163869\_7TeV\_PromptReco\_Collisions11\_JSON.txt
}

\bibitem{ElIso}
A. Vartak, M. LeBourgeois, V. Sharma, "Lepton Isolation in the CMS Tracker, ECAL and HCAL", CMS AN-2010/106.

\bibitem{PVDA}
W. Erdmann, M. LeBourgeois, B. Mangano, 
https://indico.cern.ch/getFile.py/access?contribId=5\&sessionId=3\&resId=1\&materialId=slides\&confId=127127, 
note in preparation.

\bibitem{NExpHits}
B. Mangano \textit{et al.}, "Improvement in Photon Conversion Rejection Performance Using 
Advanced Tracking Tools", AN-10-283.

\bibitem{fakeLeptonNote1}
S.~Xie, \textit{et al.}", "Study of Data-Driven Methods for Estimation of Fake Lepton Backgrounds", 
CMS AN-2009/120.

\bibitem{fakeLeptonNote2}
W.~Andrews, \textit{et al.}, "Fake Rates for dilepton Analyses", CMS AN-2010/257.

\bibitem{fakeLeptonBkgSpillage1}
 F. Golf, D. Evans, J. Mulmenstadt  \textit{et al.}, ``Expectations for observation of top quark pair production in the dilepton final state with the early CMS data'', CMS AN-2009/050.

\bibitem{dyestnote}
W. Andrews, et al., “A Method to Measure the Contribution of $\dyll$ to a di-lepton+ MET Selection”, CMS AN-2009/023 (2009).

\bibitem{jes}
CMS Collaboration, "Jet Energy Calibration with Photon+Jet Events", PAS JME-09-004.

\bibitem{jetpas}
CMS Collaboration, "Jet Performance in pp Collisions at $\sqrt{s}=7 \rm\ TeV$", PAS JME-10-003.

\bibitem{btag}
CMS collaboration, "Commissioning of b-jet identification with pp collisions at $\sqrt{s}=7~\TeV$, BTV-10-001.

\bibitem{antikt}
Cacciari, Matteo and Salam, Gavin P. and Soyez, Gregory, "The anti-$k_t$ jet clustering 
algorithm", JHEP 04,  2008.

\bibitem{ConversionNote}
W.~Andrews, \textit{et al.}, "Study of photon conversion rejection at CMS", CMS AN-2009/159.

\bibitem{tmva}
A. Hoecker, \textit{et al.}, "TMVA - Toolkit for Multivariate Data Analysis", arXiv:physics/0703039, 2007.

\bibitem{XS}
CMS Generator group, Standard Model Cross Sections for CMS at 7 TeV, 2010.

\bibitem{PDF4LHC}
PDF4LHC Working Group, 
{\tt http://www.hep.ucl.ac.uk/pdf4lhc/PDF4LHCrecom.pdf}

\bibitem{Nadolsky:2008zw}
Nadolsky, Pavel M. and others, "Implications of CTEQ global analysis for 
collider observables", Phys. Rev. D78 2008.

\bibitem{Martin:2009iq}
Martin, A. D. and Stirling, W. J. and Thorne, R. S. and Watt, G., "Parton 
distributions for the LHC, Eur. Phys. J. C63 2009.

\bibitem{Ball:2010de}
Ball, Richard D. and others, "A first unbiased global NLO determination 
of parton distributions and their uncertainties", arXiv 1002.4407.

\bibitem{bayesian}
A. O'Hagan and J.J. Forster, "Bayesian Inference", Kendall's Advanced Theory of Statistics, 
Arnold, London, 2B, 2004.

\bibitem{ref:tagprobe_mit_w}
G. Bauer {\it et. al.}, "Lepton ef?iencies for the inclusive W cross section measurement with 36.1pb$^{-1}$", AN2011/097

\bibitem{ref:tagprobe_snt_top}
W. Andrews {\it et. al.}, "Uncertainties on the Lepton Selection Efficiency for t$t\bar{t}$ Cross Section Analysis", AN2010/274

\bibitem{LHCHiggsCrossSectionWorkingGroup:2011ti}
LHC Higgs Cross Section Working Group, "Handbook of LHC Higgs Cross Sections: 
Inclusive Observables", CERN-2011-002, 2011.

\bibitem{PFMET} 
CMS Collaboration, ``CMS MET Performance in Events Containing Electroweak Bosons from pp Collisions at $\sqrt{s}=7$ TeV'', CMS PAS JME-2010-005 (2010)


\bibitem{trkMET} 
Marco Zanetti, ``MET with PU in $\hww\to2\ell$'', https://indico.cern.ch/conferenceDisplay.py?confId=131580
Benjamin Hooberman, ``MET with PU in MC and First 2011 Data'', https://indico.cern.ch/contributionDisplay.py?contribId=5\&confId=132579. 


\bibitem{lands}
Mingshui Chen and Andrey Korytov, https://mschen.web.cern.ch/mschen/lands/

\bibitem{MCFMHiggsProduction}
J. Campbell, R.K. Ellis, G. Zanderighi, ``Next-to-Leading order Higgs + 2 jet production via gluon fusion.'', JHEP 0610:028 (2006), hep-ph/0608194
\end{document}
