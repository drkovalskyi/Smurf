\subsubsection{Observed and expected upper limits corresponding to $\intlumi$ data}

In this section, we document the results on the upper-limits using shape analysis based on 
the matrix element output, described in Section~\ref{sec:LR}. 
%The matrix element output has been evaluated for 18 different values of $m_H$ between 115 and 600 \GeVcc.
Figure~\ref{fig:me_115_4700pb}-\ref{fig:me_200_4700pb} shows the likelihood ratio distributions for $m_H$~=~115, 120, 130, 140, 160 and 200\GeVcc, 
corresponding to 4.7~fb$^{-1}$. 
Note that the majority of backgrounds peak near $LR~=~0$ while the signal peaks near $LR~=~1$.  
It was noted in Section~\ref{sec:EvtSelWW} that we apply a dilepton invariant mass requirement prior to constructing the likelihood ratio. 
Figure~\ref{fig:LR_noMll} shows the likelihood ratio distribution assuming $m_{H}=160$ \GeVcc without the invariant mass requirements.
One can see from the peak at $LR~=~0$ that the Matrix Element method successfully identifies high invariant mass events as background-like, so the cut is not essential. However, by applying the cut we lose less than $1\%$ of the signal and gain significantly in processing time needed for differential cross-section calculations.


%%%%%%%%%%%%%%%%%%%%%%%%%%%%%%%%%%%%%
\begin{figure}[!hbtp]                                                                                         
\centering                                                                                                                                             
\includegraphics[width=.5\textwidth]{figures/LR_noMll.png}\\                                            
\caption{The matrix element output LR distribution after $WW$ selection but prior to $m_{ll}$ cut                      
for $m_H$=160 \GeVcc}
\label{fig:LR_noMll}                                                                                          
\end{figure}
%%%%%%%%%%%%%%%%%%%%%%%%%%%%%%%%%%%%%



%%%%%%%%%%%%%%%%%%%%%%%%%%%%%%%%%%%%%
\begin{figure}[!hbtp]
\centering
\subfigure[$e\mu$ 0-Jet]{
\centering
\label{subfig:me_115_0j_of_4700pb}
\includegraphics[width=.40\textwidth]{figures/ME_mH115_0j_of_stack_lin.pdf}}
\subfigure[$ee$/$\mu\mu$ 0-Jet]{
\centering
\label{subfig:me_115_0j_sf_4700pb}
\includegraphics[width=.40\textwidth]{figures/ME_mH115_0j_sf_stack_lin.pdf}}
\subfigure[$e\mu$ 1-Jet]{
\centering
\label{subfig:me_115_1j_of_4700pb}
\includegraphics[width=.40\textwidth]{figures/ME_mH115_1j_of_stack_lin.pdf}}
\subfigure[$ee$/$\mu\mu$ 1-Jet]{
\centering
\label{subfig:me_115_1j_sf_4700pb}
\includegraphics[width=.40\textwidth]{figures/ME_mH115_1j_sf_stack_lin.pdf}}
\caption{
ME output for $m_H$=115 GeV corresponding to \intlumi:
0-jet OF \subref{subfig:me_115_0j_of_4700pb},
0-jet SF \subref{subfig:me_115_0j_sf_4700pb},
1-jet OF \subref{subfig:me_115_1j_of_4700pb},
1-jet SF \subref{subfig:me_115_1j_sf_4700pb}
.}
\label{fig:me_115_4700pb}
\end{figure}

\begin{figure}[!hbtp]
\centering
\subfigure[$e\mu$ 0-Jet]{
\centering
\label{subfig:me_120_0j_of_4700pb}
\includegraphics[width=.40\textwidth]{figures/ME_mH120_0j_of_stack_lin.pdf}}
\subfigure[$ee/\mu\mu$ 0-Jet]{
\centering
\label{subfig:me_120_0j_sf_4700pb}
\includegraphics[width=.40\textwidth]{figures/ME_mH120_0j_sf_stack_lin.pdf}}
\subfigure[$e\mu$ 1-Jet]{
\centering
\label{subfig:me_120_1j_of_4700pb}
\includegraphics[width=.40\textwidth]{figures/ME_mH120_1j_of_stack_lin.pdf}}
\subfigure[$ee/\mu\mu$ 1-Jet]{
\centering
\label{subfig:me_120_1j_sf_4700pb}
\includegraphics[width=.40\textwidth]{figures/ME_mH120_1j_sf_stack_lin.pdf}}
\caption{
ME output for $m_H$=120 GeV corresponding to \intlumi:
0-jet OF \subref{subfig:me_120_0j_of_4700pb},
0-jet SF \subref{subfig:me_120_0j_sf_4700pb},
1-jet OF \subref{subfig:me_120_1j_of_4700pb},
1-jet SF \subref{subfig:me_120_1j_sf_4700pb}.}
\label{fig:me_120_4700pb}
\end{figure}
%%%%%%%%%%%%%%%%%%%%%%%%%%%%%%%%%%%%%

%%%%%%%%%%%%%%%%%%%%%%%%%%%%%%%%%%%%%
\begin{figure}[!hbtp]
\centering
\subfigure[$e\mu$ 0-Jet]{
\centering
\label{subfig:me_130_0j_of_4700pb}
\includegraphics[width=.40\textwidth]{figures/ME_mH130_0j_of_stack_lin.pdf}}
\subfigure[$ee/\mu\mu$ 0-Jet]{
\centering
\label{subfig:me_130_0j_sf_4700pb}
\includegraphics[width=.40\textwidth]{figures/ME_mH130_0j_sf_stack_lin.pdf}}
\subfigure[$e\mu$ 1-Jet]{
\centering
\label{subfig:me_130_1j_of_4700pb}
\includegraphics[width=.40\textwidth]{figures/ME_mH130_1j_of_stack_lin.pdf}}
\subfigure[$ee/\mu\mu$ 1-Jet]{
\centering
\label{subfig:me_130_1j_sf_4700pb}
\includegraphics[width=.40\textwidth]{figures/ME_mH130_1j_sf_stack_lin.pdf}}
\caption{
ME output for $m_H$=130 GeV corresponding to \intlumi:
0-jet OF \subref{subfig:me_130_0j_of_4700pb},
0-jet SF \subref{subfig:me_130_0j_sf_4700pb},
1-jet OF \subref{subfig:me_130_1j_of_4700pb},
1-jet SF \subref{subfig:me_130_1j_sf_4700pb}
.}
\label{fig:me_130_4700pb}
\end{figure}
%%%%%%%%%%%%%%%%%%%%%%%%%%%%%%%%%%%%%

%%%%%%%%%%%%%%%%%%%%%%%%%%%%%%%%%%%%%
\begin{figure}[!hbtp]
\centering
\subfigure[$e\mu$ 0-Jet]{
\centering
\label{subfig:me_140_0j_of_4700pb}
\includegraphics[width=.40\textwidth]{figures/ME_mH140_0j_of_stack_lin.pdf}}
\subfigure[$ee/\mu\mu$ 0-Jet]{
\centering
\label{subfig:me_140_0j_sf_4700pb}
\includegraphics[width=.40\textwidth]{figures/ME_mH140_0j_sf_stack_lin.pdf}}
\subfigure[$e\mu$ 1-Jet]{
\centering
\label{subfig:me_140_1j_of_4700pb}
\includegraphics[width=.40\textwidth]{figures/ME_mH140_1j_of_stack_lin.pdf}}
\subfigure[$ee/\mu\mu$ 1-Jet]{
\centering
\label{subfig:me_140_1j_sf_4700pb}
\includegraphics[width=.40\textwidth]{figures/ME_mH140_1j_sf_stack_lin.pdf}}
\caption{
ME output for $m_H$=140 GeV corresponding to \intlumi:
0-jet OF \subref{subfig:me_140_0j_of_4700pb},
0-jet SF \subref{subfig:me_140_0j_sf_4700pb},
1-jet OF \subref{subfig:me_140_1j_of_4700pb},
1-jet SF \subref{subfig:me_140_1j_sf_4700pb}
.}
\label{fig:me_140_4700pb}
\end{figure}
%%%%%%%%%%%%%%%%%%%%%%%%%%%%%%%%%%%%%


              
%%%%%%%%%%%%%%%%%%%%%%%%%%%%%%%%%%%%%
\begin{figure}[!hbtp]
\centering
\subfigure[$e\mu$ 0-Jet]{
\centering
\label{subfig:me_160_0j_of_4700pb}
\includegraphics[width=.40\textwidth]{figures/ME_mH160_0j_of_stack_lin.pdf}}
\subfigure[$ee/\mu\mu$ 0-Jet]{
\centering
\label{subfig:me_160_0j_sf_4700pb}
\includegraphics[width=.40\textwidth]{figures/ME_mH160_0j_sf_stack_lin.pdf}}
\subfigure[$e\mu$ 1-Jet]{
\centering
\label{subfig:me_160_1j_of_4700pb}
\includegraphics[width=.40\textwidth]{figures/ME_mH160_1j_of_stack_lin.pdf}}
\subfigure[$ee/\mu\mu$ 1-Jet]{
\centering
\label{subfig:me_160_1j_sf_4700pb}
\includegraphics[width=.40\textwidth]{figures/ME_mH160_1j_sf_stack_lin.pdf}}
\caption{
ME output for $m_H$=160 GeV corresponding to \intlumi:
0-jet OF \subref{subfig:me_160_0j_of_4700pb},
0-jet SF \subref{subfig:me_160_0j_sf_4700pb},
1-jet OF \subref{subfig:me_160_1j_of_4700pb},
1-jet SF \subref{subfig:me_160_1j_sf_4700pb}
.}
\label{fig:me_160_4700pb}
\end{figure}
%%%%%%%%%%%%%%%%%%%%%%%%%%%%%%%%%%%%%
  

              
%%%%%%%%%%%%%%%%%%%%%%%%%%%%%%%%%%%%%
\begin{figure}[!hbtp]
\centering
\subfigure[$e\mu$ 0-Jet]{
\centering
\label{subfig:me_200_0j_of_4700pb}
\includegraphics[width=.40\textwidth]{figures/ME_mH200_0j_of_stack_lin.pdf}}
\subfigure[$ee/\mu\mu$ 0-Jet]{
\centering
\label{subfig:me_200_0j_sf_4700pb}
\includegraphics[width=.40\textwidth]{figures/ME_mH200_0j_sf_stack_lin.pdf}}
\subfigure[$e\mu$ 1-Jet]{
\centering
\label{subfig:me_200_1j_of_4700pb}
\includegraphics[width=.40\textwidth]{figures/ME_mH200_1j_of_stack_lin.pdf}}
\subfigure[$ee/\mu\mu$ 1-Jet]{
\centering
\label{subfig:me_200_1j_sf_4700pb}
\includegraphics[width=.40\textwidth]{figures/ME_mH200_1j_sf_stack_lin.pdf}}
\caption{
ME output for $m_H$=200 GeV corresponding to \intlumi:
0-jet OF \subref{subfig:me_200_0j_of_4700pb},
0-jet SF \subref{subfig:me_200_0j_sf_4700pb},
1-jet OF \subref{subfig:me_200_1j_of_4700pb},
1-jet SF \subref{subfig:me_200_1j_sf_4700pb}
.}
\label{fig:me_200_4700pb}
\end{figure}
%%%%%%%%%%%%%%%%%%%%%%%%%%%%%%%%%%%%%

Systematics uncertainties applied in the search are summarized in \cite{ref:HZZ2011smurf}.
Systematic variations affecting shapes of the likelihood ratio discriminant are included in the results presented in this note,
however methods to account for them are discussed separately in \cite{ref:ShapeSmurf}. 


The expected and observed upper limits at 95\%C.L., for the dataset corresponding to $\intlumi$ for 
the shape analysis based on the matrix element outputs are shown in Table~\ref{tab:me_results_5fb} and 
Figure~\ref{fig:me_results_5fb} for the 0/1/2 jet bins combined. 
%The limits are obtained using the CLs asymptotic methods. 
We also compare the performance for the 4 sub-channels individually, shown in 
Table~\ref{tab:me_results_5fb_0jsf}-\ref{tab:me_results_5fb_1jof} and Figure~\ref{fig:me_results_5fb_subchannel}. 
%Table~\ref{tab:me_results_5fb_0j}-\ref{tab:me_results_5fb_1j} show the comparison of the performance 
%in the 0 and 1 jet bins respectively. 

Using matrix element based approach in $H\rightarrow WW \rightarrow l^{+}l^{-}\nu\bar{\nu}$ 
we exclude the Standard Model Higgs boson in the mass range [128,350] GeV, with an expected exclusion range of [128,300] GeV at 95\% C.L. 
For comparison, BDT expected and observed exclusion ranges are similar: [128-300] GeV and [128-300] GeV, respectively.  


%%%%%%%%%%%%%%%%%%%%%%%%%%%%%%
\begin{figure}[!hbtp]
\centering
\subfigure[BDT]{
\centering
\label{subfig:bdt}
\includegraphics[width=.45\textwidth]{figures/limit_nj_shape_bdt-CLs-asymptotic.pdf}}
\subfigure[ME]{
\centering
\label{subfig:me}
\includegraphics[width=.45\textwidth]{figures/limit_nj_shape_me-CLs-asymptotic.pdf}}
\caption{The expected and observed upper limits on the SM Higgs for the shape analysis based 
the BDT and matrix element outputs~\subref{subfig:me} at 95\% C.L. for $\intlumi$ data. 
The limits are obtained combining all sub-channels in 0, 1 and 2 jet bins, 
calculated with the asymptotic CLs method}
\label{fig:me_results_5fb}
\end{figure}
%%%%%%%%%%%%%%%%%%%%%%%%%%%%%%

%%%%%%%%%%%%%%%%%%%%%%%%%%%%%%
\begin{table}[!htbp]
\begin{center}
\begin{tabular}{c c c c c c}
\hline\hline
 Higgs Mass   & Observed & Median expected & Expected range for 68\% & Expected range for 95\%   \\
\hline
\multicolumn{5}{c} {BDT Based} \\
\hline
115 & 3.7 & 2.5 & [1.8, 3.5] & [1.4, 4.8] \\
120 & 2.4 & 1.5 & [1.1, 2.1] & [0.8, 2.8] \\
130 & 0.9 & 0.7 & [0.5, 1.0] & [0.4, 1.3] \\
140 & 0.6 & 0.4 & [0.3, 0.6] & [0.2, 0.8] \\
150 & 0.4 & 0.3 & [0.2, 0.4] & [0.2, 0.6] \\
160 & 0.3 & 0.2 & [0.1, 0.3] & [0.1, 0.4] \\
170 & 0.3 & 0.2 & [0.2, 0.3] & [0.1, 0.4] \\
180 & 0.3 & 0.3 & [0.2, 0.4] & [0.1, 0.5] \\
190 & 0.4 & 0.4 & [0.3, 0.6] & [0.2, 0.7] \\
200 & 0.5 & 0.5 & [0.4, 0.7] & [0.3, 1.0] \\
250 & 0.8 & 0.9 & [0.7, 1.3] & [0.5, 1.8] \\
300 & 1.0 & 1.0 & [0.7, 1.3] & [0.5, 1.8] \\
350 & 0.8 & 0.8 & [0.6, 1.2] & [0.5, 1.6] \\
400 & 1.0 & 0.9 & [0.7, 1.3] & [0.5, 1.7] \\
450 & 1.0 & 1.3 & [0.9, 1.7] & [0.7, 2.3] \\
500 & 1.2 & 1.7 & [1.2, 2.4] & [0.9, 3.2] \\
550 & 1.8 & 2.6 & [1.9, 3.6] & [1.4, 4.8] \\
600 & 2.2 & 3.6 & [2.6, 5.0] & [1.9, 6.7] \\
\hline
\multicolumn{5}{c} {Matrix Element Method} \\
\hline
115 & 2.4 & 2.6 & [1.9, 3.6] & [1.4, 4.8] \\
120 & 1.6 & 1.6 & [1.1, 2.2] & [0.9, 3.0] \\
130 & 0.8 & 0.8 & [0.6, 1.1] & [0.4, 1.4] \\
140 & 0.9 & 0.5 & [0.3, 0.6] & [0.2, 0.9] \\
150 & 0.5 & 0.3 & [0.2, 0.4] & [0.2, 0.6] \\
160 & 0.2 & 0.2 & [0.1, 0.3] & [0.1, 0.3] \\
170 & 0.2 & 0.2 & [0.1, 0.3] & [0.1, 0.4] \\
180 & 0.3 & 0.2 & [0.2, 0.3] & [0.1, 0.5] \\
190 & 0.4 & 0.4 & [0.3, 0.5] & [0.2, 0.7] \\
200 & 0.6 & 0.5 & [0.4, 0.7] & [0.3, 0.9] \\
250 & 0.7 & 0.9 & [0.6, 1.2] & [0.5, 1.7] \\
300 & 0.9 & 1.0 & [0.7, 1.4] & [0.5, 1.9] \\
350 & 1.0 & 0.9 & [0.6, 1.2] & [0.5, 1.7] \\
400 & 0.7 & 1.0 & [0.7, 1.4] & [0.5, 1.8] \\
450 & 1.0 & 1.4 & [1.0, 1.9] & [0.7, 2.5] \\
500 & 1.4 & 1.9 & [1.4, 2.7] & [1.0, 3.6] \\
550 & 1.9 & 2.7 & [2.0, 3.8] & [1.5, 5.1] \\
600 & 2.8 & 4.1 & [2.9, 5.7] & [2.2, 7.6] \\
 \hline\hline
\end{tabular}
\end{center}
\caption{Multivariate shape analysis expected and observed upper limits at 95\% C.L.
for $\intlumi$ data using the BDT and matrix element outputs for the {\bf combined 0/1/2 jet bins}.}
\label{tab:me_results_5fb}
\end{table}
%%%%%%%%%%%%%%%%%%%%%%%%%%%%%%


%%%%%%%%%%%%%%%%%%%%%%%%%%%%%%
\begin{figure}[!hbtp]
\centering
\subfigure[BDT 0-Jet SF]{
\centering
\label{subfig:bdt_0jsf}
\includegraphics[width=.42\textwidth]{figures/limit_0jsf_shape_bdt-CLs-asymptotic.pdf}}
\subfigure[ME 0-Jet SF]{
\centering
\label{subfig:me_0jsf}
\includegraphics[width=.42\textwidth]{figures/limit_0jsf_shape_me-CLs-asymptotic.pdf}}
\centering
\subfigure[BDT 0-Jet OF]{
\centering
\label{subfig:bdt_0jof}
\includegraphics[width=.42\textwidth]{figures/limit_0jof_shape_bdt-CLs-asymptotic.pdf}}
\subfigure[ME 0-Jet OF]{
\centering
\label{subfig:me_0jof}
\includegraphics[width=.42\textwidth]{figures/limit_0jof_shape_me-CLs-asymptotic.pdf}}
\centering
\subfigure[BDT 1-Jet SF]{
\centering
\label{subfig:bdt_1jsf}
\includegraphics[width=.42\textwidth]{figures/limit_1jsf_shape_bdt-CLs-asymptotic.pdf}}
\subfigure[ME 1-Jet SF]{
\centering
\label{subfig:me_1jsf}
\includegraphics[width=.42\textwidth]{figures/limit_1jsf_shape_me-CLs-asymptotic.pdf}}
\centering
\subfigure[BDT 1-Jet OF]{
\centering
\label{subfig:bdt_1jof}
\includegraphics[width=.42\textwidth]{figures/limit_1jof_shape_bdt-CLs-asymptotic.pdf}}
\subfigure[ME 1-Jet OF]{
\centering
\label{subfig:me_1jof}
\includegraphics[width=.42\textwidth]{figures/limit_1jof_shape_me-CLs-asymptotic.pdf}}
\caption{The expected and observed upper limits on the SM Higgs for the shape analysis based 
the BDT and matrix element outputs~\subref{subfig:me} at 95\% C.L. for $\intlumi$ data. 
Limits are presented separately for the 4 sub-channels for each method. All limits are calculated with the asymptotic CLs method}
\label{fig:me_results_5fb_subchannel}
\end{figure}
%%%%%%%%%%%%%%%%%%%%%%%%%%%%%%


%%%%%%%%%%%%%%%%%%%%%%%%%%%%%%
%\begin{table}[!htbp]
%\begin{center}
%\begin{tabular}{c c c c c c}
%\hline\hline
% Higgs Mass   & Observed & Median expected & Expected range for 68\% & Expected range for 95\%   \\
%\hline
%\multicolumn{5}{c} {BDT Based} \\
%\hline
%115 & 3.6 & 3.2 & [2.2, 4.8] & [1.5, 7.0] \\
%120 & 2.4 & 1.8 & [1.2, 2.7] & [0.9, 4.0] \\
%130 & 0.9 & 0.9 & [0.6, 1.3] & [0.4, 1.8] \\
%140 & 0.6 & 0.5 & [0.3, 0.8] & [0.2, 1.1] \\
%150 & 0.5 & 0.4 & [0.2, 0.5] & [0.2, 0.8] \\
%160 & 0.3 & 0.2 & [0.1, 0.3] & [0.1, 0.4] \\
%170 & 0.4 & 0.2 & [0.1, 0.3] & [0.1, 0.5] \\
%180 & 0.4 & 0.3 & [0.2, 0.5] & [0.1, 0.7] \\
%190 & 0.6 & 0.5 & [0.3, 0.7] & [0.2, 1.1] \\
%200 & 0.7 & 0.7 & [0.4, 1.0] & [0.3, 1.5] \\
%250 & 1.1 & 1.3 & [0.8, 2.0] & [0.5, 2.9] \\
%300 & 1.2 & 1.3 & [0.9, 2.0] & [0.6, 2.9] \\
%350 & 1.1 & 1.2 & [0.8, 1.8] & [0.5, 2.7] \\
%400 & 1.3 & 1.3 & [0.9, 2.1] & [0.6, 3.0] \\
%450 & 1.4 & 1.9 & [1.3, 2.9] & [0.8, 4.3] \\
%500 & 1.7 & 2.8 & [1.9, 4.3] & [1.3, 6.5] \\
%550 & 2.9 & 4.3 & [2.9, 6.6] & [1.9, 10.3] \\
%600 & 3.8 & 6.5 & [4.2, 10.2] & [2.8, 15.6] \\
%\hline
%\multicolumn{5}{c} {Matrix Element Method} \\
%\hline
%115 & 2.5 & 3.2 & [2.2, 4.8] & [1.5, 7.1] \\
%120 & 2.1 & 2.0 & [1.3, 3.0] & [0.9, 4.3] \\
%130 & 0.9 & 0.9 & [0.6, 1.4] & [0.4, 2.0] \\
%140 & 1.0 & 0.6 & [0.4, 0.9] & [0.2, 1.2] \\
%150 & 0.6 & 0.4 & [0.2, 0.5] & [0.2, 0.8] \\
%160 & 0.3 & 0.2 & [0.1, 0.3] & [0.1, 0.4] \\
%170 & 0.3 & 0.2 & [0.1, 0.3] & [0.1, 0.4] \\
%180 & 0.4 & 0.3 & [0.2, 0.4] & [0.1, 0.6] \\
%190 & 0.5 & 0.4 & [0.3, 0.7] & [0.2, 0.9] \\
%200 & 0.9 & 0.6 & [0.4, 0.9] & [0.3, 1.3] \\
%250 & 0.9 & 1.2 & [0.8, 1.8] & [0.5, 2.6] \\
%300 & 1.1 & 1.4 & [0.9, 2.1] & [0.6, 3.1] \\
%350 & 1.2 & 1.2 & [0.8, 1.9] & [0.6, 2.8] \\
%400 & 1.1 & 1.4 & [0.9, 2.1] & [0.6, 3.2] \\
%450 & 1.3 & 2.1 & [1.4, 3.2] & [0.9, 4.8] \\
%500 & 2.3 & 3.2 & [2.2, 4.9] & [1.5, 7.4] \\
%550 & 3.0 & 4.9 & [3.3, 7.7] & [2.2, 11.7] \\
%600 & 5.0 & 7.8 & [5.0, 12.1] & [3.3, 19.2] \\
%\hline\hline
%\end{tabular}
%\end{center}
%\caption{Multivariate shape analysis expected and observed upper limits at 95\% C.L.
%for $\intlumi$ data using the BDT and matrix element outputs for the {\bf 0-jet bin}.}
%\label{tab:me_results_5fb_0j}
%\end{table}
%%%%%%%%%%%%%%%%%%%%%%%%%%%%%%

%%%%%%%%%%%%%%%%%%%%%%%%%%%%%%
%\begin{table}[!htbp]
%\begin{center}
%\begin{tabular}{c c c c c c}
%\hline\hline
% Higgs Mass   & Observed & Median expected & Expected range for 68\% & Expected range for 95\%   \\
%\hline
%\multicolumn{5}{c} {BDT Based} \\
%\hline
%115 & 6.1 & 7.8 & [5.3, 11.9] & [3.8, 18.3] \\
%120 & 3.5 & 4.0 & [2.7, 6.1] & [1.9, 9.1] \\
%130 & 1.4 & 1.5 & [1.0, 2.2] & [0.7, 3.3] \\
%140 & 0.8 & 0.9 & [0.6, 1.3] & [0.4, 1.8] \\
%150 & 0.8 & 0.6 & [0.4, 0.9] & [0.3, 1.3] \\
%160 & 0.5 & 0.3 & [0.2, 0.4] & [0.1, 0.6] \\
%170 & 0.4 & 0.3 & [0.2, 0.5] & [0.2, 0.7] \\
%180 & 0.6 & 0.5 & [0.3, 0.7] & [0.2, 1.0] \\
%190 & 0.9 & 0.7 & [0.5, 1.0] & [0.3, 1.5] \\
%200 & 1.3 & 1.0 & [0.7, 1.5] & [0.5, 2.3] \\
%250 & 2.2 & 2.1 & [1.4, 3.1] & [1.0, 4.3] \\
%\hline
%\multicolumn{5}{c} {Matrix Element Method} \\
%\hline
%115 & 4.1 & 7.6 & [5.1, 12.3] & [3.6, 19.2] \\
%120 & 2.8 & 4.3 & [2.8, 6.2] & [2.0, 9.3] \\
%130 & 1.3 & 1.6 & [1.1, 2.4] & [0.7, 3.6] \\
%140 & 0.9 & 1.0 & [0.7, 1.5] & [0.4, 2.2] \\
%150 & 0.7 & 0.6 & [0.4, 0.9] & [0.3, 1.3] \\
%160 & 0.4 & 0.3 & [0.2, 0.4] & [0.1, 0.6] \\
%170 & 0.3 & 0.3 & [0.2, 0.5] & [0.1, 0.7] \\
%180 & 0.5 & 0.4 & [0.3, 0.6] & [0.2, 0.9] \\
%190 & 0.7 & 0.7 & [0.5, 1.0] & [0.3, 1.5] \\
%200 & 1.1 & 0.9 & [0.6, 1.4] & [0.4, 2.0] \\
%250 & 1.6 & 1.9 & [1.3, 2.9] & [0.9, 4.4] \\
%\hline\hline
%\end{tabular}
%\end{center}
%\caption{Multivariate shape analysis expected and observed upper limits at 95\% C.L.
%for $\intlumi$ data using the BDT and matrix element outputs for {\bf 1 jet bin}.}
%\label{tab:me_results_5fb_1j}
%\end{table}
%%%%%%%%%%%%%%%%%%%%%%%%%%%%%%
%

\subsubsection{Correlation of BDT and ME results}
\label{sec:mebdt_correlations}

The sensitivity of the matrix element method to a Higgs boson signal
is consistent with the BDT-based approach.
However we observe a difference in the observed limit from the two methods
 at low Higgs boson mass.     
The BDT shape analysis finds an excess of about 1$\sigma$ in the region $m_H<=130$\GeV,
which is not present in the matrix element based shape analysis.
When comparing these two results, we note that the two methods use different kinematic
inputs and are expected to weight the observable properties of a Higgs boson signal differently.
The inputs to the BDT are documented in detail in Ref.~\cite{ref:HWW2011smurf}.
The correlations between the BDT and Matrix Element based discriminants and the
relevant kinematic observables are shown in Table~\ref{tab:bdt_me_corr}.


\begin{table}[!htbp]
\begin{center}
\begin{tabular}{c c c}
\hline
Kinematic Observables & BDT & Matrix Element \\
\hline
$m_{ll}$ & 0.74 & 0.50 \\
leading lepton $p_T$ & 0.06 & 0.23 \\
trailing lepton $p_T$ & 0.42 & 0.57 \\
dilepton $\delta\phi$ & 0.45 & 0.15 \\
$\met$  & 0.07 & 0.13 \\
\hline
\end{tabular}
\end{center}
\label{tab:bdt_me_corr}
\caption{The absolute linear correlations between the BDT and matrix-element discriminants and the
event observables based on the Higgs MC events with mass of 120 \GeV. }
\end{table}

\subsubsection{Signal injection studies}

Motivated by the excess observed in data in the $H\to\gamma\gamma$ channel, the upper limits on the SM Higgs have been 
recalculated after the injection of the MC signal with 
$m_H=124$\GeV. In each data card, we substitute the nubmer of 
observed events with the expected events combining the SM 
background and signal. As shown in Figure~\ref{fig:me_results_hww124_5fb}, 
we observe a wide excess between 1 and 2 $\sigma$ between [115-150]\GeV, consistently 
using the cut-based analysis and shape analysis using both BDT and 
matrix element outputs. This is expected as the mass resolution of the 
$H\to WW$ is as large as 20\%. 

We thus conclude that the observed limits in the case of the presence of signal,
as well as the expected limits in the absence of signal, should respond
similarly in the BDT and matrix element methods. Because 
the output of the two methods correlates differently with the 
event observables as described in Section~\ref{sec:mebdt_correlations}, 
the two methods can be used to cross check each other.

%%%%%%%%%%%%%%%%%%%%%%%%%%%%%%
\begin{figure}[!hbtp]
\centering
\subfigure[Cut Based]{
\label{subfig:cut}
\includegraphics[width=.45\textwidth]{figures/limit_nj_cut_hww124-CLs-asymptotic.pdf}} \\
\centering
\subfigure[BDT]{
\centering
\label{subfig:bdt}
\includegraphics[width=.45\textwidth]{figures/limit_nj_shape_bdt_hww124-CLs-asymptotic.pdf}}
\subfigure[ME]{ 
\centering
\label{subfig:me}
\includegraphics[width=.45\textwidth]{figures/limit_nj_shape_me_hww124-CLs-asymptotic.pdf}}
\caption{ expected and observed upper limits on the SM Higgs for cut-based~\subref{subfig:cut} and 
shape analysis based the BDT~\subref{subfig:bdt} and matrix element outputs~\subref{subfig:me} 
at 95\% C.L. after the injection of $m_H=124$\GeVcc signal. All limits are calculated with the 
asymptotic CLs method}
\label{fig:me_results_hww124_5fb}
\end{figure}
%%%%%%%%%%%%%%%%%%%%%%%%%%%%%%
