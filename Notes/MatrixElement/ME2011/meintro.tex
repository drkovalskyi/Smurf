The Standard Model (SM) of elementary particles describes a universe in which fermions, the fundamental 
constituents of  matter, interact via fundamental forces propagated by gauge bosons. This description of elementary
particles and their interactions has been validated extensively through precision experiments and found to be incredibly
successful at describing the particle physics world. However, experimentally, there remains a missing piece of the
puzzle yet to be observed.  The SM incorporates a mechanism for symmetry breaking, known as the Higgs mechanism 
\cite{ref:HiggsMechanism,ref:HiggsMechanism1}, which naturally predicts the massive W and Z bosons as propagators of the weak 
force and provides a mechanism through which leptons and quarks are also able to acquire mass. An additional consequence of 
the Higgs mechanism is the existence of an associated massive scalar boson known as the Higgs boson, which is the only unobserved particle within
the SM framework.  Because of the central role of the Higgs mechanism in the SM, the discovery (or non-discovery) of the
Higgs boson along with additional studies of its properties will either confirm the current model or point toward new physics 
processes at the TeV energy scale.

The SM postulates the existence of the Higgs boson, but does predict its mass, which is a free parameter in the model. 
However, experimental measurements provide constraints on the Higgs mass.  Global fits to precision electroweak measurements 
set a one-sided 95$\%$ confidence level upper limit on Higgs boson mass at 157 \GeVcc \cite{ref:GlobalEwkConstraints}.
These constraints are mostly driven by measurements of the top quark and $W$ boson masses. Moreover, direct searches
at LEP have excluded a Higgs boson with mass below $114.4$ \GeVcc at 95$\%$ confidence level \cite{ref:LepExclusion}, and the Tevatron
has reported a preliminary update which extends the exclusion region for a Higgs boson mass between 156 and 177 \GeVcc \cite{ref:TevExclusion}.
Therefore, probing the mass range between 115 and 200 \GeVcc is crucial for Higgs boson searches.

Regardless of the mass, the dominant mechanism for production of the Higgs boson is gluon fusion (GF)\cite{ref:GF1,ref:GF2}, where the Higgs
is produced from a pair of gluons via a quark (primarily top-quark) loop. Another production mechanism is vector boson 
fusion (VBF) \cite{ref:VBF}, where the Higgs boson is produced from a pair of $W$ or $Z$ bosons. The relative contributions of GF and VBF
to the total Higgs boson production rate depend on the mass, but are approximately 90$\%$ and 10$\%$ over the range allowed
by electroweak constraints.  The Higgs boson can also be produced via associated production \cite{ref:VH1,ref:VH2} and top-quark fusion, however
contributions from these processes in the range of interest are small.

The Higgs decay mode is mostly determined by its mass \cite{ref:Hdecay}. While the diphoton decay channel dominates at low masses, searches 
in this channel face complications arising from large backgrounds and require excellent understanding of the electromagnetic
calorimeter response. For a SM Higgs boson mass $m_{H}>125$ GeV the $WW$ decay channel dominates and provides the opportunity of observing a 
Higgs signal in a wide range of masses between 120 and 300 \GeVcc.  The fully leptonic final state $H \rightarrow WW 
\rightarrow l^{+}l^{-}\nu\bar{\nu}$ provides a clean dilepton signature that is relatively easy to trigger on and has lower backgrounds compared to
semi-leptonic and hadronic final states. With a large signal yield and manageable backgrounds, 
this final state is the most promising discovery channel from early CMS data. 

The backgrounds in the $H \rightarrow WW \rightarrow l^{+}l^{-}\nu\bar{\nu}$ search are diboson production ($WW$, $WZ$ and $ZZ$), top-pair and 
single-top production, $W+$jet production where a hadronic jet fakes the signature of a lepton, $W+\gamma$ where a photon 
converts and produces a pair of electrons, one of which is not reconstructed, and Drell-Yan processes.  The most significant are
non-resonant $WW$ production and $W+$~jets.

While current fits to electroweak parameters prefer a low mass Higgs boson, the presence of non-SM physics at a higher mass scale can significantly alter
 the constraints. A search for the SM Higgs boson in the $H \rightarrow ZZ \rightarrow l^{+}l^{-}\nu\bar{\nu}$ final state provides sensitivity over the large
range of Higgs masses from 250 to 600 \GeVcc. The background in this mode is coming primarily from diboson processes as well as from Drell-Yan events with fake missing transverse energy due to the lepton/jet energy and direction mis-measurements. 

In this note we present a search for the standard model Higgs boson in the $WW \rightarrow ll\nu\bar{\nu}$ and $ZZ \rightarrow ll\nu\bar{\nu}$ final states 
using a Matrix Element (ME)-based discriminant. Triggers, event pre-selection, and reconstruction of high level objects are described 
in detail in notes \cite{ref:HWW2011smurf}, \cite{ref:HZZ2011smurf} and references within. Here, we introduce the ME technique, 
describe its technical implementation, summarize the event selection, present results and compare them to the ones obtained with 
other methods. The reader is first introduced to the method using the $H\rightarrow WW \rightarrow ll\nu\bar{\nu}$ search; the
$H \rightarrow ZZ \rightarrow ll\nu\bar{\nu}$ search is then described in a subsequent section without repeating details common to the two analyses.
