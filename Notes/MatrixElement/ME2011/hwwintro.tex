The kinematic properties of Higgs boson events tend to be different from those of events produced in background processes.
Namely, due to spin correlations, Higgs events produced via gluon fusion tend to have a small angular separation between 
leptons, with large missing transverse energy recoiling against the dilepton pair. Background events typically have significantly
larger dilepton angular separation.  Moreover,  the transverse momenta of leptons produced in Higgs decays tend to be smaller
than those produced in most of the backgrounds. As a consequence of this, the dilepton invariant mass is on average smaller for Higgs events
than for $q\bar{q} \rightarrow WW$ events, which dominate the background.

While a cut-based approach attempts to find an optimal set requirements on lepton $p_{T}$, lepton pair azimuthal separation ($\Delta\phi_{ll}$), 
dilepton invariant mass  ($m_{ll}$) and \met, it is beneficial to use all the information about event kinematics in a correlated way. 
For example, if the value of $\Delta\phi_{ll}$ in an event is Higgs-like for a certain $m_{H}$ hypothesis, then other parameters, such as  
\met, should be consistent with this same hypothesis. By building a multivariate discriminant, which explores the full event kinematics
in a correlated way, one can achieve better signal versus background separation than with a standard cut-based approach and, 
therefore, improve the analysis sensitivity to a standard model Higgs boson signal.

One way to create such a discriminant is by employing a Matrix Element method. This method has been previously used in precision
measurements of the top quark mass \cite{ref:CDFTopMass,ref:D0TopMass} and cross-section, diboson cross section \cite{ref:CDFDiboson}, 
as well as searches for single top \cite{ref:CDFSingleTop,ref:D0SingleTop} and Higgs boson production at the Tevatron \cite{ref:CDFHiggs,ref:D0Higgs}.

In this method we calculate the probability for each recorded event to originate from a specific physics process.  
This is done by comparing the differential cross section predicted by a Matrix Element calculation for the signal and background processes given the kinematic observables
on an event-by-event basis. The discriminating power arises because the differential cross sections for signal and background
events are largest in different regions of the available phase space. The principal difference between the ME technique and other
multivariate approaches (Artificial Neural Networks, Boosted Decision Trees, etc.) is that the final discriminant is based on the results
of physics calculations rather than various statistical methods used to solve regression problems. It is free of potential ``over-training''
dangers and can be tested on processes other than Higgs production, with larger cross-sections. However, it does have its own complication - 
theoretical Matrix Element calculations are performed at the parton level, while we reconstruct events at the detector level. 
Therefore, to achieve better discrimination, it is necessary to implement transfer functions as well as detector resolution functions for
reconstructed objects in the ME-based discriminant.

The fully leptonic $H\rightarrow WW$ final state consists of two isolated leptons and large missing energy from the two undetectable neutrinos. 
The event selections are documented in Ref.~\cite{ref:HWW2011smurf}
