In this note we have presented a technique to distinguish the Standard Model Higgs boson signal
from its primary backgrounds. The method is based on the evaluation of Matrix Element based event probabilities 
which allows the exploration of kinematic shapes of events in a correlated way. We demonstrate the
performance of the method in $H \rightarrow WW \rightarrow l^{+}l^{-}\nu\bar{\nu}$ and $H \rightarrow ZZ \rightarrow l^{+}l^{-}\nu\bar{\nu}$ searches with 4.7 fb$^{-1}$ of data and compare the achieved sensitivity to cut-and-count and shape-based analyses.  

To validate the technique, we apply it to the measurement of diboson $WZ+ZZ$ cross-sections which is extracted from
the matrix element based discriminant. The measured value of the cross-section is in good agreement with theoretical NLO prediction.

We demonstrate that, with all systematic uncertainties taken into account, use of the matrix element method 
leads to a $15-30 \%$ improvement in the sensitivity when compared to cut-and-count based approach in both analyses. 
When compared to other shape analyses, it gives performance similar to that of the BDT in $H\rightarrow WW$, and performs 10\% better 
than $m_{T}$ at 250-300 GeV Higgs mass in $H\rightarrow ZZ$, with equal to $m_{T}$ performance at higher masses. 
Overall, using full 2011 4.7 $fb^{-1}$ data sample, we exclude Standard Model Higgs boson with 
mass 128--270 GeV (in $H\rightarrow WW$) and 310--500 GeV (in $H\rightarrow ZZ$) at 95\% C.L.

