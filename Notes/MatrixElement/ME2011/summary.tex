
In this note we have presented a search for the Standard Model
Higgs boson in the $WW \rightarrow l^{+}l^{-}\nu\bar{\nu}$ and
$ZZ \rightarrow l^{+}l^{-}\nu\bar{\nu}$ decay channels,
using the Matrix Element method.  This method is based on the
evaluation of the differential cross section for signal and 
background hypotheses on an event by event basis.
To separate the Higgs boson signal from background processes,
we used the differential cross sections to calculate a likelihood
ratio discriminator.

To validate the method, we applied it to the measurement of the 
diboson $WW+ZZ$ cross section in the $l^{+}l^{-}\nu\bar{\nu}$
decay channel. To perform this measurement, we constructed a modified
likelihood ratio with the diboson processes as the signal, then
counted the number of signal like events. After subtracting the 
expected background contribution, the measured value of the 
cross section was found to agree well with the theoretical prediction.

We compared the expected performance of the shape analysis of the 
likelihod ratio discriminator in the $WW$ and $ZZ$ decay channels 
to the existing results. We find equivalent performance to the 
BDT shape analysis in the $WW$ decay channel and up to $10$\% improvement
in the $ZZ$ decay channel for Higgs boson masses from $250-300$ GeV/c$^{2}$.
Using the full 2011 data sample corresponding to $\intlumi$, 
we are able to exclude the Standard Model Higgs boson in the mass 
range $128-350$ GeV/c$^{2}$ in the $WW$ decay channel and 
$310-500$ GeV/c$^{2}$ in the $ZZ$ decay channel at the 95\% C.L.
