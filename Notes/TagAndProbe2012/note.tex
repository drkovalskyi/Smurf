\documentclass[11pt,a4paper]{article}
\usepackage{graphicx}
\usepackage{verbatim}
\usepackage{amsmath}
\usepackage{amssymb}
\usepackage{subfigure}
\usepackage{multirow}
\usepackage[pdfborder=0 0 0,
            colorlinks,
            urlcolor = blue,
            linkcolor = black,
            citecolor = black,
            menucolor = black,]
           {hyperref}
\usepackage[toc,page]{appendix}
\renewcommand{\appendixname}{Appendix}
\setlength\parindent{0pt}

\setcounter{topnumber}{1}
\setcounter{bottomnumber}{1}

%===================================================================================================
\begin{document}

%\begin{titlepage}
\title{Tag and Probe Methodology}
\author{A.~Yagil, D.~Evans, S.~Xie}
%\end{titlepage}

\maketitle

\tableofcontents
\newpage 

%===================================================================================================
\section{Methodology}

\subsection{Triggers}
{\it Which trigger paths should be used for which run ranges?}\\

\begin{itemize}
    \item For 2011 data, I think the tag and probe trigger is easier because it's
always existing while the single electron trigger fluctuated a lot
sometimes disappearing altogether from the menu.
    \item For the 2012 data, it would be nice if we can get this proposed single ele
unprescaled at 60Hz. Using this trigger is the obvious simplest choice.
\end{itemize}

{\it Do we need to take into account prescaling?}\\

\begin{itemize}
    \item For 2011 data, there is no problem if using  Ele32\_SC17 because
this trigger was always unprescaled (we think).  This trigger should not
require prescaling in the future if the tag leg is kept tight.  There may be issues
for the triggers with a lower $p_{T}$ leg.  If necessary the control dataset 
can be re-weighted to the analysis dataset according to the number of reconstructed
vertices.
    \item In general if the prescale is constant accross the entire dataset
to be analysed then no additional re-weighting is required.  It is better to have a 
trigger with a constant prescale rather than an initially small but growing prescale.
\end{itemize}


{\it Do we introduce special treatment of double L1-seeded signal triggers?}\\
\begin{itemize}
    \item If the L1 decision is required simultaneously on two legs then there is 
a correlation between the two legs and the tag must be picked a-priori to
avoid bias.  Reference somewhere what we did for muons.  It is the same problem.
\end{itemize}


\subsection{Binning}

{\it Set of variables in which we bin the probes} \\
{\it 1D, 2D, multi-dimensional?} \\
{\it Bin boundaries} \\

\begin{itemize}
    \item These are all analysis dependent.
\end{itemize}

\subsection{Tag and probe selection}

{\it The exact selection for the tag} \\
{\it Requirements for the probe (for efficiency demonimnator)} \\
{\it Require opposite sign for the tag-probe pair?} \\
{\it Invariant mass requirement} \\

\begin{itemize}    
    \item Fix the overall window within which events are selected (including
side-bands for background fitting/extraction) to 60-120 GeV.  
The low mass threshold is motivated by avoiding kinematic turn-on effects.
    \item Also there is the subtle issue of which window you use to make the
efficiency measurement (as opposed to the window used to fit). That
question is extremely delicate and is very analysis dependent.
\end{itemize}

{\it How to choose between multiple tag-probe pairs (closest to mZ / random / use all)?} \\

\begin{itemize}
    \item As long as same procedure is used in data and simulation, in principle the 
effect on extracted scale factors should be small.  
\end{itemize}


\subsection{MC treatment}

{\it generator level matching requirements}\\

\begin{itemize}
    \item It is not clear the generator level matching is required.  
The fraction of probes in simulation from fake leptons from additional jets
is likely to be small.  If absolutely necessary, an anti-matching could
be performed between partons and the reconstructed probe.
\end{itemize}

{for efficiency not binned in PU, how is the difference in PU between data and MC is taken care of?}\\

\begin{itemize}
    \item Run on the same dataset you are using in your analysis if possible (see above point 
in case prescale of control triggers changes).
\end{itemize}

{\it Which efficiencies are computed from counting and which from fits?}

\subsection{Fitting}

{\it Fit range} \\

\begin{itemize}
    \item Probably a common sense recommendation can be made.
\end{itemize}


{\it Binned or unbinned} \\

\begin{itemize}
    \item It's a technical choice. As long as 
you apply the same fit in both data and simulation either is fine
If speed is not an issue unbinned fit may be preferred.
\end{itemize}

{\it simultaneous pass+fail or separate}

\begin{itemize}
    \item The simultaneous fit is probably better, since correlated uncertainties
can be better handled, and the result (efficiency) can be constrained to be 
less than or equal to one.
\end{itemize}

{\it The exact definition of signal PDF} \\

\begin{itemize}
    \item It is necessary to take into account the expected signal shape 
from simulation with an additional smearing term to take into account difference in 
resolution and energy scale between data and simulation.
\end{itemize}


{\it The exact definition of background PDF} \\

\begin{itemize}
    \item  It has been shown (citation?) that there is no significant difference to
the result between exponential and exponential*linear models.  Other models 
could be tested as part of systematics
\end{itemize}


\subsection{Systematic uncertainties}

{\it Extrapolation of efficiencies from electrons to photons} \\

\begin{itemize}    
    \item We have not considered this.
\end{itemize}

{\it Dependence on tag selection} \\

\begin{itemize}
    \item Changing the tag will probably not have any effect, unless it is a
proxy for a background normalisation/modelling uncertainty.
\end{itemize}

{\it Dependence on choice of fit functions} \\

\begin{itemize}
    \item Probably should be investigated.
\end{itemize}

{\it Background subtraction when counting is used instead of a fit} \\

\begin{itemize}    
    \item This question is mis-phrased. if there is background it has to be
subtracted using an appropriate method. if the background is negligible,
then it's irrelvant. If there is background and you don't subtract it
then it's clearly biased, and then you have to cover the bias with a systematic error.
\end{itemize}

{\it Are the biases in measuring efficiency as a function of pT from Z->ee?} \\

\begin{itemize}    
    \item We are not clear what this question is referring to?  Isolation efficiency
extraction at low $p_{T}$?
\end{itemize}

\section{Central Measurements}

\begin{itemize}
    \item The only measurement it makes sense to do centrally with a standard
definition is the one that is common to everyone - the reconstruction
efficiency.  In other cases the definition of tag and probe is motivated
by the analysis selection used.
\end{itemize}


%===================================================================================================
\clearpage

\vspace*{-0.2cm}
\thebibliography{12}

\bibitem{pdg}
 K. Nakamura et al. (Particle Data Group), "Review of particle physics", J. Phys.G37 , 2010.

\bibitem{Higgs1}
F. Englert and R. Brout, "Broken symmetries and the masses of gauge bosons", Phys. Rev. Lett. 13,  1964.

\bibitem{Higgs2}
P. W. Higgs, "Broken symmetry and the mass of gauge vector mesons", Phys. Rev. Lett. 13, 1964.

\bibitem{Higgs3}
Guralnik, G.S. and Hagen, C.R. and Kibble, T.W.B., "Global Conservation Laws and Massless Particles", 
Phys.Rev.Lett. 13, 1964.

\bibitem{HWW2010}
CMS Collaboration, "Title: Measurement of WW Production and Search for the Higgs Boson in 
pp Collisions at $\sqrt{s}$ = 7 TeV", arXiv:1102.5429

\bibitem{VBTFCrossSectionNote}
J. Alcaraz Maestre, \textit{et al.}, "Updated Measurements of Inclusive W and Z Cross Sections 
at $\sqrt{s}=7$ TeV", CMS AN-2010/264.

\bibitem{ggWWError}
F.~ Stoeckli, "http://indico.cern.ch/getFile.py/access?contribId=0\&resId=1\&materialId=slides\&confId=49009", 
EWK Diboson meeting of March 12 2009.

\bibitem{json}
{\small
/afs/cern.ch/cms/CAF/CMSCOMM/COMM\_DQM/certification/Collisions11/7TeV/Prompt/Cert\_160404-163869\_7TeV\_PromptReco\_Collisions11\_JSON.txt
}

\bibitem{ElIso}
A. Vartak, M. LeBourgeois, V. Sharma, "Lepton Isolation in the CMS Tracker, ECAL and HCAL", CMS AN-2010/106.

\bibitem{PVDA}
W. Erdmann, M. LeBourgeois, B. Mangano, 
https://indico.cern.ch/getFile.py/access?contribId=5\&sessionId=3\&resId=1\&materialId=slides\&confId=127127, 
note in preparation.

\bibitem{NExpHits}
B. Mangano \textit{et al.}, "Improvement in Photon Conversion Rejection Performance Using 
Advanced Tracking Tools", AN-10-283.

\bibitem{fakeLeptonNote1}
S.~Xie, \textit{et al.}", "Study of Data-Driven Methods for Estimation of Fake Lepton Backgrounds", 
CMS AN-2009/120.

\bibitem{fakeLeptonNote2}
W.~Andrews, \textit{et al.}, "Fake Rates for dilepton Analyses", CMS AN-2010/257.

\bibitem{fakeLeptonBkgSpillage1}
 F. Golf, D. Evans, J. Mulmenstadt  \textit{et al.}, ``Expectations for observation of top quark pair production in the dilepton final state with the early CMS data'', CMS AN-2009/050.

\bibitem{dyestnote}
W. Andrews, et al., “A Method to Measure the Contribution of $\dyll$ to a di-lepton+ MET Selection”, CMS AN-2009/023 (2009).

\bibitem{jes}
CMS Collaboration, "Jet Energy Calibration with Photon+Jet Events", PAS JME-09-004.

\bibitem{jetpas}
CMS Collaboration, "Jet Performance in pp Collisions at $\sqrt{s}=7 \rm\ TeV$", PAS JME-10-003.

\bibitem{btag}
CMS collaboration, "Commissioning of b-jet identification with pp collisions at $\sqrt{s}=7~\TeV$, BTV-10-001.

\bibitem{antikt}
Cacciari, Matteo and Salam, Gavin P. and Soyez, Gregory, "The anti-$k_t$ jet clustering 
algorithm", JHEP 04,  2008.

\bibitem{ConversionNote}
W.~Andrews, \textit{et al.}, "Study of photon conversion rejection at CMS", CMS AN-2009/159.

\bibitem{tmva}
A. Hoecker, \textit{et al.}, "TMVA - Toolkit for Multivariate Data Analysis", arXiv:physics/0703039, 2007.

\bibitem{XS}
CMS Generator group, Standard Model Cross Sections for CMS at 7 TeV, 2010.

\bibitem{PDF4LHC}
PDF4LHC Working Group, 
{\tt http://www.hep.ucl.ac.uk/pdf4lhc/PDF4LHCrecom.pdf}

\bibitem{Nadolsky:2008zw}
Nadolsky, Pavel M. and others, "Implications of CTEQ global analysis for 
collider observables", Phys. Rev. D78 2008.

\bibitem{Martin:2009iq}
Martin, A. D. and Stirling, W. J. and Thorne, R. S. and Watt, G., "Parton 
distributions for the LHC, Eur. Phys. J. C63 2009.

\bibitem{Ball:2010de}
Ball, Richard D. and others, "A first unbiased global NLO determination 
of parton distributions and their uncertainties", arXiv 1002.4407.

\bibitem{bayesian}
A. O'Hagan and J.J. Forster, "Bayesian Inference", Kendall's Advanced Theory of Statistics, 
Arnold, London, 2B, 2004.

\bibitem{ref:tagprobe_mit_w}
G. Bauer {\it et. al.}, "Lepton ef?iencies for the inclusive W cross section measurement with 36.1pb$^{-1}$", AN2011/097

\bibitem{ref:tagprobe_snt_top}
W. Andrews {\it et. al.}, "Uncertainties on the Lepton Selection Efficiency for t$t\bar{t}$ Cross Section Analysis", AN2010/274

\bibitem{LHCHiggsCrossSectionWorkingGroup:2011ti}
LHC Higgs Cross Section Working Group, "Handbook of LHC Higgs Cross Sections: 
Inclusive Observables", CERN-2011-002, 2011.

\bibitem{PFMET} 
CMS Collaboration, ``CMS MET Performance in Events Containing Electroweak Bosons from pp Collisions at $\sqrt{s}=7$ TeV'', CMS PAS JME-2010-005 (2010)


\bibitem{trkMET} 
Marco Zanetti, ``MET with PU in $\hww\to2\ell$'', https://indico.cern.ch/conferenceDisplay.py?confId=131580
Benjamin Hooberman, ``MET with PU in MC and First 2011 Data'', https://indico.cern.ch/contributionDisplay.py?contribId=5\&confId=132579. 


\bibitem{lands}
Mingshui Chen and Andrey Korytov, https://mschen.web.cern.ch/mschen/lands/

\bibitem{MCFMHiggsProduction}
J. Campbell, R.K. Ellis, G. Zanderighi, ``Next-to-Leading order Higgs + 2 jet production via gluon fusion.'', JHEP 0610:028 (2006), hep-ph/0608194
%===================================================================================================

\end{document}

